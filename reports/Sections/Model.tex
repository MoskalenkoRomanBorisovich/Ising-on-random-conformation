\section{Описание модели}

В данной модели мы рассматриваем ансамбли конформаций на двумерной решётке: множества конформаций одинаковой длинны $L$, полученные при одинаковых температурах. Мы получаем конформации используя алгоритм SAW, где структура конформации зависит от температуры. В работе мы будем использовать обозначение $U = 1/T$ являющееся обратной температурой.
На каждой из конформаций строится модель Изинга. В каждой вершине рассматриваемой конформации размещается спин, который может принимать одно из двух значений: $+1, -1$. Соседние спины взаимодействуют друг с другом, сила взаимодействия определяется коэффициентом $J$. В данной модели соседними спинами мы считаем спины расположенные строго слева, справа, снизу или сверху друг от друга.
Гамильтониан данной системы имеет вид
\[H = -J\sum_{\langle i, j\rangle}{\sigma_i\sigma_j} - h\sum_i{\sigma_i} \]

где $i, j$ индексы соседних узлов, $J$- коэффициент взаимодействия $h$ - воздействие внешнего поля. В данной работе мы рассматриваем модель без внешнего поля, поэтому во всех вычислениях далее $h = 0$ 

Статистическая сумма
\[Z = \sum_{\{\sigma\}} e^{-H(\sigma)\beta}, \beta = \frac{1}{kT}\]
где $\{\sigma\}$ -- множество всех возможных наборов значений спинов. $\beta$ -- обратная температура.

Намагниченность и энергия каждого состояния считаются по следующим формулам

\[ 
E = -J\sum_{i, j} \sigma_i \sigma_j
\]
\[
M = \sum_i \sigma_i
\]

Средняя намагниченность системы

\[
\langle M \rangle = \frac{1}{Z}  \sum_{\{\sigma\}} M e^{-H(\sigma)\beta}
\]

Заметим, что средняя намагниченность равна 0, так как для каждого состояния спинов существует равновероятное состояние с обратными значениями. Поэтому в основном мы будем рассматривать модуль намагниченности и его начальные моменты.


\subsection{Численная модель}
Для расчёта модели Изинга используется алгоритм основанный на методе Монте-Карло. Были реализованы версии с односпиновым и кластерным апдейтом \cite{wolf_algorithm}. Код представлен в репозитории github \cite{github}. 

Алгоритм с кластерным апдейтом работает следующим образом. На каждой итерации мы выбираем случайный спин, и с него начинаем строить кластер из одинаково направленных спинов, добавляя новые спины в кластер с вероятностью $p = 1 - \exp(-2\beta)$. Затем мы меняем значения спинов в кластере на противоположные. В отличие от односпинового апдейта, где есть вероятность, что изменение значения спина не будет принято, в кластерном апдейте новые значения спинов принимаются всегда и в кластере гарантированно будет находиться как минимум один спин.
В итоге для измерений используется кластерная версия. Благодаря отказоустойчивости и изменению значения сразу нескольких спинов, она работает значительно быстрее, и быстрее сходится. Это особенно важно при низких температурах, так как большинство спинов будут сонаправлены, и за одну итерацию, будут составляться крупные кластеры, меняющие значение большого числа спинов. 
В обоих алгоритмах мы сначала случайным образом инициализируем спины, затем делаем некоторое число шагов для отжига модели. Далее на каждом шаге мы замеряем намагниченность, и после выполнения определённого числа шагов, усредняем полученные значения. Основной интерес для нас представляют начальные моменты модуля намагниченности. А именно 1, 2 и 4.

\[
\langle |M|\rangle = \frac{1}{n} \sum_{\{\sigma\}} \left| \sum_i \sigma_i \right|
\]
\[
\langle M^2\rangle = \frac{1}{n} \sum_{\{\sigma\}} \left( \sum_i \sigma_i \right)^2
\]