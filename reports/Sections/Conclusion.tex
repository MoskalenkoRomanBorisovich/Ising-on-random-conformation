\section{Заключение}

В данной работе мы представляем новую модель магнитных конформаций и описываем методы, которые будут использоваться для определения ее магнитных свойств. Мы показали, что ансамбли клубковых конформаций не являются магнитными и не имеют магнитного фазового перехода. Для глобул магнитная восприимчивость может быть полезна для определения точки перехода. А эксперименты с кумулянтом Биндера позволили найти новую корреляцию между геометрическими и магнитными свойствами конформации.
Исследование Свойств конформаций, полученных при различных температурах подтверждают, связь между геометрическим переходом в конформациях, и появлением в них ферромагнитных свойств.
Результаты этой работы могут быть применены к исследованию и производству материалов, а также к биологическим исследованиям. В будущем мы можем получить более детальное представление о магнитной восприимчивости и более длинных конформациях, усовершенствовать алгоритм моделирования Изинга, глубже изучить взаимосвязь между геометрической структурой конформаций и их магнитными свойствами. 