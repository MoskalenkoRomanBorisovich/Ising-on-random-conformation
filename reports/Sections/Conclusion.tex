\section{Заключение}

В данной работе мы представляем новую модель магнитных конформаций и описываем методы, которые будут использоваться для определения ее магнитных свойств. Мы показали, что ансамбли клубковых конформаций не являются магнитными и не имеют магнитного фазового перехода. Для глобул магнитная восприимчивость может быть полезна для определения точки перехода. А эксперименты с кумулянтом Биндера позволили найти связь между разбиением конформации на кластеры и её магнитными свойствами.
Исследование свойств конформаций, полученных при различных температурах, подтверждают связь между геометрическим переходом в конформациях и появлением в них ферромагнитных свойств. Так же видно, что при всех температурах существуют как намагничивающиеся, так и ненамагничивающиеся конформации. Пока что нельзя сказать, как именно доля этих конформаций зависит от температуры.