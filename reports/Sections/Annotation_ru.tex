\section*{Аннотация}
% Модель Изинга — относительно новая модель, используемая для определения магнитных свойств материалов и объектов. В данной работе мы изучаем свойства модели Изинга на ансамблях конформаций. Конформация — это случайное блуждание по регулярной сетке, которая может представлять молекулу. Структура конформации зависит от температуры. Другие исследования показали, что конформации имеют геометрический фазовый переход. Два состояния конформации при низких и высоких температурах называются глобулой и клубком соответственно. Геометрически конформации глобулы и клубка подобны одномерным и двумерным сеткам. Так как большинство вершин в глобулах имеют 4 соседей, а в клубках – 2. Доказано, что на одномерной сетке модель Изинга не имеет магнитного фазового перехода. Наша гипотеза состоит в том, что глобулярные конформации имеют магнитный фазовый переход. Но на двумерной сетке фазовый переход существует. Целью данного исследования является определение точки магнитного фазового перехода в глобулярных конформациях и сравнение ее с точкой геометрического фазового перехода в конформациях.


В данной работе рассмотрена модель Изинга на ансамблях конформаций. 
Интерес в данной работе представляет связь структурных свойств конформаций и магнитных свойств модели Изинга, построенной на конформациях. В частности влияние фазового геометрического перехода на появление магнитного перехода. По скольку при низких температурах структура конформаций схожа с двумерной решёткой, и известно что модель Изинга на двумерной решётке имеет фазовый переход. И при высоких температурах конформации схожи с одномерной решёткой, на которой переход в модели Изинга отсутствует.
В отличие от других работ в данной области, где спиновая подсистема модели Изинга и структура конформации меняются одновременно, мы рассматриваем модель с замороженным беспорядком. Так в нашей модели сначала генерируется конформация, и затем с фиксированной геометрией рассматривается модель Изинга.
Результатами работы являются: магнитне свойства конформаций в каждой из фаз, точка магнитного перехода в глобулярной фазе, изменение магнитных свойств конформаций при геометрическом фазовом переходе, влияние структурных особенностей конформаций на их магнитные свойства.