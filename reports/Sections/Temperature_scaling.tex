\section{Изменение температуры конформаций}
До этого мы рассматривали только конформации, строго находящиеся в одной из фаз, клубки либо глобулы, $U = 0.1$ и $U = 1.0$. Данный раздел будет посвящён влиянию температуры, при которой генерируются конформации, на магнитные свойства конформаций, и поведению модели Изинга на конформациях при переходе через критическую температуру $U_\theta$. $U_\theta$ - точка геометрического фазового перехода конформаций. Результаты различных исследований \cite{SAW_tetta_point} показывают, что точка перехода находится в области $U_\theta = 0.66$. 

Для рассмотрения мы взяли по 1000 конформаций с длинами 250, 500, 1000, 2000, полученных при $U \in \left\{0.1, 0.2 \dots 0.9, 1.0\right\}$.

Намагниченность и магнитная восприимчивость, представленные на рис. \ref{fig:temp_scaling_mag2} и рис. \ref{fig:temp_scaling_ms} соответственно, почти не меняются для $U \in \left[0.1, 0.5\right]$. Затем, после $U=0.6$ начинает расти средняя намагниченность конформаций, а количество конформаций без пиков магнитной восприимчивости начинает падать.
Данные результаты хорошо соотносится с известными значением $U_\theta$ и предполагаемым поведением модели вблизи точки перехода, где при фазовом геометрическом переходе и должны начать появляться глобулярные конформации, и соответственно появляться ферромагнитные свойства у конформаций. 



\begin{figure}[ht]
	\centering
    \begin{subfigure}[t]{0.4\textwidth}
        \includegraphics*[width=\textwidth]{temp_scaling/temp_mag2_L250.png}
        \caption*{$L = 250$}
    \end{subfigure} 
    \begin{subfigure}[t]{0.4\textwidth}
        \includegraphics*[width=\textwidth]{temp_scaling/temp_mag2_L500.png}
        \caption*{$L = 500$}

    \end{subfigure}
    \begin{subfigure}[t]{0.4\textwidth}
        \includegraphics*[width=\textwidth]{temp_scaling/temp_mag2_L1000.png}
        \caption*{$L = 1000$}
    \end{subfigure}
    \begin{subfigure}[t]{0.4\textwidth}
        \includegraphics*[width=\textwidth]{temp_scaling/temp_mag2_L2000.png}
        \caption*{$L = 2000$}
    \end{subfigure}
	\caption{Намагниченность конформаций полученных при разных температурах, цветом обозначено значение $U$ при генерации.}
	\label{fig:temp_scaling_mag2}
\end{figure}


\begin{figure}[ht]
	\centering
    \begin{subfigure}[t]{0.4\textwidth}
        \includegraphics*[width=\textwidth]{temp_scaling/temp_ms_L250.png}
        \caption*{$L = 250$}
    \end{subfigure} 
    \begin{subfigure}[t]{0.4\textwidth}
        \includegraphics*[width=\textwidth]{temp_scaling/temp_ms_L500.png}
        \caption*{$L = 500$}

    \end{subfigure}
    \begin{subfigure}[t]{0.4\textwidth}
        \includegraphics*[width=\textwidth]{temp_scaling/temp_ms_L1000.png}
        \caption*{$L = 1000$}
    \end{subfigure}
    \begin{subfigure}[t]{0.4\textwidth}
        \includegraphics*[width=\textwidth]{temp_scaling/temp_ms_L2000.png}
        \caption*{$L = 2000$}
    \end{subfigure}
	\caption{Пики магнитной восприимчивости конформаций полученных при разных температурах, цветом обозначено значение $U$ при генерации.}
	\label{fig:temp_scaling_ms}
\end{figure}


На рис. \ref{fig:temp_scaling_ms} можно заметить что при увеличении $U$ пики магнитной восприимчивости не сразу появляются около ожидаемой точки магнитного перехода $\beta \approx 0.5$ а постепенно смещаются в эту сторону из $\beta = 1.0$.
И если рассмотреть распределение значений намагниченности конформаций в точке $\beta = 1.0$ при различных $U$. На рис. \ref{fig:temp_scaling_mag2_last}, видно, что при приближении к $U_\theta$ растёт доля конформаций с намагниченностью $\in (0.1, 0.8)$, при этом распределение всё ещё смещено в сторону нуля, а после перехода через $U_\theta$ начинает расти число конформаций со намагниченностью в области $(0.8, 1.0)$.

\begin{figure}[ht]
	\centering
    \begin{subfigure}[t]{0.4\textwidth}
        \includegraphics*[width=\textwidth]{temp_scaling/temp_mag2_last_L250.png}
        \caption*{$L = 250$}
    \end{subfigure} 
    \begin{subfigure}[t]{0.4\textwidth}
        \includegraphics*[width=\textwidth]{temp_scaling/temp_mag2_last_L500.png}
        \caption*{$L = 500$}

    \end{subfigure}
    \begin{subfigure}[t]{0.4\textwidth}
        \includegraphics*[width=\textwidth]{temp_scaling/temp_mag2_last_L1000.png}
        \caption*{$L = 1000$}
    \end{subfigure}
    \begin{subfigure}[t]{0.4\textwidth}
        \includegraphics*[width=\textwidth]{temp_scaling/temp_mag2_last_L2000.png}
        \caption*{$L = 2000$}
    \end{subfigure}
	\caption{Распределение значений намагниченности конформаций полученных при разных температурах в точке $\beta = 1.0$, цветом обозначено значение $U$ при генерации.}
	\label{fig:temp_scaling_mag2_last}
\end{figure}