\section{Изменение температуры конформаций}
До этого мы рассматривали только конформации, строго находящиеся в одной из фаз, клубки либо глобулы, $U = 0.1$ и $U = 1.0$. Данный раздел будет посвящён конформациям близким к геометрической точке перехода. Мы рассмотрим, как магнитные свойства конформаций меняются при изменении температуры конформаций при их генерации. 

Для рассмотрения мы взяли по 1000 конформаций с длинами 250, 500, 1000, 2000. Полученных при $U \in \left\{0.1, 0.2 \dots 0.9, 1.0\right\}$.

Намагниченность и магнитная восприимчивость, представленные на рис. \ref{fig:temp_scaling_mag2} и рис. \ref{fig:temp_scaling_ms} соответственно, почти не меняются на  $U \in \left[0.1, 0.5\right]$. Затем после $U=0.5$ начинает растя средняя намагниченности конформаций, а количество конформаций без пиков магнитной восприимчивости начинает падать.
Можно заметить что пики магнитной восприимчивости не сразу оказываются около ожидаемой точки перехода $\beta \approx 0.5$ а постепенно смещаются в эту сторону из $\beta = 1.0$
\begin{figure}[ht]
	\centering
    \begin{subfigure}[t]{0.4\textwidth}
        \includegraphics*[width=\textwidth]{temp_scaling/temp_mag2_L250.png}
        \caption*{$L = 250$}
    \end{subfigure} 
    \begin{subfigure}[t]{0.4\textwidth}
        \includegraphics*[width=\textwidth]{temp_scaling/temp_mag2_L500.png}
        \caption*{$L = 500$}

    \end{subfigure}
    \begin{subfigure}[t]{0.4\textwidth}
        \includegraphics*[width=\textwidth]{temp_scaling/temp_mag2_L1000.png}
        \caption*{$L = 1000$}
    \end{subfigure}
    \begin{subfigure}[t]{0.4\textwidth}
        \includegraphics*[width=\textwidth]{temp_scaling/temp_mag2_L2000.png}
        \caption*{$L = 2000$}
    \end{subfigure}
	\caption{Намагниченность конформаций полученных при разных температурах, цветом обозначено значение $U$ при генерации.}
	\label{fig:temp_scaling_mag2}
\end{figure}


\begin{figure}[ht]
	\centering
    \begin{subfigure}[t]{0.4\textwidth}
        \includegraphics*[width=\textwidth]{temp_scaling/temp_ms_L250.png}
        \caption*{$L = 250$}
    \end{subfigure} 
    \begin{subfigure}[t]{0.4\textwidth}
        \includegraphics*[width=\textwidth]{temp_scaling/temp_ms_L500.png}
        \caption*{$L = 500$}

    \end{subfigure}
    \begin{subfigure}[t]{0.4\textwidth}
        \includegraphics*[width=\textwidth]{temp_scaling/temp_ms_L1000.png}
        \caption*{$L = 1000$}
    \end{subfigure}
    \begin{subfigure}[t]{0.4\textwidth}
        \includegraphics*[width=\textwidth]{temp_scaling/temp_ms_L2000.png}
        \caption*{$L = 2000$}
    \end{subfigure}
	\caption{Пики магнитной восприимчивости конформаций полученных при разных температурах, цветом обозначено значение $U$ при генерации.}
	\label{fig:temp_scaling_ms}
\end{figure}

