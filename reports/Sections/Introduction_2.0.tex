\section{Введение}
Модель Изинга \cite{ising} широко распространённая и хорошо исследованная модель, которую активно применяют в различных областях науки: биология, физика, материаловеденье и др. Она позволяет описать магнитные свойства системы, основываясь на взаимодействии магнитных моментов и внешнего магнитного поля. В частности большой интерес представляют применение данной модели к моделям макромолекул \cite{SAW_polymer}(далее конформации). Данное направление обрело популярность в последние годы. В работах рассматриваются разнообразные модели: в пространствах разной размерности, с различными правилами взаимодействия соседей\dots Однако во всех работах Модель изинга рассматривается одновременно с моделью конформаций, то есть спиновая подсистема изменяется вместе с геометрической структурой конформации. И в отличие от этих работ, в нашей модели, модель Изинга строится на фиксированной структуре, уже после генерации конформации.

\begin{figure}[h]
	\centering
    \begin{subfigure}{0.45\textwidth}
        \includegraphics[width=\textwidth]{../images/loose_conf.png}
        \caption{глобула}
    \end{subfigure}
    \begin{subfigure}{0.45\textwidth}
	    \includegraphics[width=\textwidth]{../images/dense_conf.png}
        \caption{клубок}
    \end{subfigure} 
	\caption{Примеры конформаций}
    \label{fig::globule_coil_example}
\end{figure}

Модели изинга и модель конформаций представляют для нас интерес из-за геометрических свойств конформаций, и известных свойств модели Изинга на различных решётках. А именно: точное решение для модели Изинга \cite{ising_solutions} показывает, что на одномерной сетке модель Изинга не имеет магнитного фазового перехода, одномерная модель не становится магнитной ни при каких температурах, кроме абсолютного нуля. В то время как на двумерной сетке есть фазовый переход \cite{SAW_polymer_critical}, и она становится магнитной при низких температурах. Конформации так же имеют фазовый переход. Состояния конформаций называются глобулой и клубком и соответствуют низким и высоким температурам, пример конформаций представлен на Рис. \ref{fig::globule_coil_example}. Структурно состояния конформации, которые в основном проявляются в глобулах и клубках, подобны одномерным и двумерным сеткам соответственно. Например количество соседей у вершин в глобулярных конформациях в основном 4 или 3, а в клубках у большинства вершин 2 соседа. Учитывая структурную схожесть фаз конформаций и решёток, можно предположить и магнитную схожесть.

Цель данной работы определить магнитные свойства конформаций в фазах глобула и клубок, сравнить свойства с двумерной и одномерной решёткой. Определить точку фазового магнитного перехода в глобулах, если он существует. Исследовать поведение магнитной модели вблизи геометрического перехода.