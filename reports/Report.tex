\documentclass[a4paper,12pt]{article}

%%% Работа с русским языком
\usepackage{cmap}					% поиск в PDF
\usepackage{mathtext} 				% русские буквы в фомулах
\usepackage[T2A]{fontenc}			% кодировка
\usepackage[utf8]{inputenc}			% кодировка исходного текста
\usepackage[english,russian]{babel}	% локализация и переносы

%%% Дополнительная работа с математикой
\usepackage{amsfonts,amssymb,amsthm,mathtools} % AMS
\usepackage{amsmath}
\usepackage{icomma} % "Умная" запятая: $0,2$ --- число, $0, 2$ --- перечисление

%% Номера формул
%\mathtoolsset{showonlyrefs=true} % Показывать номера только у тех формул, на которые есть \eqref{} в тексте.

\usepackage{hyperref}
\hypersetup{
    colorlinks=false,
    linkcolor=blue,
    filecolor=magenta,      
    urlcolor=cyan,
}

\usepackage{float}


%% Шрифты
\usepackage{euscript}	 % Шрифт Евклид
\usepackage{mathrsfs} % Красивый матшрифт

%%% Работа с картинками
\usepackage{graphicx}  % Для вставки рисунков
\graphicspath{{images/}{images2/}}  % папки с картинками
\setlength\fboxsep{3pt} % Отступ рамки \fbox{} от рисунка
\setlength\fboxrule{1pt} % Толщина линий рамки \fbox{}
\usepackage{wrapfig} % Обтекание рисунков и таблиц текстом
\usepackage{caption}
\usepackage{subcaption}
\captionsetup{labelsep=period} %. вместо : в рис

%\title{Научный отчёт по проекту Решеточные модели макромолекул}
\title{Информация о замерах}
\author{Москаленко Роман}
\date{24.05.2021}

\begin{document}

\maketitle

%\section{Введение}
%Модель Изинга используется для моделирования и изучения термодинамических свойств. Поведение структуры в модели Изинга сильно зависит от её геометрии. Так например на одно мерных моделях не происходит фазовый переход, но на двумерных моделях переход есть. Но что происходит в промежуточных размерностях? Например если взять какую-то последовательность узлов на двумерной решётке. Именно это и является главным вопросом в данном проекте. 
В данном проекте проводится исследование модели Изинга на фиксированной двумерной конформации. 

\section{Введение}
%Модель Изинга используется для моделирования и изучения термодинамических свойств. Поведение структуры в модели Изинга сильно зависит от её геометрии. Так например на одно мерных моделях не происходит фазовый переход, но на двумерных моделях переход есть. Но что происходит в промежуточных размерностях? Например если взять какую-то последовательность узлов на двумерной решётке. Именно это и является главным вопросом в данном проекте. 
Рассмотрим конформацию(несамопересекающуюся последовательность узлов) на двумерной решётке. Такие конформации можно рассматривать как термодинамическую систему, основанную на модели Изинга, для которых существуют две фазы: плотная(глобулярная) и развёрнутая. Эти фазы соответствуют низким и высоким температурам системы.

\begin{figure}[h]
	\centering
	\includegraphics[width=0.45\textwidth]{../images/loose_conf.png}
	\includegraphics[width=0.45\textwidth]{../images/dense_conf.png} 
	\caption{Пример неплотной и плотной конформации}
\end{figure}

Если посмотреть на изображения конформаций каждого вида, хорошо видно, что плотные конформации по структуре близки с двумерным решёткам, где у каждого узла имеется множество соседей, и развёрнутые конформации наоборот близки к одномерным структурам, где узлы у которых больше 2 соседей встречаются редко. Соответственно можно предположить, что плотные конформации будут иметь свойства схожие с двумерными решётками, а развёрнутые с одномерными. В двумерных решётках наблюдается магнитный фазовый переход, в то время как в одномерных решётках переход не происходит. Цель данного исследования определить наличие магнитного перехода в плотных конформациях.


\subsection{Модель}
В данной модели мы рассматриваем ансамбли конформаций: множества конформаций одинаковой длинны $L$, полученные при одинаковых температурах. Мы получаем конформации используя алгоритм SAW.
На каждой из конформаций строится модель Изинга \cite{ising}. В каждой вершине размещается спин, который может принимать одно из двух значений: $+1, -1$.
Гамильтониан данной системы имеет вид
\[H = -J\sum_{\langle i, j\rangle}{\sigma_i\sigma_j} - h\sum_i{\sigma_i} \]

где $i, j$ индексы соседних узлов у, $J$- коэффициент взаимодействия $h$ - воздействие внешнего поля.

Статистическая сумма
\[Z = \sum_{\{\sigma\}} e^{-H(\sigma)\beta}, \beta = \frac{1}{kT}\]
где $\{\sigma\}$ - множество всех возможных наборов значений спинов.

Намагниченность и энергия каждого состояния считаются по следующим формулам

\[ 
E = -J\sum_{i, j} \sigma_i \sigma_j, 
M = \sum_i \sigma_i
\]

Средняя намагниченность системы

\[
\langle M \rangle = \frac{1}{Z}  \sum_{\{\sigma\}} M e^{-H_{(\sigma)}\beta}
\]

\subsection{Метод Монте-Карло}
Для расчёта модели Изинга используется метод Монте-Карло. Были реализованы версии с односпиновым и кластернным апдейтом \cite{wolf_algorithm}. Код представлен в репозитории github \cite{github}. В итоге для измерений используется кластерная версия. Благодаря отказоустойчивости она работает значительно быстрее, и быстрее сходится, особенно при низких температурах.

Алгоритм с кластерным апдейтом работает следующим образом. На каждой итерации мы выбираем случайный спин и начиная с него начинаем строить кластер из одинаково направленных спинов, добавляя новые спины в кластер с определённой вероятностью. затем мы меняем значения спинов в кластере на противоположные.

Чтобы вычислить намагниченность, мы сначала случайным образом инициализируем спины, затем делаем некоторое число шагов для отжига модели. Далее на каждом шаге мы замеряем намагниченность, и после выполнения определённого числа шагов, усредняем полученные значения. Так как средняя намагниченность равна 0, имеет смысл рассматривать модуль квадрат намагниченности.

\[
\langle M^2\rangle = \frac{1}{n} \sum_{\{\sigma\}} \left( \sum_i \sigma_i \right)^2, \\
\langle |M|\rangle = \frac{1}{n} \sum_{\{\sigma\}} \left| \sum_i \sigma_i \right|
\]

\section{Используемые алгоритмы}

Реализованы два алгоритма обновления спинов. Односпиновый и кластерный апдейт. Оба алгоритма работают на произвольном графе, используя таблицу соседей. Алгоритмы реализованы как отдельные библиотеки для Python, и написаны с использованием технологии Cython для ускорения работы. Кластерный апдейт является более эффективным по времени работы и количеству шагов, которые необходимо выполнить для хорошей сходимости модели.

\subsection{Проверка алгоритмов}

Чтобы убедиться что алгоритмы работают правильно мы проверили, что оба алгоритма дают одинаковые результаты на одних и тех же конформациях, так же сравнил их с точными решениями для одномерной модели Изинга.

Результаты замеров кластерным и односпиновым апдейтом совпадают в пределах погрешности.

\begin{figure}[H]
	\centering
	\includegraphics[width = 0.45\textwidth]{../images/1spin_&_cluster_ene.png} 
	\includegraphics[width = 0.45\textwidth]{../images/1spin_&_cluster_ene_dif.png} 
	\includegraphics[width = 0.45\textwidth]{../images/1spin_&_cluster_mag.png} 
	\includegraphics[width = 0.45\textwidth]{../images/1spin_&_cluster_mag_dif.png} 
	%% add magnetization
	\caption{кластерный и односпиновый апдейт}
\end{figure}

Для сравнения с точными значениями для одномерной модели Изинга, мы используем замкнутый квадратный контур. Данная конформация по свойствам полностью совпадает с одномерной моделью Изинга с открытыми граничными условиями.

\begin{figure}[H]
	\centering
	\begin{subfigure}[t]{0.45\textwidth}
		\includegraphics[width = \textwidth]{../images/1D_conf.png} 
		\caption{Конформация эмитирующая одномерную модель}
	\end{subfigure}
	\includegraphics[width = 0.45\textwidth]{../images/1D_ene.png}
	\includegraphics[width = 0.45\textwidth]{../images/1D_ene_diff.png} 
	%% add magnetization
	\caption{Сравнение с точным решением одномерной модели}
\end{figure}

Так же был написан код, точно вычисляющий энергию системы путём полного перебора всех её состояний. Сравнение на маленьких конформациях (длина 10) даёт одинаковые результаты.

%TODO втавить результаты поного перебора

Примеры с использованием кластерного апдейта добавлены в библиотеку \texttt{mc\_lib}.

%\section{Конформации вида клубок}
До этого мы рассматривали только конформации полученные при низких температурах, где мы хотим определить точку перехода. Другая часть этого исследования заключается в проверке того, что у конформаций, полученных при низкой температуре, магнитный фазовый переход не отсутствует.

Для этого были cгенерированы 4 набора конформаций с длинами 250, 500, 1000, 2000 по 1000 конформаций в каждом наборе. Как и ожидалось средняя намагниченность по конформациям значительно меньше, чем у конформаций при $U = 1$(сравнение на Рис. \ref{fig:U0.1_mean_mag2}).

\begin{figure}[htb]
	\centering
	%% Creator: Matplotlib, PGF backend
%%
%% To include the figure in your LaTeX document, write
%%   \input{<filename>.pgf}
%%
%% Make sure the required packages are loaded in your preamble
%%   \usepackage{pgf}
%%
%% Also ensure that all the required font packages are loaded; for instance,
%% the lmodern package is sometimes necessary when using math font.
%%   \usepackage{lmodern}
%%
%% Figures using additional raster images can only be included by \input if
%% they are in the same directory as the main LaTeX file. For loading figures
%% from other directories you can use the `import` package
%%   \usepackage{import}
%%
%% and then include the figures with
%%   \import{<path to file>}{<filename>.pgf}
%%
%% Matplotlib used the following preamble
%%   
%%   \makeatletter\@ifpackageloaded{underscore}{}{\usepackage[strings]{underscore}}\makeatother
%%
\begingroup%
\makeatletter%
\begin{pgfpicture}%
\pgfpathrectangle{\pgfpointorigin}{\pgfqpoint{4.353372in}{2.871460in}}%
\pgfusepath{use as bounding box, clip}%
\begin{pgfscope}%
\pgfsetbuttcap%
\pgfsetmiterjoin%
\definecolor{currentfill}{rgb}{1.000000,1.000000,1.000000}%
\pgfsetfillcolor{currentfill}%
\pgfsetlinewidth{0.000000pt}%
\definecolor{currentstroke}{rgb}{1.000000,1.000000,1.000000}%
\pgfsetstrokecolor{currentstroke}%
\pgfsetdash{}{0pt}%
\pgfpathmoveto{\pgfqpoint{0.000000in}{0.000000in}}%
\pgfpathlineto{\pgfqpoint{4.353372in}{0.000000in}}%
\pgfpathlineto{\pgfqpoint{4.353372in}{2.871460in}}%
\pgfpathlineto{\pgfqpoint{0.000000in}{2.871460in}}%
\pgfpathlineto{\pgfqpoint{0.000000in}{0.000000in}}%
\pgfpathclose%
\pgfusepath{fill}%
\end{pgfscope}%
\begin{pgfscope}%
\pgfsetbuttcap%
\pgfsetmiterjoin%
\definecolor{currentfill}{rgb}{1.000000,1.000000,1.000000}%
\pgfsetfillcolor{currentfill}%
\pgfsetlinewidth{0.000000pt}%
\definecolor{currentstroke}{rgb}{0.000000,0.000000,0.000000}%
\pgfsetstrokecolor{currentstroke}%
\pgfsetstrokeopacity{0.000000}%
\pgfsetdash{}{0pt}%
\pgfpathmoveto{\pgfqpoint{0.553704in}{0.499691in}}%
\pgfpathlineto{\pgfqpoint{4.253372in}{0.499691in}}%
\pgfpathlineto{\pgfqpoint{4.253372in}{2.771460in}}%
\pgfpathlineto{\pgfqpoint{0.553704in}{2.771460in}}%
\pgfpathlineto{\pgfqpoint{0.553704in}{0.499691in}}%
\pgfpathclose%
\pgfusepath{fill}%
\end{pgfscope}%
\begin{pgfscope}%
\pgfpathrectangle{\pgfqpoint{0.553704in}{0.499691in}}{\pgfqpoint{3.699668in}{2.271769in}}%
\pgfusepath{clip}%
\pgfsetrectcap%
\pgfsetroundjoin%
\pgfsetlinewidth{0.803000pt}%
\definecolor{currentstroke}{rgb}{0.690196,0.690196,0.690196}%
\pgfsetstrokecolor{currentstroke}%
\pgfsetdash{}{0pt}%
\pgfpathmoveto{\pgfqpoint{1.102801in}{0.499691in}}%
\pgfpathlineto{\pgfqpoint{1.102801in}{2.771460in}}%
\pgfusepath{stroke}%
\end{pgfscope}%
\begin{pgfscope}%
\pgfsetbuttcap%
\pgfsetroundjoin%
\definecolor{currentfill}{rgb}{0.000000,0.000000,0.000000}%
\pgfsetfillcolor{currentfill}%
\pgfsetlinewidth{0.803000pt}%
\definecolor{currentstroke}{rgb}{0.000000,0.000000,0.000000}%
\pgfsetstrokecolor{currentstroke}%
\pgfsetdash{}{0pt}%
\pgfsys@defobject{currentmarker}{\pgfqpoint{0.000000in}{-0.048611in}}{\pgfqpoint{0.000000in}{0.000000in}}{%
\pgfpathmoveto{\pgfqpoint{0.000000in}{0.000000in}}%
\pgfpathlineto{\pgfqpoint{0.000000in}{-0.048611in}}%
\pgfusepath{stroke,fill}%
}%
\begin{pgfscope}%
\pgfsys@transformshift{1.102801in}{0.499691in}%
\pgfsys@useobject{currentmarker}{}%
\end{pgfscope}%
\end{pgfscope}%
\begin{pgfscope}%
\definecolor{textcolor}{rgb}{0.000000,0.000000,0.000000}%
\pgfsetstrokecolor{textcolor}%
\pgfsetfillcolor{textcolor}%
\pgftext[x=1.102801in,y=0.402469in,,top]{\color{textcolor}\sffamily\fontsize{10.000000}{12.000000}\selectfont 0.2}%
\end{pgfscope}%
\begin{pgfscope}%
\pgfpathrectangle{\pgfqpoint{0.553704in}{0.499691in}}{\pgfqpoint{3.699668in}{2.271769in}}%
\pgfusepath{clip}%
\pgfsetrectcap%
\pgfsetroundjoin%
\pgfsetlinewidth{0.803000pt}%
\definecolor{currentstroke}{rgb}{0.690196,0.690196,0.690196}%
\pgfsetstrokecolor{currentstroke}%
\pgfsetdash{}{0pt}%
\pgfpathmoveto{\pgfqpoint{1.846079in}{0.499691in}}%
\pgfpathlineto{\pgfqpoint{1.846079in}{2.771460in}}%
\pgfusepath{stroke}%
\end{pgfscope}%
\begin{pgfscope}%
\pgfsetbuttcap%
\pgfsetroundjoin%
\definecolor{currentfill}{rgb}{0.000000,0.000000,0.000000}%
\pgfsetfillcolor{currentfill}%
\pgfsetlinewidth{0.803000pt}%
\definecolor{currentstroke}{rgb}{0.000000,0.000000,0.000000}%
\pgfsetstrokecolor{currentstroke}%
\pgfsetdash{}{0pt}%
\pgfsys@defobject{currentmarker}{\pgfqpoint{0.000000in}{-0.048611in}}{\pgfqpoint{0.000000in}{0.000000in}}{%
\pgfpathmoveto{\pgfqpoint{0.000000in}{0.000000in}}%
\pgfpathlineto{\pgfqpoint{0.000000in}{-0.048611in}}%
\pgfusepath{stroke,fill}%
}%
\begin{pgfscope}%
\pgfsys@transformshift{1.846079in}{0.499691in}%
\pgfsys@useobject{currentmarker}{}%
\end{pgfscope}%
\end{pgfscope}%
\begin{pgfscope}%
\definecolor{textcolor}{rgb}{0.000000,0.000000,0.000000}%
\pgfsetstrokecolor{textcolor}%
\pgfsetfillcolor{textcolor}%
\pgftext[x=1.846079in,y=0.402469in,,top]{\color{textcolor}\sffamily\fontsize{10.000000}{12.000000}\selectfont 0.4}%
\end{pgfscope}%
\begin{pgfscope}%
\pgfpathrectangle{\pgfqpoint{0.553704in}{0.499691in}}{\pgfqpoint{3.699668in}{2.271769in}}%
\pgfusepath{clip}%
\pgfsetrectcap%
\pgfsetroundjoin%
\pgfsetlinewidth{0.803000pt}%
\definecolor{currentstroke}{rgb}{0.690196,0.690196,0.690196}%
\pgfsetstrokecolor{currentstroke}%
\pgfsetdash{}{0pt}%
\pgfpathmoveto{\pgfqpoint{2.589358in}{0.499691in}}%
\pgfpathlineto{\pgfqpoint{2.589358in}{2.771460in}}%
\pgfusepath{stroke}%
\end{pgfscope}%
\begin{pgfscope}%
\pgfsetbuttcap%
\pgfsetroundjoin%
\definecolor{currentfill}{rgb}{0.000000,0.000000,0.000000}%
\pgfsetfillcolor{currentfill}%
\pgfsetlinewidth{0.803000pt}%
\definecolor{currentstroke}{rgb}{0.000000,0.000000,0.000000}%
\pgfsetstrokecolor{currentstroke}%
\pgfsetdash{}{0pt}%
\pgfsys@defobject{currentmarker}{\pgfqpoint{0.000000in}{-0.048611in}}{\pgfqpoint{0.000000in}{0.000000in}}{%
\pgfpathmoveto{\pgfqpoint{0.000000in}{0.000000in}}%
\pgfpathlineto{\pgfqpoint{0.000000in}{-0.048611in}}%
\pgfusepath{stroke,fill}%
}%
\begin{pgfscope}%
\pgfsys@transformshift{2.589358in}{0.499691in}%
\pgfsys@useobject{currentmarker}{}%
\end{pgfscope}%
\end{pgfscope}%
\begin{pgfscope}%
\definecolor{textcolor}{rgb}{0.000000,0.000000,0.000000}%
\pgfsetstrokecolor{textcolor}%
\pgfsetfillcolor{textcolor}%
\pgftext[x=2.589358in,y=0.402469in,,top]{\color{textcolor}\sffamily\fontsize{10.000000}{12.000000}\selectfont 0.6}%
\end{pgfscope}%
\begin{pgfscope}%
\pgfpathrectangle{\pgfqpoint{0.553704in}{0.499691in}}{\pgfqpoint{3.699668in}{2.271769in}}%
\pgfusepath{clip}%
\pgfsetrectcap%
\pgfsetroundjoin%
\pgfsetlinewidth{0.803000pt}%
\definecolor{currentstroke}{rgb}{0.690196,0.690196,0.690196}%
\pgfsetstrokecolor{currentstroke}%
\pgfsetdash{}{0pt}%
\pgfpathmoveto{\pgfqpoint{3.332636in}{0.499691in}}%
\pgfpathlineto{\pgfqpoint{3.332636in}{2.771460in}}%
\pgfusepath{stroke}%
\end{pgfscope}%
\begin{pgfscope}%
\pgfsetbuttcap%
\pgfsetroundjoin%
\definecolor{currentfill}{rgb}{0.000000,0.000000,0.000000}%
\pgfsetfillcolor{currentfill}%
\pgfsetlinewidth{0.803000pt}%
\definecolor{currentstroke}{rgb}{0.000000,0.000000,0.000000}%
\pgfsetstrokecolor{currentstroke}%
\pgfsetdash{}{0pt}%
\pgfsys@defobject{currentmarker}{\pgfqpoint{0.000000in}{-0.048611in}}{\pgfqpoint{0.000000in}{0.000000in}}{%
\pgfpathmoveto{\pgfqpoint{0.000000in}{0.000000in}}%
\pgfpathlineto{\pgfqpoint{0.000000in}{-0.048611in}}%
\pgfusepath{stroke,fill}%
}%
\begin{pgfscope}%
\pgfsys@transformshift{3.332636in}{0.499691in}%
\pgfsys@useobject{currentmarker}{}%
\end{pgfscope}%
\end{pgfscope}%
\begin{pgfscope}%
\definecolor{textcolor}{rgb}{0.000000,0.000000,0.000000}%
\pgfsetstrokecolor{textcolor}%
\pgfsetfillcolor{textcolor}%
\pgftext[x=3.332636in,y=0.402469in,,top]{\color{textcolor}\sffamily\fontsize{10.000000}{12.000000}\selectfont 0.8}%
\end{pgfscope}%
\begin{pgfscope}%
\pgfpathrectangle{\pgfqpoint{0.553704in}{0.499691in}}{\pgfqpoint{3.699668in}{2.271769in}}%
\pgfusepath{clip}%
\pgfsetrectcap%
\pgfsetroundjoin%
\pgfsetlinewidth{0.803000pt}%
\definecolor{currentstroke}{rgb}{0.690196,0.690196,0.690196}%
\pgfsetstrokecolor{currentstroke}%
\pgfsetdash{}{0pt}%
\pgfpathmoveto{\pgfqpoint{4.075914in}{0.499691in}}%
\pgfpathlineto{\pgfqpoint{4.075914in}{2.771460in}}%
\pgfusepath{stroke}%
\end{pgfscope}%
\begin{pgfscope}%
\pgfsetbuttcap%
\pgfsetroundjoin%
\definecolor{currentfill}{rgb}{0.000000,0.000000,0.000000}%
\pgfsetfillcolor{currentfill}%
\pgfsetlinewidth{0.803000pt}%
\definecolor{currentstroke}{rgb}{0.000000,0.000000,0.000000}%
\pgfsetstrokecolor{currentstroke}%
\pgfsetdash{}{0pt}%
\pgfsys@defobject{currentmarker}{\pgfqpoint{0.000000in}{-0.048611in}}{\pgfqpoint{0.000000in}{0.000000in}}{%
\pgfpathmoveto{\pgfqpoint{0.000000in}{0.000000in}}%
\pgfpathlineto{\pgfqpoint{0.000000in}{-0.048611in}}%
\pgfusepath{stroke,fill}%
}%
\begin{pgfscope}%
\pgfsys@transformshift{4.075914in}{0.499691in}%
\pgfsys@useobject{currentmarker}{}%
\end{pgfscope}%
\end{pgfscope}%
\begin{pgfscope}%
\definecolor{textcolor}{rgb}{0.000000,0.000000,0.000000}%
\pgfsetstrokecolor{textcolor}%
\pgfsetfillcolor{textcolor}%
\pgftext[x=4.075914in,y=0.402469in,,top]{\color{textcolor}\sffamily\fontsize{10.000000}{12.000000}\selectfont 1.0}%
\end{pgfscope}%
\begin{pgfscope}%
\definecolor{textcolor}{rgb}{0.000000,0.000000,0.000000}%
\pgfsetstrokecolor{textcolor}%
\pgfsetfillcolor{textcolor}%
\pgftext[x=2.403538in,y=0.223457in,,top]{\color{textcolor}\sffamily\fontsize{10.000000}{12.000000}\selectfont \(\displaystyle \beta\)}%
\end{pgfscope}%
\begin{pgfscope}%
\pgfpathrectangle{\pgfqpoint{0.553704in}{0.499691in}}{\pgfqpoint{3.699668in}{2.271769in}}%
\pgfusepath{clip}%
\pgfsetrectcap%
\pgfsetroundjoin%
\pgfsetlinewidth{0.803000pt}%
\definecolor{currentstroke}{rgb}{0.690196,0.690196,0.690196}%
\pgfsetstrokecolor{currentstroke}%
\pgfsetdash{}{0pt}%
\pgfpathmoveto{\pgfqpoint{0.553704in}{0.747434in}}%
\pgfpathlineto{\pgfqpoint{4.253372in}{0.747434in}}%
\pgfusepath{stroke}%
\end{pgfscope}%
\begin{pgfscope}%
\pgfsetbuttcap%
\pgfsetroundjoin%
\definecolor{currentfill}{rgb}{0.000000,0.000000,0.000000}%
\pgfsetfillcolor{currentfill}%
\pgfsetlinewidth{0.803000pt}%
\definecolor{currentstroke}{rgb}{0.000000,0.000000,0.000000}%
\pgfsetstrokecolor{currentstroke}%
\pgfsetdash{}{0pt}%
\pgfsys@defobject{currentmarker}{\pgfqpoint{-0.048611in}{0.000000in}}{\pgfqpoint{-0.000000in}{0.000000in}}{%
\pgfpathmoveto{\pgfqpoint{-0.000000in}{0.000000in}}%
\pgfpathlineto{\pgfqpoint{-0.048611in}{0.000000in}}%
\pgfusepath{stroke,fill}%
}%
\begin{pgfscope}%
\pgfsys@transformshift{0.553704in}{0.747434in}%
\pgfsys@useobject{currentmarker}{}%
\end{pgfscope}%
\end{pgfscope}%
\begin{pgfscope}%
\definecolor{textcolor}{rgb}{0.000000,0.000000,0.000000}%
\pgfsetstrokecolor{textcolor}%
\pgfsetfillcolor{textcolor}%
\pgftext[x=0.279012in, y=0.699208in, left, base]{\color{textcolor}\sffamily\fontsize{10.000000}{12.000000}\selectfont 0.0}%
\end{pgfscope}%
\begin{pgfscope}%
\pgfpathrectangle{\pgfqpoint{0.553704in}{0.499691in}}{\pgfqpoint{3.699668in}{2.271769in}}%
\pgfusepath{clip}%
\pgfsetrectcap%
\pgfsetroundjoin%
\pgfsetlinewidth{0.803000pt}%
\definecolor{currentstroke}{rgb}{0.690196,0.690196,0.690196}%
\pgfsetstrokecolor{currentstroke}%
\pgfsetdash{}{0pt}%
\pgfpathmoveto{\pgfqpoint{0.553704in}{1.274009in}}%
\pgfpathlineto{\pgfqpoint{4.253372in}{1.274009in}}%
\pgfusepath{stroke}%
\end{pgfscope}%
\begin{pgfscope}%
\pgfsetbuttcap%
\pgfsetroundjoin%
\definecolor{currentfill}{rgb}{0.000000,0.000000,0.000000}%
\pgfsetfillcolor{currentfill}%
\pgfsetlinewidth{0.803000pt}%
\definecolor{currentstroke}{rgb}{0.000000,0.000000,0.000000}%
\pgfsetstrokecolor{currentstroke}%
\pgfsetdash{}{0pt}%
\pgfsys@defobject{currentmarker}{\pgfqpoint{-0.048611in}{0.000000in}}{\pgfqpoint{-0.000000in}{0.000000in}}{%
\pgfpathmoveto{\pgfqpoint{-0.000000in}{0.000000in}}%
\pgfpathlineto{\pgfqpoint{-0.048611in}{0.000000in}}%
\pgfusepath{stroke,fill}%
}%
\begin{pgfscope}%
\pgfsys@transformshift{0.553704in}{1.274009in}%
\pgfsys@useobject{currentmarker}{}%
\end{pgfscope}%
\end{pgfscope}%
\begin{pgfscope}%
\definecolor{textcolor}{rgb}{0.000000,0.000000,0.000000}%
\pgfsetstrokecolor{textcolor}%
\pgfsetfillcolor{textcolor}%
\pgftext[x=0.279012in, y=1.225783in, left, base]{\color{textcolor}\sffamily\fontsize{10.000000}{12.000000}\selectfont 0.2}%
\end{pgfscope}%
\begin{pgfscope}%
\pgfpathrectangle{\pgfqpoint{0.553704in}{0.499691in}}{\pgfqpoint{3.699668in}{2.271769in}}%
\pgfusepath{clip}%
\pgfsetrectcap%
\pgfsetroundjoin%
\pgfsetlinewidth{0.803000pt}%
\definecolor{currentstroke}{rgb}{0.690196,0.690196,0.690196}%
\pgfsetstrokecolor{currentstroke}%
\pgfsetdash{}{0pt}%
\pgfpathmoveto{\pgfqpoint{0.553704in}{1.800584in}}%
\pgfpathlineto{\pgfqpoint{4.253372in}{1.800584in}}%
\pgfusepath{stroke}%
\end{pgfscope}%
\begin{pgfscope}%
\pgfsetbuttcap%
\pgfsetroundjoin%
\definecolor{currentfill}{rgb}{0.000000,0.000000,0.000000}%
\pgfsetfillcolor{currentfill}%
\pgfsetlinewidth{0.803000pt}%
\definecolor{currentstroke}{rgb}{0.000000,0.000000,0.000000}%
\pgfsetstrokecolor{currentstroke}%
\pgfsetdash{}{0pt}%
\pgfsys@defobject{currentmarker}{\pgfqpoint{-0.048611in}{0.000000in}}{\pgfqpoint{-0.000000in}{0.000000in}}{%
\pgfpathmoveto{\pgfqpoint{-0.000000in}{0.000000in}}%
\pgfpathlineto{\pgfqpoint{-0.048611in}{0.000000in}}%
\pgfusepath{stroke,fill}%
}%
\begin{pgfscope}%
\pgfsys@transformshift{0.553704in}{1.800584in}%
\pgfsys@useobject{currentmarker}{}%
\end{pgfscope}%
\end{pgfscope}%
\begin{pgfscope}%
\definecolor{textcolor}{rgb}{0.000000,0.000000,0.000000}%
\pgfsetstrokecolor{textcolor}%
\pgfsetfillcolor{textcolor}%
\pgftext[x=0.279012in, y=1.752358in, left, base]{\color{textcolor}\sffamily\fontsize{10.000000}{12.000000}\selectfont 0.4}%
\end{pgfscope}%
\begin{pgfscope}%
\pgfpathrectangle{\pgfqpoint{0.553704in}{0.499691in}}{\pgfqpoint{3.699668in}{2.271769in}}%
\pgfusepath{clip}%
\pgfsetrectcap%
\pgfsetroundjoin%
\pgfsetlinewidth{0.803000pt}%
\definecolor{currentstroke}{rgb}{0.690196,0.690196,0.690196}%
\pgfsetstrokecolor{currentstroke}%
\pgfsetdash{}{0pt}%
\pgfpathmoveto{\pgfqpoint{0.553704in}{2.327159in}}%
\pgfpathlineto{\pgfqpoint{4.253372in}{2.327159in}}%
\pgfusepath{stroke}%
\end{pgfscope}%
\begin{pgfscope}%
\pgfsetbuttcap%
\pgfsetroundjoin%
\definecolor{currentfill}{rgb}{0.000000,0.000000,0.000000}%
\pgfsetfillcolor{currentfill}%
\pgfsetlinewidth{0.803000pt}%
\definecolor{currentstroke}{rgb}{0.000000,0.000000,0.000000}%
\pgfsetstrokecolor{currentstroke}%
\pgfsetdash{}{0pt}%
\pgfsys@defobject{currentmarker}{\pgfqpoint{-0.048611in}{0.000000in}}{\pgfqpoint{-0.000000in}{0.000000in}}{%
\pgfpathmoveto{\pgfqpoint{-0.000000in}{0.000000in}}%
\pgfpathlineto{\pgfqpoint{-0.048611in}{0.000000in}}%
\pgfusepath{stroke,fill}%
}%
\begin{pgfscope}%
\pgfsys@transformshift{0.553704in}{2.327159in}%
\pgfsys@useobject{currentmarker}{}%
\end{pgfscope}%
\end{pgfscope}%
\begin{pgfscope}%
\definecolor{textcolor}{rgb}{0.000000,0.000000,0.000000}%
\pgfsetstrokecolor{textcolor}%
\pgfsetfillcolor{textcolor}%
\pgftext[x=0.279012in, y=2.278933in, left, base]{\color{textcolor}\sffamily\fontsize{10.000000}{12.000000}\selectfont 0.6}%
\end{pgfscope}%
\begin{pgfscope}%
\definecolor{textcolor}{rgb}{0.000000,0.000000,0.000000}%
\pgfsetstrokecolor{textcolor}%
\pgfsetfillcolor{textcolor}%
\pgftext[x=0.223457in,y=1.635575in,,bottom,rotate=90.000000]{\color{textcolor}\sffamily\fontsize{10.000000}{12.000000}\selectfont \(\displaystyle M^2\)}%
\end{pgfscope}%
\begin{pgfscope}%
\pgfpathrectangle{\pgfqpoint{0.553704in}{0.499691in}}{\pgfqpoint{3.699668in}{2.271769in}}%
\pgfusepath{clip}%
\pgfsetbuttcap%
\pgfsetroundjoin%
\pgfsetlinewidth{1.505625pt}%
\definecolor{currentstroke}{rgb}{0.121569,0.466667,0.705882}%
\pgfsetstrokecolor{currentstroke}%
\pgfsetdash{}{0pt}%
\pgfpathmoveto{\pgfqpoint{0.721871in}{0.760549in}}%
\pgfpathlineto{\pgfqpoint{0.721871in}{0.761740in}}%
\pgfusepath{stroke}%
\end{pgfscope}%
\begin{pgfscope}%
\pgfpathrectangle{\pgfqpoint{0.553704in}{0.499691in}}{\pgfqpoint{3.699668in}{2.271769in}}%
\pgfusepath{clip}%
\pgfsetbuttcap%
\pgfsetroundjoin%
\pgfsetlinewidth{1.505625pt}%
\definecolor{currentstroke}{rgb}{0.121569,0.466667,0.705882}%
\pgfsetstrokecolor{currentstroke}%
\pgfsetdash{}{0pt}%
\pgfpathmoveto{\pgfqpoint{1.093510in}{0.764153in}}%
\pgfpathlineto{\pgfqpoint{1.093510in}{0.767878in}}%
\pgfusepath{stroke}%
\end{pgfscope}%
\begin{pgfscope}%
\pgfpathrectangle{\pgfqpoint{0.553704in}{0.499691in}}{\pgfqpoint{3.699668in}{2.271769in}}%
\pgfusepath{clip}%
\pgfsetbuttcap%
\pgfsetroundjoin%
\pgfsetlinewidth{1.505625pt}%
\definecolor{currentstroke}{rgb}{0.121569,0.466667,0.705882}%
\pgfsetstrokecolor{currentstroke}%
\pgfsetdash{}{0pt}%
\pgfpathmoveto{\pgfqpoint{1.465149in}{0.768371in}}%
\pgfpathlineto{\pgfqpoint{1.465149in}{0.779693in}}%
\pgfusepath{stroke}%
\end{pgfscope}%
\begin{pgfscope}%
\pgfpathrectangle{\pgfqpoint{0.553704in}{0.499691in}}{\pgfqpoint{3.699668in}{2.271769in}}%
\pgfusepath{clip}%
\pgfsetbuttcap%
\pgfsetroundjoin%
\pgfsetlinewidth{1.505625pt}%
\definecolor{currentstroke}{rgb}{0.121569,0.466667,0.705882}%
\pgfsetstrokecolor{currentstroke}%
\pgfsetdash{}{0pt}%
\pgfpathmoveto{\pgfqpoint{1.836788in}{0.770978in}}%
\pgfpathlineto{\pgfqpoint{1.836788in}{0.807845in}}%
\pgfusepath{stroke}%
\end{pgfscope}%
\begin{pgfscope}%
\pgfpathrectangle{\pgfqpoint{0.553704in}{0.499691in}}{\pgfqpoint{3.699668in}{2.271769in}}%
\pgfusepath{clip}%
\pgfsetbuttcap%
\pgfsetroundjoin%
\pgfsetlinewidth{1.505625pt}%
\definecolor{currentstroke}{rgb}{0.121569,0.466667,0.705882}%
\pgfsetstrokecolor{currentstroke}%
\pgfsetdash{}{0pt}%
\pgfpathmoveto{\pgfqpoint{2.208428in}{0.758303in}}%
\pgfpathlineto{\pgfqpoint{2.208428in}{0.888628in}}%
\pgfusepath{stroke}%
\end{pgfscope}%
\begin{pgfscope}%
\pgfpathrectangle{\pgfqpoint{0.553704in}{0.499691in}}{\pgfqpoint{3.699668in}{2.271769in}}%
\pgfusepath{clip}%
\pgfsetbuttcap%
\pgfsetroundjoin%
\pgfsetlinewidth{1.505625pt}%
\definecolor{currentstroke}{rgb}{0.121569,0.466667,0.705882}%
\pgfsetstrokecolor{currentstroke}%
\pgfsetdash{}{0pt}%
\pgfpathmoveto{\pgfqpoint{2.580067in}{0.719644in}}%
\pgfpathlineto{\pgfqpoint{2.580067in}{1.050782in}}%
\pgfusepath{stroke}%
\end{pgfscope}%
\begin{pgfscope}%
\pgfpathrectangle{\pgfqpoint{0.553704in}{0.499691in}}{\pgfqpoint{3.699668in}{2.271769in}}%
\pgfusepath{clip}%
\pgfsetbuttcap%
\pgfsetroundjoin%
\pgfsetlinewidth{1.505625pt}%
\definecolor{currentstroke}{rgb}{0.121569,0.466667,0.705882}%
\pgfsetstrokecolor{currentstroke}%
\pgfsetdash{}{0pt}%
\pgfpathmoveto{\pgfqpoint{2.951706in}{0.683333in}}%
\pgfpathlineto{\pgfqpoint{2.951706in}{1.233252in}}%
\pgfusepath{stroke}%
\end{pgfscope}%
\begin{pgfscope}%
\pgfpathrectangle{\pgfqpoint{0.553704in}{0.499691in}}{\pgfqpoint{3.699668in}{2.271769in}}%
\pgfusepath{clip}%
\pgfsetbuttcap%
\pgfsetroundjoin%
\pgfsetlinewidth{1.505625pt}%
\definecolor{currentstroke}{rgb}{0.121569,0.466667,0.705882}%
\pgfsetstrokecolor{currentstroke}%
\pgfsetdash{}{0pt}%
\pgfpathmoveto{\pgfqpoint{3.323345in}{0.666644in}}%
\pgfpathlineto{\pgfqpoint{3.323345in}{1.390987in}}%
\pgfusepath{stroke}%
\end{pgfscope}%
\begin{pgfscope}%
\pgfpathrectangle{\pgfqpoint{0.553704in}{0.499691in}}{\pgfqpoint{3.699668in}{2.271769in}}%
\pgfusepath{clip}%
\pgfsetbuttcap%
\pgfsetroundjoin%
\pgfsetlinewidth{1.505625pt}%
\definecolor{currentstroke}{rgb}{0.121569,0.466667,0.705882}%
\pgfsetstrokecolor{currentstroke}%
\pgfsetdash{}{0pt}%
\pgfpathmoveto{\pgfqpoint{3.694984in}{0.665905in}}%
\pgfpathlineto{\pgfqpoint{3.694984in}{1.524878in}}%
\pgfusepath{stroke}%
\end{pgfscope}%
\begin{pgfscope}%
\pgfpathrectangle{\pgfqpoint{0.553704in}{0.499691in}}{\pgfqpoint{3.699668in}{2.271769in}}%
\pgfusepath{clip}%
\pgfsetbuttcap%
\pgfsetroundjoin%
\pgfsetlinewidth{1.505625pt}%
\definecolor{currentstroke}{rgb}{0.121569,0.466667,0.705882}%
\pgfsetstrokecolor{currentstroke}%
\pgfsetdash{}{0pt}%
\pgfpathmoveto{\pgfqpoint{4.066623in}{0.677349in}}%
\pgfpathlineto{\pgfqpoint{4.066623in}{1.643496in}}%
\pgfusepath{stroke}%
\end{pgfscope}%
\begin{pgfscope}%
\pgfpathrectangle{\pgfqpoint{0.553704in}{0.499691in}}{\pgfqpoint{3.699668in}{2.271769in}}%
\pgfusepath{clip}%
\pgfsetbuttcap%
\pgfsetroundjoin%
\pgfsetlinewidth{1.505625pt}%
\definecolor{currentstroke}{rgb}{1.000000,0.498039,0.054902}%
\pgfsetstrokecolor{currentstroke}%
\pgfsetdash{}{0pt}%
\pgfpathmoveto{\pgfqpoint{0.731162in}{0.750680in}}%
\pgfpathlineto{\pgfqpoint{0.731162in}{0.751145in}}%
\pgfusepath{stroke}%
\end{pgfscope}%
\begin{pgfscope}%
\pgfpathrectangle{\pgfqpoint{0.553704in}{0.499691in}}{\pgfqpoint{3.699668in}{2.271769in}}%
\pgfusepath{clip}%
\pgfsetbuttcap%
\pgfsetroundjoin%
\pgfsetlinewidth{1.505625pt}%
\definecolor{currentstroke}{rgb}{1.000000,0.498039,0.054902}%
\pgfsetstrokecolor{currentstroke}%
\pgfsetdash{}{0pt}%
\pgfpathmoveto{\pgfqpoint{1.102801in}{0.751640in}}%
\pgfpathlineto{\pgfqpoint{1.102801in}{0.752866in}}%
\pgfusepath{stroke}%
\end{pgfscope}%
\begin{pgfscope}%
\pgfpathrectangle{\pgfqpoint{0.553704in}{0.499691in}}{\pgfqpoint{3.699668in}{2.271769in}}%
\pgfusepath{clip}%
\pgfsetbuttcap%
\pgfsetroundjoin%
\pgfsetlinewidth{1.505625pt}%
\definecolor{currentstroke}{rgb}{1.000000,0.498039,0.054902}%
\pgfsetstrokecolor{currentstroke}%
\pgfsetdash{}{0pt}%
\pgfpathmoveto{\pgfqpoint{1.474440in}{0.752699in}}%
\pgfpathlineto{\pgfqpoint{1.474440in}{0.756521in}}%
\pgfusepath{stroke}%
\end{pgfscope}%
\begin{pgfscope}%
\pgfpathrectangle{\pgfqpoint{0.553704in}{0.499691in}}{\pgfqpoint{3.699668in}{2.271769in}}%
\pgfusepath{clip}%
\pgfsetbuttcap%
\pgfsetroundjoin%
\pgfsetlinewidth{1.505625pt}%
\definecolor{currentstroke}{rgb}{1.000000,0.498039,0.054902}%
\pgfsetstrokecolor{currentstroke}%
\pgfsetdash{}{0pt}%
\pgfpathmoveto{\pgfqpoint{1.846079in}{0.752930in}}%
\pgfpathlineto{\pgfqpoint{1.846079in}{0.767666in}}%
\pgfusepath{stroke}%
\end{pgfscope}%
\begin{pgfscope}%
\pgfpathrectangle{\pgfqpoint{0.553704in}{0.499691in}}{\pgfqpoint{3.699668in}{2.271769in}}%
\pgfusepath{clip}%
\pgfsetbuttcap%
\pgfsetroundjoin%
\pgfsetlinewidth{1.505625pt}%
\definecolor{currentstroke}{rgb}{1.000000,0.498039,0.054902}%
\pgfsetstrokecolor{currentstroke}%
\pgfsetdash{}{0pt}%
\pgfpathmoveto{\pgfqpoint{2.217719in}{0.735359in}}%
\pgfpathlineto{\pgfqpoint{2.217719in}{0.829521in}}%
\pgfusepath{stroke}%
\end{pgfscope}%
\begin{pgfscope}%
\pgfpathrectangle{\pgfqpoint{0.553704in}{0.499691in}}{\pgfqpoint{3.699668in}{2.271769in}}%
\pgfusepath{clip}%
\pgfsetbuttcap%
\pgfsetroundjoin%
\pgfsetlinewidth{1.505625pt}%
\definecolor{currentstroke}{rgb}{1.000000,0.498039,0.054902}%
\pgfsetstrokecolor{currentstroke}%
\pgfsetdash{}{0pt}%
\pgfpathmoveto{\pgfqpoint{2.589358in}{0.687380in}}%
\pgfpathlineto{\pgfqpoint{2.589358in}{0.983839in}}%
\pgfusepath{stroke}%
\end{pgfscope}%
\begin{pgfscope}%
\pgfpathrectangle{\pgfqpoint{0.553704in}{0.499691in}}{\pgfqpoint{3.699668in}{2.271769in}}%
\pgfusepath{clip}%
\pgfsetbuttcap%
\pgfsetroundjoin%
\pgfsetlinewidth{1.505625pt}%
\definecolor{currentstroke}{rgb}{1.000000,0.498039,0.054902}%
\pgfsetstrokecolor{currentstroke}%
\pgfsetdash{}{0pt}%
\pgfpathmoveto{\pgfqpoint{2.960997in}{0.652628in}}%
\pgfpathlineto{\pgfqpoint{2.960997in}{1.134556in}}%
\pgfusepath{stroke}%
\end{pgfscope}%
\begin{pgfscope}%
\pgfpathrectangle{\pgfqpoint{0.553704in}{0.499691in}}{\pgfqpoint{3.699668in}{2.271769in}}%
\pgfusepath{clip}%
\pgfsetbuttcap%
\pgfsetroundjoin%
\pgfsetlinewidth{1.505625pt}%
\definecolor{currentstroke}{rgb}{1.000000,0.498039,0.054902}%
\pgfsetstrokecolor{currentstroke}%
\pgfsetdash{}{0pt}%
\pgfpathmoveto{\pgfqpoint{3.332636in}{0.629780in}}%
\pgfpathlineto{\pgfqpoint{3.332636in}{1.263651in}}%
\pgfusepath{stroke}%
\end{pgfscope}%
\begin{pgfscope}%
\pgfpathrectangle{\pgfqpoint{0.553704in}{0.499691in}}{\pgfqpoint{3.699668in}{2.271769in}}%
\pgfusepath{clip}%
\pgfsetbuttcap%
\pgfsetroundjoin%
\pgfsetlinewidth{1.505625pt}%
\definecolor{currentstroke}{rgb}{1.000000,0.498039,0.054902}%
\pgfsetstrokecolor{currentstroke}%
\pgfsetdash{}{0pt}%
\pgfpathmoveto{\pgfqpoint{3.704275in}{0.614004in}}%
\pgfpathlineto{\pgfqpoint{3.704275in}{1.376778in}}%
\pgfusepath{stroke}%
\end{pgfscope}%
\begin{pgfscope}%
\pgfpathrectangle{\pgfqpoint{0.553704in}{0.499691in}}{\pgfqpoint{3.699668in}{2.271769in}}%
\pgfusepath{clip}%
\pgfsetbuttcap%
\pgfsetroundjoin%
\pgfsetlinewidth{1.505625pt}%
\definecolor{currentstroke}{rgb}{1.000000,0.498039,0.054902}%
\pgfsetstrokecolor{currentstroke}%
\pgfsetdash{}{0pt}%
\pgfpathmoveto{\pgfqpoint{4.075914in}{0.602953in}}%
\pgfpathlineto{\pgfqpoint{4.075914in}{1.479698in}}%
\pgfusepath{stroke}%
\end{pgfscope}%
\begin{pgfscope}%
\pgfpathrectangle{\pgfqpoint{0.553704in}{0.499691in}}{\pgfqpoint{3.699668in}{2.271769in}}%
\pgfusepath{clip}%
\pgfsetbuttcap%
\pgfsetroundjoin%
\pgfsetlinewidth{1.505625pt}%
\definecolor{currentstroke}{rgb}{0.172549,0.627451,0.172549}%
\pgfsetstrokecolor{currentstroke}%
\pgfsetdash{}{0pt}%
\pgfpathmoveto{\pgfqpoint{0.740453in}{0.749035in}}%
\pgfpathlineto{\pgfqpoint{0.740453in}{0.749320in}}%
\pgfusepath{stroke}%
\end{pgfscope}%
\begin{pgfscope}%
\pgfpathrectangle{\pgfqpoint{0.553704in}{0.499691in}}{\pgfqpoint{3.699668in}{2.271769in}}%
\pgfusepath{clip}%
\pgfsetbuttcap%
\pgfsetroundjoin%
\pgfsetlinewidth{1.505625pt}%
\definecolor{currentstroke}{rgb}{0.172549,0.627451,0.172549}%
\pgfsetstrokecolor{currentstroke}%
\pgfsetdash{}{0pt}%
\pgfpathmoveto{\pgfqpoint{1.112092in}{0.749530in}}%
\pgfpathlineto{\pgfqpoint{1.112092in}{0.750220in}}%
\pgfusepath{stroke}%
\end{pgfscope}%
\begin{pgfscope}%
\pgfpathrectangle{\pgfqpoint{0.553704in}{0.499691in}}{\pgfqpoint{3.699668in}{2.271769in}}%
\pgfusepath{clip}%
\pgfsetbuttcap%
\pgfsetroundjoin%
\pgfsetlinewidth{1.505625pt}%
\definecolor{currentstroke}{rgb}{0.172549,0.627451,0.172549}%
\pgfsetstrokecolor{currentstroke}%
\pgfsetdash{}{0pt}%
\pgfpathmoveto{\pgfqpoint{1.483731in}{0.750095in}}%
\pgfpathlineto{\pgfqpoint{1.483731in}{0.752190in}}%
\pgfusepath{stroke}%
\end{pgfscope}%
\begin{pgfscope}%
\pgfpathrectangle{\pgfqpoint{0.553704in}{0.499691in}}{\pgfqpoint{3.699668in}{2.271769in}}%
\pgfusepath{clip}%
\pgfsetbuttcap%
\pgfsetroundjoin%
\pgfsetlinewidth{1.505625pt}%
\definecolor{currentstroke}{rgb}{0.172549,0.627451,0.172549}%
\pgfsetstrokecolor{currentstroke}%
\pgfsetdash{}{0pt}%
\pgfpathmoveto{\pgfqpoint{1.855370in}{0.750237in}}%
\pgfpathlineto{\pgfqpoint{1.855370in}{0.758645in}}%
\pgfusepath{stroke}%
\end{pgfscope}%
\begin{pgfscope}%
\pgfpathrectangle{\pgfqpoint{0.553704in}{0.499691in}}{\pgfqpoint{3.699668in}{2.271769in}}%
\pgfusepath{clip}%
\pgfsetbuttcap%
\pgfsetroundjoin%
\pgfsetlinewidth{1.505625pt}%
\definecolor{currentstroke}{rgb}{0.172549,0.627451,0.172549}%
\pgfsetstrokecolor{currentstroke}%
\pgfsetdash{}{0pt}%
\pgfpathmoveto{\pgfqpoint{2.227010in}{0.736896in}}%
\pgfpathlineto{\pgfqpoint{2.227010in}{0.805739in}}%
\pgfusepath{stroke}%
\end{pgfscope}%
\begin{pgfscope}%
\pgfpathrectangle{\pgfqpoint{0.553704in}{0.499691in}}{\pgfqpoint{3.699668in}{2.271769in}}%
\pgfusepath{clip}%
\pgfsetbuttcap%
\pgfsetroundjoin%
\pgfsetlinewidth{1.505625pt}%
\definecolor{currentstroke}{rgb}{0.172549,0.627451,0.172549}%
\pgfsetstrokecolor{currentstroke}%
\pgfsetdash{}{0pt}%
\pgfpathmoveto{\pgfqpoint{2.598649in}{0.695684in}}%
\pgfpathlineto{\pgfqpoint{2.598649in}{0.940576in}}%
\pgfusepath{stroke}%
\end{pgfscope}%
\begin{pgfscope}%
\pgfpathrectangle{\pgfqpoint{0.553704in}{0.499691in}}{\pgfqpoint{3.699668in}{2.271769in}}%
\pgfusepath{clip}%
\pgfsetbuttcap%
\pgfsetroundjoin%
\pgfsetlinewidth{1.505625pt}%
\definecolor{currentstroke}{rgb}{0.172549,0.627451,0.172549}%
\pgfsetstrokecolor{currentstroke}%
\pgfsetdash{}{0pt}%
\pgfpathmoveto{\pgfqpoint{2.970288in}{0.663187in}}%
\pgfpathlineto{\pgfqpoint{2.970288in}{1.076640in}}%
\pgfusepath{stroke}%
\end{pgfscope}%
\begin{pgfscope}%
\pgfpathrectangle{\pgfqpoint{0.553704in}{0.499691in}}{\pgfqpoint{3.699668in}{2.271769in}}%
\pgfusepath{clip}%
\pgfsetbuttcap%
\pgfsetroundjoin%
\pgfsetlinewidth{1.505625pt}%
\definecolor{currentstroke}{rgb}{0.172549,0.627451,0.172549}%
\pgfsetstrokecolor{currentstroke}%
\pgfsetdash{}{0pt}%
\pgfpathmoveto{\pgfqpoint{3.341927in}{0.639758in}}%
\pgfpathlineto{\pgfqpoint{3.341927in}{1.194584in}}%
\pgfusepath{stroke}%
\end{pgfscope}%
\begin{pgfscope}%
\pgfpathrectangle{\pgfqpoint{0.553704in}{0.499691in}}{\pgfqpoint{3.699668in}{2.271769in}}%
\pgfusepath{clip}%
\pgfsetbuttcap%
\pgfsetroundjoin%
\pgfsetlinewidth{1.505625pt}%
\definecolor{currentstroke}{rgb}{0.172549,0.627451,0.172549}%
\pgfsetstrokecolor{currentstroke}%
\pgfsetdash{}{0pt}%
\pgfpathmoveto{\pgfqpoint{3.713566in}{0.623579in}}%
\pgfpathlineto{\pgfqpoint{3.713566in}{1.296077in}}%
\pgfusepath{stroke}%
\end{pgfscope}%
\begin{pgfscope}%
\pgfpathrectangle{\pgfqpoint{0.553704in}{0.499691in}}{\pgfqpoint{3.699668in}{2.271769in}}%
\pgfusepath{clip}%
\pgfsetbuttcap%
\pgfsetroundjoin%
\pgfsetlinewidth{1.505625pt}%
\definecolor{currentstroke}{rgb}{0.172549,0.627451,0.172549}%
\pgfsetstrokecolor{currentstroke}%
\pgfsetdash{}{0pt}%
\pgfpathmoveto{\pgfqpoint{4.085205in}{0.611649in}}%
\pgfpathlineto{\pgfqpoint{4.085205in}{1.387923in}}%
\pgfusepath{stroke}%
\end{pgfscope}%
\begin{pgfscope}%
\pgfpathrectangle{\pgfqpoint{0.553704in}{0.499691in}}{\pgfqpoint{3.699668in}{2.271769in}}%
\pgfusepath{clip}%
\pgfsetrectcap%
\pgfsetroundjoin%
\pgfsetlinewidth{1.505625pt}%
\definecolor{currentstroke}{rgb}{0.839216,0.152941,0.156863}%
\pgfsetstrokecolor{currentstroke}%
\pgfsetdash{}{0pt}%
\pgfpathmoveto{\pgfqpoint{0.731162in}{0.751267in}}%
\pgfpathlineto{\pgfqpoint{1.102801in}{0.753631in}}%
\pgfpathlineto{\pgfqpoint{1.474440in}{0.759622in}}%
\pgfpathlineto{\pgfqpoint{1.846079in}{0.785410in}}%
\pgfpathlineto{\pgfqpoint{2.217719in}{1.119167in}}%
\pgfpathlineto{\pgfqpoint{2.589358in}{1.983889in}}%
\pgfpathlineto{\pgfqpoint{2.960997in}{2.333774in}}%
\pgfpathlineto{\pgfqpoint{3.332636in}{2.498270in}}%
\pgfpathlineto{\pgfqpoint{3.704275in}{2.597860in}}%
\pgfpathlineto{\pgfqpoint{4.075914in}{2.668198in}}%
\pgfusepath{stroke}%
\end{pgfscope}%
\begin{pgfscope}%
\pgfpathrectangle{\pgfqpoint{0.553704in}{0.499691in}}{\pgfqpoint{3.699668in}{2.271769in}}%
\pgfusepath{clip}%
\pgfsetrectcap%
\pgfsetroundjoin%
\pgfsetlinewidth{1.505625pt}%
\definecolor{currentstroke}{rgb}{0.121569,0.466667,0.705882}%
\pgfsetstrokecolor{currentstroke}%
\pgfsetdash{}{0pt}%
\pgfpathmoveto{\pgfqpoint{0.721871in}{0.761144in}}%
\pgfpathlineto{\pgfqpoint{1.093510in}{0.766015in}}%
\pgfpathlineto{\pgfqpoint{1.465149in}{0.774032in}}%
\pgfpathlineto{\pgfqpoint{1.836788in}{0.789412in}}%
\pgfpathlineto{\pgfqpoint{2.208428in}{0.823465in}}%
\pgfpathlineto{\pgfqpoint{2.580067in}{0.885213in}}%
\pgfpathlineto{\pgfqpoint{2.951706in}{0.958293in}}%
\pgfpathlineto{\pgfqpoint{3.323345in}{1.028815in}}%
\pgfpathlineto{\pgfqpoint{3.694984in}{1.095391in}}%
\pgfpathlineto{\pgfqpoint{4.066623in}{1.160422in}}%
\pgfusepath{stroke}%
\end{pgfscope}%
\begin{pgfscope}%
\pgfpathrectangle{\pgfqpoint{0.553704in}{0.499691in}}{\pgfqpoint{3.699668in}{2.271769in}}%
\pgfusepath{clip}%
\pgfsetrectcap%
\pgfsetroundjoin%
\pgfsetlinewidth{1.505625pt}%
\definecolor{currentstroke}{rgb}{1.000000,0.498039,0.054902}%
\pgfsetstrokecolor{currentstroke}%
\pgfsetdash{}{0pt}%
\pgfpathmoveto{\pgfqpoint{0.731162in}{0.750912in}}%
\pgfpathlineto{\pgfqpoint{1.102801in}{0.752253in}}%
\pgfpathlineto{\pgfqpoint{1.474440in}{0.754610in}}%
\pgfpathlineto{\pgfqpoint{1.846079in}{0.760298in}}%
\pgfpathlineto{\pgfqpoint{2.217719in}{0.782440in}}%
\pgfpathlineto{\pgfqpoint{2.589358in}{0.835610in}}%
\pgfpathlineto{\pgfqpoint{2.960997in}{0.893592in}}%
\pgfpathlineto{\pgfqpoint{3.332636in}{0.946715in}}%
\pgfpathlineto{\pgfqpoint{3.704275in}{0.995391in}}%
\pgfpathlineto{\pgfqpoint{4.075914in}{1.041325in}}%
\pgfusepath{stroke}%
\end{pgfscope}%
\begin{pgfscope}%
\pgfpathrectangle{\pgfqpoint{0.553704in}{0.499691in}}{\pgfqpoint{3.699668in}{2.271769in}}%
\pgfusepath{clip}%
\pgfsetrectcap%
\pgfsetroundjoin%
\pgfsetlinewidth{1.505625pt}%
\definecolor{currentstroke}{rgb}{0.172549,0.627451,0.172549}%
\pgfsetstrokecolor{currentstroke}%
\pgfsetdash{}{0pt}%
\pgfpathmoveto{\pgfqpoint{0.740453in}{0.749177in}}%
\pgfpathlineto{\pgfqpoint{1.112092in}{0.749875in}}%
\pgfpathlineto{\pgfqpoint{1.483731in}{0.751142in}}%
\pgfpathlineto{\pgfqpoint{1.855370in}{0.754441in}}%
\pgfpathlineto{\pgfqpoint{2.227010in}{0.771317in}}%
\pgfpathlineto{\pgfqpoint{2.598649in}{0.818130in}}%
\pgfpathlineto{\pgfqpoint{2.970288in}{0.869913in}}%
\pgfpathlineto{\pgfqpoint{3.341927in}{0.917171in}}%
\pgfpathlineto{\pgfqpoint{3.713566in}{0.959828in}}%
\pgfpathlineto{\pgfqpoint{4.085205in}{0.999786in}}%
\pgfusepath{stroke}%
\end{pgfscope}%
\begin{pgfscope}%
\pgfsetrectcap%
\pgfsetmiterjoin%
\pgfsetlinewidth{0.803000pt}%
\definecolor{currentstroke}{rgb}{0.000000,0.000000,0.000000}%
\pgfsetstrokecolor{currentstroke}%
\pgfsetdash{}{0pt}%
\pgfpathmoveto{\pgfqpoint{0.553704in}{0.499691in}}%
\pgfpathlineto{\pgfqpoint{0.553704in}{2.771460in}}%
\pgfusepath{stroke}%
\end{pgfscope}%
\begin{pgfscope}%
\pgfsetrectcap%
\pgfsetmiterjoin%
\pgfsetlinewidth{0.803000pt}%
\definecolor{currentstroke}{rgb}{0.000000,0.000000,0.000000}%
\pgfsetstrokecolor{currentstroke}%
\pgfsetdash{}{0pt}%
\pgfpathmoveto{\pgfqpoint{4.253372in}{0.499691in}}%
\pgfpathlineto{\pgfqpoint{4.253372in}{2.771460in}}%
\pgfusepath{stroke}%
\end{pgfscope}%
\begin{pgfscope}%
\pgfsetrectcap%
\pgfsetmiterjoin%
\pgfsetlinewidth{0.803000pt}%
\definecolor{currentstroke}{rgb}{0.000000,0.000000,0.000000}%
\pgfsetstrokecolor{currentstroke}%
\pgfsetdash{}{0pt}%
\pgfpathmoveto{\pgfqpoint{0.553704in}{0.499691in}}%
\pgfpathlineto{\pgfqpoint{4.253372in}{0.499691in}}%
\pgfusepath{stroke}%
\end{pgfscope}%
\begin{pgfscope}%
\pgfsetrectcap%
\pgfsetmiterjoin%
\pgfsetlinewidth{0.803000pt}%
\definecolor{currentstroke}{rgb}{0.000000,0.000000,0.000000}%
\pgfsetstrokecolor{currentstroke}%
\pgfsetdash{}{0pt}%
\pgfpathmoveto{\pgfqpoint{0.553704in}{2.771460in}}%
\pgfpathlineto{\pgfqpoint{4.253372in}{2.771460in}}%
\pgfusepath{stroke}%
\end{pgfscope}%
\begin{pgfscope}%
\pgfsetbuttcap%
\pgfsetmiterjoin%
\definecolor{currentfill}{rgb}{1.000000,1.000000,1.000000}%
\pgfsetfillcolor{currentfill}%
\pgfsetfillopacity{0.800000}%
\pgfsetlinewidth{1.003750pt}%
\definecolor{currentstroke}{rgb}{0.800000,0.800000,0.800000}%
\pgfsetstrokecolor{currentstroke}%
\pgfsetstrokeopacity{0.800000}%
\pgfsetdash{}{0pt}%
\pgfpathmoveto{\pgfqpoint{0.650926in}{1.885658in}}%
\pgfpathlineto{\pgfqpoint{1.914239in}{1.885658in}}%
\pgfpathquadraticcurveto{\pgfqpoint{1.942017in}{1.885658in}}{\pgfqpoint{1.942017in}{1.913435in}}%
\pgfpathlineto{\pgfqpoint{1.942017in}{2.674238in}}%
\pgfpathquadraticcurveto{\pgfqpoint{1.942017in}{2.702015in}}{\pgfqpoint{1.914239in}{2.702015in}}%
\pgfpathlineto{\pgfqpoint{0.650926in}{2.702015in}}%
\pgfpathquadraticcurveto{\pgfqpoint{0.623149in}{2.702015in}}{\pgfqpoint{0.623149in}{2.674238in}}%
\pgfpathlineto{\pgfqpoint{0.623149in}{1.913435in}}%
\pgfpathquadraticcurveto{\pgfqpoint{0.623149in}{1.885658in}}{\pgfqpoint{0.650926in}{1.885658in}}%
\pgfpathlineto{\pgfqpoint{0.650926in}{1.885658in}}%
\pgfpathclose%
\pgfusepath{stroke,fill}%
\end{pgfscope}%
\begin{pgfscope}%
\pgfsetrectcap%
\pgfsetroundjoin%
\pgfsetlinewidth{1.505625pt}%
\definecolor{currentstroke}{rgb}{0.839216,0.152941,0.156863}%
\pgfsetstrokecolor{currentstroke}%
\pgfsetdash{}{0pt}%
\pgfpathmoveto{\pgfqpoint{0.678704in}{2.597849in}}%
\pgfpathlineto{\pgfqpoint{0.817593in}{2.597849in}}%
\pgfpathlineto{\pgfqpoint{0.956482in}{2.597849in}}%
\pgfusepath{stroke}%
\end{pgfscope}%
\begin{pgfscope}%
\definecolor{textcolor}{rgb}{0.000000,0.000000,0.000000}%
\pgfsetstrokecolor{textcolor}%
\pgfsetfillcolor{textcolor}%
\pgftext[x=1.067593in,y=2.549238in,left,base]{\color{textcolor}\sffamily\fontsize{10.000000}{12.000000}\selectfont U=1, L=1000}%
\end{pgfscope}%
\begin{pgfscope}%
\pgfsetbuttcap%
\pgfsetroundjoin%
\pgfsetlinewidth{1.505625pt}%
\definecolor{currentstroke}{rgb}{0.121569,0.466667,0.705882}%
\pgfsetstrokecolor{currentstroke}%
\pgfsetdash{}{0pt}%
\pgfpathmoveto{\pgfqpoint{0.817593in}{2.334732in}}%
\pgfpathlineto{\pgfqpoint{0.817593in}{2.473620in}}%
\pgfusepath{stroke}%
\end{pgfscope}%
\begin{pgfscope}%
\pgfsetrectcap%
\pgfsetroundjoin%
\pgfsetlinewidth{1.505625pt}%
\definecolor{currentstroke}{rgb}{0.121569,0.466667,0.705882}%
\pgfsetstrokecolor{currentstroke}%
\pgfsetdash{}{0pt}%
\pgfpathmoveto{\pgfqpoint{0.678704in}{2.404176in}}%
\pgfpathlineto{\pgfqpoint{0.956482in}{2.404176in}}%
\pgfusepath{stroke}%
\end{pgfscope}%
\begin{pgfscope}%
\definecolor{textcolor}{rgb}{0.000000,0.000000,0.000000}%
\pgfsetstrokecolor{textcolor}%
\pgfsetfillcolor{textcolor}%
\pgftext[x=1.067593in,y=2.355565in,left,base]{\color{textcolor}\sffamily\fontsize{10.000000}{12.000000}\selectfont 250}%
\end{pgfscope}%
\begin{pgfscope}%
\pgfsetbuttcap%
\pgfsetroundjoin%
\pgfsetlinewidth{1.505625pt}%
\definecolor{currentstroke}{rgb}{1.000000,0.498039,0.054902}%
\pgfsetstrokecolor{currentstroke}%
\pgfsetdash{}{0pt}%
\pgfpathmoveto{\pgfqpoint{0.817593in}{2.141059in}}%
\pgfpathlineto{\pgfqpoint{0.817593in}{2.279948in}}%
\pgfusepath{stroke}%
\end{pgfscope}%
\begin{pgfscope}%
\pgfsetrectcap%
\pgfsetroundjoin%
\pgfsetlinewidth{1.505625pt}%
\definecolor{currentstroke}{rgb}{1.000000,0.498039,0.054902}%
\pgfsetstrokecolor{currentstroke}%
\pgfsetdash{}{0pt}%
\pgfpathmoveto{\pgfqpoint{0.678704in}{2.210503in}}%
\pgfpathlineto{\pgfqpoint{0.956482in}{2.210503in}}%
\pgfusepath{stroke}%
\end{pgfscope}%
\begin{pgfscope}%
\definecolor{textcolor}{rgb}{0.000000,0.000000,0.000000}%
\pgfsetstrokecolor{textcolor}%
\pgfsetfillcolor{textcolor}%
\pgftext[x=1.067593in,y=2.161892in,left,base]{\color{textcolor}\sffamily\fontsize{10.000000}{12.000000}\selectfont 1000}%
\end{pgfscope}%
\begin{pgfscope}%
\pgfsetbuttcap%
\pgfsetroundjoin%
\pgfsetlinewidth{1.505625pt}%
\definecolor{currentstroke}{rgb}{0.172549,0.627451,0.172549}%
\pgfsetstrokecolor{currentstroke}%
\pgfsetdash{}{0pt}%
\pgfpathmoveto{\pgfqpoint{0.817593in}{1.947386in}}%
\pgfpathlineto{\pgfqpoint{0.817593in}{2.086275in}}%
\pgfusepath{stroke}%
\end{pgfscope}%
\begin{pgfscope}%
\pgfsetrectcap%
\pgfsetroundjoin%
\pgfsetlinewidth{1.505625pt}%
\definecolor{currentstroke}{rgb}{0.172549,0.627451,0.172549}%
\pgfsetstrokecolor{currentstroke}%
\pgfsetdash{}{0pt}%
\pgfpathmoveto{\pgfqpoint{0.678704in}{2.016830in}}%
\pgfpathlineto{\pgfqpoint{0.956482in}{2.016830in}}%
\pgfusepath{stroke}%
\end{pgfscope}%
\begin{pgfscope}%
\definecolor{textcolor}{rgb}{0.000000,0.000000,0.000000}%
\pgfsetstrokecolor{textcolor}%
\pgfsetfillcolor{textcolor}%
\pgftext[x=1.067593in,y=1.968219in,left,base]{\color{textcolor}\sffamily\fontsize{10.000000}{12.000000}\selectfont 2000}%
\end{pgfscope}%
\end{pgfpicture}%
\makeatother%
\endgroup%


	\caption{Средняя намагниченность конформаций при $U=0.1$. Цветами отмечены конформации разной длины, число конформаций каждой длины - 1000. Красный график намагниченности конформаций при $U=1$, длины 1000.}
	\label{fig:U0.1_mean_mag2}
\end{figure}

Среди полученных конформаций так же встречаются намагничивающиеся. Однако если мы посмотрим на намагниченность конформаций при $\beta = 1$ то среди конформаций длины 250 будет только 4 конформации с намагниченностью больше 0.9, среди конформаций длиной 500 их 2, и в наборах с длинами 1000 и 2000 таких конформаций нет. На рис.\ref{fig:fraction_magnetization} видно, что не намагничивающиеся конформации составляют большую часть всех конформаций, и что при увеличении длины конформаций, доля намагничивающихся конформаций уменьшается. Максимальная намагниченность, достигаемая конформациями: 0.950, 0.947, 0.799, 0.788 - для длин 250, 500, 1000, 2000 соответственно.

\begin{figure}[htb]
	\centering
	%% Creator: Matplotlib, PGF backend
%%
%% To include the figure in your LaTeX document, write
%%   \input{<filename>.pgf}
%%
%% Make sure the required packages are loaded in your preamble
%%   \usepackage{pgf}
%%
%% and, on pdftex
%%   \usepackage[utf8]{inputenc}\DeclareUnicodeCharacter{2212}{-}
%%
%% or, on luatex and xetex
%%   \usepackage{unicode-math}
%%
%% Figures using additional raster images can only be included by \input if
%% they are in the same directory as the main LaTeX file. For loading figures
%% from other directories you can use the `import` package
%%   \usepackage{import}
%%
%% and then include the figures with
%%   \import{<path to file>}{<filename>.pgf}
%%
%% Matplotlib used the following preamble
%%   \usepackage{fontspec}
%%   \setmainfont{DejaVuSerif.ttf}[Path=C:/Programs/Anaconda/envs/Latest_version/Lib/site-packages/matplotlib/mpl-data/fonts/ttf/]
%%   \setsansfont{DejaVuSans.ttf}[Path=C:/Programs/Anaconda/envs/Latest_version/Lib/site-packages/matplotlib/mpl-data/fonts/ttf/]
%%   \setmonofont{DejaVuSansMono.ttf}[Path=C:/Programs/Anaconda/envs/Latest_version/Lib/site-packages/matplotlib/mpl-data/fonts/ttf/]
%%
\begingroup%
\makeatletter%
\begin{pgfpicture}%
\pgfpathrectangle{\pgfpointorigin}{\pgfqpoint{4.527082in}{2.981883in}}%
\pgfusepath{use as bounding box, clip}%
\begin{pgfscope}%
\pgfsetbuttcap%
\pgfsetmiterjoin%
\pgfsetlinewidth{0.000000pt}%
\definecolor{currentstroke}{rgb}{1.000000,1.000000,1.000000}%
\pgfsetstrokecolor{currentstroke}%
\pgfsetstrokeopacity{0.000000}%
\pgfsetdash{}{0pt}%
\pgfpathmoveto{\pgfqpoint{0.000000in}{0.000000in}}%
\pgfpathlineto{\pgfqpoint{4.527082in}{0.000000in}}%
\pgfpathlineto{\pgfqpoint{4.527082in}{2.981883in}}%
\pgfpathlineto{\pgfqpoint{0.000000in}{2.981883in}}%
\pgfpathclose%
\pgfusepath{}%
\end{pgfscope}%
\begin{pgfscope}%
\pgfsetbuttcap%
\pgfsetmiterjoin%
\definecolor{currentfill}{rgb}{1.000000,1.000000,1.000000}%
\pgfsetfillcolor{currentfill}%
\pgfsetlinewidth{0.000000pt}%
\definecolor{currentstroke}{rgb}{0.000000,0.000000,0.000000}%
\pgfsetstrokecolor{currentstroke}%
\pgfsetstrokeopacity{0.000000}%
\pgfsetdash{}{0pt}%
\pgfpathmoveto{\pgfqpoint{0.608070in}{0.521603in}}%
\pgfpathlineto{\pgfqpoint{4.427082in}{0.521603in}}%
\pgfpathlineto{\pgfqpoint{4.427082in}{2.881883in}}%
\pgfpathlineto{\pgfqpoint{0.608070in}{2.881883in}}%
\pgfpathclose%
\pgfusepath{fill}%
\end{pgfscope}%
\begin{pgfscope}%
\pgfpathrectangle{\pgfqpoint{0.608070in}{0.521603in}}{\pgfqpoint{3.819012in}{2.360279in}}%
\pgfusepath{clip}%
\pgfsetrectcap%
\pgfsetroundjoin%
\pgfsetlinewidth{0.803000pt}%
\definecolor{currentstroke}{rgb}{0.690196,0.690196,0.690196}%
\pgfsetstrokecolor{currentstroke}%
\pgfsetdash{}{0pt}%
\pgfpathmoveto{\pgfqpoint{0.781661in}{0.521603in}}%
\pgfpathlineto{\pgfqpoint{0.781661in}{2.881883in}}%
\pgfusepath{stroke}%
\end{pgfscope}%
\begin{pgfscope}%
\pgfsetbuttcap%
\pgfsetroundjoin%
\definecolor{currentfill}{rgb}{0.000000,0.000000,0.000000}%
\pgfsetfillcolor{currentfill}%
\pgfsetlinewidth{0.803000pt}%
\definecolor{currentstroke}{rgb}{0.000000,0.000000,0.000000}%
\pgfsetstrokecolor{currentstroke}%
\pgfsetdash{}{0pt}%
\pgfsys@defobject{currentmarker}{\pgfqpoint{0.000000in}{-0.048611in}}{\pgfqpoint{0.000000in}{0.000000in}}{%
\pgfpathmoveto{\pgfqpoint{0.000000in}{0.000000in}}%
\pgfpathlineto{\pgfqpoint{0.000000in}{-0.048611in}}%
\pgfusepath{stroke,fill}%
}%
\begin{pgfscope}%
\pgfsys@transformshift{0.781661in}{0.521603in}%
\pgfsys@useobject{currentmarker}{}%
\end{pgfscope}%
\end{pgfscope}%
\begin{pgfscope}%
\definecolor{textcolor}{rgb}{0.000000,0.000000,0.000000}%
\pgfsetstrokecolor{textcolor}%
\pgfsetfillcolor{textcolor}%
\pgftext[x=0.781661in,y=0.424381in,,top]{\color{textcolor}\sffamily\fontsize{10.000000}{12.000000}\selectfont 0.0}%
\end{pgfscope}%
\begin{pgfscope}%
\pgfpathrectangle{\pgfqpoint{0.608070in}{0.521603in}}{\pgfqpoint{3.819012in}{2.360279in}}%
\pgfusepath{clip}%
\pgfsetrectcap%
\pgfsetroundjoin%
\pgfsetlinewidth{0.803000pt}%
\definecolor{currentstroke}{rgb}{0.690196,0.690196,0.690196}%
\pgfsetstrokecolor{currentstroke}%
\pgfsetdash{}{0pt}%
\pgfpathmoveto{\pgfqpoint{1.476027in}{0.521603in}}%
\pgfpathlineto{\pgfqpoint{1.476027in}{2.881883in}}%
\pgfusepath{stroke}%
\end{pgfscope}%
\begin{pgfscope}%
\pgfsetbuttcap%
\pgfsetroundjoin%
\definecolor{currentfill}{rgb}{0.000000,0.000000,0.000000}%
\pgfsetfillcolor{currentfill}%
\pgfsetlinewidth{0.803000pt}%
\definecolor{currentstroke}{rgb}{0.000000,0.000000,0.000000}%
\pgfsetstrokecolor{currentstroke}%
\pgfsetdash{}{0pt}%
\pgfsys@defobject{currentmarker}{\pgfqpoint{0.000000in}{-0.048611in}}{\pgfqpoint{0.000000in}{0.000000in}}{%
\pgfpathmoveto{\pgfqpoint{0.000000in}{0.000000in}}%
\pgfpathlineto{\pgfqpoint{0.000000in}{-0.048611in}}%
\pgfusepath{stroke,fill}%
}%
\begin{pgfscope}%
\pgfsys@transformshift{1.476027in}{0.521603in}%
\pgfsys@useobject{currentmarker}{}%
\end{pgfscope}%
\end{pgfscope}%
\begin{pgfscope}%
\definecolor{textcolor}{rgb}{0.000000,0.000000,0.000000}%
\pgfsetstrokecolor{textcolor}%
\pgfsetfillcolor{textcolor}%
\pgftext[x=1.476027in,y=0.424381in,,top]{\color{textcolor}\sffamily\fontsize{10.000000}{12.000000}\selectfont 0.2}%
\end{pgfscope}%
\begin{pgfscope}%
\pgfpathrectangle{\pgfqpoint{0.608070in}{0.521603in}}{\pgfqpoint{3.819012in}{2.360279in}}%
\pgfusepath{clip}%
\pgfsetrectcap%
\pgfsetroundjoin%
\pgfsetlinewidth{0.803000pt}%
\definecolor{currentstroke}{rgb}{0.690196,0.690196,0.690196}%
\pgfsetstrokecolor{currentstroke}%
\pgfsetdash{}{0pt}%
\pgfpathmoveto{\pgfqpoint{2.170393in}{0.521603in}}%
\pgfpathlineto{\pgfqpoint{2.170393in}{2.881883in}}%
\pgfusepath{stroke}%
\end{pgfscope}%
\begin{pgfscope}%
\pgfsetbuttcap%
\pgfsetroundjoin%
\definecolor{currentfill}{rgb}{0.000000,0.000000,0.000000}%
\pgfsetfillcolor{currentfill}%
\pgfsetlinewidth{0.803000pt}%
\definecolor{currentstroke}{rgb}{0.000000,0.000000,0.000000}%
\pgfsetstrokecolor{currentstroke}%
\pgfsetdash{}{0pt}%
\pgfsys@defobject{currentmarker}{\pgfqpoint{0.000000in}{-0.048611in}}{\pgfqpoint{0.000000in}{0.000000in}}{%
\pgfpathmoveto{\pgfqpoint{0.000000in}{0.000000in}}%
\pgfpathlineto{\pgfqpoint{0.000000in}{-0.048611in}}%
\pgfusepath{stroke,fill}%
}%
\begin{pgfscope}%
\pgfsys@transformshift{2.170393in}{0.521603in}%
\pgfsys@useobject{currentmarker}{}%
\end{pgfscope}%
\end{pgfscope}%
\begin{pgfscope}%
\definecolor{textcolor}{rgb}{0.000000,0.000000,0.000000}%
\pgfsetstrokecolor{textcolor}%
\pgfsetfillcolor{textcolor}%
\pgftext[x=2.170393in,y=0.424381in,,top]{\color{textcolor}\sffamily\fontsize{10.000000}{12.000000}\selectfont 0.4}%
\end{pgfscope}%
\begin{pgfscope}%
\pgfpathrectangle{\pgfqpoint{0.608070in}{0.521603in}}{\pgfqpoint{3.819012in}{2.360279in}}%
\pgfusepath{clip}%
\pgfsetrectcap%
\pgfsetroundjoin%
\pgfsetlinewidth{0.803000pt}%
\definecolor{currentstroke}{rgb}{0.690196,0.690196,0.690196}%
\pgfsetstrokecolor{currentstroke}%
\pgfsetdash{}{0pt}%
\pgfpathmoveto{\pgfqpoint{2.864759in}{0.521603in}}%
\pgfpathlineto{\pgfqpoint{2.864759in}{2.881883in}}%
\pgfusepath{stroke}%
\end{pgfscope}%
\begin{pgfscope}%
\pgfsetbuttcap%
\pgfsetroundjoin%
\definecolor{currentfill}{rgb}{0.000000,0.000000,0.000000}%
\pgfsetfillcolor{currentfill}%
\pgfsetlinewidth{0.803000pt}%
\definecolor{currentstroke}{rgb}{0.000000,0.000000,0.000000}%
\pgfsetstrokecolor{currentstroke}%
\pgfsetdash{}{0pt}%
\pgfsys@defobject{currentmarker}{\pgfqpoint{0.000000in}{-0.048611in}}{\pgfqpoint{0.000000in}{0.000000in}}{%
\pgfpathmoveto{\pgfqpoint{0.000000in}{0.000000in}}%
\pgfpathlineto{\pgfqpoint{0.000000in}{-0.048611in}}%
\pgfusepath{stroke,fill}%
}%
\begin{pgfscope}%
\pgfsys@transformshift{2.864759in}{0.521603in}%
\pgfsys@useobject{currentmarker}{}%
\end{pgfscope}%
\end{pgfscope}%
\begin{pgfscope}%
\definecolor{textcolor}{rgb}{0.000000,0.000000,0.000000}%
\pgfsetstrokecolor{textcolor}%
\pgfsetfillcolor{textcolor}%
\pgftext[x=2.864759in,y=0.424381in,,top]{\color{textcolor}\sffamily\fontsize{10.000000}{12.000000}\selectfont 0.6}%
\end{pgfscope}%
\begin{pgfscope}%
\pgfpathrectangle{\pgfqpoint{0.608070in}{0.521603in}}{\pgfqpoint{3.819012in}{2.360279in}}%
\pgfusepath{clip}%
\pgfsetrectcap%
\pgfsetroundjoin%
\pgfsetlinewidth{0.803000pt}%
\definecolor{currentstroke}{rgb}{0.690196,0.690196,0.690196}%
\pgfsetstrokecolor{currentstroke}%
\pgfsetdash{}{0pt}%
\pgfpathmoveto{\pgfqpoint{3.559125in}{0.521603in}}%
\pgfpathlineto{\pgfqpoint{3.559125in}{2.881883in}}%
\pgfusepath{stroke}%
\end{pgfscope}%
\begin{pgfscope}%
\pgfsetbuttcap%
\pgfsetroundjoin%
\definecolor{currentfill}{rgb}{0.000000,0.000000,0.000000}%
\pgfsetfillcolor{currentfill}%
\pgfsetlinewidth{0.803000pt}%
\definecolor{currentstroke}{rgb}{0.000000,0.000000,0.000000}%
\pgfsetstrokecolor{currentstroke}%
\pgfsetdash{}{0pt}%
\pgfsys@defobject{currentmarker}{\pgfqpoint{0.000000in}{-0.048611in}}{\pgfqpoint{0.000000in}{0.000000in}}{%
\pgfpathmoveto{\pgfqpoint{0.000000in}{0.000000in}}%
\pgfpathlineto{\pgfqpoint{0.000000in}{-0.048611in}}%
\pgfusepath{stroke,fill}%
}%
\begin{pgfscope}%
\pgfsys@transformshift{3.559125in}{0.521603in}%
\pgfsys@useobject{currentmarker}{}%
\end{pgfscope}%
\end{pgfscope}%
\begin{pgfscope}%
\definecolor{textcolor}{rgb}{0.000000,0.000000,0.000000}%
\pgfsetstrokecolor{textcolor}%
\pgfsetfillcolor{textcolor}%
\pgftext[x=3.559125in,y=0.424381in,,top]{\color{textcolor}\sffamily\fontsize{10.000000}{12.000000}\selectfont 0.8}%
\end{pgfscope}%
\begin{pgfscope}%
\pgfpathrectangle{\pgfqpoint{0.608070in}{0.521603in}}{\pgfqpoint{3.819012in}{2.360279in}}%
\pgfusepath{clip}%
\pgfsetrectcap%
\pgfsetroundjoin%
\pgfsetlinewidth{0.803000pt}%
\definecolor{currentstroke}{rgb}{0.690196,0.690196,0.690196}%
\pgfsetstrokecolor{currentstroke}%
\pgfsetdash{}{0pt}%
\pgfpathmoveto{\pgfqpoint{4.253491in}{0.521603in}}%
\pgfpathlineto{\pgfqpoint{4.253491in}{2.881883in}}%
\pgfusepath{stroke}%
\end{pgfscope}%
\begin{pgfscope}%
\pgfsetbuttcap%
\pgfsetroundjoin%
\definecolor{currentfill}{rgb}{0.000000,0.000000,0.000000}%
\pgfsetfillcolor{currentfill}%
\pgfsetlinewidth{0.803000pt}%
\definecolor{currentstroke}{rgb}{0.000000,0.000000,0.000000}%
\pgfsetstrokecolor{currentstroke}%
\pgfsetdash{}{0pt}%
\pgfsys@defobject{currentmarker}{\pgfqpoint{0.000000in}{-0.048611in}}{\pgfqpoint{0.000000in}{0.000000in}}{%
\pgfpathmoveto{\pgfqpoint{0.000000in}{0.000000in}}%
\pgfpathlineto{\pgfqpoint{0.000000in}{-0.048611in}}%
\pgfusepath{stroke,fill}%
}%
\begin{pgfscope}%
\pgfsys@transformshift{4.253491in}{0.521603in}%
\pgfsys@useobject{currentmarker}{}%
\end{pgfscope}%
\end{pgfscope}%
\begin{pgfscope}%
\definecolor{textcolor}{rgb}{0.000000,0.000000,0.000000}%
\pgfsetstrokecolor{textcolor}%
\pgfsetfillcolor{textcolor}%
\pgftext[x=4.253491in,y=0.424381in,,top]{\color{textcolor}\sffamily\fontsize{10.000000}{12.000000}\selectfont 1.0}%
\end{pgfscope}%
\begin{pgfscope}%
\definecolor{textcolor}{rgb}{0.000000,0.000000,0.000000}%
\pgfsetstrokecolor{textcolor}%
\pgfsetfillcolor{textcolor}%
\pgftext[x=2.517576in,y=0.234413in,,top]{\color{textcolor}\sffamily\fontsize{10.000000}{12.000000}\selectfont \(\displaystyle M^2\)}%
\end{pgfscope}%
\begin{pgfscope}%
\pgfpathrectangle{\pgfqpoint{0.608070in}{0.521603in}}{\pgfqpoint{3.819012in}{2.360279in}}%
\pgfusepath{clip}%
\pgfsetrectcap%
\pgfsetroundjoin%
\pgfsetlinewidth{0.803000pt}%
\definecolor{currentstroke}{rgb}{0.690196,0.690196,0.690196}%
\pgfsetstrokecolor{currentstroke}%
\pgfsetdash{}{0pt}%
\pgfpathmoveto{\pgfqpoint{0.608070in}{0.628889in}}%
\pgfpathlineto{\pgfqpoint{4.427082in}{0.628889in}}%
\pgfusepath{stroke}%
\end{pgfscope}%
\begin{pgfscope}%
\pgfsetbuttcap%
\pgfsetroundjoin%
\definecolor{currentfill}{rgb}{0.000000,0.000000,0.000000}%
\pgfsetfillcolor{currentfill}%
\pgfsetlinewidth{0.803000pt}%
\definecolor{currentstroke}{rgb}{0.000000,0.000000,0.000000}%
\pgfsetstrokecolor{currentstroke}%
\pgfsetdash{}{0pt}%
\pgfsys@defobject{currentmarker}{\pgfqpoint{-0.048611in}{0.000000in}}{\pgfqpoint{-0.000000in}{0.000000in}}{%
\pgfpathmoveto{\pgfqpoint{-0.000000in}{0.000000in}}%
\pgfpathlineto{\pgfqpoint{-0.048611in}{0.000000in}}%
\pgfusepath{stroke,fill}%
}%
\begin{pgfscope}%
\pgfsys@transformshift{0.608070in}{0.628889in}%
\pgfsys@useobject{currentmarker}{}%
\end{pgfscope}%
\end{pgfscope}%
\begin{pgfscope}%
\definecolor{textcolor}{rgb}{0.000000,0.000000,0.000000}%
\pgfsetstrokecolor{textcolor}%
\pgfsetfillcolor{textcolor}%
\pgftext[x=0.289968in, y=0.576127in, left, base]{\color{textcolor}\sffamily\fontsize{10.000000}{12.000000}\selectfont 0.0}%
\end{pgfscope}%
\begin{pgfscope}%
\pgfpathrectangle{\pgfqpoint{0.608070in}{0.521603in}}{\pgfqpoint{3.819012in}{2.360279in}}%
\pgfusepath{clip}%
\pgfsetrectcap%
\pgfsetroundjoin%
\pgfsetlinewidth{0.803000pt}%
\definecolor{currentstroke}{rgb}{0.690196,0.690196,0.690196}%
\pgfsetstrokecolor{currentstroke}%
\pgfsetdash{}{0pt}%
\pgfpathmoveto{\pgfqpoint{0.608070in}{1.058030in}}%
\pgfpathlineto{\pgfqpoint{4.427082in}{1.058030in}}%
\pgfusepath{stroke}%
\end{pgfscope}%
\begin{pgfscope}%
\pgfsetbuttcap%
\pgfsetroundjoin%
\definecolor{currentfill}{rgb}{0.000000,0.000000,0.000000}%
\pgfsetfillcolor{currentfill}%
\pgfsetlinewidth{0.803000pt}%
\definecolor{currentstroke}{rgb}{0.000000,0.000000,0.000000}%
\pgfsetstrokecolor{currentstroke}%
\pgfsetdash{}{0pt}%
\pgfsys@defobject{currentmarker}{\pgfqpoint{-0.048611in}{0.000000in}}{\pgfqpoint{-0.000000in}{0.000000in}}{%
\pgfpathmoveto{\pgfqpoint{-0.000000in}{0.000000in}}%
\pgfpathlineto{\pgfqpoint{-0.048611in}{0.000000in}}%
\pgfusepath{stroke,fill}%
}%
\begin{pgfscope}%
\pgfsys@transformshift{0.608070in}{1.058030in}%
\pgfsys@useobject{currentmarker}{}%
\end{pgfscope}%
\end{pgfscope}%
\begin{pgfscope}%
\definecolor{textcolor}{rgb}{0.000000,0.000000,0.000000}%
\pgfsetstrokecolor{textcolor}%
\pgfsetfillcolor{textcolor}%
\pgftext[x=0.289968in, y=1.005269in, left, base]{\color{textcolor}\sffamily\fontsize{10.000000}{12.000000}\selectfont 0.2}%
\end{pgfscope}%
\begin{pgfscope}%
\pgfpathrectangle{\pgfqpoint{0.608070in}{0.521603in}}{\pgfqpoint{3.819012in}{2.360279in}}%
\pgfusepath{clip}%
\pgfsetrectcap%
\pgfsetroundjoin%
\pgfsetlinewidth{0.803000pt}%
\definecolor{currentstroke}{rgb}{0.690196,0.690196,0.690196}%
\pgfsetstrokecolor{currentstroke}%
\pgfsetdash{}{0pt}%
\pgfpathmoveto{\pgfqpoint{0.608070in}{1.487172in}}%
\pgfpathlineto{\pgfqpoint{4.427082in}{1.487172in}}%
\pgfusepath{stroke}%
\end{pgfscope}%
\begin{pgfscope}%
\pgfsetbuttcap%
\pgfsetroundjoin%
\definecolor{currentfill}{rgb}{0.000000,0.000000,0.000000}%
\pgfsetfillcolor{currentfill}%
\pgfsetlinewidth{0.803000pt}%
\definecolor{currentstroke}{rgb}{0.000000,0.000000,0.000000}%
\pgfsetstrokecolor{currentstroke}%
\pgfsetdash{}{0pt}%
\pgfsys@defobject{currentmarker}{\pgfqpoint{-0.048611in}{0.000000in}}{\pgfqpoint{-0.000000in}{0.000000in}}{%
\pgfpathmoveto{\pgfqpoint{-0.000000in}{0.000000in}}%
\pgfpathlineto{\pgfqpoint{-0.048611in}{0.000000in}}%
\pgfusepath{stroke,fill}%
}%
\begin{pgfscope}%
\pgfsys@transformshift{0.608070in}{1.487172in}%
\pgfsys@useobject{currentmarker}{}%
\end{pgfscope}%
\end{pgfscope}%
\begin{pgfscope}%
\definecolor{textcolor}{rgb}{0.000000,0.000000,0.000000}%
\pgfsetstrokecolor{textcolor}%
\pgfsetfillcolor{textcolor}%
\pgftext[x=0.289968in, y=1.434411in, left, base]{\color{textcolor}\sffamily\fontsize{10.000000}{12.000000}\selectfont 0.4}%
\end{pgfscope}%
\begin{pgfscope}%
\pgfpathrectangle{\pgfqpoint{0.608070in}{0.521603in}}{\pgfqpoint{3.819012in}{2.360279in}}%
\pgfusepath{clip}%
\pgfsetrectcap%
\pgfsetroundjoin%
\pgfsetlinewidth{0.803000pt}%
\definecolor{currentstroke}{rgb}{0.690196,0.690196,0.690196}%
\pgfsetstrokecolor{currentstroke}%
\pgfsetdash{}{0pt}%
\pgfpathmoveto{\pgfqpoint{0.608070in}{1.916314in}}%
\pgfpathlineto{\pgfqpoint{4.427082in}{1.916314in}}%
\pgfusepath{stroke}%
\end{pgfscope}%
\begin{pgfscope}%
\pgfsetbuttcap%
\pgfsetroundjoin%
\definecolor{currentfill}{rgb}{0.000000,0.000000,0.000000}%
\pgfsetfillcolor{currentfill}%
\pgfsetlinewidth{0.803000pt}%
\definecolor{currentstroke}{rgb}{0.000000,0.000000,0.000000}%
\pgfsetstrokecolor{currentstroke}%
\pgfsetdash{}{0pt}%
\pgfsys@defobject{currentmarker}{\pgfqpoint{-0.048611in}{0.000000in}}{\pgfqpoint{-0.000000in}{0.000000in}}{%
\pgfpathmoveto{\pgfqpoint{-0.000000in}{0.000000in}}%
\pgfpathlineto{\pgfqpoint{-0.048611in}{0.000000in}}%
\pgfusepath{stroke,fill}%
}%
\begin{pgfscope}%
\pgfsys@transformshift{0.608070in}{1.916314in}%
\pgfsys@useobject{currentmarker}{}%
\end{pgfscope}%
\end{pgfscope}%
\begin{pgfscope}%
\definecolor{textcolor}{rgb}{0.000000,0.000000,0.000000}%
\pgfsetstrokecolor{textcolor}%
\pgfsetfillcolor{textcolor}%
\pgftext[x=0.289968in, y=1.863552in, left, base]{\color{textcolor}\sffamily\fontsize{10.000000}{12.000000}\selectfont 0.6}%
\end{pgfscope}%
\begin{pgfscope}%
\pgfpathrectangle{\pgfqpoint{0.608070in}{0.521603in}}{\pgfqpoint{3.819012in}{2.360279in}}%
\pgfusepath{clip}%
\pgfsetrectcap%
\pgfsetroundjoin%
\pgfsetlinewidth{0.803000pt}%
\definecolor{currentstroke}{rgb}{0.690196,0.690196,0.690196}%
\pgfsetstrokecolor{currentstroke}%
\pgfsetdash{}{0pt}%
\pgfpathmoveto{\pgfqpoint{0.608070in}{2.345455in}}%
\pgfpathlineto{\pgfqpoint{4.427082in}{2.345455in}}%
\pgfusepath{stroke}%
\end{pgfscope}%
\begin{pgfscope}%
\pgfsetbuttcap%
\pgfsetroundjoin%
\definecolor{currentfill}{rgb}{0.000000,0.000000,0.000000}%
\pgfsetfillcolor{currentfill}%
\pgfsetlinewidth{0.803000pt}%
\definecolor{currentstroke}{rgb}{0.000000,0.000000,0.000000}%
\pgfsetstrokecolor{currentstroke}%
\pgfsetdash{}{0pt}%
\pgfsys@defobject{currentmarker}{\pgfqpoint{-0.048611in}{0.000000in}}{\pgfqpoint{-0.000000in}{0.000000in}}{%
\pgfpathmoveto{\pgfqpoint{-0.000000in}{0.000000in}}%
\pgfpathlineto{\pgfqpoint{-0.048611in}{0.000000in}}%
\pgfusepath{stroke,fill}%
}%
\begin{pgfscope}%
\pgfsys@transformshift{0.608070in}{2.345455in}%
\pgfsys@useobject{currentmarker}{}%
\end{pgfscope}%
\end{pgfscope}%
\begin{pgfscope}%
\definecolor{textcolor}{rgb}{0.000000,0.000000,0.000000}%
\pgfsetstrokecolor{textcolor}%
\pgfsetfillcolor{textcolor}%
\pgftext[x=0.289968in, y=2.292694in, left, base]{\color{textcolor}\sffamily\fontsize{10.000000}{12.000000}\selectfont 0.8}%
\end{pgfscope}%
\begin{pgfscope}%
\pgfpathrectangle{\pgfqpoint{0.608070in}{0.521603in}}{\pgfqpoint{3.819012in}{2.360279in}}%
\pgfusepath{clip}%
\pgfsetrectcap%
\pgfsetroundjoin%
\pgfsetlinewidth{0.803000pt}%
\definecolor{currentstroke}{rgb}{0.690196,0.690196,0.690196}%
\pgfsetstrokecolor{currentstroke}%
\pgfsetdash{}{0pt}%
\pgfpathmoveto{\pgfqpoint{0.608070in}{2.774597in}}%
\pgfpathlineto{\pgfqpoint{4.427082in}{2.774597in}}%
\pgfusepath{stroke}%
\end{pgfscope}%
\begin{pgfscope}%
\pgfsetbuttcap%
\pgfsetroundjoin%
\definecolor{currentfill}{rgb}{0.000000,0.000000,0.000000}%
\pgfsetfillcolor{currentfill}%
\pgfsetlinewidth{0.803000pt}%
\definecolor{currentstroke}{rgb}{0.000000,0.000000,0.000000}%
\pgfsetstrokecolor{currentstroke}%
\pgfsetdash{}{0pt}%
\pgfsys@defobject{currentmarker}{\pgfqpoint{-0.048611in}{0.000000in}}{\pgfqpoint{-0.000000in}{0.000000in}}{%
\pgfpathmoveto{\pgfqpoint{-0.000000in}{0.000000in}}%
\pgfpathlineto{\pgfqpoint{-0.048611in}{0.000000in}}%
\pgfusepath{stroke,fill}%
}%
\begin{pgfscope}%
\pgfsys@transformshift{0.608070in}{2.774597in}%
\pgfsys@useobject{currentmarker}{}%
\end{pgfscope}%
\end{pgfscope}%
\begin{pgfscope}%
\definecolor{textcolor}{rgb}{0.000000,0.000000,0.000000}%
\pgfsetstrokecolor{textcolor}%
\pgfsetfillcolor{textcolor}%
\pgftext[x=0.289968in, y=2.721836in, left, base]{\color{textcolor}\sffamily\fontsize{10.000000}{12.000000}\selectfont 1.0}%
\end{pgfscope}%
\begin{pgfscope}%
\definecolor{textcolor}{rgb}{0.000000,0.000000,0.000000}%
\pgfsetstrokecolor{textcolor}%
\pgfsetfillcolor{textcolor}%
\pgftext[x=0.234413in,y=1.701743in,,bottom,rotate=90.000000]{\color{textcolor}\sffamily\fontsize{10.000000}{12.000000}\selectfont proportion of conformations}%
\end{pgfscope}%
\begin{pgfscope}%
\pgfpathrectangle{\pgfqpoint{0.608070in}{0.521603in}}{\pgfqpoint{3.819012in}{2.360279in}}%
\pgfusepath{clip}%
\pgfsetrectcap%
\pgfsetroundjoin%
\pgfsetlinewidth{1.505625pt}%
\definecolor{currentstroke}{rgb}{0.121569,0.466667,0.705882}%
\pgfsetstrokecolor{currentstroke}%
\pgfsetdash{}{0pt}%
\pgfpathmoveto{\pgfqpoint{0.781661in}{2.774597in}}%
\pgfpathlineto{\pgfqpoint{0.816730in}{2.774597in}}%
\pgfpathlineto{\pgfqpoint{0.851799in}{2.774597in}}%
\pgfpathlineto{\pgfqpoint{0.886868in}{2.774597in}}%
\pgfpathlineto{\pgfqpoint{0.921937in}{2.774597in}}%
\pgfpathlineto{\pgfqpoint{0.957006in}{2.572901in}}%
\pgfpathlineto{\pgfqpoint{0.992075in}{1.967811in}}%
\pgfpathlineto{\pgfqpoint{1.027144in}{1.542961in}}%
\pgfpathlineto{\pgfqpoint{1.062213in}{1.351992in}}%
\pgfpathlineto{\pgfqpoint{1.097282in}{1.287621in}}%
\pgfpathlineto{\pgfqpoint{1.132351in}{1.251144in}}%
\pgfpathlineto{\pgfqpoint{1.167420in}{1.244707in}}%
\pgfpathlineto{\pgfqpoint{1.202489in}{1.233979in}}%
\pgfpathlineto{\pgfqpoint{1.237558in}{1.229687in}}%
\pgfpathlineto{\pgfqpoint{1.272627in}{1.227541in}}%
\pgfpathlineto{\pgfqpoint{1.307696in}{1.206084in}}%
\pgfpathlineto{\pgfqpoint{1.342765in}{1.195356in}}%
\pgfpathlineto{\pgfqpoint{1.377834in}{1.176044in}}%
\pgfpathlineto{\pgfqpoint{1.412903in}{1.167462in}}%
\pgfpathlineto{\pgfqpoint{1.447972in}{1.156733in}}%
\pgfpathlineto{\pgfqpoint{1.483041in}{1.139567in}}%
\pgfpathlineto{\pgfqpoint{1.518110in}{1.126693in}}%
\pgfpathlineto{\pgfqpoint{1.553179in}{1.111673in}}%
\pgfpathlineto{\pgfqpoint{1.588248in}{1.096653in}}%
\pgfpathlineto{\pgfqpoint{1.623317in}{1.085925in}}%
\pgfpathlineto{\pgfqpoint{1.658386in}{1.070905in}}%
\pgfpathlineto{\pgfqpoint{1.693455in}{1.051593in}}%
\pgfpathlineto{\pgfqpoint{1.728524in}{1.036573in}}%
\pgfpathlineto{\pgfqpoint{1.763593in}{1.023699in}}%
\pgfpathlineto{\pgfqpoint{1.798662in}{1.008679in}}%
\pgfpathlineto{\pgfqpoint{1.833731in}{0.993659in}}%
\pgfpathlineto{\pgfqpoint{1.868800in}{0.987222in}}%
\pgfpathlineto{\pgfqpoint{1.903869in}{0.976494in}}%
\pgfpathlineto{\pgfqpoint{1.938938in}{0.955036in}}%
\pgfpathlineto{\pgfqpoint{1.974007in}{0.944308in}}%
\pgfpathlineto{\pgfqpoint{2.009076in}{0.929288in}}%
\pgfpathlineto{\pgfqpoint{2.044145in}{0.912122in}}%
\pgfpathlineto{\pgfqpoint{2.079214in}{0.899248in}}%
\pgfpathlineto{\pgfqpoint{2.114283in}{0.888519in}}%
\pgfpathlineto{\pgfqpoint{2.149352in}{0.886374in}}%
\pgfpathlineto{\pgfqpoint{2.184421in}{0.869208in}}%
\pgfpathlineto{\pgfqpoint{2.219490in}{0.864917in}}%
\pgfpathlineto{\pgfqpoint{2.254559in}{0.856334in}}%
\pgfpathlineto{\pgfqpoint{2.289628in}{0.849897in}}%
\pgfpathlineto{\pgfqpoint{2.324697in}{0.839168in}}%
\pgfpathlineto{\pgfqpoint{2.359766in}{0.834877in}}%
\pgfpathlineto{\pgfqpoint{2.394835in}{0.828440in}}%
\pgfpathlineto{\pgfqpoint{2.429904in}{0.811274in}}%
\pgfpathlineto{\pgfqpoint{2.464973in}{0.809128in}}%
\pgfpathlineto{\pgfqpoint{2.500042in}{0.804837in}}%
\pgfpathlineto{\pgfqpoint{2.535111in}{0.787671in}}%
\pgfpathlineto{\pgfqpoint{2.570179in}{0.781234in}}%
\pgfpathlineto{\pgfqpoint{2.605248in}{0.772651in}}%
\pgfpathlineto{\pgfqpoint{2.640317in}{0.768360in}}%
\pgfpathlineto{\pgfqpoint{2.675386in}{0.759777in}}%
\pgfpathlineto{\pgfqpoint{2.710455in}{0.742611in}}%
\pgfpathlineto{\pgfqpoint{2.745524in}{0.736174in}}%
\pgfpathlineto{\pgfqpoint{2.780593in}{0.734028in}}%
\pgfpathlineto{\pgfqpoint{2.815662in}{0.725446in}}%
\pgfpathlineto{\pgfqpoint{2.850731in}{0.723300in}}%
\pgfpathlineto{\pgfqpoint{2.885800in}{0.716863in}}%
\pgfpathlineto{\pgfqpoint{2.920869in}{0.710426in}}%
\pgfpathlineto{\pgfqpoint{2.955938in}{0.708280in}}%
\pgfpathlineto{\pgfqpoint{2.991007in}{0.708280in}}%
\pgfpathlineto{\pgfqpoint{3.026076in}{0.706134in}}%
\pgfpathlineto{\pgfqpoint{3.061145in}{0.703989in}}%
\pgfpathlineto{\pgfqpoint{3.096214in}{0.697551in}}%
\pgfpathlineto{\pgfqpoint{3.131283in}{0.697551in}}%
\pgfpathlineto{\pgfqpoint{3.166352in}{0.695406in}}%
\pgfpathlineto{\pgfqpoint{3.201421in}{0.695406in}}%
\pgfpathlineto{\pgfqpoint{3.236490in}{0.693260in}}%
\pgfpathlineto{\pgfqpoint{3.271559in}{0.691114in}}%
\pgfpathlineto{\pgfqpoint{3.306628in}{0.691114in}}%
\pgfpathlineto{\pgfqpoint{3.341697in}{0.686823in}}%
\pgfpathlineto{\pgfqpoint{3.376766in}{0.678240in}}%
\pgfpathlineto{\pgfqpoint{3.411835in}{0.673949in}}%
\pgfpathlineto{\pgfqpoint{3.446904in}{0.665366in}}%
\pgfpathlineto{\pgfqpoint{3.481973in}{0.663220in}}%
\pgfpathlineto{\pgfqpoint{3.517042in}{0.661074in}}%
\pgfpathlineto{\pgfqpoint{3.552111in}{0.658929in}}%
\pgfpathlineto{\pgfqpoint{3.587180in}{0.658929in}}%
\pgfpathlineto{\pgfqpoint{3.622249in}{0.656783in}}%
\pgfpathlineto{\pgfqpoint{3.657318in}{0.656783in}}%
\pgfpathlineto{\pgfqpoint{3.692387in}{0.650346in}}%
\pgfpathlineto{\pgfqpoint{3.727456in}{0.643909in}}%
\pgfpathlineto{\pgfqpoint{3.762525in}{0.643909in}}%
\pgfpathlineto{\pgfqpoint{3.797594in}{0.643909in}}%
\pgfpathlineto{\pgfqpoint{3.832663in}{0.641763in}}%
\pgfpathlineto{\pgfqpoint{3.867732in}{0.641763in}}%
\pgfpathlineto{\pgfqpoint{3.902801in}{0.637472in}}%
\pgfpathlineto{\pgfqpoint{3.937870in}{0.633180in}}%
\pgfpathlineto{\pgfqpoint{3.972939in}{0.633180in}}%
\pgfpathlineto{\pgfqpoint{4.008008in}{0.633180in}}%
\pgfpathlineto{\pgfqpoint{4.043077in}{0.633180in}}%
\pgfpathlineto{\pgfqpoint{4.078146in}{0.631034in}}%
\pgfpathlineto{\pgfqpoint{4.113215in}{0.628889in}}%
\pgfpathlineto{\pgfqpoint{4.148284in}{0.628889in}}%
\pgfpathlineto{\pgfqpoint{4.183353in}{0.628889in}}%
\pgfpathlineto{\pgfqpoint{4.218422in}{0.628889in}}%
\pgfpathlineto{\pgfqpoint{4.253491in}{0.628889in}}%
\pgfusepath{stroke}%
\end{pgfscope}%
\begin{pgfscope}%
\pgfpathrectangle{\pgfqpoint{0.608070in}{0.521603in}}{\pgfqpoint{3.819012in}{2.360279in}}%
\pgfusepath{clip}%
\pgfsetrectcap%
\pgfsetroundjoin%
\pgfsetlinewidth{1.505625pt}%
\definecolor{currentstroke}{rgb}{1.000000,0.498039,0.054902}%
\pgfsetstrokecolor{currentstroke}%
\pgfsetdash{}{0pt}%
\pgfpathmoveto{\pgfqpoint{0.781661in}{2.774597in}}%
\pgfpathlineto{\pgfqpoint{0.816730in}{2.774597in}}%
\pgfpathlineto{\pgfqpoint{0.851799in}{2.774597in}}%
\pgfpathlineto{\pgfqpoint{0.886868in}{2.150196in}}%
\pgfpathlineto{\pgfqpoint{0.921937in}{1.375595in}}%
\pgfpathlineto{\pgfqpoint{0.957006in}{1.328390in}}%
\pgfpathlineto{\pgfqpoint{0.992075in}{1.328390in}}%
\pgfpathlineto{\pgfqpoint{1.027144in}{1.326244in}}%
\pgfpathlineto{\pgfqpoint{1.062213in}{1.324098in}}%
\pgfpathlineto{\pgfqpoint{1.097282in}{1.321953in}}%
\pgfpathlineto{\pgfqpoint{1.132351in}{1.311224in}}%
\pgfpathlineto{\pgfqpoint{1.167420in}{1.306933in}}%
\pgfpathlineto{\pgfqpoint{1.202489in}{1.291913in}}%
\pgfpathlineto{\pgfqpoint{1.237558in}{1.285476in}}%
\pgfpathlineto{\pgfqpoint{1.272627in}{1.274747in}}%
\pgfpathlineto{\pgfqpoint{1.307696in}{1.266164in}}%
\pgfpathlineto{\pgfqpoint{1.342765in}{1.238270in}}%
\pgfpathlineto{\pgfqpoint{1.377834in}{1.221104in}}%
\pgfpathlineto{\pgfqpoint{1.412903in}{1.184627in}}%
\pgfpathlineto{\pgfqpoint{1.447972in}{1.169607in}}%
\pgfpathlineto{\pgfqpoint{1.483041in}{1.148150in}}%
\pgfpathlineto{\pgfqpoint{1.518110in}{1.130985in}}%
\pgfpathlineto{\pgfqpoint{1.553179in}{1.111673in}}%
\pgfpathlineto{\pgfqpoint{1.588248in}{1.092362in}}%
\pgfpathlineto{\pgfqpoint{1.623317in}{1.079488in}}%
\pgfpathlineto{\pgfqpoint{1.658386in}{1.064468in}}%
\pgfpathlineto{\pgfqpoint{1.693455in}{1.049448in}}%
\pgfpathlineto{\pgfqpoint{1.728524in}{1.025845in}}%
\pgfpathlineto{\pgfqpoint{1.763593in}{1.000096in}}%
\pgfpathlineto{\pgfqpoint{1.798662in}{0.974348in}}%
\pgfpathlineto{\pgfqpoint{1.833731in}{0.963619in}}%
\pgfpathlineto{\pgfqpoint{1.868800in}{0.952891in}}%
\pgfpathlineto{\pgfqpoint{1.903869in}{0.940016in}}%
\pgfpathlineto{\pgfqpoint{1.938938in}{0.933579in}}%
\pgfpathlineto{\pgfqpoint{1.974007in}{0.924997in}}%
\pgfpathlineto{\pgfqpoint{2.009076in}{0.909977in}}%
\pgfpathlineto{\pgfqpoint{2.044145in}{0.894957in}}%
\pgfpathlineto{\pgfqpoint{2.079214in}{0.886374in}}%
\pgfpathlineto{\pgfqpoint{2.114283in}{0.882082in}}%
\pgfpathlineto{\pgfqpoint{2.149352in}{0.867062in}}%
\pgfpathlineto{\pgfqpoint{2.184421in}{0.860625in}}%
\pgfpathlineto{\pgfqpoint{2.219490in}{0.854188in}}%
\pgfpathlineto{\pgfqpoint{2.254559in}{0.839168in}}%
\pgfpathlineto{\pgfqpoint{2.289628in}{0.830585in}}%
\pgfpathlineto{\pgfqpoint{2.324697in}{0.819857in}}%
\pgfpathlineto{\pgfqpoint{2.359766in}{0.813420in}}%
\pgfpathlineto{\pgfqpoint{2.394835in}{0.798400in}}%
\pgfpathlineto{\pgfqpoint{2.429904in}{0.787671in}}%
\pgfpathlineto{\pgfqpoint{2.464973in}{0.783380in}}%
\pgfpathlineto{\pgfqpoint{2.500042in}{0.776943in}}%
\pgfpathlineto{\pgfqpoint{2.535111in}{0.768360in}}%
\pgfpathlineto{\pgfqpoint{2.570179in}{0.759777in}}%
\pgfpathlineto{\pgfqpoint{2.605248in}{0.751194in}}%
\pgfpathlineto{\pgfqpoint{2.640317in}{0.749048in}}%
\pgfpathlineto{\pgfqpoint{2.675386in}{0.744757in}}%
\pgfpathlineto{\pgfqpoint{2.710455in}{0.744757in}}%
\pgfpathlineto{\pgfqpoint{2.745524in}{0.742611in}}%
\pgfpathlineto{\pgfqpoint{2.780593in}{0.738320in}}%
\pgfpathlineto{\pgfqpoint{2.815662in}{0.734028in}}%
\pgfpathlineto{\pgfqpoint{2.850731in}{0.727591in}}%
\pgfpathlineto{\pgfqpoint{2.885800in}{0.725446in}}%
\pgfpathlineto{\pgfqpoint{2.920869in}{0.710426in}}%
\pgfpathlineto{\pgfqpoint{2.955938in}{0.708280in}}%
\pgfpathlineto{\pgfqpoint{2.991007in}{0.703989in}}%
\pgfpathlineto{\pgfqpoint{3.026076in}{0.697551in}}%
\pgfpathlineto{\pgfqpoint{3.061145in}{0.695406in}}%
\pgfpathlineto{\pgfqpoint{3.096214in}{0.691114in}}%
\pgfpathlineto{\pgfqpoint{3.131283in}{0.686823in}}%
\pgfpathlineto{\pgfqpoint{3.166352in}{0.680386in}}%
\pgfpathlineto{\pgfqpoint{3.201421in}{0.673949in}}%
\pgfpathlineto{\pgfqpoint{3.236490in}{0.671803in}}%
\pgfpathlineto{\pgfqpoint{3.271559in}{0.671803in}}%
\pgfpathlineto{\pgfqpoint{3.306628in}{0.669657in}}%
\pgfpathlineto{\pgfqpoint{3.341697in}{0.669657in}}%
\pgfpathlineto{\pgfqpoint{3.376766in}{0.667512in}}%
\pgfpathlineto{\pgfqpoint{3.411835in}{0.665366in}}%
\pgfpathlineto{\pgfqpoint{3.446904in}{0.663220in}}%
\pgfpathlineto{\pgfqpoint{3.481973in}{0.663220in}}%
\pgfpathlineto{\pgfqpoint{3.517042in}{0.658929in}}%
\pgfpathlineto{\pgfqpoint{3.552111in}{0.652492in}}%
\pgfpathlineto{\pgfqpoint{3.587180in}{0.652492in}}%
\pgfpathlineto{\pgfqpoint{3.622249in}{0.652492in}}%
\pgfpathlineto{\pgfqpoint{3.657318in}{0.650346in}}%
\pgfpathlineto{\pgfqpoint{3.692387in}{0.648200in}}%
\pgfpathlineto{\pgfqpoint{3.727456in}{0.648200in}}%
\pgfpathlineto{\pgfqpoint{3.762525in}{0.643909in}}%
\pgfpathlineto{\pgfqpoint{3.797594in}{0.641763in}}%
\pgfpathlineto{\pgfqpoint{3.832663in}{0.639617in}}%
\pgfpathlineto{\pgfqpoint{3.867732in}{0.635326in}}%
\pgfpathlineto{\pgfqpoint{3.902801in}{0.633180in}}%
\pgfpathlineto{\pgfqpoint{3.937870in}{0.631034in}}%
\pgfpathlineto{\pgfqpoint{3.972939in}{0.631034in}}%
\pgfpathlineto{\pgfqpoint{4.008008in}{0.631034in}}%
\pgfpathlineto{\pgfqpoint{4.043077in}{0.631034in}}%
\pgfpathlineto{\pgfqpoint{4.078146in}{0.628889in}}%
\pgfpathlineto{\pgfqpoint{4.113215in}{0.628889in}}%
\pgfpathlineto{\pgfqpoint{4.148284in}{0.628889in}}%
\pgfpathlineto{\pgfqpoint{4.183353in}{0.628889in}}%
\pgfpathlineto{\pgfqpoint{4.218422in}{0.628889in}}%
\pgfpathlineto{\pgfqpoint{4.253491in}{0.628889in}}%
\pgfusepath{stroke}%
\end{pgfscope}%
\begin{pgfscope}%
\pgfpathrectangle{\pgfqpoint{0.608070in}{0.521603in}}{\pgfqpoint{3.819012in}{2.360279in}}%
\pgfusepath{clip}%
\pgfsetrectcap%
\pgfsetroundjoin%
\pgfsetlinewidth{1.505625pt}%
\definecolor{currentstroke}{rgb}{0.172549,0.627451,0.172549}%
\pgfsetstrokecolor{currentstroke}%
\pgfsetdash{}{0pt}%
\pgfpathmoveto{\pgfqpoint{0.781661in}{2.774597in}}%
\pgfpathlineto{\pgfqpoint{0.816730in}{2.774597in}}%
\pgfpathlineto{\pgfqpoint{0.851799in}{1.392761in}}%
\pgfpathlineto{\pgfqpoint{0.886868in}{1.379887in}}%
\pgfpathlineto{\pgfqpoint{0.921937in}{1.379887in}}%
\pgfpathlineto{\pgfqpoint{0.957006in}{1.375595in}}%
\pgfpathlineto{\pgfqpoint{0.992075in}{1.375595in}}%
\pgfpathlineto{\pgfqpoint{1.027144in}{1.375595in}}%
\pgfpathlineto{\pgfqpoint{1.062213in}{1.358430in}}%
\pgfpathlineto{\pgfqpoint{1.097282in}{1.351992in}}%
\pgfpathlineto{\pgfqpoint{1.132351in}{1.336973in}}%
\pgfpathlineto{\pgfqpoint{1.167420in}{1.304787in}}%
\pgfpathlineto{\pgfqpoint{1.202489in}{1.276893in}}%
\pgfpathlineto{\pgfqpoint{1.237558in}{1.253290in}}%
\pgfpathlineto{\pgfqpoint{1.272627in}{1.218959in}}%
\pgfpathlineto{\pgfqpoint{1.307696in}{1.197501in}}%
\pgfpathlineto{\pgfqpoint{1.342765in}{1.173899in}}%
\pgfpathlineto{\pgfqpoint{1.377834in}{1.152442in}}%
\pgfpathlineto{\pgfqpoint{1.412903in}{1.130985in}}%
\pgfpathlineto{\pgfqpoint{1.447972in}{1.109527in}}%
\pgfpathlineto{\pgfqpoint{1.483041in}{1.079488in}}%
\pgfpathlineto{\pgfqpoint{1.518110in}{1.062322in}}%
\pgfpathlineto{\pgfqpoint{1.553179in}{1.038719in}}%
\pgfpathlineto{\pgfqpoint{1.588248in}{1.015116in}}%
\pgfpathlineto{\pgfqpoint{1.623317in}{0.989368in}}%
\pgfpathlineto{\pgfqpoint{1.658386in}{0.974348in}}%
\pgfpathlineto{\pgfqpoint{1.693455in}{0.950745in}}%
\pgfpathlineto{\pgfqpoint{1.728524in}{0.944308in}}%
\pgfpathlineto{\pgfqpoint{1.763593in}{0.916414in}}%
\pgfpathlineto{\pgfqpoint{1.798662in}{0.901394in}}%
\pgfpathlineto{\pgfqpoint{1.833731in}{0.892811in}}%
\pgfpathlineto{\pgfqpoint{1.868800in}{0.877791in}}%
\pgfpathlineto{\pgfqpoint{1.903869in}{0.875645in}}%
\pgfpathlineto{\pgfqpoint{1.938938in}{0.858480in}}%
\pgfpathlineto{\pgfqpoint{1.974007in}{0.845605in}}%
\pgfpathlineto{\pgfqpoint{2.009076in}{0.839168in}}%
\pgfpathlineto{\pgfqpoint{2.044145in}{0.834877in}}%
\pgfpathlineto{\pgfqpoint{2.079214in}{0.822003in}}%
\pgfpathlineto{\pgfqpoint{2.114283in}{0.815565in}}%
\pgfpathlineto{\pgfqpoint{2.149352in}{0.804837in}}%
\pgfpathlineto{\pgfqpoint{2.184421in}{0.798400in}}%
\pgfpathlineto{\pgfqpoint{2.219490in}{0.794108in}}%
\pgfpathlineto{\pgfqpoint{2.254559in}{0.791963in}}%
\pgfpathlineto{\pgfqpoint{2.289628in}{0.789817in}}%
\pgfpathlineto{\pgfqpoint{2.324697in}{0.785525in}}%
\pgfpathlineto{\pgfqpoint{2.359766in}{0.774797in}}%
\pgfpathlineto{\pgfqpoint{2.394835in}{0.768360in}}%
\pgfpathlineto{\pgfqpoint{2.429904in}{0.761923in}}%
\pgfpathlineto{\pgfqpoint{2.464973in}{0.753340in}}%
\pgfpathlineto{\pgfqpoint{2.500042in}{0.744757in}}%
\pgfpathlineto{\pgfqpoint{2.535111in}{0.736174in}}%
\pgfpathlineto{\pgfqpoint{2.570179in}{0.729737in}}%
\pgfpathlineto{\pgfqpoint{2.605248in}{0.723300in}}%
\pgfpathlineto{\pgfqpoint{2.640317in}{0.716863in}}%
\pgfpathlineto{\pgfqpoint{2.675386in}{0.710426in}}%
\pgfpathlineto{\pgfqpoint{2.710455in}{0.708280in}}%
\pgfpathlineto{\pgfqpoint{2.745524in}{0.701843in}}%
\pgfpathlineto{\pgfqpoint{2.780593in}{0.697551in}}%
\pgfpathlineto{\pgfqpoint{2.815662in}{0.691114in}}%
\pgfpathlineto{\pgfqpoint{2.850731in}{0.686823in}}%
\pgfpathlineto{\pgfqpoint{2.885800in}{0.682531in}}%
\pgfpathlineto{\pgfqpoint{2.920869in}{0.673949in}}%
\pgfpathlineto{\pgfqpoint{2.955938in}{0.673949in}}%
\pgfpathlineto{\pgfqpoint{2.991007in}{0.673949in}}%
\pgfpathlineto{\pgfqpoint{3.026076in}{0.671803in}}%
\pgfpathlineto{\pgfqpoint{3.061145in}{0.669657in}}%
\pgfpathlineto{\pgfqpoint{3.096214in}{0.665366in}}%
\pgfpathlineto{\pgfqpoint{3.131283in}{0.663220in}}%
\pgfpathlineto{\pgfqpoint{3.166352in}{0.658929in}}%
\pgfpathlineto{\pgfqpoint{3.201421in}{0.658929in}}%
\pgfpathlineto{\pgfqpoint{3.236490in}{0.654637in}}%
\pgfpathlineto{\pgfqpoint{3.271559in}{0.650346in}}%
\pgfpathlineto{\pgfqpoint{3.306628in}{0.648200in}}%
\pgfpathlineto{\pgfqpoint{3.341697in}{0.643909in}}%
\pgfpathlineto{\pgfqpoint{3.376766in}{0.639617in}}%
\pgfpathlineto{\pgfqpoint{3.411835in}{0.639617in}}%
\pgfpathlineto{\pgfqpoint{3.446904in}{0.639617in}}%
\pgfpathlineto{\pgfqpoint{3.481973in}{0.639617in}}%
\pgfpathlineto{\pgfqpoint{3.517042in}{0.631034in}}%
\pgfpathlineto{\pgfqpoint{3.552111in}{0.631034in}}%
\pgfpathlineto{\pgfqpoint{3.587180in}{0.628889in}}%
\pgfpathlineto{\pgfqpoint{3.622249in}{0.628889in}}%
\pgfpathlineto{\pgfqpoint{3.657318in}{0.628889in}}%
\pgfpathlineto{\pgfqpoint{3.692387in}{0.628889in}}%
\pgfpathlineto{\pgfqpoint{3.727456in}{0.628889in}}%
\pgfpathlineto{\pgfqpoint{3.762525in}{0.628889in}}%
\pgfpathlineto{\pgfqpoint{3.797594in}{0.628889in}}%
\pgfpathlineto{\pgfqpoint{3.832663in}{0.628889in}}%
\pgfpathlineto{\pgfqpoint{3.867732in}{0.628889in}}%
\pgfpathlineto{\pgfqpoint{3.902801in}{0.628889in}}%
\pgfpathlineto{\pgfqpoint{3.937870in}{0.628889in}}%
\pgfpathlineto{\pgfqpoint{3.972939in}{0.628889in}}%
\pgfpathlineto{\pgfqpoint{4.008008in}{0.628889in}}%
\pgfpathlineto{\pgfqpoint{4.043077in}{0.628889in}}%
\pgfpathlineto{\pgfqpoint{4.078146in}{0.628889in}}%
\pgfpathlineto{\pgfqpoint{4.113215in}{0.628889in}}%
\pgfpathlineto{\pgfqpoint{4.148284in}{0.628889in}}%
\pgfpathlineto{\pgfqpoint{4.183353in}{0.628889in}}%
\pgfpathlineto{\pgfqpoint{4.218422in}{0.628889in}}%
\pgfpathlineto{\pgfqpoint{4.253491in}{0.628889in}}%
\pgfusepath{stroke}%
\end{pgfscope}%
\begin{pgfscope}%
\pgfpathrectangle{\pgfqpoint{0.608070in}{0.521603in}}{\pgfqpoint{3.819012in}{2.360279in}}%
\pgfusepath{clip}%
\pgfsetrectcap%
\pgfsetroundjoin%
\pgfsetlinewidth{1.505625pt}%
\definecolor{currentstroke}{rgb}{0.839216,0.152941,0.156863}%
\pgfsetstrokecolor{currentstroke}%
\pgfsetdash{}{0pt}%
\pgfpathmoveto{\pgfqpoint{0.781661in}{2.774597in}}%
\pgfpathlineto{\pgfqpoint{0.816730in}{1.431384in}}%
\pgfpathlineto{\pgfqpoint{0.851799in}{1.427092in}}%
\pgfpathlineto{\pgfqpoint{0.886868in}{1.427092in}}%
\pgfpathlineto{\pgfqpoint{0.921937in}{1.427092in}}%
\pgfpathlineto{\pgfqpoint{0.957006in}{1.420655in}}%
\pgfpathlineto{\pgfqpoint{0.992075in}{1.407781in}}%
\pgfpathlineto{\pgfqpoint{1.027144in}{1.399198in}}%
\pgfpathlineto{\pgfqpoint{1.062213in}{1.382032in}}%
\pgfpathlineto{\pgfqpoint{1.097282in}{1.354138in}}%
\pgfpathlineto{\pgfqpoint{1.132351in}{1.326244in}}%
\pgfpathlineto{\pgfqpoint{1.167420in}{1.287621in}}%
\pgfpathlineto{\pgfqpoint{1.202489in}{1.257581in}}%
\pgfpathlineto{\pgfqpoint{1.237558in}{1.216813in}}%
\pgfpathlineto{\pgfqpoint{1.272627in}{1.193210in}}%
\pgfpathlineto{\pgfqpoint{1.307696in}{1.152442in}}%
\pgfpathlineto{\pgfqpoint{1.342765in}{1.126693in}}%
\pgfpathlineto{\pgfqpoint{1.377834in}{1.100945in}}%
\pgfpathlineto{\pgfqpoint{1.412903in}{1.075196in}}%
\pgfpathlineto{\pgfqpoint{1.447972in}{1.045156in}}%
\pgfpathlineto{\pgfqpoint{1.483041in}{1.023699in}}%
\pgfpathlineto{\pgfqpoint{1.518110in}{1.000096in}}%
\pgfpathlineto{\pgfqpoint{1.553179in}{0.974348in}}%
\pgfpathlineto{\pgfqpoint{1.588248in}{0.955036in}}%
\pgfpathlineto{\pgfqpoint{1.623317in}{0.933579in}}%
\pgfpathlineto{\pgfqpoint{1.658386in}{0.920705in}}%
\pgfpathlineto{\pgfqpoint{1.693455in}{0.899248in}}%
\pgfpathlineto{\pgfqpoint{1.728524in}{0.884228in}}%
\pgfpathlineto{\pgfqpoint{1.763593in}{0.873500in}}%
\pgfpathlineto{\pgfqpoint{1.798662in}{0.856334in}}%
\pgfpathlineto{\pgfqpoint{1.833731in}{0.852042in}}%
\pgfpathlineto{\pgfqpoint{1.868800in}{0.841314in}}%
\pgfpathlineto{\pgfqpoint{1.903869in}{0.832731in}}%
\pgfpathlineto{\pgfqpoint{1.938938in}{0.817711in}}%
\pgfpathlineto{\pgfqpoint{1.974007in}{0.804837in}}%
\pgfpathlineto{\pgfqpoint{2.009076in}{0.794108in}}%
\pgfpathlineto{\pgfqpoint{2.044145in}{0.789817in}}%
\pgfpathlineto{\pgfqpoint{2.079214in}{0.779088in}}%
\pgfpathlineto{\pgfqpoint{2.114283in}{0.770506in}}%
\pgfpathlineto{\pgfqpoint{2.149352in}{0.759777in}}%
\pgfpathlineto{\pgfqpoint{2.184421in}{0.746903in}}%
\pgfpathlineto{\pgfqpoint{2.219490in}{0.744757in}}%
\pgfpathlineto{\pgfqpoint{2.254559in}{0.734028in}}%
\pgfpathlineto{\pgfqpoint{2.289628in}{0.727591in}}%
\pgfpathlineto{\pgfqpoint{2.324697in}{0.725446in}}%
\pgfpathlineto{\pgfqpoint{2.359766in}{0.712571in}}%
\pgfpathlineto{\pgfqpoint{2.394835in}{0.701843in}}%
\pgfpathlineto{\pgfqpoint{2.429904in}{0.697551in}}%
\pgfpathlineto{\pgfqpoint{2.464973in}{0.695406in}}%
\pgfpathlineto{\pgfqpoint{2.500042in}{0.693260in}}%
\pgfpathlineto{\pgfqpoint{2.535111in}{0.688969in}}%
\pgfpathlineto{\pgfqpoint{2.570179in}{0.682531in}}%
\pgfpathlineto{\pgfqpoint{2.605248in}{0.682531in}}%
\pgfpathlineto{\pgfqpoint{2.640317in}{0.680386in}}%
\pgfpathlineto{\pgfqpoint{2.675386in}{0.669657in}}%
\pgfpathlineto{\pgfqpoint{2.710455in}{0.665366in}}%
\pgfpathlineto{\pgfqpoint{2.745524in}{0.663220in}}%
\pgfpathlineto{\pgfqpoint{2.780593in}{0.661074in}}%
\pgfpathlineto{\pgfqpoint{2.815662in}{0.656783in}}%
\pgfpathlineto{\pgfqpoint{2.850731in}{0.656783in}}%
\pgfpathlineto{\pgfqpoint{2.885800in}{0.656783in}}%
\pgfpathlineto{\pgfqpoint{2.920869in}{0.656783in}}%
\pgfpathlineto{\pgfqpoint{2.955938in}{0.656783in}}%
\pgfpathlineto{\pgfqpoint{2.991007in}{0.656783in}}%
\pgfpathlineto{\pgfqpoint{3.026076in}{0.652492in}}%
\pgfpathlineto{\pgfqpoint{3.061145in}{0.652492in}}%
\pgfpathlineto{\pgfqpoint{3.096214in}{0.650346in}}%
\pgfpathlineto{\pgfqpoint{3.131283in}{0.648200in}}%
\pgfpathlineto{\pgfqpoint{3.166352in}{0.646054in}}%
\pgfpathlineto{\pgfqpoint{3.201421in}{0.646054in}}%
\pgfpathlineto{\pgfqpoint{3.236490in}{0.641763in}}%
\pgfpathlineto{\pgfqpoint{3.271559in}{0.641763in}}%
\pgfpathlineto{\pgfqpoint{3.306628in}{0.637472in}}%
\pgfpathlineto{\pgfqpoint{3.341697in}{0.633180in}}%
\pgfpathlineto{\pgfqpoint{3.376766in}{0.633180in}}%
\pgfpathlineto{\pgfqpoint{3.411835in}{0.633180in}}%
\pgfpathlineto{\pgfqpoint{3.446904in}{0.633180in}}%
\pgfpathlineto{\pgfqpoint{3.481973in}{0.631034in}}%
\pgfpathlineto{\pgfqpoint{3.517042in}{0.631034in}}%
\pgfpathlineto{\pgfqpoint{3.552111in}{0.628889in}}%
\pgfpathlineto{\pgfqpoint{3.587180in}{0.628889in}}%
\pgfpathlineto{\pgfqpoint{3.622249in}{0.628889in}}%
\pgfpathlineto{\pgfqpoint{3.657318in}{0.628889in}}%
\pgfpathlineto{\pgfqpoint{3.692387in}{0.628889in}}%
\pgfpathlineto{\pgfqpoint{3.727456in}{0.628889in}}%
\pgfpathlineto{\pgfqpoint{3.762525in}{0.628889in}}%
\pgfpathlineto{\pgfqpoint{3.797594in}{0.628889in}}%
\pgfpathlineto{\pgfqpoint{3.832663in}{0.628889in}}%
\pgfpathlineto{\pgfqpoint{3.867732in}{0.628889in}}%
\pgfpathlineto{\pgfqpoint{3.902801in}{0.628889in}}%
\pgfpathlineto{\pgfqpoint{3.937870in}{0.628889in}}%
\pgfpathlineto{\pgfqpoint{3.972939in}{0.628889in}}%
\pgfpathlineto{\pgfqpoint{4.008008in}{0.628889in}}%
\pgfpathlineto{\pgfqpoint{4.043077in}{0.628889in}}%
\pgfpathlineto{\pgfqpoint{4.078146in}{0.628889in}}%
\pgfpathlineto{\pgfqpoint{4.113215in}{0.628889in}}%
\pgfpathlineto{\pgfqpoint{4.148284in}{0.628889in}}%
\pgfpathlineto{\pgfqpoint{4.183353in}{0.628889in}}%
\pgfpathlineto{\pgfqpoint{4.218422in}{0.628889in}}%
\pgfpathlineto{\pgfqpoint{4.253491in}{0.628889in}}%
\pgfusepath{stroke}%
\end{pgfscope}%
\begin{pgfscope}%
\pgfsetrectcap%
\pgfsetmiterjoin%
\pgfsetlinewidth{0.803000pt}%
\definecolor{currentstroke}{rgb}{0.000000,0.000000,0.000000}%
\pgfsetstrokecolor{currentstroke}%
\pgfsetdash{}{0pt}%
\pgfpathmoveto{\pgfqpoint{0.608070in}{0.521603in}}%
\pgfpathlineto{\pgfqpoint{0.608070in}{2.881883in}}%
\pgfusepath{stroke}%
\end{pgfscope}%
\begin{pgfscope}%
\pgfsetrectcap%
\pgfsetmiterjoin%
\pgfsetlinewidth{0.803000pt}%
\definecolor{currentstroke}{rgb}{0.000000,0.000000,0.000000}%
\pgfsetstrokecolor{currentstroke}%
\pgfsetdash{}{0pt}%
\pgfpathmoveto{\pgfqpoint{4.427082in}{0.521603in}}%
\pgfpathlineto{\pgfqpoint{4.427082in}{2.881883in}}%
\pgfusepath{stroke}%
\end{pgfscope}%
\begin{pgfscope}%
\pgfsetrectcap%
\pgfsetmiterjoin%
\pgfsetlinewidth{0.803000pt}%
\definecolor{currentstroke}{rgb}{0.000000,0.000000,0.000000}%
\pgfsetstrokecolor{currentstroke}%
\pgfsetdash{}{0pt}%
\pgfpathmoveto{\pgfqpoint{0.608070in}{0.521603in}}%
\pgfpathlineto{\pgfqpoint{4.427082in}{0.521603in}}%
\pgfusepath{stroke}%
\end{pgfscope}%
\begin{pgfscope}%
\pgfsetrectcap%
\pgfsetmiterjoin%
\pgfsetlinewidth{0.803000pt}%
\definecolor{currentstroke}{rgb}{0.000000,0.000000,0.000000}%
\pgfsetstrokecolor{currentstroke}%
\pgfsetdash{}{0pt}%
\pgfpathmoveto{\pgfqpoint{0.608070in}{2.881883in}}%
\pgfpathlineto{\pgfqpoint{4.427082in}{2.881883in}}%
\pgfusepath{stroke}%
\end{pgfscope}%
\begin{pgfscope}%
\pgfsetbuttcap%
\pgfsetmiterjoin%
\definecolor{currentfill}{rgb}{1.000000,1.000000,1.000000}%
\pgfsetfillcolor{currentfill}%
\pgfsetfillopacity{0.800000}%
\pgfsetlinewidth{1.003750pt}%
\definecolor{currentstroke}{rgb}{0.800000,0.800000,0.800000}%
\pgfsetstrokecolor{currentstroke}%
\pgfsetstrokeopacity{0.800000}%
\pgfsetdash{}{0pt}%
\pgfpathmoveto{\pgfqpoint{3.531954in}{1.955343in}}%
\pgfpathlineto{\pgfqpoint{4.329860in}{1.955343in}}%
\pgfpathquadraticcurveto{\pgfqpoint{4.357638in}{1.955343in}}{\pgfqpoint{4.357638in}{1.983120in}}%
\pgfpathlineto{\pgfqpoint{4.357638in}{2.784660in}}%
\pgfpathquadraticcurveto{\pgfqpoint{4.357638in}{2.812438in}}{\pgfqpoint{4.329860in}{2.812438in}}%
\pgfpathlineto{\pgfqpoint{3.531954in}{2.812438in}}%
\pgfpathquadraticcurveto{\pgfqpoint{3.504176in}{2.812438in}}{\pgfqpoint{3.504176in}{2.784660in}}%
\pgfpathlineto{\pgfqpoint{3.504176in}{1.983120in}}%
\pgfpathquadraticcurveto{\pgfqpoint{3.504176in}{1.955343in}}{\pgfqpoint{3.531954in}{1.955343in}}%
\pgfpathclose%
\pgfusepath{stroke,fill}%
\end{pgfscope}%
\begin{pgfscope}%
\pgfsetrectcap%
\pgfsetroundjoin%
\pgfsetlinewidth{1.505625pt}%
\definecolor{currentstroke}{rgb}{0.121569,0.466667,0.705882}%
\pgfsetstrokecolor{currentstroke}%
\pgfsetdash{}{0pt}%
\pgfpathmoveto{\pgfqpoint{3.559732in}{2.699971in}}%
\pgfpathlineto{\pgfqpoint{3.837510in}{2.699971in}}%
\pgfusepath{stroke}%
\end{pgfscope}%
\begin{pgfscope}%
\definecolor{textcolor}{rgb}{0.000000,0.000000,0.000000}%
\pgfsetstrokecolor{textcolor}%
\pgfsetfillcolor{textcolor}%
\pgftext[x=3.948621in,y=2.651360in,left,base]{\color{textcolor}\sffamily\fontsize{10.000000}{12.000000}\selectfont 250}%
\end{pgfscope}%
\begin{pgfscope}%
\pgfsetrectcap%
\pgfsetroundjoin%
\pgfsetlinewidth{1.505625pt}%
\definecolor{currentstroke}{rgb}{1.000000,0.498039,0.054902}%
\pgfsetstrokecolor{currentstroke}%
\pgfsetdash{}{0pt}%
\pgfpathmoveto{\pgfqpoint{3.559732in}{2.496113in}}%
\pgfpathlineto{\pgfqpoint{3.837510in}{2.496113in}}%
\pgfusepath{stroke}%
\end{pgfscope}%
\begin{pgfscope}%
\definecolor{textcolor}{rgb}{0.000000,0.000000,0.000000}%
\pgfsetstrokecolor{textcolor}%
\pgfsetfillcolor{textcolor}%
\pgftext[x=3.948621in,y=2.447502in,left,base]{\color{textcolor}\sffamily\fontsize{10.000000}{12.000000}\selectfont 500}%
\end{pgfscope}%
\begin{pgfscope}%
\pgfsetrectcap%
\pgfsetroundjoin%
\pgfsetlinewidth{1.505625pt}%
\definecolor{currentstroke}{rgb}{0.172549,0.627451,0.172549}%
\pgfsetstrokecolor{currentstroke}%
\pgfsetdash{}{0pt}%
\pgfpathmoveto{\pgfqpoint{3.559732in}{2.292256in}}%
\pgfpathlineto{\pgfqpoint{3.837510in}{2.292256in}}%
\pgfusepath{stroke}%
\end{pgfscope}%
\begin{pgfscope}%
\definecolor{textcolor}{rgb}{0.000000,0.000000,0.000000}%
\pgfsetstrokecolor{textcolor}%
\pgfsetfillcolor{textcolor}%
\pgftext[x=3.948621in,y=2.243645in,left,base]{\color{textcolor}\sffamily\fontsize{10.000000}{12.000000}\selectfont 1000}%
\end{pgfscope}%
\begin{pgfscope}%
\pgfsetrectcap%
\pgfsetroundjoin%
\pgfsetlinewidth{1.505625pt}%
\definecolor{currentstroke}{rgb}{0.839216,0.152941,0.156863}%
\pgfsetstrokecolor{currentstroke}%
\pgfsetdash{}{0pt}%
\pgfpathmoveto{\pgfqpoint{3.559732in}{2.088399in}}%
\pgfpathlineto{\pgfqpoint{3.837510in}{2.088399in}}%
\pgfusepath{stroke}%
\end{pgfscope}%
\begin{pgfscope}%
\definecolor{textcolor}{rgb}{0.000000,0.000000,0.000000}%
\pgfsetstrokecolor{textcolor}%
\pgfsetfillcolor{textcolor}%
\pgftext[x=3.948621in,y=2.039788in,left,base]{\color{textcolor}\sffamily\fontsize{10.000000}{12.000000}\selectfont 2000}%
\end{pgfscope}%
\end{pgfpicture}%
\makeatother%
\endgroup%

	\caption{Доля конформаций, намагниченность которых в точке $\beta = 1$ больше чем заданное значение. Цветами отмечены конформации разной длины, число конформаций каждой длины - 1000.}
	\label{fig:fraction_magnetization}
\end{figure}

При увеличении длины конформаций средняя намагниченность, и максимальная достигаемая намагниченность уменьшаются. Что подтверждает предположение о том, что при $L\to \infty$ конформации не будут намагничиваться.


\subsection{Магнитная восприимчивость}
Выше мы использовали магнитную восприимчивость для определения точки магнитного перехода, в глобулярных конформациях. В конформациях типа клубок, магнитная восприимчивость не должна иметь пиков, так как в них отсутствует магнитный переход, и в целом магнитная восприимчивость, как и другие свойства этих конформаций, должна быть схожа с одномерной моделью Изинга.

Действительно,у большинства конформаций полученных при $U=0.1$ отсутствуют пики, и график магнитной восприимчивости имеет такой же вид как и у одномерной модели изинига. Пример графиков представлен на рис. \ref{fig:MS_1D_comparison}. На этом же графике представлен пример масштабирования магнитной восприимчивости конформации, для сравнения с одномерной цепочкой.

\begin{figure}[ht]
	\centering
	%% Creator: Matplotlib, PGF backend
%%
%% To include the figure in your LaTeX document, write
%%   \input{<filename>.pgf}
%%
%% Make sure the required packages are loaded in your preamble
%%   \usepackage{pgf}
%%
%% Also ensure that all the required font packages are loaded; for instance,
%% the lmodern package is sometimes necessary when using math font.
%%   \usepackage{lmodern}
%%
%% Figures using additional raster images can only be included by \input if
%% they are in the same directory as the main LaTeX file. For loading figures
%% from other directories you can use the `import` package
%%   \usepackage{import}
%%
%% and then include the figures with
%%   \import{<path to file>}{<filename>.pgf}
%%
%% Matplotlib used the following preamble
%%   
%%   \usepackage{fontspec}
%%   \setmainfont{DejaVuSerif.ttf}[Path=\detokenize{/home/roman/anaconda3/envs/ising/lib/python3.8/site-packages/matplotlib/mpl-data/fonts/ttf/}]
%%   \setsansfont{DejaVuSans.ttf}[Path=\detokenize{/home/roman/anaconda3/envs/ising/lib/python3.8/site-packages/matplotlib/mpl-data/fonts/ttf/}]
%%   \setmonofont{DejaVuSansMono.ttf}[Path=\detokenize{/home/roman/anaconda3/envs/ising/lib/python3.8/site-packages/matplotlib/mpl-data/fonts/ttf/}]
%%   \makeatletter\@ifpackageloaded{underscore}{}{\usepackage[strings]{underscore}}\makeatother
%%
\begingroup%
\makeatletter%
\begin{pgfpicture}%
\pgfpathrectangle{\pgfpointorigin}{\pgfqpoint{5.217998in}{1.930900in}}%
\pgfusepath{use as bounding box, clip}%
\begin{pgfscope}%
\pgfsetbuttcap%
\pgfsetmiterjoin%
\definecolor{currentfill}{rgb}{1.000000,1.000000,1.000000}%
\pgfsetfillcolor{currentfill}%
\pgfsetlinewidth{0.000000pt}%
\definecolor{currentstroke}{rgb}{1.000000,1.000000,1.000000}%
\pgfsetstrokecolor{currentstroke}%
\pgfsetdash{}{0pt}%
\pgfpathmoveto{\pgfqpoint{0.000000in}{0.000000in}}%
\pgfpathlineto{\pgfqpoint{5.217998in}{0.000000in}}%
\pgfpathlineto{\pgfqpoint{5.217998in}{1.930900in}}%
\pgfpathlineto{\pgfqpoint{0.000000in}{1.930900in}}%
\pgfpathlineto{\pgfqpoint{0.000000in}{0.000000in}}%
\pgfpathclose%
\pgfusepath{fill}%
\end{pgfscope}%
\begin{pgfscope}%
\pgfsetbuttcap%
\pgfsetmiterjoin%
\definecolor{currentfill}{rgb}{1.000000,1.000000,1.000000}%
\pgfsetfillcolor{currentfill}%
\pgfsetlinewidth{0.000000pt}%
\definecolor{currentstroke}{rgb}{0.000000,0.000000,0.000000}%
\pgfsetstrokecolor{currentstroke}%
\pgfsetstrokeopacity{0.000000}%
\pgfsetdash{}{0pt}%
\pgfpathmoveto{\pgfqpoint{0.678396in}{0.467838in}}%
\pgfpathlineto{\pgfqpoint{2.696397in}{0.467838in}}%
\pgfpathlineto{\pgfqpoint{2.696397in}{1.830900in}}%
\pgfpathlineto{\pgfqpoint{0.678396in}{1.830900in}}%
\pgfpathlineto{\pgfqpoint{0.678396in}{0.467838in}}%
\pgfpathclose%
\pgfusepath{fill}%
\end{pgfscope}%
\begin{pgfscope}%
\pgfsetbuttcap%
\pgfsetroundjoin%
\definecolor{currentfill}{rgb}{0.000000,0.000000,0.000000}%
\pgfsetfillcolor{currentfill}%
\pgfsetlinewidth{0.803000pt}%
\definecolor{currentstroke}{rgb}{0.000000,0.000000,0.000000}%
\pgfsetstrokecolor{currentstroke}%
\pgfsetdash{}{0pt}%
\pgfsys@defobject{currentmarker}{\pgfqpoint{0.000000in}{-0.048611in}}{\pgfqpoint{0.000000in}{0.000000in}}{%
\pgfpathmoveto{\pgfqpoint{0.000000in}{0.000000in}}%
\pgfpathlineto{\pgfqpoint{0.000000in}{-0.048611in}}%
\pgfusepath{stroke,fill}%
}%
\begin{pgfscope}%
\pgfsys@transformshift{0.973962in}{0.467838in}%
\pgfsys@useobject{currentmarker}{}%
\end{pgfscope}%
\end{pgfscope}%
\begin{pgfscope}%
\definecolor{textcolor}{rgb}{0.000000,0.000000,0.000000}%
\pgfsetstrokecolor{textcolor}%
\pgfsetfillcolor{textcolor}%
\pgftext[x=0.973962in,y=0.370616in,,top]{\color{textcolor}\sffamily\fontsize{8.000000}{9.600000}\selectfont 0.2}%
\end{pgfscope}%
\begin{pgfscope}%
\pgfsetbuttcap%
\pgfsetroundjoin%
\definecolor{currentfill}{rgb}{0.000000,0.000000,0.000000}%
\pgfsetfillcolor{currentfill}%
\pgfsetlinewidth{0.803000pt}%
\definecolor{currentstroke}{rgb}{0.000000,0.000000,0.000000}%
\pgfsetstrokecolor{currentstroke}%
\pgfsetdash{}{0pt}%
\pgfsys@defobject{currentmarker}{\pgfqpoint{0.000000in}{-0.048611in}}{\pgfqpoint{0.000000in}{0.000000in}}{%
\pgfpathmoveto{\pgfqpoint{0.000000in}{0.000000in}}%
\pgfpathlineto{\pgfqpoint{0.000000in}{-0.048611in}}%
\pgfusepath{stroke,fill}%
}%
\begin{pgfscope}%
\pgfsys@transformshift{1.381639in}{0.467838in}%
\pgfsys@useobject{currentmarker}{}%
\end{pgfscope}%
\end{pgfscope}%
\begin{pgfscope}%
\definecolor{textcolor}{rgb}{0.000000,0.000000,0.000000}%
\pgfsetstrokecolor{textcolor}%
\pgfsetfillcolor{textcolor}%
\pgftext[x=1.381639in,y=0.370616in,,top]{\color{textcolor}\sffamily\fontsize{8.000000}{9.600000}\selectfont 0.4}%
\end{pgfscope}%
\begin{pgfscope}%
\pgfsetbuttcap%
\pgfsetroundjoin%
\definecolor{currentfill}{rgb}{0.000000,0.000000,0.000000}%
\pgfsetfillcolor{currentfill}%
\pgfsetlinewidth{0.803000pt}%
\definecolor{currentstroke}{rgb}{0.000000,0.000000,0.000000}%
\pgfsetstrokecolor{currentstroke}%
\pgfsetdash{}{0pt}%
\pgfsys@defobject{currentmarker}{\pgfqpoint{0.000000in}{-0.048611in}}{\pgfqpoint{0.000000in}{0.000000in}}{%
\pgfpathmoveto{\pgfqpoint{0.000000in}{0.000000in}}%
\pgfpathlineto{\pgfqpoint{0.000000in}{-0.048611in}}%
\pgfusepath{stroke,fill}%
}%
\begin{pgfscope}%
\pgfsys@transformshift{1.789316in}{0.467838in}%
\pgfsys@useobject{currentmarker}{}%
\end{pgfscope}%
\end{pgfscope}%
\begin{pgfscope}%
\definecolor{textcolor}{rgb}{0.000000,0.000000,0.000000}%
\pgfsetstrokecolor{textcolor}%
\pgfsetfillcolor{textcolor}%
\pgftext[x=1.789316in,y=0.370616in,,top]{\color{textcolor}\sffamily\fontsize{8.000000}{9.600000}\selectfont 0.6}%
\end{pgfscope}%
\begin{pgfscope}%
\pgfsetbuttcap%
\pgfsetroundjoin%
\definecolor{currentfill}{rgb}{0.000000,0.000000,0.000000}%
\pgfsetfillcolor{currentfill}%
\pgfsetlinewidth{0.803000pt}%
\definecolor{currentstroke}{rgb}{0.000000,0.000000,0.000000}%
\pgfsetstrokecolor{currentstroke}%
\pgfsetdash{}{0pt}%
\pgfsys@defobject{currentmarker}{\pgfqpoint{0.000000in}{-0.048611in}}{\pgfqpoint{0.000000in}{0.000000in}}{%
\pgfpathmoveto{\pgfqpoint{0.000000in}{0.000000in}}%
\pgfpathlineto{\pgfqpoint{0.000000in}{-0.048611in}}%
\pgfusepath{stroke,fill}%
}%
\begin{pgfscope}%
\pgfsys@transformshift{2.196993in}{0.467838in}%
\pgfsys@useobject{currentmarker}{}%
\end{pgfscope}%
\end{pgfscope}%
\begin{pgfscope}%
\definecolor{textcolor}{rgb}{0.000000,0.000000,0.000000}%
\pgfsetstrokecolor{textcolor}%
\pgfsetfillcolor{textcolor}%
\pgftext[x=2.196993in,y=0.370616in,,top]{\color{textcolor}\sffamily\fontsize{8.000000}{9.600000}\selectfont 0.8}%
\end{pgfscope}%
\begin{pgfscope}%
\pgfsetbuttcap%
\pgfsetroundjoin%
\definecolor{currentfill}{rgb}{0.000000,0.000000,0.000000}%
\pgfsetfillcolor{currentfill}%
\pgfsetlinewidth{0.803000pt}%
\definecolor{currentstroke}{rgb}{0.000000,0.000000,0.000000}%
\pgfsetstrokecolor{currentstroke}%
\pgfsetdash{}{0pt}%
\pgfsys@defobject{currentmarker}{\pgfqpoint{0.000000in}{-0.048611in}}{\pgfqpoint{0.000000in}{0.000000in}}{%
\pgfpathmoveto{\pgfqpoint{0.000000in}{0.000000in}}%
\pgfpathlineto{\pgfqpoint{0.000000in}{-0.048611in}}%
\pgfusepath{stroke,fill}%
}%
\begin{pgfscope}%
\pgfsys@transformshift{2.604669in}{0.467838in}%
\pgfsys@useobject{currentmarker}{}%
\end{pgfscope}%
\end{pgfscope}%
\begin{pgfscope}%
\definecolor{textcolor}{rgb}{0.000000,0.000000,0.000000}%
\pgfsetstrokecolor{textcolor}%
\pgfsetfillcolor{textcolor}%
\pgftext[x=2.604669in,y=0.370616in,,top]{\color{textcolor}\sffamily\fontsize{8.000000}{9.600000}\selectfont 1.0}%
\end{pgfscope}%
\begin{pgfscope}%
\definecolor{textcolor}{rgb}{0.000000,0.000000,0.000000}%
\pgfsetstrokecolor{textcolor}%
\pgfsetfillcolor{textcolor}%
\pgftext[x=1.687396in,y=0.207530in,,top]{\color{textcolor}\sffamily\fontsize{8.000000}{9.600000}\selectfont \(\displaystyle \beta\)}%
\end{pgfscope}%
\begin{pgfscope}%
\pgfsetbuttcap%
\pgfsetroundjoin%
\definecolor{currentfill}{rgb}{0.000000,0.000000,0.000000}%
\pgfsetfillcolor{currentfill}%
\pgfsetlinewidth{0.803000pt}%
\definecolor{currentstroke}{rgb}{0.000000,0.000000,0.000000}%
\pgfsetstrokecolor{currentstroke}%
\pgfsetdash{}{0pt}%
\pgfsys@defobject{currentmarker}{\pgfqpoint{-0.048611in}{0.000000in}}{\pgfqpoint{-0.000000in}{0.000000in}}{%
\pgfpathmoveto{\pgfqpoint{-0.000000in}{0.000000in}}%
\pgfpathlineto{\pgfqpoint{-0.048611in}{0.000000in}}%
\pgfusepath{stroke,fill}%
}%
\begin{pgfscope}%
\pgfsys@transformshift{0.678396in}{0.521526in}%
\pgfsys@useobject{currentmarker}{}%
\end{pgfscope}%
\end{pgfscope}%
\begin{pgfscope}%
\definecolor{textcolor}{rgb}{0.000000,0.000000,0.000000}%
\pgfsetstrokecolor{textcolor}%
\pgfsetfillcolor{textcolor}%
\pgftext[x=0.263086in, y=0.479317in, left, base]{\color{textcolor}\sffamily\fontsize{8.000000}{9.600000}\selectfont 0.000}%
\end{pgfscope}%
\begin{pgfscope}%
\pgfsetbuttcap%
\pgfsetroundjoin%
\definecolor{currentfill}{rgb}{0.000000,0.000000,0.000000}%
\pgfsetfillcolor{currentfill}%
\pgfsetlinewidth{0.803000pt}%
\definecolor{currentstroke}{rgb}{0.000000,0.000000,0.000000}%
\pgfsetstrokecolor{currentstroke}%
\pgfsetdash{}{0pt}%
\pgfsys@defobject{currentmarker}{\pgfqpoint{-0.048611in}{0.000000in}}{\pgfqpoint{-0.000000in}{0.000000in}}{%
\pgfpathmoveto{\pgfqpoint{-0.000000in}{0.000000in}}%
\pgfpathlineto{\pgfqpoint{-0.048611in}{0.000000in}}%
\pgfusepath{stroke,fill}%
}%
\begin{pgfscope}%
\pgfsys@transformshift{0.678396in}{0.860732in}%
\pgfsys@useobject{currentmarker}{}%
\end{pgfscope}%
\end{pgfscope}%
\begin{pgfscope}%
\definecolor{textcolor}{rgb}{0.000000,0.000000,0.000000}%
\pgfsetstrokecolor{textcolor}%
\pgfsetfillcolor{textcolor}%
\pgftext[x=0.263086in, y=0.818522in, left, base]{\color{textcolor}\sffamily\fontsize{8.000000}{9.600000}\selectfont 0.002}%
\end{pgfscope}%
\begin{pgfscope}%
\pgfsetbuttcap%
\pgfsetroundjoin%
\definecolor{currentfill}{rgb}{0.000000,0.000000,0.000000}%
\pgfsetfillcolor{currentfill}%
\pgfsetlinewidth{0.803000pt}%
\definecolor{currentstroke}{rgb}{0.000000,0.000000,0.000000}%
\pgfsetstrokecolor{currentstroke}%
\pgfsetdash{}{0pt}%
\pgfsys@defobject{currentmarker}{\pgfqpoint{-0.048611in}{0.000000in}}{\pgfqpoint{-0.000000in}{0.000000in}}{%
\pgfpathmoveto{\pgfqpoint{-0.000000in}{0.000000in}}%
\pgfpathlineto{\pgfqpoint{-0.048611in}{0.000000in}}%
\pgfusepath{stroke,fill}%
}%
\begin{pgfscope}%
\pgfsys@transformshift{0.678396in}{1.199937in}%
\pgfsys@useobject{currentmarker}{}%
\end{pgfscope}%
\end{pgfscope}%
\begin{pgfscope}%
\definecolor{textcolor}{rgb}{0.000000,0.000000,0.000000}%
\pgfsetstrokecolor{textcolor}%
\pgfsetfillcolor{textcolor}%
\pgftext[x=0.263086in, y=1.157728in, left, base]{\color{textcolor}\sffamily\fontsize{8.000000}{9.600000}\selectfont 0.004}%
\end{pgfscope}%
\begin{pgfscope}%
\pgfsetbuttcap%
\pgfsetroundjoin%
\definecolor{currentfill}{rgb}{0.000000,0.000000,0.000000}%
\pgfsetfillcolor{currentfill}%
\pgfsetlinewidth{0.803000pt}%
\definecolor{currentstroke}{rgb}{0.000000,0.000000,0.000000}%
\pgfsetstrokecolor{currentstroke}%
\pgfsetdash{}{0pt}%
\pgfsys@defobject{currentmarker}{\pgfqpoint{-0.048611in}{0.000000in}}{\pgfqpoint{-0.000000in}{0.000000in}}{%
\pgfpathmoveto{\pgfqpoint{-0.000000in}{0.000000in}}%
\pgfpathlineto{\pgfqpoint{-0.048611in}{0.000000in}}%
\pgfusepath{stroke,fill}%
}%
\begin{pgfscope}%
\pgfsys@transformshift{0.678396in}{1.539143in}%
\pgfsys@useobject{currentmarker}{}%
\end{pgfscope}%
\end{pgfscope}%
\begin{pgfscope}%
\definecolor{textcolor}{rgb}{0.000000,0.000000,0.000000}%
\pgfsetstrokecolor{textcolor}%
\pgfsetfillcolor{textcolor}%
\pgftext[x=0.263086in, y=1.496933in, left, base]{\color{textcolor}\sffamily\fontsize{8.000000}{9.600000}\selectfont 0.006}%
\end{pgfscope}%
\begin{pgfscope}%
\definecolor{textcolor}{rgb}{0.000000,0.000000,0.000000}%
\pgfsetstrokecolor{textcolor}%
\pgfsetfillcolor{textcolor}%
\pgftext[x=0.207530in,y=1.149369in,,bottom,rotate=90.000000]{\color{textcolor}\sffamily\fontsize{8.000000}{9.600000}\selectfont \(\displaystyle X\)}%
\end{pgfscope}%
\begin{pgfscope}%
\pgfpathrectangle{\pgfqpoint{0.678396in}{0.467838in}}{\pgfqpoint{2.018001in}{1.363061in}}%
\pgfusepath{clip}%
\pgfsetrectcap%
\pgfsetroundjoin%
\pgfsetlinewidth{1.505625pt}%
\definecolor{currentstroke}{rgb}{0.000000,0.000000,0.000000}%
\pgfsetstrokecolor{currentstroke}%
\pgfsetdash{}{0pt}%
\pgfpathmoveto{\pgfqpoint{0.770123in}{0.542217in}}%
\pgfpathlineto{\pgfqpoint{0.973962in}{0.572058in}}%
\pgfpathlineto{\pgfqpoint{1.177800in}{0.614085in}}%
\pgfpathlineto{\pgfqpoint{1.381639in}{0.672225in}}%
\pgfpathlineto{\pgfqpoint{1.585477in}{0.751539in}}%
\pgfpathlineto{\pgfqpoint{1.789316in}{0.858540in}}%
\pgfpathlineto{\pgfqpoint{1.993154in}{1.001571in}}%
\pgfpathlineto{\pgfqpoint{2.196993in}{1.191300in}}%
\pgfpathlineto{\pgfqpoint{2.400831in}{1.441325in}}%
\pgfpathlineto{\pgfqpoint{2.604669in}{1.768942in}}%
\pgfusepath{stroke}%
\end{pgfscope}%
\begin{pgfscope}%
\pgfpathrectangle{\pgfqpoint{0.678396in}{0.467838in}}{\pgfqpoint{2.018001in}{1.363061in}}%
\pgfusepath{clip}%
\pgfsetrectcap%
\pgfsetroundjoin%
\pgfsetlinewidth{1.505625pt}%
\definecolor{currentstroke}{rgb}{0.121569,0.466667,0.705882}%
\pgfsetstrokecolor{currentstroke}%
\pgfsetdash{}{0pt}%
\pgfpathmoveto{\pgfqpoint{0.770123in}{0.529796in}}%
\pgfpathlineto{\pgfqpoint{0.973962in}{0.542041in}}%
\pgfpathlineto{\pgfqpoint{1.177800in}{0.563116in}}%
\pgfpathlineto{\pgfqpoint{1.381639in}{0.596044in}}%
\pgfpathlineto{\pgfqpoint{1.585477in}{0.651218in}}%
\pgfpathlineto{\pgfqpoint{1.789316in}{0.735119in}}%
\pgfpathlineto{\pgfqpoint{1.993154in}{0.860588in}}%
\pgfpathlineto{\pgfqpoint{2.196993in}{1.015296in}}%
\pgfpathlineto{\pgfqpoint{2.400831in}{1.213409in}}%
\pgfpathlineto{\pgfqpoint{2.604669in}{1.487094in}}%
\pgfusepath{stroke}%
\end{pgfscope}%
\begin{pgfscope}%
\pgfsetrectcap%
\pgfsetmiterjoin%
\pgfsetlinewidth{0.803000pt}%
\definecolor{currentstroke}{rgb}{0.000000,0.000000,0.000000}%
\pgfsetstrokecolor{currentstroke}%
\pgfsetdash{}{0pt}%
\pgfpathmoveto{\pgfqpoint{0.678396in}{0.467838in}}%
\pgfpathlineto{\pgfqpoint{0.678396in}{1.830900in}}%
\pgfusepath{stroke}%
\end{pgfscope}%
\begin{pgfscope}%
\pgfsetrectcap%
\pgfsetmiterjoin%
\pgfsetlinewidth{0.803000pt}%
\definecolor{currentstroke}{rgb}{0.000000,0.000000,0.000000}%
\pgfsetstrokecolor{currentstroke}%
\pgfsetdash{}{0pt}%
\pgfpathmoveto{\pgfqpoint{2.696397in}{0.467838in}}%
\pgfpathlineto{\pgfqpoint{2.696397in}{1.830900in}}%
\pgfusepath{stroke}%
\end{pgfscope}%
\begin{pgfscope}%
\pgfsetrectcap%
\pgfsetmiterjoin%
\pgfsetlinewidth{0.803000pt}%
\definecolor{currentstroke}{rgb}{0.000000,0.000000,0.000000}%
\pgfsetstrokecolor{currentstroke}%
\pgfsetdash{}{0pt}%
\pgfpathmoveto{\pgfqpoint{0.678396in}{0.467838in}}%
\pgfpathlineto{\pgfqpoint{2.696397in}{0.467838in}}%
\pgfusepath{stroke}%
\end{pgfscope}%
\begin{pgfscope}%
\pgfsetrectcap%
\pgfsetmiterjoin%
\pgfsetlinewidth{0.803000pt}%
\definecolor{currentstroke}{rgb}{0.000000,0.000000,0.000000}%
\pgfsetstrokecolor{currentstroke}%
\pgfsetdash{}{0pt}%
\pgfpathmoveto{\pgfqpoint{0.678396in}{1.830900in}}%
\pgfpathlineto{\pgfqpoint{2.696397in}{1.830900in}}%
\pgfusepath{stroke}%
\end{pgfscope}%
\begin{pgfscope}%
\pgfsetbuttcap%
\pgfsetmiterjoin%
\definecolor{currentfill}{rgb}{1.000000,1.000000,1.000000}%
\pgfsetfillcolor{currentfill}%
\pgfsetfillopacity{0.800000}%
\pgfsetlinewidth{1.003750pt}%
\definecolor{currentstroke}{rgb}{0.800000,0.800000,0.800000}%
\pgfsetstrokecolor{currentstroke}%
\pgfsetstrokeopacity{0.800000}%
\pgfsetdash{}{0pt}%
\pgfpathmoveto{\pgfqpoint{0.756174in}{1.415839in}}%
\pgfpathlineto{\pgfqpoint{1.851204in}{1.415839in}}%
\pgfpathquadraticcurveto{\pgfqpoint{1.873427in}{1.415839in}}{\pgfqpoint{1.873427in}{1.438061in}}%
\pgfpathlineto{\pgfqpoint{1.873427in}{1.753122in}}%
\pgfpathquadraticcurveto{\pgfqpoint{1.873427in}{1.775344in}}{\pgfqpoint{1.851204in}{1.775344in}}%
\pgfpathlineto{\pgfqpoint{0.756174in}{1.775344in}}%
\pgfpathquadraticcurveto{\pgfqpoint{0.733952in}{1.775344in}}{\pgfqpoint{0.733952in}{1.753122in}}%
\pgfpathlineto{\pgfqpoint{0.733952in}{1.438061in}}%
\pgfpathquadraticcurveto{\pgfqpoint{0.733952in}{1.415839in}}{\pgfqpoint{0.756174in}{1.415839in}}%
\pgfpathlineto{\pgfqpoint{0.756174in}{1.415839in}}%
\pgfpathclose%
\pgfusepath{stroke,fill}%
\end{pgfscope}%
\begin{pgfscope}%
\pgfsetrectcap%
\pgfsetroundjoin%
\pgfsetlinewidth{1.505625pt}%
\definecolor{currentstroke}{rgb}{0.000000,0.000000,0.000000}%
\pgfsetstrokecolor{currentstroke}%
\pgfsetdash{}{0pt}%
\pgfpathmoveto{\pgfqpoint{0.778396in}{1.685370in}}%
\pgfpathlineto{\pgfqpoint{0.889507in}{1.685370in}}%
\pgfpathlineto{\pgfqpoint{1.000618in}{1.685370in}}%
\pgfusepath{stroke}%
\end{pgfscope}%
\begin{pgfscope}%
\definecolor{textcolor}{rgb}{0.000000,0.000000,0.000000}%
\pgfsetstrokecolor{textcolor}%
\pgfsetfillcolor{textcolor}%
\pgftext[x=1.089507in,y=1.646481in,left,base]{\color{textcolor}\sffamily\fontsize{8.000000}{9.600000}\selectfont exact 1D}%
\end{pgfscope}%
\begin{pgfscope}%
\pgfsetrectcap%
\pgfsetroundjoin%
\pgfsetlinewidth{1.505625pt}%
\definecolor{currentstroke}{rgb}{0.121569,0.466667,0.705882}%
\pgfsetstrokecolor{currentstroke}%
\pgfsetdash{}{0pt}%
\pgfpathmoveto{\pgfqpoint{0.778396in}{1.522284in}}%
\pgfpathlineto{\pgfqpoint{0.889507in}{1.522284in}}%
\pgfpathlineto{\pgfqpoint{1.000618in}{1.522284in}}%
\pgfusepath{stroke}%
\end{pgfscope}%
\begin{pgfscope}%
\definecolor{textcolor}{rgb}{0.000000,0.000000,0.000000}%
\pgfsetstrokecolor{textcolor}%
\pgfsetfillcolor{textcolor}%
\pgftext[x=1.089507in,y=1.483395in,left,base]{\color{textcolor}\sffamily\fontsize{8.000000}{9.600000}\selectfont conformation}%
\end{pgfscope}%
\begin{pgfscope}%
\pgfsetbuttcap%
\pgfsetmiterjoin%
\definecolor{currentfill}{rgb}{1.000000,1.000000,1.000000}%
\pgfsetfillcolor{currentfill}%
\pgfsetlinewidth{0.000000pt}%
\definecolor{currentstroke}{rgb}{0.000000,0.000000,0.000000}%
\pgfsetstrokecolor{currentstroke}%
\pgfsetstrokeopacity{0.000000}%
\pgfsetdash{}{0pt}%
\pgfpathmoveto{\pgfqpoint{3.099997in}{0.467838in}}%
\pgfpathlineto{\pgfqpoint{5.117998in}{0.467838in}}%
\pgfpathlineto{\pgfqpoint{5.117998in}{1.830900in}}%
\pgfpathlineto{\pgfqpoint{3.099997in}{1.830900in}}%
\pgfpathlineto{\pgfqpoint{3.099997in}{0.467838in}}%
\pgfpathclose%
\pgfusepath{fill}%
\end{pgfscope}%
\begin{pgfscope}%
\pgfsetbuttcap%
\pgfsetroundjoin%
\definecolor{currentfill}{rgb}{0.000000,0.000000,0.000000}%
\pgfsetfillcolor{currentfill}%
\pgfsetlinewidth{0.803000pt}%
\definecolor{currentstroke}{rgb}{0.000000,0.000000,0.000000}%
\pgfsetstrokecolor{currentstroke}%
\pgfsetdash{}{0pt}%
\pgfsys@defobject{currentmarker}{\pgfqpoint{0.000000in}{-0.048611in}}{\pgfqpoint{0.000000in}{0.000000in}}{%
\pgfpathmoveto{\pgfqpoint{0.000000in}{0.000000in}}%
\pgfpathlineto{\pgfqpoint{0.000000in}{-0.048611in}}%
\pgfusepath{stroke,fill}%
}%
\begin{pgfscope}%
\pgfsys@transformshift{3.395563in}{0.467838in}%
\pgfsys@useobject{currentmarker}{}%
\end{pgfscope}%
\end{pgfscope}%
\begin{pgfscope}%
\definecolor{textcolor}{rgb}{0.000000,0.000000,0.000000}%
\pgfsetstrokecolor{textcolor}%
\pgfsetfillcolor{textcolor}%
\pgftext[x=3.395563in,y=0.370616in,,top]{\color{textcolor}\sffamily\fontsize{8.000000}{9.600000}\selectfont 0.2}%
\end{pgfscope}%
\begin{pgfscope}%
\pgfsetbuttcap%
\pgfsetroundjoin%
\definecolor{currentfill}{rgb}{0.000000,0.000000,0.000000}%
\pgfsetfillcolor{currentfill}%
\pgfsetlinewidth{0.803000pt}%
\definecolor{currentstroke}{rgb}{0.000000,0.000000,0.000000}%
\pgfsetstrokecolor{currentstroke}%
\pgfsetdash{}{0pt}%
\pgfsys@defobject{currentmarker}{\pgfqpoint{0.000000in}{-0.048611in}}{\pgfqpoint{0.000000in}{0.000000in}}{%
\pgfpathmoveto{\pgfqpoint{0.000000in}{0.000000in}}%
\pgfpathlineto{\pgfqpoint{0.000000in}{-0.048611in}}%
\pgfusepath{stroke,fill}%
}%
\begin{pgfscope}%
\pgfsys@transformshift{3.803240in}{0.467838in}%
\pgfsys@useobject{currentmarker}{}%
\end{pgfscope}%
\end{pgfscope}%
\begin{pgfscope}%
\definecolor{textcolor}{rgb}{0.000000,0.000000,0.000000}%
\pgfsetstrokecolor{textcolor}%
\pgfsetfillcolor{textcolor}%
\pgftext[x=3.803240in,y=0.370616in,,top]{\color{textcolor}\sffamily\fontsize{8.000000}{9.600000}\selectfont 0.4}%
\end{pgfscope}%
\begin{pgfscope}%
\pgfsetbuttcap%
\pgfsetroundjoin%
\definecolor{currentfill}{rgb}{0.000000,0.000000,0.000000}%
\pgfsetfillcolor{currentfill}%
\pgfsetlinewidth{0.803000pt}%
\definecolor{currentstroke}{rgb}{0.000000,0.000000,0.000000}%
\pgfsetstrokecolor{currentstroke}%
\pgfsetdash{}{0pt}%
\pgfsys@defobject{currentmarker}{\pgfqpoint{0.000000in}{-0.048611in}}{\pgfqpoint{0.000000in}{0.000000in}}{%
\pgfpathmoveto{\pgfqpoint{0.000000in}{0.000000in}}%
\pgfpathlineto{\pgfqpoint{0.000000in}{-0.048611in}}%
\pgfusepath{stroke,fill}%
}%
\begin{pgfscope}%
\pgfsys@transformshift{4.210916in}{0.467838in}%
\pgfsys@useobject{currentmarker}{}%
\end{pgfscope}%
\end{pgfscope}%
\begin{pgfscope}%
\definecolor{textcolor}{rgb}{0.000000,0.000000,0.000000}%
\pgfsetstrokecolor{textcolor}%
\pgfsetfillcolor{textcolor}%
\pgftext[x=4.210916in,y=0.370616in,,top]{\color{textcolor}\sffamily\fontsize{8.000000}{9.600000}\selectfont 0.6}%
\end{pgfscope}%
\begin{pgfscope}%
\pgfsetbuttcap%
\pgfsetroundjoin%
\definecolor{currentfill}{rgb}{0.000000,0.000000,0.000000}%
\pgfsetfillcolor{currentfill}%
\pgfsetlinewidth{0.803000pt}%
\definecolor{currentstroke}{rgb}{0.000000,0.000000,0.000000}%
\pgfsetstrokecolor{currentstroke}%
\pgfsetdash{}{0pt}%
\pgfsys@defobject{currentmarker}{\pgfqpoint{0.000000in}{-0.048611in}}{\pgfqpoint{0.000000in}{0.000000in}}{%
\pgfpathmoveto{\pgfqpoint{0.000000in}{0.000000in}}%
\pgfpathlineto{\pgfqpoint{0.000000in}{-0.048611in}}%
\pgfusepath{stroke,fill}%
}%
\begin{pgfscope}%
\pgfsys@transformshift{4.618593in}{0.467838in}%
\pgfsys@useobject{currentmarker}{}%
\end{pgfscope}%
\end{pgfscope}%
\begin{pgfscope}%
\definecolor{textcolor}{rgb}{0.000000,0.000000,0.000000}%
\pgfsetstrokecolor{textcolor}%
\pgfsetfillcolor{textcolor}%
\pgftext[x=4.618593in,y=0.370616in,,top]{\color{textcolor}\sffamily\fontsize{8.000000}{9.600000}\selectfont 0.8}%
\end{pgfscope}%
\begin{pgfscope}%
\pgfsetbuttcap%
\pgfsetroundjoin%
\definecolor{currentfill}{rgb}{0.000000,0.000000,0.000000}%
\pgfsetfillcolor{currentfill}%
\pgfsetlinewidth{0.803000pt}%
\definecolor{currentstroke}{rgb}{0.000000,0.000000,0.000000}%
\pgfsetstrokecolor{currentstroke}%
\pgfsetdash{}{0pt}%
\pgfsys@defobject{currentmarker}{\pgfqpoint{0.000000in}{-0.048611in}}{\pgfqpoint{0.000000in}{0.000000in}}{%
\pgfpathmoveto{\pgfqpoint{0.000000in}{0.000000in}}%
\pgfpathlineto{\pgfqpoint{0.000000in}{-0.048611in}}%
\pgfusepath{stroke,fill}%
}%
\begin{pgfscope}%
\pgfsys@transformshift{5.026270in}{0.467838in}%
\pgfsys@useobject{currentmarker}{}%
\end{pgfscope}%
\end{pgfscope}%
\begin{pgfscope}%
\definecolor{textcolor}{rgb}{0.000000,0.000000,0.000000}%
\pgfsetstrokecolor{textcolor}%
\pgfsetfillcolor{textcolor}%
\pgftext[x=5.026270in,y=0.370616in,,top]{\color{textcolor}\sffamily\fontsize{8.000000}{9.600000}\selectfont 1.0}%
\end{pgfscope}%
\begin{pgfscope}%
\definecolor{textcolor}{rgb}{0.000000,0.000000,0.000000}%
\pgfsetstrokecolor{textcolor}%
\pgfsetfillcolor{textcolor}%
\pgftext[x=4.108997in,y=0.207530in,,top]{\color{textcolor}\sffamily\fontsize{8.000000}{9.600000}\selectfont \(\displaystyle \beta\)}%
\end{pgfscope}%
\begin{pgfscope}%
\pgfsetbuttcap%
\pgfsetroundjoin%
\definecolor{currentfill}{rgb}{0.000000,0.000000,0.000000}%
\pgfsetfillcolor{currentfill}%
\pgfsetlinewidth{0.803000pt}%
\definecolor{currentstroke}{rgb}{0.000000,0.000000,0.000000}%
\pgfsetstrokecolor{currentstroke}%
\pgfsetdash{}{0pt}%
\pgfsys@defobject{currentmarker}{\pgfqpoint{-0.048611in}{0.000000in}}{\pgfqpoint{-0.000000in}{0.000000in}}{%
\pgfpathmoveto{\pgfqpoint{-0.000000in}{0.000000in}}%
\pgfpathlineto{\pgfqpoint{-0.048611in}{0.000000in}}%
\pgfusepath{stroke,fill}%
}%
\begin{pgfscope}%
\pgfsys@transformshift{3.099997in}{0.519091in}%
\pgfsys@useobject{currentmarker}{}%
\end{pgfscope}%
\end{pgfscope}%
\begin{pgfscope}%
\definecolor{textcolor}{rgb}{0.000000,0.000000,0.000000}%
\pgfsetstrokecolor{textcolor}%
\pgfsetfillcolor{textcolor}%
\pgftext[x=2.684687in, y=0.476882in, left, base]{\color{textcolor}\sffamily\fontsize{8.000000}{9.600000}\selectfont 0.000}%
\end{pgfscope}%
\begin{pgfscope}%
\pgfsetbuttcap%
\pgfsetroundjoin%
\definecolor{currentfill}{rgb}{0.000000,0.000000,0.000000}%
\pgfsetfillcolor{currentfill}%
\pgfsetlinewidth{0.803000pt}%
\definecolor{currentstroke}{rgb}{0.000000,0.000000,0.000000}%
\pgfsetstrokecolor{currentstroke}%
\pgfsetdash{}{0pt}%
\pgfsys@defobject{currentmarker}{\pgfqpoint{-0.048611in}{0.000000in}}{\pgfqpoint{-0.000000in}{0.000000in}}{%
\pgfpathmoveto{\pgfqpoint{-0.000000in}{0.000000in}}%
\pgfpathlineto{\pgfqpoint{-0.048611in}{0.000000in}}%
\pgfusepath{stroke,fill}%
}%
\begin{pgfscope}%
\pgfsys@transformshift{3.099997in}{0.848519in}%
\pgfsys@useobject{currentmarker}{}%
\end{pgfscope}%
\end{pgfscope}%
\begin{pgfscope}%
\definecolor{textcolor}{rgb}{0.000000,0.000000,0.000000}%
\pgfsetstrokecolor{textcolor}%
\pgfsetfillcolor{textcolor}%
\pgftext[x=2.684687in, y=0.806310in, left, base]{\color{textcolor}\sffamily\fontsize{8.000000}{9.600000}\selectfont 0.002}%
\end{pgfscope}%
\begin{pgfscope}%
\pgfsetbuttcap%
\pgfsetroundjoin%
\definecolor{currentfill}{rgb}{0.000000,0.000000,0.000000}%
\pgfsetfillcolor{currentfill}%
\pgfsetlinewidth{0.803000pt}%
\definecolor{currentstroke}{rgb}{0.000000,0.000000,0.000000}%
\pgfsetstrokecolor{currentstroke}%
\pgfsetdash{}{0pt}%
\pgfsys@defobject{currentmarker}{\pgfqpoint{-0.048611in}{0.000000in}}{\pgfqpoint{-0.000000in}{0.000000in}}{%
\pgfpathmoveto{\pgfqpoint{-0.000000in}{0.000000in}}%
\pgfpathlineto{\pgfqpoint{-0.048611in}{0.000000in}}%
\pgfusepath{stroke,fill}%
}%
\begin{pgfscope}%
\pgfsys@transformshift{3.099997in}{1.177947in}%
\pgfsys@useobject{currentmarker}{}%
\end{pgfscope}%
\end{pgfscope}%
\begin{pgfscope}%
\definecolor{textcolor}{rgb}{0.000000,0.000000,0.000000}%
\pgfsetstrokecolor{textcolor}%
\pgfsetfillcolor{textcolor}%
\pgftext[x=2.684687in, y=1.135737in, left, base]{\color{textcolor}\sffamily\fontsize{8.000000}{9.600000}\selectfont 0.004}%
\end{pgfscope}%
\begin{pgfscope}%
\pgfsetbuttcap%
\pgfsetroundjoin%
\definecolor{currentfill}{rgb}{0.000000,0.000000,0.000000}%
\pgfsetfillcolor{currentfill}%
\pgfsetlinewidth{0.803000pt}%
\definecolor{currentstroke}{rgb}{0.000000,0.000000,0.000000}%
\pgfsetstrokecolor{currentstroke}%
\pgfsetdash{}{0pt}%
\pgfsys@defobject{currentmarker}{\pgfqpoint{-0.048611in}{0.000000in}}{\pgfqpoint{-0.000000in}{0.000000in}}{%
\pgfpathmoveto{\pgfqpoint{-0.000000in}{0.000000in}}%
\pgfpathlineto{\pgfqpoint{-0.048611in}{0.000000in}}%
\pgfusepath{stroke,fill}%
}%
\begin{pgfscope}%
\pgfsys@transformshift{3.099997in}{1.507374in}%
\pgfsys@useobject{currentmarker}{}%
\end{pgfscope}%
\end{pgfscope}%
\begin{pgfscope}%
\definecolor{textcolor}{rgb}{0.000000,0.000000,0.000000}%
\pgfsetstrokecolor{textcolor}%
\pgfsetfillcolor{textcolor}%
\pgftext[x=2.684687in, y=1.465165in, left, base]{\color{textcolor}\sffamily\fontsize{8.000000}{9.600000}\selectfont 0.006}%
\end{pgfscope}%
\begin{pgfscope}%
\pgfpathrectangle{\pgfqpoint{3.099997in}{0.467838in}}{\pgfqpoint{2.018001in}{1.363061in}}%
\pgfusepath{clip}%
\pgfsetrectcap%
\pgfsetroundjoin%
\pgfsetlinewidth{1.505625pt}%
\definecolor{currentstroke}{rgb}{0.000000,0.000000,0.000000}%
\pgfsetstrokecolor{currentstroke}%
\pgfsetdash{}{0pt}%
\pgfpathmoveto{\pgfqpoint{3.191724in}{0.539185in}}%
\pgfpathlineto{\pgfqpoint{3.395563in}{0.568167in}}%
\pgfpathlineto{\pgfqpoint{3.599401in}{0.608983in}}%
\pgfpathlineto{\pgfqpoint{3.803240in}{0.665446in}}%
\pgfpathlineto{\pgfqpoint{4.007078in}{0.742474in}}%
\pgfpathlineto{\pgfqpoint{4.210916in}{0.846391in}}%
\pgfpathlineto{\pgfqpoint{4.414755in}{0.985299in}}%
\pgfpathlineto{\pgfqpoint{4.618593in}{1.169558in}}%
\pgfpathlineto{\pgfqpoint{4.822432in}{1.412377in}}%
\pgfpathlineto{\pgfqpoint{5.026270in}{1.730550in}}%
\pgfusepath{stroke}%
\end{pgfscope}%
\begin{pgfscope}%
\pgfpathrectangle{\pgfqpoint{3.099997in}{0.467838in}}{\pgfqpoint{2.018001in}{1.363061in}}%
\pgfusepath{clip}%
\pgfsetrectcap%
\pgfsetroundjoin%
\pgfsetlinewidth{1.505625pt}%
\definecolor{currentstroke}{rgb}{0.121569,0.466667,0.705882}%
\pgfsetstrokecolor{currentstroke}%
\pgfsetdash{}{0pt}%
\pgfpathmoveto{\pgfqpoint{3.191724in}{0.529796in}}%
\pgfpathlineto{\pgfqpoint{3.395563in}{0.545647in}}%
\pgfpathlineto{\pgfqpoint{3.599401in}{0.572927in}}%
\pgfpathlineto{\pgfqpoint{3.803240in}{0.615550in}}%
\pgfpathlineto{\pgfqpoint{4.007078in}{0.686967in}}%
\pgfpathlineto{\pgfqpoint{4.210916in}{0.795571in}}%
\pgfpathlineto{\pgfqpoint{4.414755in}{0.957980in}}%
\pgfpathlineto{\pgfqpoint{4.618593in}{1.158238in}}%
\pgfpathlineto{\pgfqpoint{4.822432in}{1.414679in}}%
\pgfpathlineto{\pgfqpoint{5.026270in}{1.768942in}}%
\pgfusepath{stroke}%
\end{pgfscope}%
\begin{pgfscope}%
\pgfsetrectcap%
\pgfsetmiterjoin%
\pgfsetlinewidth{0.803000pt}%
\definecolor{currentstroke}{rgb}{0.000000,0.000000,0.000000}%
\pgfsetstrokecolor{currentstroke}%
\pgfsetdash{}{0pt}%
\pgfpathmoveto{\pgfqpoint{3.099997in}{0.467838in}}%
\pgfpathlineto{\pgfqpoint{3.099997in}{1.830900in}}%
\pgfusepath{stroke}%
\end{pgfscope}%
\begin{pgfscope}%
\pgfsetrectcap%
\pgfsetmiterjoin%
\pgfsetlinewidth{0.803000pt}%
\definecolor{currentstroke}{rgb}{0.000000,0.000000,0.000000}%
\pgfsetstrokecolor{currentstroke}%
\pgfsetdash{}{0pt}%
\pgfpathmoveto{\pgfqpoint{5.117998in}{0.467838in}}%
\pgfpathlineto{\pgfqpoint{5.117998in}{1.830900in}}%
\pgfusepath{stroke}%
\end{pgfscope}%
\begin{pgfscope}%
\pgfsetrectcap%
\pgfsetmiterjoin%
\pgfsetlinewidth{0.803000pt}%
\definecolor{currentstroke}{rgb}{0.000000,0.000000,0.000000}%
\pgfsetstrokecolor{currentstroke}%
\pgfsetdash{}{0pt}%
\pgfpathmoveto{\pgfqpoint{3.099997in}{0.467838in}}%
\pgfpathlineto{\pgfqpoint{5.117998in}{0.467838in}}%
\pgfusepath{stroke}%
\end{pgfscope}%
\begin{pgfscope}%
\pgfsetrectcap%
\pgfsetmiterjoin%
\pgfsetlinewidth{0.803000pt}%
\definecolor{currentstroke}{rgb}{0.000000,0.000000,0.000000}%
\pgfsetstrokecolor{currentstroke}%
\pgfsetdash{}{0pt}%
\pgfpathmoveto{\pgfqpoint{3.099997in}{1.830900in}}%
\pgfpathlineto{\pgfqpoint{5.117998in}{1.830900in}}%
\pgfusepath{stroke}%
\end{pgfscope}%
\begin{pgfscope}%
\pgfsetbuttcap%
\pgfsetmiterjoin%
\definecolor{currentfill}{rgb}{1.000000,1.000000,1.000000}%
\pgfsetfillcolor{currentfill}%
\pgfsetfillopacity{0.800000}%
\pgfsetlinewidth{1.003750pt}%
\definecolor{currentstroke}{rgb}{0.800000,0.800000,0.800000}%
\pgfsetstrokecolor{currentstroke}%
\pgfsetstrokeopacity{0.800000}%
\pgfsetdash{}{0pt}%
\pgfpathmoveto{\pgfqpoint{3.177775in}{1.415839in}}%
\pgfpathlineto{\pgfqpoint{4.664949in}{1.415839in}}%
\pgfpathquadraticcurveto{\pgfqpoint{4.687171in}{1.415839in}}{\pgfqpoint{4.687171in}{1.438061in}}%
\pgfpathlineto{\pgfqpoint{4.687171in}{1.753122in}}%
\pgfpathquadraticcurveto{\pgfqpoint{4.687171in}{1.775344in}}{\pgfqpoint{4.664949in}{1.775344in}}%
\pgfpathlineto{\pgfqpoint{3.177775in}{1.775344in}}%
\pgfpathquadraticcurveto{\pgfqpoint{3.155552in}{1.775344in}}{\pgfqpoint{3.155552in}{1.753122in}}%
\pgfpathlineto{\pgfqpoint{3.155552in}{1.438061in}}%
\pgfpathquadraticcurveto{\pgfqpoint{3.155552in}{1.415839in}}{\pgfqpoint{3.177775in}{1.415839in}}%
\pgfpathlineto{\pgfqpoint{3.177775in}{1.415839in}}%
\pgfpathclose%
\pgfusepath{stroke,fill}%
\end{pgfscope}%
\begin{pgfscope}%
\pgfsetrectcap%
\pgfsetroundjoin%
\pgfsetlinewidth{1.505625pt}%
\definecolor{currentstroke}{rgb}{0.000000,0.000000,0.000000}%
\pgfsetstrokecolor{currentstroke}%
\pgfsetdash{}{0pt}%
\pgfpathmoveto{\pgfqpoint{3.199997in}{1.685370in}}%
\pgfpathlineto{\pgfqpoint{3.311108in}{1.685370in}}%
\pgfpathlineto{\pgfqpoint{3.422219in}{1.685370in}}%
\pgfusepath{stroke}%
\end{pgfscope}%
\begin{pgfscope}%
\definecolor{textcolor}{rgb}{0.000000,0.000000,0.000000}%
\pgfsetstrokecolor{textcolor}%
\pgfsetfillcolor{textcolor}%
\pgftext[x=3.511108in,y=1.646481in,left,base]{\color{textcolor}\sffamily\fontsize{8.000000}{9.600000}\selectfont exact 1D}%
\end{pgfscope}%
\begin{pgfscope}%
\pgfsetrectcap%
\pgfsetroundjoin%
\pgfsetlinewidth{1.505625pt}%
\definecolor{currentstroke}{rgb}{0.121569,0.466667,0.705882}%
\pgfsetstrokecolor{currentstroke}%
\pgfsetdash{}{0pt}%
\pgfpathmoveto{\pgfqpoint{3.199997in}{1.522284in}}%
\pgfpathlineto{\pgfqpoint{3.311108in}{1.522284in}}%
\pgfpathlineto{\pgfqpoint{3.422219in}{1.522284in}}%
\pgfusepath{stroke}%
\end{pgfscope}%
\begin{pgfscope}%
\definecolor{textcolor}{rgb}{0.000000,0.000000,0.000000}%
\pgfsetstrokecolor{textcolor}%
\pgfsetfillcolor{textcolor}%
\pgftext[x=3.511108in,y=1.483395in,left,base]{\color{textcolor}\sffamily\fontsize{8.000000}{9.600000}\selectfont conformation scaled}%
\end{pgfscope}%
\end{pgfpicture}%
\makeatother%
\endgroup%

	\caption{сравнение магнитной восприимчивости типичной конформации при $U=0.1$ и магнитной восприимчивости одномерной моделии изинга, с открытыми граничными условиями. На втором графике магнитная восприимчивость конформации домножена на коэффициент, подобранный методом наименьших квадратов.}
	\label{fig:MS_1D_comparison}
\end{figure}


Чтобы убедиться что большинство конформаций имеют магнитную восприимчивость схожую с одномерной цепочкой мы вычислили среднеквадратичное отклонение магнитной восприимчивости конформаций от одномерной цепочки. Распределение полученных значений представлено на рис. \ref{fig:MS_1D_dif_distr}. На нём мы можем видеть пик в 0, означающий что большинство конформаций имеют магнитную восприимчивость близкую к одномерной модели. Так же интересным наблюдением является форма распределения, которая схожа с распределениями полученными при исследования кластеров и мостов в конформациях пи $U=1$.

\begin{figure}[ht]
	\centering
	%% Creator: Matplotlib, PGF backend
%%
%% To include the figure in your LaTeX document, write
%%   \input{<filename>.pgf}
%%
%% Make sure the required packages are loaded in your preamble
%%   \usepackage{pgf}
%%
%% Also ensure that all the required font packages are loaded; for instance,
%% the lmodern package is sometimes necessary when using math font.
%%   \usepackage{lmodern}
%%
%% Figures using additional raster images can only be included by \input if
%% they are in the same directory as the main LaTeX file. For loading figures
%% from other directories you can use the `import` package
%%   \usepackage{import}
%%
%% and then include the figures with
%%   \import{<path to file>}{<filename>.pgf}
%%
%% Matplotlib used the following preamble
%%   
%%   \usepackage{fontspec}
%%   \setmainfont{DejaVuSerif.ttf}[Path=\detokenize{/home/roman/anaconda3/envs/ising/lib/python3.8/site-packages/matplotlib/mpl-data/fonts/ttf/}]
%%   \setsansfont{DejaVuSans.ttf}[Path=\detokenize{/home/roman/anaconda3/envs/ising/lib/python3.8/site-packages/matplotlib/mpl-data/fonts/ttf/}]
%%   \setmonofont{DejaVuSansMono.ttf}[Path=\detokenize{/home/roman/anaconda3/envs/ising/lib/python3.8/site-packages/matplotlib/mpl-data/fonts/ttf/}]
%%   \makeatletter\@ifpackageloaded{underscore}{}{\usepackage[strings]{underscore}}\makeatother
%%
\begingroup%
\makeatletter%
\begin{pgfpicture}%
\pgfpathrectangle{\pgfpointorigin}{\pgfqpoint{4.469938in}{2.648979in}}%
\pgfusepath{use as bounding box, clip}%
\begin{pgfscope}%
\pgfsetbuttcap%
\pgfsetmiterjoin%
\definecolor{currentfill}{rgb}{1.000000,1.000000,1.000000}%
\pgfsetfillcolor{currentfill}%
\pgfsetlinewidth{0.000000pt}%
\definecolor{currentstroke}{rgb}{1.000000,1.000000,1.000000}%
\pgfsetstrokecolor{currentstroke}%
\pgfsetdash{}{0pt}%
\pgfpathmoveto{\pgfqpoint{-0.000000in}{0.000000in}}%
\pgfpathlineto{\pgfqpoint{4.469938in}{0.000000in}}%
\pgfpathlineto{\pgfqpoint{4.469938in}{2.648979in}}%
\pgfpathlineto{\pgfqpoint{-0.000000in}{2.648979in}}%
\pgfpathlineto{\pgfqpoint{-0.000000in}{0.000000in}}%
\pgfpathclose%
\pgfusepath{fill}%
\end{pgfscope}%
\begin{pgfscope}%
\pgfsetbuttcap%
\pgfsetmiterjoin%
\definecolor{currentfill}{rgb}{1.000000,1.000000,1.000000}%
\pgfsetfillcolor{currentfill}%
\pgfsetlinewidth{0.000000pt}%
\definecolor{currentstroke}{rgb}{0.000000,0.000000,0.000000}%
\pgfsetstrokecolor{currentstroke}%
\pgfsetstrokeopacity{0.000000}%
\pgfsetdash{}{0pt}%
\pgfpathmoveto{\pgfqpoint{0.755102in}{0.521603in}}%
\pgfpathlineto{\pgfqpoint{3.714836in}{0.521603in}}%
\pgfpathlineto{\pgfqpoint{3.714836in}{2.339018in}}%
\pgfpathlineto{\pgfqpoint{0.755102in}{2.339018in}}%
\pgfpathlineto{\pgfqpoint{0.755102in}{0.521603in}}%
\pgfpathclose%
\pgfusepath{fill}%
\end{pgfscope}%
\begin{pgfscope}%
\pgfpathrectangle{\pgfqpoint{0.755102in}{0.521603in}}{\pgfqpoint{2.959734in}{1.817415in}}%
\pgfusepath{clip}%
\pgfsetbuttcap%
\pgfsetmiterjoin%
\definecolor{currentfill}{rgb}{0.121569,0.466667,0.705882}%
\pgfsetfillcolor{currentfill}%
\pgfsetlinewidth{0.000000pt}%
\definecolor{currentstroke}{rgb}{0.000000,0.000000,0.000000}%
\pgfsetstrokecolor{currentstroke}%
\pgfsetstrokeopacity{0.000000}%
\pgfsetdash{}{0pt}%
\pgfpathmoveto{\pgfqpoint{0.889635in}{0.521603in}}%
\pgfpathlineto{\pgfqpoint{0.993123in}{0.521603in}}%
\pgfpathlineto{\pgfqpoint{0.993123in}{2.252475in}}%
\pgfpathlineto{\pgfqpoint{0.889635in}{2.252475in}}%
\pgfpathlineto{\pgfqpoint{0.889635in}{0.521603in}}%
\pgfpathclose%
\pgfusepath{fill}%
\end{pgfscope}%
\begin{pgfscope}%
\pgfpathrectangle{\pgfqpoint{0.755102in}{0.521603in}}{\pgfqpoint{2.959734in}{1.817415in}}%
\pgfusepath{clip}%
\pgfsetbuttcap%
\pgfsetmiterjoin%
\definecolor{currentfill}{rgb}{0.121569,0.466667,0.705882}%
\pgfsetfillcolor{currentfill}%
\pgfsetlinewidth{0.000000pt}%
\definecolor{currentstroke}{rgb}{0.000000,0.000000,0.000000}%
\pgfsetstrokecolor{currentstroke}%
\pgfsetstrokeopacity{0.000000}%
\pgfsetdash{}{0pt}%
\pgfpathmoveto{\pgfqpoint{0.993123in}{0.521603in}}%
\pgfpathlineto{\pgfqpoint{1.096610in}{0.521603in}}%
\pgfpathlineto{\pgfqpoint{1.096610in}{0.564209in}}%
\pgfpathlineto{\pgfqpoint{0.993123in}{0.564209in}}%
\pgfpathlineto{\pgfqpoint{0.993123in}{0.521603in}}%
\pgfpathclose%
\pgfusepath{fill}%
\end{pgfscope}%
\begin{pgfscope}%
\pgfpathrectangle{\pgfqpoint{0.755102in}{0.521603in}}{\pgfqpoint{2.959734in}{1.817415in}}%
\pgfusepath{clip}%
\pgfsetbuttcap%
\pgfsetmiterjoin%
\definecolor{currentfill}{rgb}{0.121569,0.466667,0.705882}%
\pgfsetfillcolor{currentfill}%
\pgfsetlinewidth{0.000000pt}%
\definecolor{currentstroke}{rgb}{0.000000,0.000000,0.000000}%
\pgfsetstrokecolor{currentstroke}%
\pgfsetstrokeopacity{0.000000}%
\pgfsetdash{}{0pt}%
\pgfpathmoveto{\pgfqpoint{1.096610in}{0.521603in}}%
\pgfpathlineto{\pgfqpoint{1.200097in}{0.521603in}}%
\pgfpathlineto{\pgfqpoint{1.200097in}{0.569535in}}%
\pgfpathlineto{\pgfqpoint{1.096610in}{0.569535in}}%
\pgfpathlineto{\pgfqpoint{1.096610in}{0.521603in}}%
\pgfpathclose%
\pgfusepath{fill}%
\end{pgfscope}%
\begin{pgfscope}%
\pgfpathrectangle{\pgfqpoint{0.755102in}{0.521603in}}{\pgfqpoint{2.959734in}{1.817415in}}%
\pgfusepath{clip}%
\pgfsetbuttcap%
\pgfsetmiterjoin%
\definecolor{currentfill}{rgb}{0.121569,0.466667,0.705882}%
\pgfsetfillcolor{currentfill}%
\pgfsetlinewidth{0.000000pt}%
\definecolor{currentstroke}{rgb}{0.000000,0.000000,0.000000}%
\pgfsetstrokecolor{currentstroke}%
\pgfsetstrokeopacity{0.000000}%
\pgfsetdash{}{0pt}%
\pgfpathmoveto{\pgfqpoint{1.200097in}{0.521603in}}%
\pgfpathlineto{\pgfqpoint{1.303584in}{0.521603in}}%
\pgfpathlineto{\pgfqpoint{1.303584in}{0.572198in}}%
\pgfpathlineto{\pgfqpoint{1.200097in}{0.572198in}}%
\pgfpathlineto{\pgfqpoint{1.200097in}{0.521603in}}%
\pgfpathclose%
\pgfusepath{fill}%
\end{pgfscope}%
\begin{pgfscope}%
\pgfpathrectangle{\pgfqpoint{0.755102in}{0.521603in}}{\pgfqpoint{2.959734in}{1.817415in}}%
\pgfusepath{clip}%
\pgfsetbuttcap%
\pgfsetmiterjoin%
\definecolor{currentfill}{rgb}{0.121569,0.466667,0.705882}%
\pgfsetfillcolor{currentfill}%
\pgfsetlinewidth{0.000000pt}%
\definecolor{currentstroke}{rgb}{0.000000,0.000000,0.000000}%
\pgfsetstrokecolor{currentstroke}%
\pgfsetstrokeopacity{0.000000}%
\pgfsetdash{}{0pt}%
\pgfpathmoveto{\pgfqpoint{1.303584in}{0.521603in}}%
\pgfpathlineto{\pgfqpoint{1.407071in}{0.521603in}}%
\pgfpathlineto{\pgfqpoint{1.407071in}{0.604153in}}%
\pgfpathlineto{\pgfqpoint{1.303584in}{0.604153in}}%
\pgfpathlineto{\pgfqpoint{1.303584in}{0.521603in}}%
\pgfpathclose%
\pgfusepath{fill}%
\end{pgfscope}%
\begin{pgfscope}%
\pgfpathrectangle{\pgfqpoint{0.755102in}{0.521603in}}{\pgfqpoint{2.959734in}{1.817415in}}%
\pgfusepath{clip}%
\pgfsetbuttcap%
\pgfsetmiterjoin%
\definecolor{currentfill}{rgb}{0.121569,0.466667,0.705882}%
\pgfsetfillcolor{currentfill}%
\pgfsetlinewidth{0.000000pt}%
\definecolor{currentstroke}{rgb}{0.000000,0.000000,0.000000}%
\pgfsetstrokecolor{currentstroke}%
\pgfsetstrokeopacity{0.000000}%
\pgfsetdash{}{0pt}%
\pgfpathmoveto{\pgfqpoint{1.407071in}{0.521603in}}%
\pgfpathlineto{\pgfqpoint{1.510559in}{0.521603in}}%
\pgfpathlineto{\pgfqpoint{1.510559in}{0.646759in}}%
\pgfpathlineto{\pgfqpoint{1.407071in}{0.646759in}}%
\pgfpathlineto{\pgfqpoint{1.407071in}{0.521603in}}%
\pgfpathclose%
\pgfusepath{fill}%
\end{pgfscope}%
\begin{pgfscope}%
\pgfpathrectangle{\pgfqpoint{0.755102in}{0.521603in}}{\pgfqpoint{2.959734in}{1.817415in}}%
\pgfusepath{clip}%
\pgfsetbuttcap%
\pgfsetmiterjoin%
\definecolor{currentfill}{rgb}{0.121569,0.466667,0.705882}%
\pgfsetfillcolor{currentfill}%
\pgfsetlinewidth{0.000000pt}%
\definecolor{currentstroke}{rgb}{0.000000,0.000000,0.000000}%
\pgfsetstrokecolor{currentstroke}%
\pgfsetstrokeopacity{0.000000}%
\pgfsetdash{}{0pt}%
\pgfpathmoveto{\pgfqpoint{1.510559in}{0.521603in}}%
\pgfpathlineto{\pgfqpoint{1.614046in}{0.521603in}}%
\pgfpathlineto{\pgfqpoint{1.614046in}{0.668062in}}%
\pgfpathlineto{\pgfqpoint{1.510559in}{0.668062in}}%
\pgfpathlineto{\pgfqpoint{1.510559in}{0.521603in}}%
\pgfpathclose%
\pgfusepath{fill}%
\end{pgfscope}%
\begin{pgfscope}%
\pgfpathrectangle{\pgfqpoint{0.755102in}{0.521603in}}{\pgfqpoint{2.959734in}{1.817415in}}%
\pgfusepath{clip}%
\pgfsetbuttcap%
\pgfsetmiterjoin%
\definecolor{currentfill}{rgb}{0.121569,0.466667,0.705882}%
\pgfsetfillcolor{currentfill}%
\pgfsetlinewidth{0.000000pt}%
\definecolor{currentstroke}{rgb}{0.000000,0.000000,0.000000}%
\pgfsetstrokecolor{currentstroke}%
\pgfsetstrokeopacity{0.000000}%
\pgfsetdash{}{0pt}%
\pgfpathmoveto{\pgfqpoint{1.614046in}{0.521603in}}%
\pgfpathlineto{\pgfqpoint{1.717533in}{0.521603in}}%
\pgfpathlineto{\pgfqpoint{1.717533in}{0.609478in}}%
\pgfpathlineto{\pgfqpoint{1.614046in}{0.609478in}}%
\pgfpathlineto{\pgfqpoint{1.614046in}{0.521603in}}%
\pgfpathclose%
\pgfusepath{fill}%
\end{pgfscope}%
\begin{pgfscope}%
\pgfpathrectangle{\pgfqpoint{0.755102in}{0.521603in}}{\pgfqpoint{2.959734in}{1.817415in}}%
\pgfusepath{clip}%
\pgfsetbuttcap%
\pgfsetmiterjoin%
\definecolor{currentfill}{rgb}{0.121569,0.466667,0.705882}%
\pgfsetfillcolor{currentfill}%
\pgfsetlinewidth{0.000000pt}%
\definecolor{currentstroke}{rgb}{0.000000,0.000000,0.000000}%
\pgfsetstrokecolor{currentstroke}%
\pgfsetstrokeopacity{0.000000}%
\pgfsetdash{}{0pt}%
\pgfpathmoveto{\pgfqpoint{1.717533in}{0.521603in}}%
\pgfpathlineto{\pgfqpoint{1.821020in}{0.521603in}}%
\pgfpathlineto{\pgfqpoint{1.821020in}{0.598827in}}%
\pgfpathlineto{\pgfqpoint{1.717533in}{0.598827in}}%
\pgfpathlineto{\pgfqpoint{1.717533in}{0.521603in}}%
\pgfpathclose%
\pgfusepath{fill}%
\end{pgfscope}%
\begin{pgfscope}%
\pgfpathrectangle{\pgfqpoint{0.755102in}{0.521603in}}{\pgfqpoint{2.959734in}{1.817415in}}%
\pgfusepath{clip}%
\pgfsetbuttcap%
\pgfsetmiterjoin%
\definecolor{currentfill}{rgb}{0.121569,0.466667,0.705882}%
\pgfsetfillcolor{currentfill}%
\pgfsetlinewidth{0.000000pt}%
\definecolor{currentstroke}{rgb}{0.000000,0.000000,0.000000}%
\pgfsetstrokecolor{currentstroke}%
\pgfsetstrokeopacity{0.000000}%
\pgfsetdash{}{0pt}%
\pgfpathmoveto{\pgfqpoint{1.821020in}{0.521603in}}%
\pgfpathlineto{\pgfqpoint{1.924507in}{0.521603in}}%
\pgfpathlineto{\pgfqpoint{1.924507in}{0.574861in}}%
\pgfpathlineto{\pgfqpoint{1.821020in}{0.574861in}}%
\pgfpathlineto{\pgfqpoint{1.821020in}{0.521603in}}%
\pgfpathclose%
\pgfusepath{fill}%
\end{pgfscope}%
\begin{pgfscope}%
\pgfpathrectangle{\pgfqpoint{0.755102in}{0.521603in}}{\pgfqpoint{2.959734in}{1.817415in}}%
\pgfusepath{clip}%
\pgfsetbuttcap%
\pgfsetmiterjoin%
\definecolor{currentfill}{rgb}{0.121569,0.466667,0.705882}%
\pgfsetfillcolor{currentfill}%
\pgfsetlinewidth{0.000000pt}%
\definecolor{currentstroke}{rgb}{0.000000,0.000000,0.000000}%
\pgfsetstrokecolor{currentstroke}%
\pgfsetstrokeopacity{0.000000}%
\pgfsetdash{}{0pt}%
\pgfpathmoveto{\pgfqpoint{1.924507in}{0.521603in}}%
\pgfpathlineto{\pgfqpoint{2.027995in}{0.521603in}}%
\pgfpathlineto{\pgfqpoint{2.027995in}{0.585512in}}%
\pgfpathlineto{\pgfqpoint{1.924507in}{0.585512in}}%
\pgfpathlineto{\pgfqpoint{1.924507in}{0.521603in}}%
\pgfpathclose%
\pgfusepath{fill}%
\end{pgfscope}%
\begin{pgfscope}%
\pgfpathrectangle{\pgfqpoint{0.755102in}{0.521603in}}{\pgfqpoint{2.959734in}{1.817415in}}%
\pgfusepath{clip}%
\pgfsetbuttcap%
\pgfsetmiterjoin%
\definecolor{currentfill}{rgb}{0.121569,0.466667,0.705882}%
\pgfsetfillcolor{currentfill}%
\pgfsetlinewidth{0.000000pt}%
\definecolor{currentstroke}{rgb}{0.000000,0.000000,0.000000}%
\pgfsetstrokecolor{currentstroke}%
\pgfsetstrokeopacity{0.000000}%
\pgfsetdash{}{0pt}%
\pgfpathmoveto{\pgfqpoint{2.027995in}{0.521603in}}%
\pgfpathlineto{\pgfqpoint{2.131482in}{0.521603in}}%
\pgfpathlineto{\pgfqpoint{2.131482in}{0.558884in}}%
\pgfpathlineto{\pgfqpoint{2.027995in}{0.558884in}}%
\pgfpathlineto{\pgfqpoint{2.027995in}{0.521603in}}%
\pgfpathclose%
\pgfusepath{fill}%
\end{pgfscope}%
\begin{pgfscope}%
\pgfpathrectangle{\pgfqpoint{0.755102in}{0.521603in}}{\pgfqpoint{2.959734in}{1.817415in}}%
\pgfusepath{clip}%
\pgfsetbuttcap%
\pgfsetmiterjoin%
\definecolor{currentfill}{rgb}{0.121569,0.466667,0.705882}%
\pgfsetfillcolor{currentfill}%
\pgfsetlinewidth{0.000000pt}%
\definecolor{currentstroke}{rgb}{0.000000,0.000000,0.000000}%
\pgfsetstrokecolor{currentstroke}%
\pgfsetstrokeopacity{0.000000}%
\pgfsetdash{}{0pt}%
\pgfpathmoveto{\pgfqpoint{2.131482in}{0.521603in}}%
\pgfpathlineto{\pgfqpoint{2.234969in}{0.521603in}}%
\pgfpathlineto{\pgfqpoint{2.234969in}{0.548232in}}%
\pgfpathlineto{\pgfqpoint{2.131482in}{0.548232in}}%
\pgfpathlineto{\pgfqpoint{2.131482in}{0.521603in}}%
\pgfpathclose%
\pgfusepath{fill}%
\end{pgfscope}%
\begin{pgfscope}%
\pgfpathrectangle{\pgfqpoint{0.755102in}{0.521603in}}{\pgfqpoint{2.959734in}{1.817415in}}%
\pgfusepath{clip}%
\pgfsetbuttcap%
\pgfsetmiterjoin%
\definecolor{currentfill}{rgb}{0.121569,0.466667,0.705882}%
\pgfsetfillcolor{currentfill}%
\pgfsetlinewidth{0.000000pt}%
\definecolor{currentstroke}{rgb}{0.000000,0.000000,0.000000}%
\pgfsetstrokecolor{currentstroke}%
\pgfsetstrokeopacity{0.000000}%
\pgfsetdash{}{0pt}%
\pgfpathmoveto{\pgfqpoint{2.234969in}{0.521603in}}%
\pgfpathlineto{\pgfqpoint{2.338456in}{0.521603in}}%
\pgfpathlineto{\pgfqpoint{2.338456in}{0.545569in}}%
\pgfpathlineto{\pgfqpoint{2.234969in}{0.545569in}}%
\pgfpathlineto{\pgfqpoint{2.234969in}{0.521603in}}%
\pgfpathclose%
\pgfusepath{fill}%
\end{pgfscope}%
\begin{pgfscope}%
\pgfpathrectangle{\pgfqpoint{0.755102in}{0.521603in}}{\pgfqpoint{2.959734in}{1.817415in}}%
\pgfusepath{clip}%
\pgfsetbuttcap%
\pgfsetmiterjoin%
\definecolor{currentfill}{rgb}{0.121569,0.466667,0.705882}%
\pgfsetfillcolor{currentfill}%
\pgfsetlinewidth{0.000000pt}%
\definecolor{currentstroke}{rgb}{0.000000,0.000000,0.000000}%
\pgfsetstrokecolor{currentstroke}%
\pgfsetstrokeopacity{0.000000}%
\pgfsetdash{}{0pt}%
\pgfpathmoveto{\pgfqpoint{2.338456in}{0.521603in}}%
\pgfpathlineto{\pgfqpoint{2.441944in}{0.521603in}}%
\pgfpathlineto{\pgfqpoint{2.441944in}{0.540243in}}%
\pgfpathlineto{\pgfqpoint{2.338456in}{0.540243in}}%
\pgfpathlineto{\pgfqpoint{2.338456in}{0.521603in}}%
\pgfpathclose%
\pgfusepath{fill}%
\end{pgfscope}%
\begin{pgfscope}%
\pgfpathrectangle{\pgfqpoint{0.755102in}{0.521603in}}{\pgfqpoint{2.959734in}{1.817415in}}%
\pgfusepath{clip}%
\pgfsetbuttcap%
\pgfsetmiterjoin%
\definecolor{currentfill}{rgb}{0.121569,0.466667,0.705882}%
\pgfsetfillcolor{currentfill}%
\pgfsetlinewidth{0.000000pt}%
\definecolor{currentstroke}{rgb}{0.000000,0.000000,0.000000}%
\pgfsetstrokecolor{currentstroke}%
\pgfsetstrokeopacity{0.000000}%
\pgfsetdash{}{0pt}%
\pgfpathmoveto{\pgfqpoint{2.441944in}{0.521603in}}%
\pgfpathlineto{\pgfqpoint{2.545431in}{0.521603in}}%
\pgfpathlineto{\pgfqpoint{2.545431in}{0.537581in}}%
\pgfpathlineto{\pgfqpoint{2.441944in}{0.537581in}}%
\pgfpathlineto{\pgfqpoint{2.441944in}{0.521603in}}%
\pgfpathclose%
\pgfusepath{fill}%
\end{pgfscope}%
\begin{pgfscope}%
\pgfpathrectangle{\pgfqpoint{0.755102in}{0.521603in}}{\pgfqpoint{2.959734in}{1.817415in}}%
\pgfusepath{clip}%
\pgfsetbuttcap%
\pgfsetmiterjoin%
\definecolor{currentfill}{rgb}{0.121569,0.466667,0.705882}%
\pgfsetfillcolor{currentfill}%
\pgfsetlinewidth{0.000000pt}%
\definecolor{currentstroke}{rgb}{0.000000,0.000000,0.000000}%
\pgfsetstrokecolor{currentstroke}%
\pgfsetstrokeopacity{0.000000}%
\pgfsetdash{}{0pt}%
\pgfpathmoveto{\pgfqpoint{2.545431in}{0.521603in}}%
\pgfpathlineto{\pgfqpoint{2.648918in}{0.521603in}}%
\pgfpathlineto{\pgfqpoint{2.648918in}{0.529592in}}%
\pgfpathlineto{\pgfqpoint{2.545431in}{0.529592in}}%
\pgfpathlineto{\pgfqpoint{2.545431in}{0.521603in}}%
\pgfpathclose%
\pgfusepath{fill}%
\end{pgfscope}%
\begin{pgfscope}%
\pgfpathrectangle{\pgfqpoint{0.755102in}{0.521603in}}{\pgfqpoint{2.959734in}{1.817415in}}%
\pgfusepath{clip}%
\pgfsetbuttcap%
\pgfsetmiterjoin%
\definecolor{currentfill}{rgb}{0.121569,0.466667,0.705882}%
\pgfsetfillcolor{currentfill}%
\pgfsetlinewidth{0.000000pt}%
\definecolor{currentstroke}{rgb}{0.000000,0.000000,0.000000}%
\pgfsetstrokecolor{currentstroke}%
\pgfsetstrokeopacity{0.000000}%
\pgfsetdash{}{0pt}%
\pgfpathmoveto{\pgfqpoint{2.648918in}{0.521603in}}%
\pgfpathlineto{\pgfqpoint{2.752405in}{0.521603in}}%
\pgfpathlineto{\pgfqpoint{2.752405in}{0.526929in}}%
\pgfpathlineto{\pgfqpoint{2.648918in}{0.526929in}}%
\pgfpathlineto{\pgfqpoint{2.648918in}{0.521603in}}%
\pgfpathclose%
\pgfusepath{fill}%
\end{pgfscope}%
\begin{pgfscope}%
\pgfpathrectangle{\pgfqpoint{0.755102in}{0.521603in}}{\pgfqpoint{2.959734in}{1.817415in}}%
\pgfusepath{clip}%
\pgfsetbuttcap%
\pgfsetmiterjoin%
\definecolor{currentfill}{rgb}{0.121569,0.466667,0.705882}%
\pgfsetfillcolor{currentfill}%
\pgfsetlinewidth{0.000000pt}%
\definecolor{currentstroke}{rgb}{0.000000,0.000000,0.000000}%
\pgfsetstrokecolor{currentstroke}%
\pgfsetstrokeopacity{0.000000}%
\pgfsetdash{}{0pt}%
\pgfpathmoveto{\pgfqpoint{2.752405in}{0.521603in}}%
\pgfpathlineto{\pgfqpoint{2.855892in}{0.521603in}}%
\pgfpathlineto{\pgfqpoint{2.855892in}{0.524266in}}%
\pgfpathlineto{\pgfqpoint{2.752405in}{0.524266in}}%
\pgfpathlineto{\pgfqpoint{2.752405in}{0.521603in}}%
\pgfpathclose%
\pgfusepath{fill}%
\end{pgfscope}%
\begin{pgfscope}%
\pgfpathrectangle{\pgfqpoint{0.755102in}{0.521603in}}{\pgfqpoint{2.959734in}{1.817415in}}%
\pgfusepath{clip}%
\pgfsetbuttcap%
\pgfsetmiterjoin%
\definecolor{currentfill}{rgb}{0.121569,0.466667,0.705882}%
\pgfsetfillcolor{currentfill}%
\pgfsetlinewidth{0.000000pt}%
\definecolor{currentstroke}{rgb}{0.000000,0.000000,0.000000}%
\pgfsetstrokecolor{currentstroke}%
\pgfsetstrokeopacity{0.000000}%
\pgfsetdash{}{0pt}%
\pgfpathmoveto{\pgfqpoint{2.855892in}{0.521603in}}%
\pgfpathlineto{\pgfqpoint{2.959380in}{0.521603in}}%
\pgfpathlineto{\pgfqpoint{2.959380in}{0.524266in}}%
\pgfpathlineto{\pgfqpoint{2.855892in}{0.524266in}}%
\pgfpathlineto{\pgfqpoint{2.855892in}{0.521603in}}%
\pgfpathclose%
\pgfusepath{fill}%
\end{pgfscope}%
\begin{pgfscope}%
\pgfpathrectangle{\pgfqpoint{0.755102in}{0.521603in}}{\pgfqpoint{2.959734in}{1.817415in}}%
\pgfusepath{clip}%
\pgfsetbuttcap%
\pgfsetmiterjoin%
\definecolor{currentfill}{rgb}{0.121569,0.466667,0.705882}%
\pgfsetfillcolor{currentfill}%
\pgfsetlinewidth{0.000000pt}%
\definecolor{currentstroke}{rgb}{0.000000,0.000000,0.000000}%
\pgfsetstrokecolor{currentstroke}%
\pgfsetstrokeopacity{0.000000}%
\pgfsetdash{}{0pt}%
\pgfpathmoveto{\pgfqpoint{2.959380in}{0.521603in}}%
\pgfpathlineto{\pgfqpoint{3.062867in}{0.521603in}}%
\pgfpathlineto{\pgfqpoint{3.062867in}{0.524266in}}%
\pgfpathlineto{\pgfqpoint{2.959380in}{0.524266in}}%
\pgfpathlineto{\pgfqpoint{2.959380in}{0.521603in}}%
\pgfpathclose%
\pgfusepath{fill}%
\end{pgfscope}%
\begin{pgfscope}%
\pgfpathrectangle{\pgfqpoint{0.755102in}{0.521603in}}{\pgfqpoint{2.959734in}{1.817415in}}%
\pgfusepath{clip}%
\pgfsetbuttcap%
\pgfsetmiterjoin%
\definecolor{currentfill}{rgb}{0.121569,0.466667,0.705882}%
\pgfsetfillcolor{currentfill}%
\pgfsetlinewidth{0.000000pt}%
\definecolor{currentstroke}{rgb}{0.000000,0.000000,0.000000}%
\pgfsetstrokecolor{currentstroke}%
\pgfsetstrokeopacity{0.000000}%
\pgfsetdash{}{0pt}%
\pgfpathmoveto{\pgfqpoint{3.062867in}{0.521603in}}%
\pgfpathlineto{\pgfqpoint{3.166354in}{0.521603in}}%
\pgfpathlineto{\pgfqpoint{3.166354in}{0.521603in}}%
\pgfpathlineto{\pgfqpoint{3.062867in}{0.521603in}}%
\pgfpathlineto{\pgfqpoint{3.062867in}{0.521603in}}%
\pgfpathclose%
\pgfusepath{fill}%
\end{pgfscope}%
\begin{pgfscope}%
\pgfpathrectangle{\pgfqpoint{0.755102in}{0.521603in}}{\pgfqpoint{2.959734in}{1.817415in}}%
\pgfusepath{clip}%
\pgfsetbuttcap%
\pgfsetmiterjoin%
\definecolor{currentfill}{rgb}{0.121569,0.466667,0.705882}%
\pgfsetfillcolor{currentfill}%
\pgfsetlinewidth{0.000000pt}%
\definecolor{currentstroke}{rgb}{0.000000,0.000000,0.000000}%
\pgfsetstrokecolor{currentstroke}%
\pgfsetstrokeopacity{0.000000}%
\pgfsetdash{}{0pt}%
\pgfpathmoveto{\pgfqpoint{3.166354in}{0.521603in}}%
\pgfpathlineto{\pgfqpoint{3.269841in}{0.521603in}}%
\pgfpathlineto{\pgfqpoint{3.269841in}{0.524266in}}%
\pgfpathlineto{\pgfqpoint{3.166354in}{0.524266in}}%
\pgfpathlineto{\pgfqpoint{3.166354in}{0.521603in}}%
\pgfpathclose%
\pgfusepath{fill}%
\end{pgfscope}%
\begin{pgfscope}%
\pgfpathrectangle{\pgfqpoint{0.755102in}{0.521603in}}{\pgfqpoint{2.959734in}{1.817415in}}%
\pgfusepath{clip}%
\pgfsetbuttcap%
\pgfsetmiterjoin%
\definecolor{currentfill}{rgb}{0.121569,0.466667,0.705882}%
\pgfsetfillcolor{currentfill}%
\pgfsetlinewidth{0.000000pt}%
\definecolor{currentstroke}{rgb}{0.000000,0.000000,0.000000}%
\pgfsetstrokecolor{currentstroke}%
\pgfsetstrokeopacity{0.000000}%
\pgfsetdash{}{0pt}%
\pgfpathmoveto{\pgfqpoint{3.269841in}{0.521603in}}%
\pgfpathlineto{\pgfqpoint{3.373328in}{0.521603in}}%
\pgfpathlineto{\pgfqpoint{3.373328in}{0.521603in}}%
\pgfpathlineto{\pgfqpoint{3.269841in}{0.521603in}}%
\pgfpathlineto{\pgfqpoint{3.269841in}{0.521603in}}%
\pgfpathclose%
\pgfusepath{fill}%
\end{pgfscope}%
\begin{pgfscope}%
\pgfpathrectangle{\pgfqpoint{0.755102in}{0.521603in}}{\pgfqpoint{2.959734in}{1.817415in}}%
\pgfusepath{clip}%
\pgfsetbuttcap%
\pgfsetmiterjoin%
\definecolor{currentfill}{rgb}{0.121569,0.466667,0.705882}%
\pgfsetfillcolor{currentfill}%
\pgfsetlinewidth{0.000000pt}%
\definecolor{currentstroke}{rgb}{0.000000,0.000000,0.000000}%
\pgfsetstrokecolor{currentstroke}%
\pgfsetstrokeopacity{0.000000}%
\pgfsetdash{}{0pt}%
\pgfpathmoveto{\pgfqpoint{3.373328in}{0.521603in}}%
\pgfpathlineto{\pgfqpoint{3.476816in}{0.521603in}}%
\pgfpathlineto{\pgfqpoint{3.476816in}{0.524266in}}%
\pgfpathlineto{\pgfqpoint{3.373328in}{0.524266in}}%
\pgfpathlineto{\pgfqpoint{3.373328in}{0.521603in}}%
\pgfpathclose%
\pgfusepath{fill}%
\end{pgfscope}%
\begin{pgfscope}%
\pgfpathrectangle{\pgfqpoint{0.755102in}{0.521603in}}{\pgfqpoint{2.959734in}{1.817415in}}%
\pgfusepath{clip}%
\pgfsetbuttcap%
\pgfsetmiterjoin%
\definecolor{currentfill}{rgb}{0.121569,0.466667,0.705882}%
\pgfsetfillcolor{currentfill}%
\pgfsetlinewidth{0.000000pt}%
\definecolor{currentstroke}{rgb}{0.000000,0.000000,0.000000}%
\pgfsetstrokecolor{currentstroke}%
\pgfsetstrokeopacity{0.000000}%
\pgfsetdash{}{0pt}%
\pgfpathmoveto{\pgfqpoint{3.476816in}{0.521603in}}%
\pgfpathlineto{\pgfqpoint{3.580303in}{0.521603in}}%
\pgfpathlineto{\pgfqpoint{3.580303in}{0.526929in}}%
\pgfpathlineto{\pgfqpoint{3.476816in}{0.526929in}}%
\pgfpathlineto{\pgfqpoint{3.476816in}{0.521603in}}%
\pgfpathclose%
\pgfusepath{fill}%
\end{pgfscope}%
\begin{pgfscope}%
\pgfsetbuttcap%
\pgfsetroundjoin%
\definecolor{currentfill}{rgb}{0.000000,0.000000,0.000000}%
\pgfsetfillcolor{currentfill}%
\pgfsetlinewidth{0.803000pt}%
\definecolor{currentstroke}{rgb}{0.000000,0.000000,0.000000}%
\pgfsetstrokecolor{currentstroke}%
\pgfsetdash{}{0pt}%
\pgfsys@defobject{currentmarker}{\pgfqpoint{0.000000in}{-0.048611in}}{\pgfqpoint{0.000000in}{0.000000in}}{%
\pgfpathmoveto{\pgfqpoint{0.000000in}{0.000000in}}%
\pgfpathlineto{\pgfqpoint{0.000000in}{-0.048611in}}%
\pgfusepath{stroke,fill}%
}%
\begin{pgfscope}%
\pgfsys@transformshift{0.877312in}{0.521603in}%
\pgfsys@useobject{currentmarker}{}%
\end{pgfscope}%
\end{pgfscope}%
\begin{pgfscope}%
\definecolor{textcolor}{rgb}{0.000000,0.000000,0.000000}%
\pgfsetstrokecolor{textcolor}%
\pgfsetfillcolor{textcolor}%
\pgftext[x=0.877312in,y=0.424381in,,top]{\color{textcolor}\sffamily\fontsize{10.000000}{12.000000}\selectfont 0.00}%
\end{pgfscope}%
\begin{pgfscope}%
\pgfsetbuttcap%
\pgfsetroundjoin%
\definecolor{currentfill}{rgb}{0.000000,0.000000,0.000000}%
\pgfsetfillcolor{currentfill}%
\pgfsetlinewidth{0.803000pt}%
\definecolor{currentstroke}{rgb}{0.000000,0.000000,0.000000}%
\pgfsetstrokecolor{currentstroke}%
\pgfsetdash{}{0pt}%
\pgfsys@defobject{currentmarker}{\pgfqpoint{0.000000in}{-0.048611in}}{\pgfqpoint{0.000000in}{0.000000in}}{%
\pgfpathmoveto{\pgfqpoint{0.000000in}{0.000000in}}%
\pgfpathlineto{\pgfqpoint{0.000000in}{-0.048611in}}%
\pgfusepath{stroke,fill}%
}%
\begin{pgfscope}%
\pgfsys@transformshift{1.709563in}{0.521603in}%
\pgfsys@useobject{currentmarker}{}%
\end{pgfscope}%
\end{pgfscope}%
\begin{pgfscope}%
\definecolor{textcolor}{rgb}{0.000000,0.000000,0.000000}%
\pgfsetstrokecolor{textcolor}%
\pgfsetfillcolor{textcolor}%
\pgftext[x=1.709563in,y=0.424381in,,top]{\color{textcolor}\sffamily\fontsize{10.000000}{12.000000}\selectfont 0.02}%
\end{pgfscope}%
\begin{pgfscope}%
\pgfsetbuttcap%
\pgfsetroundjoin%
\definecolor{currentfill}{rgb}{0.000000,0.000000,0.000000}%
\pgfsetfillcolor{currentfill}%
\pgfsetlinewidth{0.803000pt}%
\definecolor{currentstroke}{rgb}{0.000000,0.000000,0.000000}%
\pgfsetstrokecolor{currentstroke}%
\pgfsetdash{}{0pt}%
\pgfsys@defobject{currentmarker}{\pgfqpoint{0.000000in}{-0.048611in}}{\pgfqpoint{0.000000in}{0.000000in}}{%
\pgfpathmoveto{\pgfqpoint{0.000000in}{0.000000in}}%
\pgfpathlineto{\pgfqpoint{0.000000in}{-0.048611in}}%
\pgfusepath{stroke,fill}%
}%
\begin{pgfscope}%
\pgfsys@transformshift{2.541814in}{0.521603in}%
\pgfsys@useobject{currentmarker}{}%
\end{pgfscope}%
\end{pgfscope}%
\begin{pgfscope}%
\definecolor{textcolor}{rgb}{0.000000,0.000000,0.000000}%
\pgfsetstrokecolor{textcolor}%
\pgfsetfillcolor{textcolor}%
\pgftext[x=2.541814in,y=0.424381in,,top]{\color{textcolor}\sffamily\fontsize{10.000000}{12.000000}\selectfont 0.04}%
\end{pgfscope}%
\begin{pgfscope}%
\pgfsetbuttcap%
\pgfsetroundjoin%
\definecolor{currentfill}{rgb}{0.000000,0.000000,0.000000}%
\pgfsetfillcolor{currentfill}%
\pgfsetlinewidth{0.803000pt}%
\definecolor{currentstroke}{rgb}{0.000000,0.000000,0.000000}%
\pgfsetstrokecolor{currentstroke}%
\pgfsetdash{}{0pt}%
\pgfsys@defobject{currentmarker}{\pgfqpoint{0.000000in}{-0.048611in}}{\pgfqpoint{0.000000in}{0.000000in}}{%
\pgfpathmoveto{\pgfqpoint{0.000000in}{0.000000in}}%
\pgfpathlineto{\pgfqpoint{0.000000in}{-0.048611in}}%
\pgfusepath{stroke,fill}%
}%
\begin{pgfscope}%
\pgfsys@transformshift{3.374065in}{0.521603in}%
\pgfsys@useobject{currentmarker}{}%
\end{pgfscope}%
\end{pgfscope}%
\begin{pgfscope}%
\definecolor{textcolor}{rgb}{0.000000,0.000000,0.000000}%
\pgfsetstrokecolor{textcolor}%
\pgfsetfillcolor{textcolor}%
\pgftext[x=3.374065in,y=0.424381in,,top]{\color{textcolor}\sffamily\fontsize{10.000000}{12.000000}\selectfont 0.06}%
\end{pgfscope}%
\begin{pgfscope}%
\definecolor{textcolor}{rgb}{0.000000,0.000000,0.000000}%
\pgfsetstrokecolor{textcolor}%
\pgfsetfillcolor{textcolor}%
\pgftext[x=2.234969in,y=0.234413in,,top]{\color{textcolor}\sffamily\fontsize{10.000000}{12.000000}\selectfont error}%
\end{pgfscope}%
\begin{pgfscope}%
\pgfsetbuttcap%
\pgfsetroundjoin%
\definecolor{currentfill}{rgb}{0.000000,0.000000,0.000000}%
\pgfsetfillcolor{currentfill}%
\pgfsetlinewidth{0.803000pt}%
\definecolor{currentstroke}{rgb}{0.000000,0.000000,0.000000}%
\pgfsetstrokecolor{currentstroke}%
\pgfsetdash{}{0pt}%
\pgfsys@defobject{currentmarker}{\pgfqpoint{-0.048611in}{0.000000in}}{\pgfqpoint{-0.000000in}{0.000000in}}{%
\pgfpathmoveto{\pgfqpoint{-0.000000in}{0.000000in}}%
\pgfpathlineto{\pgfqpoint{-0.048611in}{0.000000in}}%
\pgfusepath{stroke,fill}%
}%
\begin{pgfscope}%
\pgfsys@transformshift{0.755102in}{0.521603in}%
\pgfsys@useobject{currentmarker}{}%
\end{pgfscope}%
\end{pgfscope}%
\begin{pgfscope}%
\definecolor{textcolor}{rgb}{0.000000,0.000000,0.000000}%
\pgfsetstrokecolor{textcolor}%
\pgfsetfillcolor{textcolor}%
\pgftext[x=0.569514in, y=0.468842in, left, base]{\color{textcolor}\sffamily\fontsize{10.000000}{12.000000}\selectfont 0}%
\end{pgfscope}%
\begin{pgfscope}%
\pgfsetbuttcap%
\pgfsetroundjoin%
\definecolor{currentfill}{rgb}{0.000000,0.000000,0.000000}%
\pgfsetfillcolor{currentfill}%
\pgfsetlinewidth{0.803000pt}%
\definecolor{currentstroke}{rgb}{0.000000,0.000000,0.000000}%
\pgfsetstrokecolor{currentstroke}%
\pgfsetdash{}{0pt}%
\pgfsys@defobject{currentmarker}{\pgfqpoint{-0.048611in}{0.000000in}}{\pgfqpoint{-0.000000in}{0.000000in}}{%
\pgfpathmoveto{\pgfqpoint{-0.000000in}{0.000000in}}%
\pgfpathlineto{\pgfqpoint{-0.048611in}{0.000000in}}%
\pgfusepath{stroke,fill}%
}%
\begin{pgfscope}%
\pgfsys@transformshift{0.755102in}{1.054179in}%
\pgfsys@useobject{currentmarker}{}%
\end{pgfscope}%
\end{pgfscope}%
\begin{pgfscope}%
\definecolor{textcolor}{rgb}{0.000000,0.000000,0.000000}%
\pgfsetstrokecolor{textcolor}%
\pgfsetfillcolor{textcolor}%
\pgftext[x=0.392784in, y=1.001418in, left, base]{\color{textcolor}\sffamily\fontsize{10.000000}{12.000000}\selectfont 200}%
\end{pgfscope}%
\begin{pgfscope}%
\pgfsetbuttcap%
\pgfsetroundjoin%
\definecolor{currentfill}{rgb}{0.000000,0.000000,0.000000}%
\pgfsetfillcolor{currentfill}%
\pgfsetlinewidth{0.803000pt}%
\definecolor{currentstroke}{rgb}{0.000000,0.000000,0.000000}%
\pgfsetstrokecolor{currentstroke}%
\pgfsetdash{}{0pt}%
\pgfsys@defobject{currentmarker}{\pgfqpoint{-0.048611in}{0.000000in}}{\pgfqpoint{-0.000000in}{0.000000in}}{%
\pgfpathmoveto{\pgfqpoint{-0.000000in}{0.000000in}}%
\pgfpathlineto{\pgfqpoint{-0.048611in}{0.000000in}}%
\pgfusepath{stroke,fill}%
}%
\begin{pgfscope}%
\pgfsys@transformshift{0.755102in}{1.586755in}%
\pgfsys@useobject{currentmarker}{}%
\end{pgfscope}%
\end{pgfscope}%
\begin{pgfscope}%
\definecolor{textcolor}{rgb}{0.000000,0.000000,0.000000}%
\pgfsetstrokecolor{textcolor}%
\pgfsetfillcolor{textcolor}%
\pgftext[x=0.392784in, y=1.533993in, left, base]{\color{textcolor}\sffamily\fontsize{10.000000}{12.000000}\selectfont 400}%
\end{pgfscope}%
\begin{pgfscope}%
\pgfsetbuttcap%
\pgfsetroundjoin%
\definecolor{currentfill}{rgb}{0.000000,0.000000,0.000000}%
\pgfsetfillcolor{currentfill}%
\pgfsetlinewidth{0.803000pt}%
\definecolor{currentstroke}{rgb}{0.000000,0.000000,0.000000}%
\pgfsetstrokecolor{currentstroke}%
\pgfsetdash{}{0pt}%
\pgfsys@defobject{currentmarker}{\pgfqpoint{-0.048611in}{0.000000in}}{\pgfqpoint{-0.000000in}{0.000000in}}{%
\pgfpathmoveto{\pgfqpoint{-0.000000in}{0.000000in}}%
\pgfpathlineto{\pgfqpoint{-0.048611in}{0.000000in}}%
\pgfusepath{stroke,fill}%
}%
\begin{pgfscope}%
\pgfsys@transformshift{0.755102in}{2.119331in}%
\pgfsys@useobject{currentmarker}{}%
\end{pgfscope}%
\end{pgfscope}%
\begin{pgfscope}%
\definecolor{textcolor}{rgb}{0.000000,0.000000,0.000000}%
\pgfsetstrokecolor{textcolor}%
\pgfsetfillcolor{textcolor}%
\pgftext[x=0.392784in, y=2.066569in, left, base]{\color{textcolor}\sffamily\fontsize{10.000000}{12.000000}\selectfont 600}%
\end{pgfscope}%
\begin{pgfscope}%
\definecolor{textcolor}{rgb}{0.000000,0.000000,0.000000}%
\pgfsetstrokecolor{textcolor}%
\pgfsetfillcolor{textcolor}%
\pgftext[x=0.337228in,y=1.430311in,,bottom,rotate=90.000000]{\color{textcolor}\sffamily\fontsize{10.000000}{12.000000}\selectfont count}%
\end{pgfscope}%
\begin{pgfscope}%
\pgfsetrectcap%
\pgfsetmiterjoin%
\pgfsetlinewidth{0.803000pt}%
\definecolor{currentstroke}{rgb}{0.000000,0.000000,0.000000}%
\pgfsetstrokecolor{currentstroke}%
\pgfsetdash{}{0pt}%
\pgfpathmoveto{\pgfqpoint{0.755102in}{0.521603in}}%
\pgfpathlineto{\pgfqpoint{0.755102in}{2.339018in}}%
\pgfusepath{stroke}%
\end{pgfscope}%
\begin{pgfscope}%
\pgfsetrectcap%
\pgfsetmiterjoin%
\pgfsetlinewidth{0.803000pt}%
\definecolor{currentstroke}{rgb}{0.000000,0.000000,0.000000}%
\pgfsetstrokecolor{currentstroke}%
\pgfsetdash{}{0pt}%
\pgfpathmoveto{\pgfqpoint{3.714836in}{0.521603in}}%
\pgfpathlineto{\pgfqpoint{3.714836in}{2.339018in}}%
\pgfusepath{stroke}%
\end{pgfscope}%
\begin{pgfscope}%
\pgfsetrectcap%
\pgfsetmiterjoin%
\pgfsetlinewidth{0.803000pt}%
\definecolor{currentstroke}{rgb}{0.000000,0.000000,0.000000}%
\pgfsetstrokecolor{currentstroke}%
\pgfsetdash{}{0pt}%
\pgfpathmoveto{\pgfqpoint{0.755102in}{0.521603in}}%
\pgfpathlineto{\pgfqpoint{3.714836in}{0.521603in}}%
\pgfusepath{stroke}%
\end{pgfscope}%
\begin{pgfscope}%
\pgfsetrectcap%
\pgfsetmiterjoin%
\pgfsetlinewidth{0.803000pt}%
\definecolor{currentstroke}{rgb}{0.000000,0.000000,0.000000}%
\pgfsetstrokecolor{currentstroke}%
\pgfsetdash{}{0pt}%
\pgfpathmoveto{\pgfqpoint{0.755102in}{2.339018in}}%
\pgfpathlineto{\pgfqpoint{3.714836in}{2.339018in}}%
\pgfusepath{stroke}%
\end{pgfscope}%
\begin{pgfscope}%
\definecolor{textcolor}{rgb}{0.000000,0.000000,0.000000}%
\pgfsetstrokecolor{textcolor}%
\pgfsetfillcolor{textcolor}%
\pgftext[x=2.234969in,y=2.422352in,,base]{\color{textcolor}\sffamily\fontsize{12.000000}{14.400000}\selectfont MSE of the magnetic susceptibility from a 1D chain}%
\end{pgfscope}%
\end{pgfpicture}%
\makeatother%
\endgroup%

	\caption{Распределение значений среднеквадратичного отклонения магнитной восприимчивости конформаций от одномерной цепочки длины 1000.}
	\label{fig:MS_1D_dif_distr}
\end{figure}

При рассмотрении пиков магнитной восприимчивости, распределение которых представлено на рис. \ref{fig:MS_peaks_distr}, видно что у большинства конформаций (больше 90\%) пик отсутствует и магнитная восприимчивость достигает максимума при $\beta = 1$. По данному графику можно предположить, что при увеличении длины конформаций так же увеличивается доля конформаций с пиком в $\beta = 1$, но особенность при $L=500$ не позволяет утверждать что-то однозначно. Для подтверждения требуется повторить замеры с большим количеством конформаций


\begin{figure}[ht]
	\centering
	%% Creator: Matplotlib, PGF backend
%%
%% To include the figure in your LaTeX document, write
%%   \input{<filename>.pgf}
%%
%% Make sure the required packages are loaded in your preamble
%%   \usepackage{pgf}
%%
%% Also ensure that all the required font packages are loaded; for instance,
%% the lmodern package is sometimes necessary when using math font.
%%   \usepackage{lmodern}
%%
%% Figures using additional raster images can only be included by \input if
%% they are in the same directory as the main LaTeX file. For loading figures
%% from other directories you can use the `import` package
%%   \usepackage{import}
%%
%% and then include the figures with
%%   \import{<path to file>}{<filename>.pgf}
%%
%% Matplotlib used the following preamble
%%   
%%   \usepackage{fontspec}
%%   \setmainfont{DejaVuSerif.ttf}[Path=\detokenize{/home/roman/anaconda3/envs/ising/lib/python3.8/site-packages/matplotlib/mpl-data/fonts/ttf/}]
%%   \setsansfont{DejaVuSans.ttf}[Path=\detokenize{/home/roman/anaconda3/envs/ising/lib/python3.8/site-packages/matplotlib/mpl-data/fonts/ttf/}]
%%   \setmonofont{DejaVuSansMono.ttf}[Path=\detokenize{/home/roman/anaconda3/envs/ising/lib/python3.8/site-packages/matplotlib/mpl-data/fonts/ttf/}]
%%   \makeatletter\@ifpackageloaded{underscore}{}{\usepackage[strings]{underscore}}\makeatother
%%
\begingroup%
\makeatletter%
\begin{pgfpicture}%
\pgfpathrectangle{\pgfpointorigin}{\pgfqpoint{3.702812in}{2.392613in}}%
\pgfusepath{use as bounding box, clip}%
\begin{pgfscope}%
\pgfsetbuttcap%
\pgfsetmiterjoin%
\definecolor{currentfill}{rgb}{1.000000,1.000000,1.000000}%
\pgfsetfillcolor{currentfill}%
\pgfsetlinewidth{0.000000pt}%
\definecolor{currentstroke}{rgb}{1.000000,1.000000,1.000000}%
\pgfsetstrokecolor{currentstroke}%
\pgfsetdash{}{0pt}%
\pgfpathmoveto{\pgfqpoint{0.000000in}{0.000000in}}%
\pgfpathlineto{\pgfqpoint{3.702812in}{0.000000in}}%
\pgfpathlineto{\pgfqpoint{3.702812in}{2.392613in}}%
\pgfpathlineto{\pgfqpoint{0.000000in}{2.392613in}}%
\pgfpathlineto{\pgfqpoint{0.000000in}{0.000000in}}%
\pgfpathclose%
\pgfusepath{fill}%
\end{pgfscope}%
\begin{pgfscope}%
\pgfsetbuttcap%
\pgfsetmiterjoin%
\definecolor{currentfill}{rgb}{1.000000,1.000000,1.000000}%
\pgfsetfillcolor{currentfill}%
\pgfsetlinewidth{0.000000pt}%
\definecolor{currentstroke}{rgb}{0.000000,0.000000,0.000000}%
\pgfsetstrokecolor{currentstroke}%
\pgfsetstrokeopacity{0.000000}%
\pgfsetdash{}{0pt}%
\pgfpathmoveto{\pgfqpoint{0.643077in}{0.467838in}}%
\pgfpathlineto{\pgfqpoint{3.602812in}{0.467838in}}%
\pgfpathlineto{\pgfqpoint{3.602812in}{2.285253in}}%
\pgfpathlineto{\pgfqpoint{0.643077in}{2.285253in}}%
\pgfpathlineto{\pgfqpoint{0.643077in}{0.467838in}}%
\pgfpathclose%
\pgfusepath{fill}%
\end{pgfscope}%
\begin{pgfscope}%
\pgfpathrectangle{\pgfqpoint{0.643077in}{0.467838in}}{\pgfqpoint{2.959734in}{1.817415in}}%
\pgfusepath{clip}%
\pgfsetbuttcap%
\pgfsetmiterjoin%
\definecolor{currentfill}{rgb}{0.121569,0.466667,0.705882}%
\pgfsetfillcolor{currentfill}%
\pgfsetlinewidth{0.000000pt}%
\definecolor{currentstroke}{rgb}{0.000000,0.000000,0.000000}%
\pgfsetstrokecolor{currentstroke}%
\pgfsetstrokeopacity{0.000000}%
\pgfsetdash{}{0pt}%
\pgfpathmoveto{\pgfqpoint{0.777611in}{0.467838in}}%
\pgfpathlineto{\pgfqpoint{0.819652in}{0.467838in}}%
\pgfpathlineto{\pgfqpoint{0.819652in}{0.471403in}}%
\pgfpathlineto{\pgfqpoint{0.777611in}{0.471403in}}%
\pgfpathlineto{\pgfqpoint{0.777611in}{0.467838in}}%
\pgfpathclose%
\pgfusepath{fill}%
\end{pgfscope}%
\begin{pgfscope}%
\pgfpathrectangle{\pgfqpoint{0.643077in}{0.467838in}}{\pgfqpoint{2.959734in}{1.817415in}}%
\pgfusepath{clip}%
\pgfsetbuttcap%
\pgfsetmiterjoin%
\definecolor{currentfill}{rgb}{0.121569,0.466667,0.705882}%
\pgfsetfillcolor{currentfill}%
\pgfsetlinewidth{0.000000pt}%
\definecolor{currentstroke}{rgb}{0.000000,0.000000,0.000000}%
\pgfsetstrokecolor{currentstroke}%
\pgfsetstrokeopacity{0.000000}%
\pgfsetdash{}{0pt}%
\pgfpathmoveto{\pgfqpoint{0.987819in}{0.467838in}}%
\pgfpathlineto{\pgfqpoint{1.029861in}{0.467838in}}%
\pgfpathlineto{\pgfqpoint{1.029861in}{0.467838in}}%
\pgfpathlineto{\pgfqpoint{0.987819in}{0.467838in}}%
\pgfpathlineto{\pgfqpoint{0.987819in}{0.467838in}}%
\pgfpathclose%
\pgfusepath{fill}%
\end{pgfscope}%
\begin{pgfscope}%
\pgfpathrectangle{\pgfqpoint{0.643077in}{0.467838in}}{\pgfqpoint{2.959734in}{1.817415in}}%
\pgfusepath{clip}%
\pgfsetbuttcap%
\pgfsetmiterjoin%
\definecolor{currentfill}{rgb}{0.121569,0.466667,0.705882}%
\pgfsetfillcolor{currentfill}%
\pgfsetlinewidth{0.000000pt}%
\definecolor{currentstroke}{rgb}{0.000000,0.000000,0.000000}%
\pgfsetstrokecolor{currentstroke}%
\pgfsetstrokeopacity{0.000000}%
\pgfsetdash{}{0pt}%
\pgfpathmoveto{\pgfqpoint{1.198027in}{0.467838in}}%
\pgfpathlineto{\pgfqpoint{1.240069in}{0.467838in}}%
\pgfpathlineto{\pgfqpoint{1.240069in}{0.467838in}}%
\pgfpathlineto{\pgfqpoint{1.198027in}{0.467838in}}%
\pgfpathlineto{\pgfqpoint{1.198027in}{0.467838in}}%
\pgfpathclose%
\pgfusepath{fill}%
\end{pgfscope}%
\begin{pgfscope}%
\pgfpathrectangle{\pgfqpoint{0.643077in}{0.467838in}}{\pgfqpoint{2.959734in}{1.817415in}}%
\pgfusepath{clip}%
\pgfsetbuttcap%
\pgfsetmiterjoin%
\definecolor{currentfill}{rgb}{0.121569,0.466667,0.705882}%
\pgfsetfillcolor{currentfill}%
\pgfsetlinewidth{0.000000pt}%
\definecolor{currentstroke}{rgb}{0.000000,0.000000,0.000000}%
\pgfsetstrokecolor{currentstroke}%
\pgfsetstrokeopacity{0.000000}%
\pgfsetdash{}{0pt}%
\pgfpathmoveto{\pgfqpoint{1.408236in}{0.467838in}}%
\pgfpathlineto{\pgfqpoint{1.450277in}{0.467838in}}%
\pgfpathlineto{\pgfqpoint{1.450277in}{0.498142in}}%
\pgfpathlineto{\pgfqpoint{1.408236in}{0.498142in}}%
\pgfpathlineto{\pgfqpoint{1.408236in}{0.467838in}}%
\pgfpathclose%
\pgfusepath{fill}%
\end{pgfscope}%
\begin{pgfscope}%
\pgfpathrectangle{\pgfqpoint{0.643077in}{0.467838in}}{\pgfqpoint{2.959734in}{1.817415in}}%
\pgfusepath{clip}%
\pgfsetbuttcap%
\pgfsetmiterjoin%
\definecolor{currentfill}{rgb}{0.121569,0.466667,0.705882}%
\pgfsetfillcolor{currentfill}%
\pgfsetlinewidth{0.000000pt}%
\definecolor{currentstroke}{rgb}{0.000000,0.000000,0.000000}%
\pgfsetstrokecolor{currentstroke}%
\pgfsetstrokeopacity{0.000000}%
\pgfsetdash{}{0pt}%
\pgfpathmoveto{\pgfqpoint{1.618444in}{0.467838in}}%
\pgfpathlineto{\pgfqpoint{1.660486in}{0.467838in}}%
\pgfpathlineto{\pgfqpoint{1.660486in}{0.467838in}}%
\pgfpathlineto{\pgfqpoint{1.618444in}{0.467838in}}%
\pgfpathlineto{\pgfqpoint{1.618444in}{0.467838in}}%
\pgfpathclose%
\pgfusepath{fill}%
\end{pgfscope}%
\begin{pgfscope}%
\pgfpathrectangle{\pgfqpoint{0.643077in}{0.467838in}}{\pgfqpoint{2.959734in}{1.817415in}}%
\pgfusepath{clip}%
\pgfsetbuttcap%
\pgfsetmiterjoin%
\definecolor{currentfill}{rgb}{0.121569,0.466667,0.705882}%
\pgfsetfillcolor{currentfill}%
\pgfsetlinewidth{0.000000pt}%
\definecolor{currentstroke}{rgb}{0.000000,0.000000,0.000000}%
\pgfsetstrokecolor{currentstroke}%
\pgfsetstrokeopacity{0.000000}%
\pgfsetdash{}{0pt}%
\pgfpathmoveto{\pgfqpoint{1.828653in}{0.467838in}}%
\pgfpathlineto{\pgfqpoint{1.870694in}{0.467838in}}%
\pgfpathlineto{\pgfqpoint{1.870694in}{0.489229in}}%
\pgfpathlineto{\pgfqpoint{1.828653in}{0.489229in}}%
\pgfpathlineto{\pgfqpoint{1.828653in}{0.467838in}}%
\pgfpathclose%
\pgfusepath{fill}%
\end{pgfscope}%
\begin{pgfscope}%
\pgfpathrectangle{\pgfqpoint{0.643077in}{0.467838in}}{\pgfqpoint{2.959734in}{1.817415in}}%
\pgfusepath{clip}%
\pgfsetbuttcap%
\pgfsetmiterjoin%
\definecolor{currentfill}{rgb}{0.121569,0.466667,0.705882}%
\pgfsetfillcolor{currentfill}%
\pgfsetlinewidth{0.000000pt}%
\definecolor{currentstroke}{rgb}{0.000000,0.000000,0.000000}%
\pgfsetstrokecolor{currentstroke}%
\pgfsetstrokeopacity{0.000000}%
\pgfsetdash{}{0pt}%
\pgfpathmoveto{\pgfqpoint{2.038861in}{0.467838in}}%
\pgfpathlineto{\pgfqpoint{2.080903in}{0.467838in}}%
\pgfpathlineto{\pgfqpoint{2.080903in}{0.467838in}}%
\pgfpathlineto{\pgfqpoint{2.038861in}{0.467838in}}%
\pgfpathlineto{\pgfqpoint{2.038861in}{0.467838in}}%
\pgfpathclose%
\pgfusepath{fill}%
\end{pgfscope}%
\begin{pgfscope}%
\pgfpathrectangle{\pgfqpoint{0.643077in}{0.467838in}}{\pgfqpoint{2.959734in}{1.817415in}}%
\pgfusepath{clip}%
\pgfsetbuttcap%
\pgfsetmiterjoin%
\definecolor{currentfill}{rgb}{0.121569,0.466667,0.705882}%
\pgfsetfillcolor{currentfill}%
\pgfsetlinewidth{0.000000pt}%
\definecolor{currentstroke}{rgb}{0.000000,0.000000,0.000000}%
\pgfsetstrokecolor{currentstroke}%
\pgfsetstrokeopacity{0.000000}%
\pgfsetdash{}{0pt}%
\pgfpathmoveto{\pgfqpoint{2.249069in}{0.467838in}}%
\pgfpathlineto{\pgfqpoint{2.291111in}{0.467838in}}%
\pgfpathlineto{\pgfqpoint{2.291111in}{0.467838in}}%
\pgfpathlineto{\pgfqpoint{2.249069in}{0.467838in}}%
\pgfpathlineto{\pgfqpoint{2.249069in}{0.467838in}}%
\pgfpathclose%
\pgfusepath{fill}%
\end{pgfscope}%
\begin{pgfscope}%
\pgfpathrectangle{\pgfqpoint{0.643077in}{0.467838in}}{\pgfqpoint{2.959734in}{1.817415in}}%
\pgfusepath{clip}%
\pgfsetbuttcap%
\pgfsetmiterjoin%
\definecolor{currentfill}{rgb}{0.121569,0.466667,0.705882}%
\pgfsetfillcolor{currentfill}%
\pgfsetlinewidth{0.000000pt}%
\definecolor{currentstroke}{rgb}{0.000000,0.000000,0.000000}%
\pgfsetstrokecolor{currentstroke}%
\pgfsetstrokeopacity{0.000000}%
\pgfsetdash{}{0pt}%
\pgfpathmoveto{\pgfqpoint{2.459278in}{0.467838in}}%
\pgfpathlineto{\pgfqpoint{2.501320in}{0.467838in}}%
\pgfpathlineto{\pgfqpoint{2.501320in}{0.483881in}}%
\pgfpathlineto{\pgfqpoint{2.459278in}{0.483881in}}%
\pgfpathlineto{\pgfqpoint{2.459278in}{0.467838in}}%
\pgfpathclose%
\pgfusepath{fill}%
\end{pgfscope}%
\begin{pgfscope}%
\pgfpathrectangle{\pgfqpoint{0.643077in}{0.467838in}}{\pgfqpoint{2.959734in}{1.817415in}}%
\pgfusepath{clip}%
\pgfsetbuttcap%
\pgfsetmiterjoin%
\definecolor{currentfill}{rgb}{0.121569,0.466667,0.705882}%
\pgfsetfillcolor{currentfill}%
\pgfsetlinewidth{0.000000pt}%
\definecolor{currentstroke}{rgb}{0.000000,0.000000,0.000000}%
\pgfsetstrokecolor{currentstroke}%
\pgfsetstrokeopacity{0.000000}%
\pgfsetdash{}{0pt}%
\pgfpathmoveto{\pgfqpoint{2.669486in}{0.467838in}}%
\pgfpathlineto{\pgfqpoint{2.711528in}{0.467838in}}%
\pgfpathlineto{\pgfqpoint{2.711528in}{0.467838in}}%
\pgfpathlineto{\pgfqpoint{2.669486in}{0.467838in}}%
\pgfpathlineto{\pgfqpoint{2.669486in}{0.467838in}}%
\pgfpathclose%
\pgfusepath{fill}%
\end{pgfscope}%
\begin{pgfscope}%
\pgfpathrectangle{\pgfqpoint{0.643077in}{0.467838in}}{\pgfqpoint{2.959734in}{1.817415in}}%
\pgfusepath{clip}%
\pgfsetbuttcap%
\pgfsetmiterjoin%
\definecolor{currentfill}{rgb}{0.121569,0.466667,0.705882}%
\pgfsetfillcolor{currentfill}%
\pgfsetlinewidth{0.000000pt}%
\definecolor{currentstroke}{rgb}{0.000000,0.000000,0.000000}%
\pgfsetstrokecolor{currentstroke}%
\pgfsetstrokeopacity{0.000000}%
\pgfsetdash{}{0pt}%
\pgfpathmoveto{\pgfqpoint{2.879695in}{0.467838in}}%
\pgfpathlineto{\pgfqpoint{2.921736in}{0.467838in}}%
\pgfpathlineto{\pgfqpoint{2.921736in}{0.487447in}}%
\pgfpathlineto{\pgfqpoint{2.879695in}{0.487447in}}%
\pgfpathlineto{\pgfqpoint{2.879695in}{0.467838in}}%
\pgfpathclose%
\pgfusepath{fill}%
\end{pgfscope}%
\begin{pgfscope}%
\pgfpathrectangle{\pgfqpoint{0.643077in}{0.467838in}}{\pgfqpoint{2.959734in}{1.817415in}}%
\pgfusepath{clip}%
\pgfsetbuttcap%
\pgfsetmiterjoin%
\definecolor{currentfill}{rgb}{0.121569,0.466667,0.705882}%
\pgfsetfillcolor{currentfill}%
\pgfsetlinewidth{0.000000pt}%
\definecolor{currentstroke}{rgb}{0.000000,0.000000,0.000000}%
\pgfsetstrokecolor{currentstroke}%
\pgfsetstrokeopacity{0.000000}%
\pgfsetdash{}{0pt}%
\pgfpathmoveto{\pgfqpoint{3.089903in}{0.467838in}}%
\pgfpathlineto{\pgfqpoint{3.131945in}{0.467838in}}%
\pgfpathlineto{\pgfqpoint{3.131945in}{0.467838in}}%
\pgfpathlineto{\pgfqpoint{3.089903in}{0.467838in}}%
\pgfpathlineto{\pgfqpoint{3.089903in}{0.467838in}}%
\pgfpathclose%
\pgfusepath{fill}%
\end{pgfscope}%
\begin{pgfscope}%
\pgfpathrectangle{\pgfqpoint{0.643077in}{0.467838in}}{\pgfqpoint{2.959734in}{1.817415in}}%
\pgfusepath{clip}%
\pgfsetbuttcap%
\pgfsetmiterjoin%
\definecolor{currentfill}{rgb}{0.121569,0.466667,0.705882}%
\pgfsetfillcolor{currentfill}%
\pgfsetlinewidth{0.000000pt}%
\definecolor{currentstroke}{rgb}{0.000000,0.000000,0.000000}%
\pgfsetstrokecolor{currentstroke}%
\pgfsetstrokeopacity{0.000000}%
\pgfsetdash{}{0pt}%
\pgfpathmoveto{\pgfqpoint{3.300111in}{0.467838in}}%
\pgfpathlineto{\pgfqpoint{3.342153in}{0.467838in}}%
\pgfpathlineto{\pgfqpoint{3.342153in}{2.159493in}}%
\pgfpathlineto{\pgfqpoint{3.300111in}{2.159493in}}%
\pgfpathlineto{\pgfqpoint{3.300111in}{0.467838in}}%
\pgfpathclose%
\pgfusepath{fill}%
\end{pgfscope}%
\begin{pgfscope}%
\pgfpathrectangle{\pgfqpoint{0.643077in}{0.467838in}}{\pgfqpoint{2.959734in}{1.817415in}}%
\pgfusepath{clip}%
\pgfsetbuttcap%
\pgfsetmiterjoin%
\definecolor{currentfill}{rgb}{1.000000,0.498039,0.054902}%
\pgfsetfillcolor{currentfill}%
\pgfsetlinewidth{0.000000pt}%
\definecolor{currentstroke}{rgb}{0.000000,0.000000,0.000000}%
\pgfsetstrokecolor{currentstroke}%
\pgfsetstrokeopacity{0.000000}%
\pgfsetdash{}{0pt}%
\pgfpathmoveto{\pgfqpoint{0.819652in}{0.467838in}}%
\pgfpathlineto{\pgfqpoint{0.861694in}{0.467838in}}%
\pgfpathlineto{\pgfqpoint{0.861694in}{0.469621in}}%
\pgfpathlineto{\pgfqpoint{0.819652in}{0.469621in}}%
\pgfpathlineto{\pgfqpoint{0.819652in}{0.467838in}}%
\pgfpathclose%
\pgfusepath{fill}%
\end{pgfscope}%
\begin{pgfscope}%
\pgfpathrectangle{\pgfqpoint{0.643077in}{0.467838in}}{\pgfqpoint{2.959734in}{1.817415in}}%
\pgfusepath{clip}%
\pgfsetbuttcap%
\pgfsetmiterjoin%
\definecolor{currentfill}{rgb}{1.000000,0.498039,0.054902}%
\pgfsetfillcolor{currentfill}%
\pgfsetlinewidth{0.000000pt}%
\definecolor{currentstroke}{rgb}{0.000000,0.000000,0.000000}%
\pgfsetstrokecolor{currentstroke}%
\pgfsetstrokeopacity{0.000000}%
\pgfsetdash{}{0pt}%
\pgfpathmoveto{\pgfqpoint{1.029861in}{0.467838in}}%
\pgfpathlineto{\pgfqpoint{1.071902in}{0.467838in}}%
\pgfpathlineto{\pgfqpoint{1.071902in}{0.467838in}}%
\pgfpathlineto{\pgfqpoint{1.029861in}{0.467838in}}%
\pgfpathlineto{\pgfqpoint{1.029861in}{0.467838in}}%
\pgfpathclose%
\pgfusepath{fill}%
\end{pgfscope}%
\begin{pgfscope}%
\pgfpathrectangle{\pgfqpoint{0.643077in}{0.467838in}}{\pgfqpoint{2.959734in}{1.817415in}}%
\pgfusepath{clip}%
\pgfsetbuttcap%
\pgfsetmiterjoin%
\definecolor{currentfill}{rgb}{1.000000,0.498039,0.054902}%
\pgfsetfillcolor{currentfill}%
\pgfsetlinewidth{0.000000pt}%
\definecolor{currentstroke}{rgb}{0.000000,0.000000,0.000000}%
\pgfsetstrokecolor{currentstroke}%
\pgfsetstrokeopacity{0.000000}%
\pgfsetdash{}{0pt}%
\pgfpathmoveto{\pgfqpoint{1.240069in}{0.467838in}}%
\pgfpathlineto{\pgfqpoint{1.282111in}{0.467838in}}%
\pgfpathlineto{\pgfqpoint{1.282111in}{0.487447in}}%
\pgfpathlineto{\pgfqpoint{1.240069in}{0.487447in}}%
\pgfpathlineto{\pgfqpoint{1.240069in}{0.467838in}}%
\pgfpathclose%
\pgfusepath{fill}%
\end{pgfscope}%
\begin{pgfscope}%
\pgfpathrectangle{\pgfqpoint{0.643077in}{0.467838in}}{\pgfqpoint{2.959734in}{1.817415in}}%
\pgfusepath{clip}%
\pgfsetbuttcap%
\pgfsetmiterjoin%
\definecolor{currentfill}{rgb}{1.000000,0.498039,0.054902}%
\pgfsetfillcolor{currentfill}%
\pgfsetlinewidth{0.000000pt}%
\definecolor{currentstroke}{rgb}{0.000000,0.000000,0.000000}%
\pgfsetstrokecolor{currentstroke}%
\pgfsetstrokeopacity{0.000000}%
\pgfsetdash{}{0pt}%
\pgfpathmoveto{\pgfqpoint{1.450277in}{0.467838in}}%
\pgfpathlineto{\pgfqpoint{1.492319in}{0.467838in}}%
\pgfpathlineto{\pgfqpoint{1.492319in}{0.467838in}}%
\pgfpathlineto{\pgfqpoint{1.450277in}{0.467838in}}%
\pgfpathlineto{\pgfqpoint{1.450277in}{0.467838in}}%
\pgfpathclose%
\pgfusepath{fill}%
\end{pgfscope}%
\begin{pgfscope}%
\pgfpathrectangle{\pgfqpoint{0.643077in}{0.467838in}}{\pgfqpoint{2.959734in}{1.817415in}}%
\pgfusepath{clip}%
\pgfsetbuttcap%
\pgfsetmiterjoin%
\definecolor{currentfill}{rgb}{1.000000,0.498039,0.054902}%
\pgfsetfillcolor{currentfill}%
\pgfsetlinewidth{0.000000pt}%
\definecolor{currentstroke}{rgb}{0.000000,0.000000,0.000000}%
\pgfsetstrokecolor{currentstroke}%
\pgfsetstrokeopacity{0.000000}%
\pgfsetdash{}{0pt}%
\pgfpathmoveto{\pgfqpoint{1.660486in}{0.467838in}}%
\pgfpathlineto{\pgfqpoint{1.702528in}{0.467838in}}%
\pgfpathlineto{\pgfqpoint{1.702528in}{0.487447in}}%
\pgfpathlineto{\pgfqpoint{1.660486in}{0.487447in}}%
\pgfpathlineto{\pgfqpoint{1.660486in}{0.467838in}}%
\pgfpathclose%
\pgfusepath{fill}%
\end{pgfscope}%
\begin{pgfscope}%
\pgfpathrectangle{\pgfqpoint{0.643077in}{0.467838in}}{\pgfqpoint{2.959734in}{1.817415in}}%
\pgfusepath{clip}%
\pgfsetbuttcap%
\pgfsetmiterjoin%
\definecolor{currentfill}{rgb}{1.000000,0.498039,0.054902}%
\pgfsetfillcolor{currentfill}%
\pgfsetlinewidth{0.000000pt}%
\definecolor{currentstroke}{rgb}{0.000000,0.000000,0.000000}%
\pgfsetstrokecolor{currentstroke}%
\pgfsetstrokeopacity{0.000000}%
\pgfsetdash{}{0pt}%
\pgfpathmoveto{\pgfqpoint{1.870694in}{0.467838in}}%
\pgfpathlineto{\pgfqpoint{1.912736in}{0.467838in}}%
\pgfpathlineto{\pgfqpoint{1.912736in}{0.467838in}}%
\pgfpathlineto{\pgfqpoint{1.870694in}{0.467838in}}%
\pgfpathlineto{\pgfqpoint{1.870694in}{0.467838in}}%
\pgfpathclose%
\pgfusepath{fill}%
\end{pgfscope}%
\begin{pgfscope}%
\pgfpathrectangle{\pgfqpoint{0.643077in}{0.467838in}}{\pgfqpoint{2.959734in}{1.817415in}}%
\pgfusepath{clip}%
\pgfsetbuttcap%
\pgfsetmiterjoin%
\definecolor{currentfill}{rgb}{1.000000,0.498039,0.054902}%
\pgfsetfillcolor{currentfill}%
\pgfsetlinewidth{0.000000pt}%
\definecolor{currentstroke}{rgb}{0.000000,0.000000,0.000000}%
\pgfsetstrokecolor{currentstroke}%
\pgfsetstrokeopacity{0.000000}%
\pgfsetdash{}{0pt}%
\pgfpathmoveto{\pgfqpoint{2.080903in}{0.467838in}}%
\pgfpathlineto{\pgfqpoint{2.122944in}{0.467838in}}%
\pgfpathlineto{\pgfqpoint{2.122944in}{0.491012in}}%
\pgfpathlineto{\pgfqpoint{2.080903in}{0.491012in}}%
\pgfpathlineto{\pgfqpoint{2.080903in}{0.467838in}}%
\pgfpathclose%
\pgfusepath{fill}%
\end{pgfscope}%
\begin{pgfscope}%
\pgfpathrectangle{\pgfqpoint{0.643077in}{0.467838in}}{\pgfqpoint{2.959734in}{1.817415in}}%
\pgfusepath{clip}%
\pgfsetbuttcap%
\pgfsetmiterjoin%
\definecolor{currentfill}{rgb}{1.000000,0.498039,0.054902}%
\pgfsetfillcolor{currentfill}%
\pgfsetlinewidth{0.000000pt}%
\definecolor{currentstroke}{rgb}{0.000000,0.000000,0.000000}%
\pgfsetstrokecolor{currentstroke}%
\pgfsetstrokeopacity{0.000000}%
\pgfsetdash{}{0pt}%
\pgfpathmoveto{\pgfqpoint{2.291111in}{0.467838in}}%
\pgfpathlineto{\pgfqpoint{2.333153in}{0.467838in}}%
\pgfpathlineto{\pgfqpoint{2.333153in}{0.467838in}}%
\pgfpathlineto{\pgfqpoint{2.291111in}{0.467838in}}%
\pgfpathlineto{\pgfqpoint{2.291111in}{0.467838in}}%
\pgfpathclose%
\pgfusepath{fill}%
\end{pgfscope}%
\begin{pgfscope}%
\pgfpathrectangle{\pgfqpoint{0.643077in}{0.467838in}}{\pgfqpoint{2.959734in}{1.817415in}}%
\pgfusepath{clip}%
\pgfsetbuttcap%
\pgfsetmiterjoin%
\definecolor{currentfill}{rgb}{1.000000,0.498039,0.054902}%
\pgfsetfillcolor{currentfill}%
\pgfsetlinewidth{0.000000pt}%
\definecolor{currentstroke}{rgb}{0.000000,0.000000,0.000000}%
\pgfsetstrokecolor{currentstroke}%
\pgfsetstrokeopacity{0.000000}%
\pgfsetdash{}{0pt}%
\pgfpathmoveto{\pgfqpoint{2.501320in}{0.467838in}}%
\pgfpathlineto{\pgfqpoint{2.543361in}{0.467838in}}%
\pgfpathlineto{\pgfqpoint{2.543361in}{0.485664in}}%
\pgfpathlineto{\pgfqpoint{2.501320in}{0.485664in}}%
\pgfpathlineto{\pgfqpoint{2.501320in}{0.467838in}}%
\pgfpathclose%
\pgfusepath{fill}%
\end{pgfscope}%
\begin{pgfscope}%
\pgfpathrectangle{\pgfqpoint{0.643077in}{0.467838in}}{\pgfqpoint{2.959734in}{1.817415in}}%
\pgfusepath{clip}%
\pgfsetbuttcap%
\pgfsetmiterjoin%
\definecolor{currentfill}{rgb}{1.000000,0.498039,0.054902}%
\pgfsetfillcolor{currentfill}%
\pgfsetlinewidth{0.000000pt}%
\definecolor{currentstroke}{rgb}{0.000000,0.000000,0.000000}%
\pgfsetstrokecolor{currentstroke}%
\pgfsetstrokeopacity{0.000000}%
\pgfsetdash{}{0pt}%
\pgfpathmoveto{\pgfqpoint{2.711528in}{0.467838in}}%
\pgfpathlineto{\pgfqpoint{2.753570in}{0.467838in}}%
\pgfpathlineto{\pgfqpoint{2.753570in}{0.467838in}}%
\pgfpathlineto{\pgfqpoint{2.711528in}{0.467838in}}%
\pgfpathlineto{\pgfqpoint{2.711528in}{0.467838in}}%
\pgfpathclose%
\pgfusepath{fill}%
\end{pgfscope}%
\begin{pgfscope}%
\pgfpathrectangle{\pgfqpoint{0.643077in}{0.467838in}}{\pgfqpoint{2.959734in}{1.817415in}}%
\pgfusepath{clip}%
\pgfsetbuttcap%
\pgfsetmiterjoin%
\definecolor{currentfill}{rgb}{1.000000,0.498039,0.054902}%
\pgfsetfillcolor{currentfill}%
\pgfsetlinewidth{0.000000pt}%
\definecolor{currentstroke}{rgb}{0.000000,0.000000,0.000000}%
\pgfsetstrokecolor{currentstroke}%
\pgfsetstrokeopacity{0.000000}%
\pgfsetdash{}{0pt}%
\pgfpathmoveto{\pgfqpoint{2.921736in}{0.467838in}}%
\pgfpathlineto{\pgfqpoint{2.963778in}{0.467838in}}%
\pgfpathlineto{\pgfqpoint{2.963778in}{0.492794in}}%
\pgfpathlineto{\pgfqpoint{2.921736in}{0.492794in}}%
\pgfpathlineto{\pgfqpoint{2.921736in}{0.467838in}}%
\pgfpathclose%
\pgfusepath{fill}%
\end{pgfscope}%
\begin{pgfscope}%
\pgfpathrectangle{\pgfqpoint{0.643077in}{0.467838in}}{\pgfqpoint{2.959734in}{1.817415in}}%
\pgfusepath{clip}%
\pgfsetbuttcap%
\pgfsetmiterjoin%
\definecolor{currentfill}{rgb}{1.000000,0.498039,0.054902}%
\pgfsetfillcolor{currentfill}%
\pgfsetlinewidth{0.000000pt}%
\definecolor{currentstroke}{rgb}{0.000000,0.000000,0.000000}%
\pgfsetstrokecolor{currentstroke}%
\pgfsetstrokeopacity{0.000000}%
\pgfsetdash{}{0pt}%
\pgfpathmoveto{\pgfqpoint{3.131945in}{0.467838in}}%
\pgfpathlineto{\pgfqpoint{3.173986in}{0.467838in}}%
\pgfpathlineto{\pgfqpoint{3.173986in}{0.467838in}}%
\pgfpathlineto{\pgfqpoint{3.131945in}{0.467838in}}%
\pgfpathlineto{\pgfqpoint{3.131945in}{0.467838in}}%
\pgfpathclose%
\pgfusepath{fill}%
\end{pgfscope}%
\begin{pgfscope}%
\pgfpathrectangle{\pgfqpoint{0.643077in}{0.467838in}}{\pgfqpoint{2.959734in}{1.817415in}}%
\pgfusepath{clip}%
\pgfsetbuttcap%
\pgfsetmiterjoin%
\definecolor{currentfill}{rgb}{1.000000,0.498039,0.054902}%
\pgfsetfillcolor{currentfill}%
\pgfsetlinewidth{0.000000pt}%
\definecolor{currentstroke}{rgb}{0.000000,0.000000,0.000000}%
\pgfsetstrokecolor{currentstroke}%
\pgfsetstrokeopacity{0.000000}%
\pgfsetdash{}{0pt}%
\pgfpathmoveto{\pgfqpoint{3.342153in}{0.467838in}}%
\pgfpathlineto{\pgfqpoint{3.384195in}{0.467838in}}%
\pgfpathlineto{\pgfqpoint{3.384195in}{2.143450in}}%
\pgfpathlineto{\pgfqpoint{3.342153in}{2.143450in}}%
\pgfpathlineto{\pgfqpoint{3.342153in}{0.467838in}}%
\pgfpathclose%
\pgfusepath{fill}%
\end{pgfscope}%
\begin{pgfscope}%
\pgfpathrectangle{\pgfqpoint{0.643077in}{0.467838in}}{\pgfqpoint{2.959734in}{1.817415in}}%
\pgfusepath{clip}%
\pgfsetbuttcap%
\pgfsetmiterjoin%
\definecolor{currentfill}{rgb}{0.172549,0.627451,0.172549}%
\pgfsetfillcolor{currentfill}%
\pgfsetlinewidth{0.000000pt}%
\definecolor{currentstroke}{rgb}{0.000000,0.000000,0.000000}%
\pgfsetstrokecolor{currentstroke}%
\pgfsetstrokeopacity{0.000000}%
\pgfsetdash{}{0pt}%
\pgfpathmoveto{\pgfqpoint{0.861694in}{0.467838in}}%
\pgfpathlineto{\pgfqpoint{0.903736in}{0.467838in}}%
\pgfpathlineto{\pgfqpoint{0.903736in}{0.478534in}}%
\pgfpathlineto{\pgfqpoint{0.861694in}{0.478534in}}%
\pgfpathlineto{\pgfqpoint{0.861694in}{0.467838in}}%
\pgfpathclose%
\pgfusepath{fill}%
\end{pgfscope}%
\begin{pgfscope}%
\pgfpathrectangle{\pgfqpoint{0.643077in}{0.467838in}}{\pgfqpoint{2.959734in}{1.817415in}}%
\pgfusepath{clip}%
\pgfsetbuttcap%
\pgfsetmiterjoin%
\definecolor{currentfill}{rgb}{0.172549,0.627451,0.172549}%
\pgfsetfillcolor{currentfill}%
\pgfsetlinewidth{0.000000pt}%
\definecolor{currentstroke}{rgb}{0.000000,0.000000,0.000000}%
\pgfsetstrokecolor{currentstroke}%
\pgfsetstrokeopacity{0.000000}%
\pgfsetdash{}{0pt}%
\pgfpathmoveto{\pgfqpoint{1.071902in}{0.467838in}}%
\pgfpathlineto{\pgfqpoint{1.113944in}{0.467838in}}%
\pgfpathlineto{\pgfqpoint{1.113944in}{0.467838in}}%
\pgfpathlineto{\pgfqpoint{1.071902in}{0.467838in}}%
\pgfpathlineto{\pgfqpoint{1.071902in}{0.467838in}}%
\pgfpathclose%
\pgfusepath{fill}%
\end{pgfscope}%
\begin{pgfscope}%
\pgfpathrectangle{\pgfqpoint{0.643077in}{0.467838in}}{\pgfqpoint{2.959734in}{1.817415in}}%
\pgfusepath{clip}%
\pgfsetbuttcap%
\pgfsetmiterjoin%
\definecolor{currentfill}{rgb}{0.172549,0.627451,0.172549}%
\pgfsetfillcolor{currentfill}%
\pgfsetlinewidth{0.000000pt}%
\definecolor{currentstroke}{rgb}{0.000000,0.000000,0.000000}%
\pgfsetstrokecolor{currentstroke}%
\pgfsetstrokeopacity{0.000000}%
\pgfsetdash{}{0pt}%
\pgfpathmoveto{\pgfqpoint{1.282111in}{0.467838in}}%
\pgfpathlineto{\pgfqpoint{1.324152in}{0.467838in}}%
\pgfpathlineto{\pgfqpoint{1.324152in}{0.467838in}}%
\pgfpathlineto{\pgfqpoint{1.282111in}{0.467838in}}%
\pgfpathlineto{\pgfqpoint{1.282111in}{0.467838in}}%
\pgfpathclose%
\pgfusepath{fill}%
\end{pgfscope}%
\begin{pgfscope}%
\pgfpathrectangle{\pgfqpoint{0.643077in}{0.467838in}}{\pgfqpoint{2.959734in}{1.817415in}}%
\pgfusepath{clip}%
\pgfsetbuttcap%
\pgfsetmiterjoin%
\definecolor{currentfill}{rgb}{0.172549,0.627451,0.172549}%
\pgfsetfillcolor{currentfill}%
\pgfsetlinewidth{0.000000pt}%
\definecolor{currentstroke}{rgb}{0.000000,0.000000,0.000000}%
\pgfsetstrokecolor{currentstroke}%
\pgfsetstrokeopacity{0.000000}%
\pgfsetdash{}{0pt}%
\pgfpathmoveto{\pgfqpoint{1.492319in}{0.467838in}}%
\pgfpathlineto{\pgfqpoint{1.534361in}{0.467838in}}%
\pgfpathlineto{\pgfqpoint{1.534361in}{0.476751in}}%
\pgfpathlineto{\pgfqpoint{1.492319in}{0.476751in}}%
\pgfpathlineto{\pgfqpoint{1.492319in}{0.467838in}}%
\pgfpathclose%
\pgfusepath{fill}%
\end{pgfscope}%
\begin{pgfscope}%
\pgfpathrectangle{\pgfqpoint{0.643077in}{0.467838in}}{\pgfqpoint{2.959734in}{1.817415in}}%
\pgfusepath{clip}%
\pgfsetbuttcap%
\pgfsetmiterjoin%
\definecolor{currentfill}{rgb}{0.172549,0.627451,0.172549}%
\pgfsetfillcolor{currentfill}%
\pgfsetlinewidth{0.000000pt}%
\definecolor{currentstroke}{rgb}{0.000000,0.000000,0.000000}%
\pgfsetstrokecolor{currentstroke}%
\pgfsetstrokeopacity{0.000000}%
\pgfsetdash{}{0pt}%
\pgfpathmoveto{\pgfqpoint{1.702528in}{0.467838in}}%
\pgfpathlineto{\pgfqpoint{1.744569in}{0.467838in}}%
\pgfpathlineto{\pgfqpoint{1.744569in}{0.467838in}}%
\pgfpathlineto{\pgfqpoint{1.702528in}{0.467838in}}%
\pgfpathlineto{\pgfqpoint{1.702528in}{0.467838in}}%
\pgfpathclose%
\pgfusepath{fill}%
\end{pgfscope}%
\begin{pgfscope}%
\pgfpathrectangle{\pgfqpoint{0.643077in}{0.467838in}}{\pgfqpoint{2.959734in}{1.817415in}}%
\pgfusepath{clip}%
\pgfsetbuttcap%
\pgfsetmiterjoin%
\definecolor{currentfill}{rgb}{0.172549,0.627451,0.172549}%
\pgfsetfillcolor{currentfill}%
\pgfsetlinewidth{0.000000pt}%
\definecolor{currentstroke}{rgb}{0.000000,0.000000,0.000000}%
\pgfsetstrokecolor{currentstroke}%
\pgfsetstrokeopacity{0.000000}%
\pgfsetdash{}{0pt}%
\pgfpathmoveto{\pgfqpoint{1.912736in}{0.467838in}}%
\pgfpathlineto{\pgfqpoint{1.954778in}{0.467838in}}%
\pgfpathlineto{\pgfqpoint{1.954778in}{0.487447in}}%
\pgfpathlineto{\pgfqpoint{1.912736in}{0.487447in}}%
\pgfpathlineto{\pgfqpoint{1.912736in}{0.467838in}}%
\pgfpathclose%
\pgfusepath{fill}%
\end{pgfscope}%
\begin{pgfscope}%
\pgfpathrectangle{\pgfqpoint{0.643077in}{0.467838in}}{\pgfqpoint{2.959734in}{1.817415in}}%
\pgfusepath{clip}%
\pgfsetbuttcap%
\pgfsetmiterjoin%
\definecolor{currentfill}{rgb}{0.172549,0.627451,0.172549}%
\pgfsetfillcolor{currentfill}%
\pgfsetlinewidth{0.000000pt}%
\definecolor{currentstroke}{rgb}{0.000000,0.000000,0.000000}%
\pgfsetstrokecolor{currentstroke}%
\pgfsetstrokeopacity{0.000000}%
\pgfsetdash{}{0pt}%
\pgfpathmoveto{\pgfqpoint{2.122944in}{0.467838in}}%
\pgfpathlineto{\pgfqpoint{2.164986in}{0.467838in}}%
\pgfpathlineto{\pgfqpoint{2.164986in}{0.467838in}}%
\pgfpathlineto{\pgfqpoint{2.122944in}{0.467838in}}%
\pgfpathlineto{\pgfqpoint{2.122944in}{0.467838in}}%
\pgfpathclose%
\pgfusepath{fill}%
\end{pgfscope}%
\begin{pgfscope}%
\pgfpathrectangle{\pgfqpoint{0.643077in}{0.467838in}}{\pgfqpoint{2.959734in}{1.817415in}}%
\pgfusepath{clip}%
\pgfsetbuttcap%
\pgfsetmiterjoin%
\definecolor{currentfill}{rgb}{0.172549,0.627451,0.172549}%
\pgfsetfillcolor{currentfill}%
\pgfsetlinewidth{0.000000pt}%
\definecolor{currentstroke}{rgb}{0.000000,0.000000,0.000000}%
\pgfsetstrokecolor{currentstroke}%
\pgfsetstrokeopacity{0.000000}%
\pgfsetdash{}{0pt}%
\pgfpathmoveto{\pgfqpoint{2.333153in}{0.467838in}}%
\pgfpathlineto{\pgfqpoint{2.375194in}{0.467838in}}%
\pgfpathlineto{\pgfqpoint{2.375194in}{0.467838in}}%
\pgfpathlineto{\pgfqpoint{2.333153in}{0.467838in}}%
\pgfpathlineto{\pgfqpoint{2.333153in}{0.467838in}}%
\pgfpathclose%
\pgfusepath{fill}%
\end{pgfscope}%
\begin{pgfscope}%
\pgfpathrectangle{\pgfqpoint{0.643077in}{0.467838in}}{\pgfqpoint{2.959734in}{1.817415in}}%
\pgfusepath{clip}%
\pgfsetbuttcap%
\pgfsetmiterjoin%
\definecolor{currentfill}{rgb}{0.172549,0.627451,0.172549}%
\pgfsetfillcolor{currentfill}%
\pgfsetlinewidth{0.000000pt}%
\definecolor{currentstroke}{rgb}{0.000000,0.000000,0.000000}%
\pgfsetstrokecolor{currentstroke}%
\pgfsetstrokeopacity{0.000000}%
\pgfsetdash{}{0pt}%
\pgfpathmoveto{\pgfqpoint{2.543361in}{0.467838in}}%
\pgfpathlineto{\pgfqpoint{2.585403in}{0.467838in}}%
\pgfpathlineto{\pgfqpoint{2.585403in}{0.482099in}}%
\pgfpathlineto{\pgfqpoint{2.543361in}{0.482099in}}%
\pgfpathlineto{\pgfqpoint{2.543361in}{0.467838in}}%
\pgfpathclose%
\pgfusepath{fill}%
\end{pgfscope}%
\begin{pgfscope}%
\pgfpathrectangle{\pgfqpoint{0.643077in}{0.467838in}}{\pgfqpoint{2.959734in}{1.817415in}}%
\pgfusepath{clip}%
\pgfsetbuttcap%
\pgfsetmiterjoin%
\definecolor{currentfill}{rgb}{0.172549,0.627451,0.172549}%
\pgfsetfillcolor{currentfill}%
\pgfsetlinewidth{0.000000pt}%
\definecolor{currentstroke}{rgb}{0.000000,0.000000,0.000000}%
\pgfsetstrokecolor{currentstroke}%
\pgfsetstrokeopacity{0.000000}%
\pgfsetdash{}{0pt}%
\pgfpathmoveto{\pgfqpoint{2.753570in}{0.467838in}}%
\pgfpathlineto{\pgfqpoint{2.795611in}{0.467838in}}%
\pgfpathlineto{\pgfqpoint{2.795611in}{0.467838in}}%
\pgfpathlineto{\pgfqpoint{2.753570in}{0.467838in}}%
\pgfpathlineto{\pgfqpoint{2.753570in}{0.467838in}}%
\pgfpathclose%
\pgfusepath{fill}%
\end{pgfscope}%
\begin{pgfscope}%
\pgfpathrectangle{\pgfqpoint{0.643077in}{0.467838in}}{\pgfqpoint{2.959734in}{1.817415in}}%
\pgfusepath{clip}%
\pgfsetbuttcap%
\pgfsetmiterjoin%
\definecolor{currentfill}{rgb}{0.172549,0.627451,0.172549}%
\pgfsetfillcolor{currentfill}%
\pgfsetlinewidth{0.000000pt}%
\definecolor{currentstroke}{rgb}{0.000000,0.000000,0.000000}%
\pgfsetstrokecolor{currentstroke}%
\pgfsetstrokeopacity{0.000000}%
\pgfsetdash{}{0pt}%
\pgfpathmoveto{\pgfqpoint{2.963778in}{0.467838in}}%
\pgfpathlineto{\pgfqpoint{3.005820in}{0.467838in}}%
\pgfpathlineto{\pgfqpoint{3.005820in}{0.483881in}}%
\pgfpathlineto{\pgfqpoint{2.963778in}{0.483881in}}%
\pgfpathlineto{\pgfqpoint{2.963778in}{0.467838in}}%
\pgfpathclose%
\pgfusepath{fill}%
\end{pgfscope}%
\begin{pgfscope}%
\pgfpathrectangle{\pgfqpoint{0.643077in}{0.467838in}}{\pgfqpoint{2.959734in}{1.817415in}}%
\pgfusepath{clip}%
\pgfsetbuttcap%
\pgfsetmiterjoin%
\definecolor{currentfill}{rgb}{0.172549,0.627451,0.172549}%
\pgfsetfillcolor{currentfill}%
\pgfsetlinewidth{0.000000pt}%
\definecolor{currentstroke}{rgb}{0.000000,0.000000,0.000000}%
\pgfsetstrokecolor{currentstroke}%
\pgfsetstrokeopacity{0.000000}%
\pgfsetdash{}{0pt}%
\pgfpathmoveto{\pgfqpoint{3.173986in}{0.467838in}}%
\pgfpathlineto{\pgfqpoint{3.216028in}{0.467838in}}%
\pgfpathlineto{\pgfqpoint{3.216028in}{0.467838in}}%
\pgfpathlineto{\pgfqpoint{3.173986in}{0.467838in}}%
\pgfpathlineto{\pgfqpoint{3.173986in}{0.467838in}}%
\pgfpathclose%
\pgfusepath{fill}%
\end{pgfscope}%
\begin{pgfscope}%
\pgfpathrectangle{\pgfqpoint{0.643077in}{0.467838in}}{\pgfqpoint{2.959734in}{1.817415in}}%
\pgfusepath{clip}%
\pgfsetbuttcap%
\pgfsetmiterjoin%
\definecolor{currentfill}{rgb}{0.172549,0.627451,0.172549}%
\pgfsetfillcolor{currentfill}%
\pgfsetlinewidth{0.000000pt}%
\definecolor{currentstroke}{rgb}{0.000000,0.000000,0.000000}%
\pgfsetstrokecolor{currentstroke}%
\pgfsetstrokeopacity{0.000000}%
\pgfsetdash{}{0pt}%
\pgfpathmoveto{\pgfqpoint{3.384195in}{0.467838in}}%
\pgfpathlineto{\pgfqpoint{3.426236in}{0.467838in}}%
\pgfpathlineto{\pgfqpoint{3.426236in}{2.180884in}}%
\pgfpathlineto{\pgfqpoint{3.384195in}{2.180884in}}%
\pgfpathlineto{\pgfqpoint{3.384195in}{0.467838in}}%
\pgfpathclose%
\pgfusepath{fill}%
\end{pgfscope}%
\begin{pgfscope}%
\pgfpathrectangle{\pgfqpoint{0.643077in}{0.467838in}}{\pgfqpoint{2.959734in}{1.817415in}}%
\pgfusepath{clip}%
\pgfsetbuttcap%
\pgfsetmiterjoin%
\definecolor{currentfill}{rgb}{0.839216,0.152941,0.156863}%
\pgfsetfillcolor{currentfill}%
\pgfsetlinewidth{0.000000pt}%
\definecolor{currentstroke}{rgb}{0.000000,0.000000,0.000000}%
\pgfsetstrokecolor{currentstroke}%
\pgfsetstrokeopacity{0.000000}%
\pgfsetdash{}{0pt}%
\pgfpathmoveto{\pgfqpoint{0.903736in}{0.467838in}}%
\pgfpathlineto{\pgfqpoint{0.945777in}{0.467838in}}%
\pgfpathlineto{\pgfqpoint{0.945777in}{0.471403in}}%
\pgfpathlineto{\pgfqpoint{0.903736in}{0.471403in}}%
\pgfpathlineto{\pgfqpoint{0.903736in}{0.467838in}}%
\pgfpathclose%
\pgfusepath{fill}%
\end{pgfscope}%
\begin{pgfscope}%
\pgfpathrectangle{\pgfqpoint{0.643077in}{0.467838in}}{\pgfqpoint{2.959734in}{1.817415in}}%
\pgfusepath{clip}%
\pgfsetbuttcap%
\pgfsetmiterjoin%
\definecolor{currentfill}{rgb}{0.839216,0.152941,0.156863}%
\pgfsetfillcolor{currentfill}%
\pgfsetlinewidth{0.000000pt}%
\definecolor{currentstroke}{rgb}{0.000000,0.000000,0.000000}%
\pgfsetstrokecolor{currentstroke}%
\pgfsetstrokeopacity{0.000000}%
\pgfsetdash{}{0pt}%
\pgfpathmoveto{\pgfqpoint{1.113944in}{0.467838in}}%
\pgfpathlineto{\pgfqpoint{1.155986in}{0.467838in}}%
\pgfpathlineto{\pgfqpoint{1.155986in}{0.467838in}}%
\pgfpathlineto{\pgfqpoint{1.113944in}{0.467838in}}%
\pgfpathlineto{\pgfqpoint{1.113944in}{0.467838in}}%
\pgfpathclose%
\pgfusepath{fill}%
\end{pgfscope}%
\begin{pgfscope}%
\pgfpathrectangle{\pgfqpoint{0.643077in}{0.467838in}}{\pgfqpoint{2.959734in}{1.817415in}}%
\pgfusepath{clip}%
\pgfsetbuttcap%
\pgfsetmiterjoin%
\definecolor{currentfill}{rgb}{0.839216,0.152941,0.156863}%
\pgfsetfillcolor{currentfill}%
\pgfsetlinewidth{0.000000pt}%
\definecolor{currentstroke}{rgb}{0.000000,0.000000,0.000000}%
\pgfsetstrokecolor{currentstroke}%
\pgfsetstrokeopacity{0.000000}%
\pgfsetdash{}{0pt}%
\pgfpathmoveto{\pgfqpoint{1.324152in}{0.467838in}}%
\pgfpathlineto{\pgfqpoint{1.366194in}{0.467838in}}%
\pgfpathlineto{\pgfqpoint{1.366194in}{0.467838in}}%
\pgfpathlineto{\pgfqpoint{1.324152in}{0.467838in}}%
\pgfpathlineto{\pgfqpoint{1.324152in}{0.467838in}}%
\pgfpathclose%
\pgfusepath{fill}%
\end{pgfscope}%
\begin{pgfscope}%
\pgfpathrectangle{\pgfqpoint{0.643077in}{0.467838in}}{\pgfqpoint{2.959734in}{1.817415in}}%
\pgfusepath{clip}%
\pgfsetbuttcap%
\pgfsetmiterjoin%
\definecolor{currentfill}{rgb}{0.839216,0.152941,0.156863}%
\pgfsetfillcolor{currentfill}%
\pgfsetlinewidth{0.000000pt}%
\definecolor{currentstroke}{rgb}{0.000000,0.000000,0.000000}%
\pgfsetstrokecolor{currentstroke}%
\pgfsetstrokeopacity{0.000000}%
\pgfsetdash{}{0pt}%
\pgfpathmoveto{\pgfqpoint{1.534361in}{0.467838in}}%
\pgfpathlineto{\pgfqpoint{1.576403in}{0.467838in}}%
\pgfpathlineto{\pgfqpoint{1.576403in}{0.480316in}}%
\pgfpathlineto{\pgfqpoint{1.534361in}{0.480316in}}%
\pgfpathlineto{\pgfqpoint{1.534361in}{0.467838in}}%
\pgfpathclose%
\pgfusepath{fill}%
\end{pgfscope}%
\begin{pgfscope}%
\pgfpathrectangle{\pgfqpoint{0.643077in}{0.467838in}}{\pgfqpoint{2.959734in}{1.817415in}}%
\pgfusepath{clip}%
\pgfsetbuttcap%
\pgfsetmiterjoin%
\definecolor{currentfill}{rgb}{0.839216,0.152941,0.156863}%
\pgfsetfillcolor{currentfill}%
\pgfsetlinewidth{0.000000pt}%
\definecolor{currentstroke}{rgb}{0.000000,0.000000,0.000000}%
\pgfsetstrokecolor{currentstroke}%
\pgfsetstrokeopacity{0.000000}%
\pgfsetdash{}{0pt}%
\pgfpathmoveto{\pgfqpoint{1.744569in}{0.467838in}}%
\pgfpathlineto{\pgfqpoint{1.786611in}{0.467838in}}%
\pgfpathlineto{\pgfqpoint{1.786611in}{0.467838in}}%
\pgfpathlineto{\pgfqpoint{1.744569in}{0.467838in}}%
\pgfpathlineto{\pgfqpoint{1.744569in}{0.467838in}}%
\pgfpathclose%
\pgfusepath{fill}%
\end{pgfscope}%
\begin{pgfscope}%
\pgfpathrectangle{\pgfqpoint{0.643077in}{0.467838in}}{\pgfqpoint{2.959734in}{1.817415in}}%
\pgfusepath{clip}%
\pgfsetbuttcap%
\pgfsetmiterjoin%
\definecolor{currentfill}{rgb}{0.839216,0.152941,0.156863}%
\pgfsetfillcolor{currentfill}%
\pgfsetlinewidth{0.000000pt}%
\definecolor{currentstroke}{rgb}{0.000000,0.000000,0.000000}%
\pgfsetstrokecolor{currentstroke}%
\pgfsetstrokeopacity{0.000000}%
\pgfsetdash{}{0pt}%
\pgfpathmoveto{\pgfqpoint{1.954778in}{0.467838in}}%
\pgfpathlineto{\pgfqpoint{1.996819in}{0.467838in}}%
\pgfpathlineto{\pgfqpoint{1.996819in}{0.485664in}}%
\pgfpathlineto{\pgfqpoint{1.954778in}{0.485664in}}%
\pgfpathlineto{\pgfqpoint{1.954778in}{0.467838in}}%
\pgfpathclose%
\pgfusepath{fill}%
\end{pgfscope}%
\begin{pgfscope}%
\pgfpathrectangle{\pgfqpoint{0.643077in}{0.467838in}}{\pgfqpoint{2.959734in}{1.817415in}}%
\pgfusepath{clip}%
\pgfsetbuttcap%
\pgfsetmiterjoin%
\definecolor{currentfill}{rgb}{0.839216,0.152941,0.156863}%
\pgfsetfillcolor{currentfill}%
\pgfsetlinewidth{0.000000pt}%
\definecolor{currentstroke}{rgb}{0.000000,0.000000,0.000000}%
\pgfsetstrokecolor{currentstroke}%
\pgfsetstrokeopacity{0.000000}%
\pgfsetdash{}{0pt}%
\pgfpathmoveto{\pgfqpoint{2.164986in}{0.467838in}}%
\pgfpathlineto{\pgfqpoint{2.207028in}{0.467838in}}%
\pgfpathlineto{\pgfqpoint{2.207028in}{0.467838in}}%
\pgfpathlineto{\pgfqpoint{2.164986in}{0.467838in}}%
\pgfpathlineto{\pgfqpoint{2.164986in}{0.467838in}}%
\pgfpathclose%
\pgfusepath{fill}%
\end{pgfscope}%
\begin{pgfscope}%
\pgfpathrectangle{\pgfqpoint{0.643077in}{0.467838in}}{\pgfqpoint{2.959734in}{1.817415in}}%
\pgfusepath{clip}%
\pgfsetbuttcap%
\pgfsetmiterjoin%
\definecolor{currentfill}{rgb}{0.839216,0.152941,0.156863}%
\pgfsetfillcolor{currentfill}%
\pgfsetlinewidth{0.000000pt}%
\definecolor{currentstroke}{rgb}{0.000000,0.000000,0.000000}%
\pgfsetstrokecolor{currentstroke}%
\pgfsetstrokeopacity{0.000000}%
\pgfsetdash{}{0pt}%
\pgfpathmoveto{\pgfqpoint{2.375194in}{0.467838in}}%
\pgfpathlineto{\pgfqpoint{2.417236in}{0.467838in}}%
\pgfpathlineto{\pgfqpoint{2.417236in}{0.467838in}}%
\pgfpathlineto{\pgfqpoint{2.375194in}{0.467838in}}%
\pgfpathlineto{\pgfqpoint{2.375194in}{0.467838in}}%
\pgfpathclose%
\pgfusepath{fill}%
\end{pgfscope}%
\begin{pgfscope}%
\pgfpathrectangle{\pgfqpoint{0.643077in}{0.467838in}}{\pgfqpoint{2.959734in}{1.817415in}}%
\pgfusepath{clip}%
\pgfsetbuttcap%
\pgfsetmiterjoin%
\definecolor{currentfill}{rgb}{0.839216,0.152941,0.156863}%
\pgfsetfillcolor{currentfill}%
\pgfsetlinewidth{0.000000pt}%
\definecolor{currentstroke}{rgb}{0.000000,0.000000,0.000000}%
\pgfsetstrokecolor{currentstroke}%
\pgfsetstrokeopacity{0.000000}%
\pgfsetdash{}{0pt}%
\pgfpathmoveto{\pgfqpoint{2.585403in}{0.467838in}}%
\pgfpathlineto{\pgfqpoint{2.627445in}{0.467838in}}%
\pgfpathlineto{\pgfqpoint{2.627445in}{0.474969in}}%
\pgfpathlineto{\pgfqpoint{2.585403in}{0.474969in}}%
\pgfpathlineto{\pgfqpoint{2.585403in}{0.467838in}}%
\pgfpathclose%
\pgfusepath{fill}%
\end{pgfscope}%
\begin{pgfscope}%
\pgfpathrectangle{\pgfqpoint{0.643077in}{0.467838in}}{\pgfqpoint{2.959734in}{1.817415in}}%
\pgfusepath{clip}%
\pgfsetbuttcap%
\pgfsetmiterjoin%
\definecolor{currentfill}{rgb}{0.839216,0.152941,0.156863}%
\pgfsetfillcolor{currentfill}%
\pgfsetlinewidth{0.000000pt}%
\definecolor{currentstroke}{rgb}{0.000000,0.000000,0.000000}%
\pgfsetstrokecolor{currentstroke}%
\pgfsetstrokeopacity{0.000000}%
\pgfsetdash{}{0pt}%
\pgfpathmoveto{\pgfqpoint{2.795611in}{0.467838in}}%
\pgfpathlineto{\pgfqpoint{2.837653in}{0.467838in}}%
\pgfpathlineto{\pgfqpoint{2.837653in}{0.467838in}}%
\pgfpathlineto{\pgfqpoint{2.795611in}{0.467838in}}%
\pgfpathlineto{\pgfqpoint{2.795611in}{0.467838in}}%
\pgfpathclose%
\pgfusepath{fill}%
\end{pgfscope}%
\begin{pgfscope}%
\pgfpathrectangle{\pgfqpoint{0.643077in}{0.467838in}}{\pgfqpoint{2.959734in}{1.817415in}}%
\pgfusepath{clip}%
\pgfsetbuttcap%
\pgfsetmiterjoin%
\definecolor{currentfill}{rgb}{0.839216,0.152941,0.156863}%
\pgfsetfillcolor{currentfill}%
\pgfsetlinewidth{0.000000pt}%
\definecolor{currentstroke}{rgb}{0.000000,0.000000,0.000000}%
\pgfsetstrokecolor{currentstroke}%
\pgfsetstrokeopacity{0.000000}%
\pgfsetdash{}{0pt}%
\pgfpathmoveto{\pgfqpoint{3.005820in}{0.467838in}}%
\pgfpathlineto{\pgfqpoint{3.047861in}{0.467838in}}%
\pgfpathlineto{\pgfqpoint{3.047861in}{0.478534in}}%
\pgfpathlineto{\pgfqpoint{3.005820in}{0.478534in}}%
\pgfpathlineto{\pgfqpoint{3.005820in}{0.467838in}}%
\pgfpathclose%
\pgfusepath{fill}%
\end{pgfscope}%
\begin{pgfscope}%
\pgfpathrectangle{\pgfqpoint{0.643077in}{0.467838in}}{\pgfqpoint{2.959734in}{1.817415in}}%
\pgfusepath{clip}%
\pgfsetbuttcap%
\pgfsetmiterjoin%
\definecolor{currentfill}{rgb}{0.839216,0.152941,0.156863}%
\pgfsetfillcolor{currentfill}%
\pgfsetlinewidth{0.000000pt}%
\definecolor{currentstroke}{rgb}{0.000000,0.000000,0.000000}%
\pgfsetstrokecolor{currentstroke}%
\pgfsetstrokeopacity{0.000000}%
\pgfsetdash{}{0pt}%
\pgfpathmoveto{\pgfqpoint{3.216028in}{0.467838in}}%
\pgfpathlineto{\pgfqpoint{3.258070in}{0.467838in}}%
\pgfpathlineto{\pgfqpoint{3.258070in}{0.467838in}}%
\pgfpathlineto{\pgfqpoint{3.216028in}{0.467838in}}%
\pgfpathlineto{\pgfqpoint{3.216028in}{0.467838in}}%
\pgfpathclose%
\pgfusepath{fill}%
\end{pgfscope}%
\begin{pgfscope}%
\pgfpathrectangle{\pgfqpoint{0.643077in}{0.467838in}}{\pgfqpoint{2.959734in}{1.817415in}}%
\pgfusepath{clip}%
\pgfsetbuttcap%
\pgfsetmiterjoin%
\definecolor{currentfill}{rgb}{0.839216,0.152941,0.156863}%
\pgfsetfillcolor{currentfill}%
\pgfsetlinewidth{0.000000pt}%
\definecolor{currentstroke}{rgb}{0.000000,0.000000,0.000000}%
\pgfsetstrokecolor{currentstroke}%
\pgfsetstrokeopacity{0.000000}%
\pgfsetdash{}{0pt}%
\pgfpathmoveto{\pgfqpoint{3.426236in}{0.467838in}}%
\pgfpathlineto{\pgfqpoint{3.468278in}{0.467838in}}%
\pgfpathlineto{\pgfqpoint{3.468278in}{2.198710in}}%
\pgfpathlineto{\pgfqpoint{3.426236in}{2.198710in}}%
\pgfpathlineto{\pgfqpoint{3.426236in}{0.467838in}}%
\pgfpathclose%
\pgfusepath{fill}%
\end{pgfscope}%
\begin{pgfscope}%
\pgfsetbuttcap%
\pgfsetroundjoin%
\definecolor{currentfill}{rgb}{0.000000,0.000000,0.000000}%
\pgfsetfillcolor{currentfill}%
\pgfsetlinewidth{0.803000pt}%
\definecolor{currentstroke}{rgb}{0.000000,0.000000,0.000000}%
\pgfsetstrokecolor{currentstroke}%
\pgfsetdash{}{0pt}%
\pgfsys@defobject{currentmarker}{\pgfqpoint{0.000000in}{-0.048611in}}{\pgfqpoint{0.000000in}{0.000000in}}{%
\pgfpathmoveto{\pgfqpoint{0.000000in}{0.000000in}}%
\pgfpathlineto{\pgfqpoint{0.000000in}{-0.048611in}}%
\pgfusepath{stroke,fill}%
}%
\begin{pgfscope}%
\pgfsys@transformshift{0.959013in}{0.467838in}%
\pgfsys@useobject{currentmarker}{}%
\end{pgfscope}%
\end{pgfscope}%
\begin{pgfscope}%
\definecolor{textcolor}{rgb}{0.000000,0.000000,0.000000}%
\pgfsetstrokecolor{textcolor}%
\pgfsetfillcolor{textcolor}%
\pgftext[x=0.959013in,y=0.370616in,,top]{\color{textcolor}\sffamily\fontsize{8.000000}{9.600000}\selectfont 0.5}%
\end{pgfscope}%
\begin{pgfscope}%
\pgfsetbuttcap%
\pgfsetroundjoin%
\definecolor{currentfill}{rgb}{0.000000,0.000000,0.000000}%
\pgfsetfillcolor{currentfill}%
\pgfsetlinewidth{0.803000pt}%
\definecolor{currentstroke}{rgb}{0.000000,0.000000,0.000000}%
\pgfsetstrokecolor{currentstroke}%
\pgfsetdash{}{0pt}%
\pgfsys@defobject{currentmarker}{\pgfqpoint{0.000000in}{-0.048611in}}{\pgfqpoint{0.000000in}{0.000000in}}{%
\pgfpathmoveto{\pgfqpoint{0.000000in}{0.000000in}}%
\pgfpathlineto{\pgfqpoint{0.000000in}{-0.048611in}}%
\pgfusepath{stroke,fill}%
}%
\begin{pgfscope}%
\pgfsys@transformshift{1.465070in}{0.467838in}%
\pgfsys@useobject{currentmarker}{}%
\end{pgfscope}%
\end{pgfscope}%
\begin{pgfscope}%
\definecolor{textcolor}{rgb}{0.000000,0.000000,0.000000}%
\pgfsetstrokecolor{textcolor}%
\pgfsetfillcolor{textcolor}%
\pgftext[x=1.465070in,y=0.370616in,,top]{\color{textcolor}\sffamily\fontsize{8.000000}{9.600000}\selectfont 0.6}%
\end{pgfscope}%
\begin{pgfscope}%
\pgfsetbuttcap%
\pgfsetroundjoin%
\definecolor{currentfill}{rgb}{0.000000,0.000000,0.000000}%
\pgfsetfillcolor{currentfill}%
\pgfsetlinewidth{0.803000pt}%
\definecolor{currentstroke}{rgb}{0.000000,0.000000,0.000000}%
\pgfsetstrokecolor{currentstroke}%
\pgfsetdash{}{0pt}%
\pgfsys@defobject{currentmarker}{\pgfqpoint{0.000000in}{-0.048611in}}{\pgfqpoint{0.000000in}{0.000000in}}{%
\pgfpathmoveto{\pgfqpoint{0.000000in}{0.000000in}}%
\pgfpathlineto{\pgfqpoint{0.000000in}{-0.048611in}}%
\pgfusepath{stroke,fill}%
}%
\begin{pgfscope}%
\pgfsys@transformshift{1.971127in}{0.467838in}%
\pgfsys@useobject{currentmarker}{}%
\end{pgfscope}%
\end{pgfscope}%
\begin{pgfscope}%
\definecolor{textcolor}{rgb}{0.000000,0.000000,0.000000}%
\pgfsetstrokecolor{textcolor}%
\pgfsetfillcolor{textcolor}%
\pgftext[x=1.971127in,y=0.370616in,,top]{\color{textcolor}\sffamily\fontsize{8.000000}{9.600000}\selectfont 0.7}%
\end{pgfscope}%
\begin{pgfscope}%
\pgfsetbuttcap%
\pgfsetroundjoin%
\definecolor{currentfill}{rgb}{0.000000,0.000000,0.000000}%
\pgfsetfillcolor{currentfill}%
\pgfsetlinewidth{0.803000pt}%
\definecolor{currentstroke}{rgb}{0.000000,0.000000,0.000000}%
\pgfsetstrokecolor{currentstroke}%
\pgfsetdash{}{0pt}%
\pgfsys@defobject{currentmarker}{\pgfqpoint{0.000000in}{-0.048611in}}{\pgfqpoint{0.000000in}{0.000000in}}{%
\pgfpathmoveto{\pgfqpoint{0.000000in}{0.000000in}}%
\pgfpathlineto{\pgfqpoint{0.000000in}{-0.048611in}}%
\pgfusepath{stroke,fill}%
}%
\begin{pgfscope}%
\pgfsys@transformshift{2.477184in}{0.467838in}%
\pgfsys@useobject{currentmarker}{}%
\end{pgfscope}%
\end{pgfscope}%
\begin{pgfscope}%
\definecolor{textcolor}{rgb}{0.000000,0.000000,0.000000}%
\pgfsetstrokecolor{textcolor}%
\pgfsetfillcolor{textcolor}%
\pgftext[x=2.477184in,y=0.370616in,,top]{\color{textcolor}\sffamily\fontsize{8.000000}{9.600000}\selectfont 0.8}%
\end{pgfscope}%
\begin{pgfscope}%
\pgfsetbuttcap%
\pgfsetroundjoin%
\definecolor{currentfill}{rgb}{0.000000,0.000000,0.000000}%
\pgfsetfillcolor{currentfill}%
\pgfsetlinewidth{0.803000pt}%
\definecolor{currentstroke}{rgb}{0.000000,0.000000,0.000000}%
\pgfsetstrokecolor{currentstroke}%
\pgfsetdash{}{0pt}%
\pgfsys@defobject{currentmarker}{\pgfqpoint{0.000000in}{-0.048611in}}{\pgfqpoint{0.000000in}{0.000000in}}{%
\pgfpathmoveto{\pgfqpoint{0.000000in}{0.000000in}}%
\pgfpathlineto{\pgfqpoint{0.000000in}{-0.048611in}}%
\pgfusepath{stroke,fill}%
}%
\begin{pgfscope}%
\pgfsys@transformshift{2.983242in}{0.467838in}%
\pgfsys@useobject{currentmarker}{}%
\end{pgfscope}%
\end{pgfscope}%
\begin{pgfscope}%
\definecolor{textcolor}{rgb}{0.000000,0.000000,0.000000}%
\pgfsetstrokecolor{textcolor}%
\pgfsetfillcolor{textcolor}%
\pgftext[x=2.983242in,y=0.370616in,,top]{\color{textcolor}\sffamily\fontsize{8.000000}{9.600000}\selectfont 0.9}%
\end{pgfscope}%
\begin{pgfscope}%
\pgfsetbuttcap%
\pgfsetroundjoin%
\definecolor{currentfill}{rgb}{0.000000,0.000000,0.000000}%
\pgfsetfillcolor{currentfill}%
\pgfsetlinewidth{0.803000pt}%
\definecolor{currentstroke}{rgb}{0.000000,0.000000,0.000000}%
\pgfsetstrokecolor{currentstroke}%
\pgfsetdash{}{0pt}%
\pgfsys@defobject{currentmarker}{\pgfqpoint{0.000000in}{-0.048611in}}{\pgfqpoint{0.000000in}{0.000000in}}{%
\pgfpathmoveto{\pgfqpoint{0.000000in}{0.000000in}}%
\pgfpathlineto{\pgfqpoint{0.000000in}{-0.048611in}}%
\pgfusepath{stroke,fill}%
}%
\begin{pgfscope}%
\pgfsys@transformshift{3.489299in}{0.467838in}%
\pgfsys@useobject{currentmarker}{}%
\end{pgfscope}%
\end{pgfscope}%
\begin{pgfscope}%
\definecolor{textcolor}{rgb}{0.000000,0.000000,0.000000}%
\pgfsetstrokecolor{textcolor}%
\pgfsetfillcolor{textcolor}%
\pgftext[x=3.489299in,y=0.370616in,,top]{\color{textcolor}\sffamily\fontsize{8.000000}{9.600000}\selectfont 1.0}%
\end{pgfscope}%
\begin{pgfscope}%
\definecolor{textcolor}{rgb}{0.000000,0.000000,0.000000}%
\pgfsetstrokecolor{textcolor}%
\pgfsetfillcolor{textcolor}%
\pgftext[x=2.122944in,y=0.207530in,,top]{\color{textcolor}\sffamily\fontsize{8.000000}{9.600000}\selectfont \(\displaystyle \beta\)}%
\end{pgfscope}%
\begin{pgfscope}%
\pgfsetbuttcap%
\pgfsetroundjoin%
\definecolor{currentfill}{rgb}{0.000000,0.000000,0.000000}%
\pgfsetfillcolor{currentfill}%
\pgfsetlinewidth{0.803000pt}%
\definecolor{currentstroke}{rgb}{0.000000,0.000000,0.000000}%
\pgfsetstrokecolor{currentstroke}%
\pgfsetdash{}{0pt}%
\pgfsys@defobject{currentmarker}{\pgfqpoint{-0.048611in}{0.000000in}}{\pgfqpoint{-0.000000in}{0.000000in}}{%
\pgfpathmoveto{\pgfqpoint{-0.000000in}{0.000000in}}%
\pgfpathlineto{\pgfqpoint{-0.048611in}{0.000000in}}%
\pgfusepath{stroke,fill}%
}%
\begin{pgfscope}%
\pgfsys@transformshift{0.643077in}{0.467838in}%
\pgfsys@useobject{currentmarker}{}%
\end{pgfscope}%
\end{pgfscope}%
\begin{pgfscope}%
\definecolor{textcolor}{rgb}{0.000000,0.000000,0.000000}%
\pgfsetstrokecolor{textcolor}%
\pgfsetfillcolor{textcolor}%
\pgftext[x=0.475163in, y=0.425629in, left, base]{\color{textcolor}\sffamily\fontsize{8.000000}{9.600000}\selectfont 0}%
\end{pgfscope}%
\begin{pgfscope}%
\pgfsetbuttcap%
\pgfsetroundjoin%
\definecolor{currentfill}{rgb}{0.000000,0.000000,0.000000}%
\pgfsetfillcolor{currentfill}%
\pgfsetlinewidth{0.803000pt}%
\definecolor{currentstroke}{rgb}{0.000000,0.000000,0.000000}%
\pgfsetstrokecolor{currentstroke}%
\pgfsetdash{}{0pt}%
\pgfsys@defobject{currentmarker}{\pgfqpoint{-0.048611in}{0.000000in}}{\pgfqpoint{-0.000000in}{0.000000in}}{%
\pgfpathmoveto{\pgfqpoint{-0.000000in}{0.000000in}}%
\pgfpathlineto{\pgfqpoint{-0.048611in}{0.000000in}}%
\pgfusepath{stroke,fill}%
}%
\begin{pgfscope}%
\pgfsys@transformshift{0.643077in}{0.824352in}%
\pgfsys@useobject{currentmarker}{}%
\end{pgfscope}%
\end{pgfscope}%
\begin{pgfscope}%
\definecolor{textcolor}{rgb}{0.000000,0.000000,0.000000}%
\pgfsetstrokecolor{textcolor}%
\pgfsetfillcolor{textcolor}%
\pgftext[x=0.333778in, y=0.782142in, left, base]{\color{textcolor}\sffamily\fontsize{8.000000}{9.600000}\selectfont 200}%
\end{pgfscope}%
\begin{pgfscope}%
\pgfsetbuttcap%
\pgfsetroundjoin%
\definecolor{currentfill}{rgb}{0.000000,0.000000,0.000000}%
\pgfsetfillcolor{currentfill}%
\pgfsetlinewidth{0.803000pt}%
\definecolor{currentstroke}{rgb}{0.000000,0.000000,0.000000}%
\pgfsetstrokecolor{currentstroke}%
\pgfsetdash{}{0pt}%
\pgfsys@defobject{currentmarker}{\pgfqpoint{-0.048611in}{0.000000in}}{\pgfqpoint{-0.000000in}{0.000000in}}{%
\pgfpathmoveto{\pgfqpoint{-0.000000in}{0.000000in}}%
\pgfpathlineto{\pgfqpoint{-0.048611in}{0.000000in}}%
\pgfusepath{stroke,fill}%
}%
\begin{pgfscope}%
\pgfsys@transformshift{0.643077in}{1.180865in}%
\pgfsys@useobject{currentmarker}{}%
\end{pgfscope}%
\end{pgfscope}%
\begin{pgfscope}%
\definecolor{textcolor}{rgb}{0.000000,0.000000,0.000000}%
\pgfsetstrokecolor{textcolor}%
\pgfsetfillcolor{textcolor}%
\pgftext[x=0.333778in, y=1.138655in, left, base]{\color{textcolor}\sffamily\fontsize{8.000000}{9.600000}\selectfont 400}%
\end{pgfscope}%
\begin{pgfscope}%
\pgfsetbuttcap%
\pgfsetroundjoin%
\definecolor{currentfill}{rgb}{0.000000,0.000000,0.000000}%
\pgfsetfillcolor{currentfill}%
\pgfsetlinewidth{0.803000pt}%
\definecolor{currentstroke}{rgb}{0.000000,0.000000,0.000000}%
\pgfsetstrokecolor{currentstroke}%
\pgfsetdash{}{0pt}%
\pgfsys@defobject{currentmarker}{\pgfqpoint{-0.048611in}{0.000000in}}{\pgfqpoint{-0.000000in}{0.000000in}}{%
\pgfpathmoveto{\pgfqpoint{-0.000000in}{0.000000in}}%
\pgfpathlineto{\pgfqpoint{-0.048611in}{0.000000in}}%
\pgfusepath{stroke,fill}%
}%
\begin{pgfscope}%
\pgfsys@transformshift{0.643077in}{1.537378in}%
\pgfsys@useobject{currentmarker}{}%
\end{pgfscope}%
\end{pgfscope}%
\begin{pgfscope}%
\definecolor{textcolor}{rgb}{0.000000,0.000000,0.000000}%
\pgfsetstrokecolor{textcolor}%
\pgfsetfillcolor{textcolor}%
\pgftext[x=0.333778in, y=1.495169in, left, base]{\color{textcolor}\sffamily\fontsize{8.000000}{9.600000}\selectfont 600}%
\end{pgfscope}%
\begin{pgfscope}%
\pgfsetbuttcap%
\pgfsetroundjoin%
\definecolor{currentfill}{rgb}{0.000000,0.000000,0.000000}%
\pgfsetfillcolor{currentfill}%
\pgfsetlinewidth{0.803000pt}%
\definecolor{currentstroke}{rgb}{0.000000,0.000000,0.000000}%
\pgfsetstrokecolor{currentstroke}%
\pgfsetdash{}{0pt}%
\pgfsys@defobject{currentmarker}{\pgfqpoint{-0.048611in}{0.000000in}}{\pgfqpoint{-0.000000in}{0.000000in}}{%
\pgfpathmoveto{\pgfqpoint{-0.000000in}{0.000000in}}%
\pgfpathlineto{\pgfqpoint{-0.048611in}{0.000000in}}%
\pgfusepath{stroke,fill}%
}%
\begin{pgfscope}%
\pgfsys@transformshift{0.643077in}{1.893891in}%
\pgfsys@useobject{currentmarker}{}%
\end{pgfscope}%
\end{pgfscope}%
\begin{pgfscope}%
\definecolor{textcolor}{rgb}{0.000000,0.000000,0.000000}%
\pgfsetstrokecolor{textcolor}%
\pgfsetfillcolor{textcolor}%
\pgftext[x=0.333778in, y=1.851682in, left, base]{\color{textcolor}\sffamily\fontsize{8.000000}{9.600000}\selectfont 800}%
\end{pgfscope}%
\begin{pgfscope}%
\pgfsetbuttcap%
\pgfsetroundjoin%
\definecolor{currentfill}{rgb}{0.000000,0.000000,0.000000}%
\pgfsetfillcolor{currentfill}%
\pgfsetlinewidth{0.803000pt}%
\definecolor{currentstroke}{rgb}{0.000000,0.000000,0.000000}%
\pgfsetstrokecolor{currentstroke}%
\pgfsetdash{}{0pt}%
\pgfsys@defobject{currentmarker}{\pgfqpoint{-0.048611in}{0.000000in}}{\pgfqpoint{-0.000000in}{0.000000in}}{%
\pgfpathmoveto{\pgfqpoint{-0.000000in}{0.000000in}}%
\pgfpathlineto{\pgfqpoint{-0.048611in}{0.000000in}}%
\pgfusepath{stroke,fill}%
}%
\begin{pgfscope}%
\pgfsys@transformshift{0.643077in}{2.250404in}%
\pgfsys@useobject{currentmarker}{}%
\end{pgfscope}%
\end{pgfscope}%
\begin{pgfscope}%
\definecolor{textcolor}{rgb}{0.000000,0.000000,0.000000}%
\pgfsetstrokecolor{textcolor}%
\pgfsetfillcolor{textcolor}%
\pgftext[x=0.263086in, y=2.208195in, left, base]{\color{textcolor}\sffamily\fontsize{8.000000}{9.600000}\selectfont 1000}%
\end{pgfscope}%
\begin{pgfscope}%
\definecolor{textcolor}{rgb}{0.000000,0.000000,0.000000}%
\pgfsetstrokecolor{textcolor}%
\pgfsetfillcolor{textcolor}%
\pgftext[x=0.207530in,y=1.376546in,,bottom,rotate=90.000000]{\color{textcolor}\sffamily\fontsize{8.000000}{9.600000}\selectfont count}%
\end{pgfscope}%
\begin{pgfscope}%
\pgfsetrectcap%
\pgfsetmiterjoin%
\pgfsetlinewidth{0.803000pt}%
\definecolor{currentstroke}{rgb}{0.000000,0.000000,0.000000}%
\pgfsetstrokecolor{currentstroke}%
\pgfsetdash{}{0pt}%
\pgfpathmoveto{\pgfqpoint{0.643077in}{0.467838in}}%
\pgfpathlineto{\pgfqpoint{0.643077in}{2.285253in}}%
\pgfusepath{stroke}%
\end{pgfscope}%
\begin{pgfscope}%
\pgfsetrectcap%
\pgfsetmiterjoin%
\pgfsetlinewidth{0.803000pt}%
\definecolor{currentstroke}{rgb}{0.000000,0.000000,0.000000}%
\pgfsetstrokecolor{currentstroke}%
\pgfsetdash{}{0pt}%
\pgfpathmoveto{\pgfqpoint{3.602812in}{0.467838in}}%
\pgfpathlineto{\pgfqpoint{3.602812in}{2.285253in}}%
\pgfusepath{stroke}%
\end{pgfscope}%
\begin{pgfscope}%
\pgfsetrectcap%
\pgfsetmiterjoin%
\pgfsetlinewidth{0.803000pt}%
\definecolor{currentstroke}{rgb}{0.000000,0.000000,0.000000}%
\pgfsetstrokecolor{currentstroke}%
\pgfsetdash{}{0pt}%
\pgfpathmoveto{\pgfqpoint{0.643077in}{0.467838in}}%
\pgfpathlineto{\pgfqpoint{3.602812in}{0.467838in}}%
\pgfusepath{stroke}%
\end{pgfscope}%
\begin{pgfscope}%
\pgfsetrectcap%
\pgfsetmiterjoin%
\pgfsetlinewidth{0.803000pt}%
\definecolor{currentstroke}{rgb}{0.000000,0.000000,0.000000}%
\pgfsetstrokecolor{currentstroke}%
\pgfsetdash{}{0pt}%
\pgfpathmoveto{\pgfqpoint{0.643077in}{2.285253in}}%
\pgfpathlineto{\pgfqpoint{3.602812in}{2.285253in}}%
\pgfusepath{stroke}%
\end{pgfscope}%
\begin{pgfscope}%
\pgfsetbuttcap%
\pgfsetmiterjoin%
\definecolor{currentfill}{rgb}{1.000000,1.000000,1.000000}%
\pgfsetfillcolor{currentfill}%
\pgfsetfillopacity{0.800000}%
\pgfsetlinewidth{1.003750pt}%
\definecolor{currentstroke}{rgb}{0.800000,0.800000,0.800000}%
\pgfsetstrokecolor{currentstroke}%
\pgfsetstrokeopacity{0.800000}%
\pgfsetdash{}{0pt}%
\pgfpathmoveto{\pgfqpoint{0.720855in}{1.544021in}}%
\pgfpathlineto{\pgfqpoint{1.514182in}{1.544021in}}%
\pgfpathquadraticcurveto{\pgfqpoint{1.536404in}{1.544021in}}{\pgfqpoint{1.536404in}{1.566243in}}%
\pgfpathlineto{\pgfqpoint{1.536404in}{2.207476in}}%
\pgfpathquadraticcurveto{\pgfqpoint{1.536404in}{2.229698in}}{\pgfqpoint{1.514182in}{2.229698in}}%
\pgfpathlineto{\pgfqpoint{0.720855in}{2.229698in}}%
\pgfpathquadraticcurveto{\pgfqpoint{0.698633in}{2.229698in}}{\pgfqpoint{0.698633in}{2.207476in}}%
\pgfpathlineto{\pgfqpoint{0.698633in}{1.566243in}}%
\pgfpathquadraticcurveto{\pgfqpoint{0.698633in}{1.544021in}}{\pgfqpoint{0.720855in}{1.544021in}}%
\pgfpathlineto{\pgfqpoint{0.720855in}{1.544021in}}%
\pgfpathclose%
\pgfusepath{stroke,fill}%
\end{pgfscope}%
\begin{pgfscope}%
\pgfsetbuttcap%
\pgfsetmiterjoin%
\definecolor{currentfill}{rgb}{0.121569,0.466667,0.705882}%
\pgfsetfillcolor{currentfill}%
\pgfsetlinewidth{0.000000pt}%
\definecolor{currentstroke}{rgb}{0.000000,0.000000,0.000000}%
\pgfsetstrokecolor{currentstroke}%
\pgfsetstrokeopacity{0.000000}%
\pgfsetdash{}{0pt}%
\pgfpathmoveto{\pgfqpoint{0.743077in}{2.100835in}}%
\pgfpathlineto{\pgfqpoint{0.965299in}{2.100835in}}%
\pgfpathlineto{\pgfqpoint{0.965299in}{2.178613in}}%
\pgfpathlineto{\pgfqpoint{0.743077in}{2.178613in}}%
\pgfpathlineto{\pgfqpoint{0.743077in}{2.100835in}}%
\pgfpathclose%
\pgfusepath{fill}%
\end{pgfscope}%
\begin{pgfscope}%
\definecolor{textcolor}{rgb}{0.000000,0.000000,0.000000}%
\pgfsetstrokecolor{textcolor}%
\pgfsetfillcolor{textcolor}%
\pgftext[x=1.054188in,y=2.100835in,left,base]{\color{textcolor}\sffamily\fontsize{8.000000}{9.600000}\selectfont L=250}%
\end{pgfscope}%
\begin{pgfscope}%
\pgfsetbuttcap%
\pgfsetmiterjoin%
\definecolor{currentfill}{rgb}{1.000000,0.498039,0.054902}%
\pgfsetfillcolor{currentfill}%
\pgfsetlinewidth{0.000000pt}%
\definecolor{currentstroke}{rgb}{0.000000,0.000000,0.000000}%
\pgfsetstrokecolor{currentstroke}%
\pgfsetstrokeopacity{0.000000}%
\pgfsetdash{}{0pt}%
\pgfpathmoveto{\pgfqpoint{0.743077in}{1.937749in}}%
\pgfpathlineto{\pgfqpoint{0.965299in}{1.937749in}}%
\pgfpathlineto{\pgfqpoint{0.965299in}{2.015527in}}%
\pgfpathlineto{\pgfqpoint{0.743077in}{2.015527in}}%
\pgfpathlineto{\pgfqpoint{0.743077in}{1.937749in}}%
\pgfpathclose%
\pgfusepath{fill}%
\end{pgfscope}%
\begin{pgfscope}%
\definecolor{textcolor}{rgb}{0.000000,0.000000,0.000000}%
\pgfsetstrokecolor{textcolor}%
\pgfsetfillcolor{textcolor}%
\pgftext[x=1.054188in,y=1.937749in,left,base]{\color{textcolor}\sffamily\fontsize{8.000000}{9.600000}\selectfont L=500}%
\end{pgfscope}%
\begin{pgfscope}%
\pgfsetbuttcap%
\pgfsetmiterjoin%
\definecolor{currentfill}{rgb}{0.172549,0.627451,0.172549}%
\pgfsetfillcolor{currentfill}%
\pgfsetlinewidth{0.000000pt}%
\definecolor{currentstroke}{rgb}{0.000000,0.000000,0.000000}%
\pgfsetstrokecolor{currentstroke}%
\pgfsetstrokeopacity{0.000000}%
\pgfsetdash{}{0pt}%
\pgfpathmoveto{\pgfqpoint{0.743077in}{1.774663in}}%
\pgfpathlineto{\pgfqpoint{0.965299in}{1.774663in}}%
\pgfpathlineto{\pgfqpoint{0.965299in}{1.852441in}}%
\pgfpathlineto{\pgfqpoint{0.743077in}{1.852441in}}%
\pgfpathlineto{\pgfqpoint{0.743077in}{1.774663in}}%
\pgfpathclose%
\pgfusepath{fill}%
\end{pgfscope}%
\begin{pgfscope}%
\definecolor{textcolor}{rgb}{0.000000,0.000000,0.000000}%
\pgfsetstrokecolor{textcolor}%
\pgfsetfillcolor{textcolor}%
\pgftext[x=1.054188in,y=1.774663in,left,base]{\color{textcolor}\sffamily\fontsize{8.000000}{9.600000}\selectfont L=1000}%
\end{pgfscope}%
\begin{pgfscope}%
\pgfsetbuttcap%
\pgfsetmiterjoin%
\definecolor{currentfill}{rgb}{0.839216,0.152941,0.156863}%
\pgfsetfillcolor{currentfill}%
\pgfsetlinewidth{0.000000pt}%
\definecolor{currentstroke}{rgb}{0.000000,0.000000,0.000000}%
\pgfsetstrokecolor{currentstroke}%
\pgfsetstrokeopacity{0.000000}%
\pgfsetdash{}{0pt}%
\pgfpathmoveto{\pgfqpoint{0.743077in}{1.611577in}}%
\pgfpathlineto{\pgfqpoint{0.965299in}{1.611577in}}%
\pgfpathlineto{\pgfqpoint{0.965299in}{1.689355in}}%
\pgfpathlineto{\pgfqpoint{0.743077in}{1.689355in}}%
\pgfpathlineto{\pgfqpoint{0.743077in}{1.611577in}}%
\pgfpathclose%
\pgfusepath{fill}%
\end{pgfscope}%
\begin{pgfscope}%
\definecolor{textcolor}{rgb}{0.000000,0.000000,0.000000}%
\pgfsetstrokecolor{textcolor}%
\pgfsetfillcolor{textcolor}%
\pgftext[x=1.054188in,y=1.611577in,left,base]{\color{textcolor}\sffamily\fontsize{8.000000}{9.600000}\selectfont L=2000}%
\end{pgfscope}%
\end{pgfpicture}%
\makeatother%
\endgroup%

	\caption{Распределение пиков магнитной восприимчивости}
	\label{fig:MS_peaks_distr}
\end{figure}

\section{Замеры}

\subsection{$U = 1$}
Для вычислений я сгенерировал по 1000 реплик длины 250, 500, 1000, 2000. При моделировании методом Монте-Карло делал 100000--300000 шагов на отжиг, и 500000--1000000 шагов для замеров. 
Оказалось что достаточно большая часть этих конформаций неплотные, то есть их свойства ближе к свойствам одномерной решётки, чем двумерной. При попытке посчитать среднее значения кумулянта Биндера неплотные конформации Сильно влияли на значение кумулянта, увеличивая погрешность от реплики к реплике.

\begin{figure}[h]
	\centering
	\begin{subfigure}[t]{0.48\textwidth}
		\includegraphics[width=\textwidth]{../images/dense_cumulant.png} 
		\includegraphics[width=\textwidth]{../images/dense_magnetization.png} 
		\caption{Плотная}
	\end{subfigure}
	\begin{subfigure}[t]{0.48\textwidth}
		\includegraphics[width=\textwidth]{../images/loose_cumulant.png} 
		\includegraphics[width=\textwidth]{../images/loose_magnetization.png} 
		\caption{Неплотная}
	\end{subfigure}
	\caption{Пример кумулянта и намагниченности плотной и неплотной конформаций}
\end{figure}


\subsubsection{Разделение конформаций}

Для отделения плотных конформаций от остальных было предложено вычислять их радиус инерции.$R = \sqrt{\frac{1}{n}\sum_{i=1}^{n}r_{i}^{2}}$, где $r_i$ это расстояние от узла конформации до её центра масс. Однако при рассмотрении большого количества конформаций оказалось, что маленький радиус инерции не гарантирует хорошую намагниченность конформации

\begin{figure}[h]
	\centering
	\includegraphics[width=\textwidth]{../images/mag2_to_R_L250.png} 
	\caption{Корреляция намагниченности конформаций при $\beta = 1$ и радиуса инерции для конформаций длины $L = 250$}
	\label{fig:mag2_to_R} 
\end{figure}

На рис.\ref{fig:mag2_to_R}, при $R \approx 0.6$ $m^2$ принимают любые значения от $0.2$ до $1.0$. Значит, при разделении конформации только по радиусу инерции, мы либо будем отбрасывать намагничивающиеся конформации, либо оставлять не намагничивающиеся

\paragraph{Кластеризованные конформации}

\begin{figure}[h!]
	\centering
	\includegraphics[width=0.47\textwidth]{../images/2Cluster_conformation.png}
	\includegraphics[width=0.47\textwidth]{../images/3Cluster_conformation.png} 
	\caption{Пример плотной немагнитной конформации}
	\label{fig:synth_cluster_conf}
\end{figure}

На искусственном примере рис.\ref{fig:synth_cluster_conf} показана одна из причин, по которой плотная конформация может плохо намагничиваться. Тут имеется несколько крупных двумерных кластеров, соединённых одномерной цепочкой. И не смотря на то, что сами по себе эти кластеры намагничиваются, направление спинов в них слабо связано, из-за чего спины в разных кластерах с большой вероятностью будут направлены в противоположные стороны.


Следующей задачей стало проанализировать конформации на количество и размеры кластеров, а так же мостов(одномерных сегментов) которые и соединяют. Первым вариантом было искать классические мосты -- спины, при удалении увеличивается число компонент связанности. Однако такой способ не дал желаемого эффекта, так как кластеры могут быть соединены более чем одним мостом. И например на конформации из рис. \ref{fig:clusters_and_bridges} данный способ не выделяет ни одного моста. Следующий алгоритм выделял как мосты все цепочки спинов у которых 1 или 2 соседа, однако при таком подходе мы получаем мосты, которые соединяют один и тот же кластер. Такие мосты не разделяют кластеры и незначительно влияют на намагниченность конформации.

Итоговая версия алгоритма выделяет как мосты все спины, которые имеют 1 или 2 соседа, и затем добавляет мосты, которые соединяют один и тот же кластер, к этому же кластеру.

\paragraph{Алгоритм разбиения на мосты и кластеры}
\begin{enumerate}
	\item Отметить все спины с 1 или 2 соседями как мосты.
	\item Создаём массив, где отмечаем посещённые спины. Создаём массив где для каждого спина будем писать номер его кластера. И переменную отвечающую за текущую длину моста $l$. Изначально все спины не посещены, $l = 0$.
	\item Начинаем идти по конформации от первой вершины.
	\begin{enumerate}
	
		\item Если спин отмечен как мост, то увеличиваем $l$ на 1
		\item Если спин не отмечен как мост, и не посещён. Увеличиваем счётчик кластеров на 1 и запускаем DFS(Алгоритм DFS описан ниже). Если $l > 0$ увеличиваем счётчик мостов на 1, длина нового моста $= l$. Обнуляем $l$
		\item Если спин не мост, уже посещён, последний встреченный спин, не являющийся мостом, принадлежит тому же кластеру и текущая длина моста $l > 0$. Значит этот мост соединяет один и тот же кластер. Поэтому добавляем предыдущие $l$ спинов к этому кластеру, обнуляем $l$.
		\item Если спин не мост, посещён, но номер кластера отличается от последнего встреченного кластера. Если $l > 0$ увеличиваем счётчик мостов на 1, длина нового моста $= l$. Обнуляем $l$
	\end{enumerate}
	\item пройдя всю конформацию, если $l > 0$, создаём ещё один мост
\end{enumerate}

\textbf{Алгоритм DFS}
\begin{enumerate}
	\item Заходим в вершину.
	\item Отмечаем вершину как посещённую.
	\item Отмечаем номер её кластера.
	\item Увеличиваем счётчик размера текущего кластера на 1.
	\item Заходим во все соседние не посещённые вершины не мосты.
\end{enumerate}

В данном алгоритме мы пользуемся тем, что мосты обязательно образуются из подряд идущих спинов конформации. Поэтому чтобы определить соединяет ли мост один и тот же кластер, нам достаточно, идя по конформации, запоминать последний встреченный кластер и сравнивать его с новым.

\begin{figure}[h]
	\centering
	\includegraphics[width=0.70\textwidth]{../images/bridges_example_1.png}  
	\caption{Пример реальных конформаций с маленьким радиусом инерции и маленькой намагниченностью, с отмеченными мостами}
	\label{fig:clusters_and_bridges}
\end{figure}


\subsection{Результаты разбиения на кластеры}
Результаты анализа связи между намагниченностью и количеством и размерами кластеров и мостов подтверждают сказанное выше. У конформаций с большим числом кластеров обычно намагниченность ниже чем у конформаций с одним большим кластером. 

Я рассмотрел несколько параметров: количество мостов, количество кластеров, суммарная длина мостов, размер наибольшего кластера. Наилучшим способом разделения конформаций на магнитные и немагнитные сейчас выглядит именно разделение по размеру наибольшего кластера. Как видно на рис \ref{fig:mag_from_max_cluster} при разбиении по данному параметру разброс намагниченности значительно ниже, чем при разбиении по радиусу инерции. Данный параметр можно легко масштабировать для разных длин конформаций.

\begin{figure}[h]
	\centering
	\caption{График размера наибольшего кластера и квадрата намагниченности для 10000 конформаций длины 1000}
	\includegraphics[scale=1]{../images/mag_from_cluster_size.png} 
	\label{fig:mag_from_max_cluster}
\end{figure}




\subsection{Кумулянт и точка перехода}

Кумулянт Биндера для одной реплики при заданной температуре вычисляется по формуле $U = 1 - \frac{\langle m^4\rangle}{3\langle m^2\rangle ^2}$. Дальше Значения усредняются между репликами при каждой температуре $\langle U\rangle = \frac{1}{n}\sum_{i=1}^{n}U_i$ 
Погрешность кумулянта от реплики к реплике вычисляется как среднеквадратичное отклонение по формуле $\sqrt{\frac{1}{n}\sum_{i=1}^{n}(\langle U\rangle - U_i)^2}$


Эксперименты с разделением конформаций по радиусу инерции показали, что таким образом можно выделить наборы конформаций разных длин так, что для них будет возможно найти точку перехода.

\begin{figure}[h]
	\centering
	\includegraphics[width=1\textwidth]{../images/Cumulant_big.png} 
	\includegraphics[width=1\textwidth]{../images/Cumulant_beta0.4_0.6.png} 
	\caption{Значения куулянтов после отбрасывания конформаций по радиусам}
\end{figure}


\begin{thebibliography}{9}
\bibitem{github}
 Ссылка на репозиторий: https://github.com/MoskalenkoRomanBorisovich/Ising-on-random-conformation 
\end{thebibliography}

\end{document}