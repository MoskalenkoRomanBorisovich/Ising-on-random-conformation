\documentclass[14pt]{extarticle}

%%% Работа с русским языком
\usepackage{cmap}					% поиск в PDF
\usepackage{mathtext} 				% русские буквы в фомулах
\usepackage[T2A]{fontenc}			% кодировка
\usepackage[utf8]{inputenc}			% кодировка исходного текста
\usepackage[english,russian]{babel}	% локализация и переносы

%%% Дополнительная работа с математикой
\usepackage{amsfonts,amssymb,amsthm,mathtools} % AMS
\usepackage{amsmath}
\usepackage{icomma} % "Умная" запятая: $0,2$ --- число, $0, 2$ --- перечисление


% matplotlib plotss
\usepackage{pgf}
\usepackage[utf8]{inputenc}\DeclareUnicodeCharacter{2212}{-}

%% Номера формул
%\mathtoolsset{showonlyrefs=true} % Показывать номера только у тех формул, на которые есть \eqref{} в тексте.

\usepackage{hyperref}
\hypersetup{
    colorlinks=false,
    linkcolor=blue,
    filecolor=magenta,      
    urlcolor=cyan,
}

\usepackage{float}


%% Шрифты
\usepackage{euscript}	 % Шрифт Евклид
\usepackage{mathrsfs} % Красивый матшрифт

%%% Работа с картинками
\usepackage{graphicx}  % Для вставки рисунков
\graphicspath{{images/}{images2/}}  % папки с картинками
\setlength\fboxsep{3pt} % Отступ рамки \fbox{} от рисунка
\setlength\fboxrule{1pt} % Толщина линий рамки \fbox{}
\usepackage{wrapfig} % Обтекание рисунков и таблиц текстом
\usepackage{caption}
\usepackage{subcaption}

\usepackage{pdfpages}

\captionsetup{labelsep=period} %. вместо : в рис

\title{Отчёт по проекту Решеточные модели макромолекул}
%\title{Информация о замерах}
\author{Москаленко Р. Б.}
\date{24.05.2021}

\begin{document}

\includepdf{otchet_header.pdf}

\section{Введение}
%Модель Изинга используется для моделирования и изучения термодинамических свойств. Поведение структуры в модели Изинга сильно зависит от её геометрии. Так например на одно мерных моделях не происходит фазовый переход, но на двумерных моделях переход есть. Но что происходит в промежуточных размерностях? Например если взять какую-то последовательность узлов на двумерной решётке. Именно это и является главным вопросом в данном проекте. 
Рассмотрим конформацию(несамопересекающуюся последовательность узлов) на двумерной решётке. Такие конформации можно рассматривать как термодинамическую систему, основанную на модели Изинга, для которых существуют две фазы: плотная(глобулярная) и развёрнутая. Эти фазы соответствуют низким и высоким температурам системы.

\begin{figure}[h]
	\centering
	\includegraphics[width=0.45\textwidth]{../images/loose_conf.png}
	\includegraphics[width=0.45\textwidth]{../images/dense_conf.png} 
	\caption{Пример неплотной и плотной конформации}
\end{figure}

Если посмотреть на изображения конформаций каждого вида, хорошо видно, что плотные конформации по структуре близки с двумерным решёткам, где у каждого узла имеется множество соседей, и развёрнутые конформации наоборот близки к одномерным структурам, где узлы у которых больше 2 соседей встречаются редко. Соответственно можно предположить, что плотные конформации будут иметь свойства схожие с двумерными решётками, а развёрнутые с одномерными. В двумерных решётках наблюдается магнитный фазовый переход, в то время как в одномерных решётках переход не происходит. Цель данного исследования определить наличие магнитного перехода в плотных конформациях.


\subsection{Модель}
В данной модели мы рассматриваем ансамбли конформаций: множества конформаций одинаковой длинны $L$, полученные при одинаковых температурах. Мы получаем конформации используя алгоритм SAW.
На каждой из конформаций строится модель Изинга \cite{ising}. В каждой вершине размещается спин, который может принимать одно из двух значений: $+1, -1$.
Гамильтониан данной системы имеет вид
\[H = -J\sum_{\langle i, j\rangle}{\sigma_i\sigma_j} - h\sum_i{\sigma_i} \]

где $i, j$ индексы соседних узлов у, $J$- коэффициент взаимодействия $h$ - воздействие внешнего поля.

Статистическая сумма
\[Z = \sum_{\{\sigma\}} e^{-H(\sigma)\beta}, \beta = \frac{1}{kT}\]
где $\{\sigma\}$ - множество всех возможных наборов значений спинов.

Намагниченность и энергия каждого состояния считаются по следующим формулам

\[ 
E = -J\sum_{i, j} \sigma_i \sigma_j, 
M = \sum_i \sigma_i
\]

Средняя намагниченность системы

\[
\langle M \rangle = \frac{1}{Z}  \sum_{\{\sigma\}} M e^{-H_{(\sigma)}\beta}
\]

\subsection{Метод Монте-Карло}
Для расчёта модели Изинга используется метод Монте-Карло. Были реализованы версии с односпиновым и кластернным апдейтом \cite{wolf_algorithm}. Код представлен в репозитории github \cite{github}. В итоге для измерений используется кластерная версия. Благодаря отказоустойчивости она работает значительно быстрее, и быстрее сходится, особенно при низких температурах.

Алгоритм с кластерным апдейтом работает следующим образом. На каждой итерации мы выбираем случайный спин и начиная с него начинаем строить кластер из одинаково направленных спинов, добавляя новые спины в кластер с определённой вероятностью. затем мы меняем значения спинов в кластере на противоположные.

Чтобы вычислить намагниченность, мы сначала случайным образом инициализируем спины, затем делаем некоторое число шагов для отжига модели. Далее на каждом шаге мы замеряем намагниченность, и после выполнения определённого числа шагов, усредняем полученные значения. Так как средняя намагниченность равна 0, имеет смысл рассматривать модуль квадрат намагниченности.

\[
\langle M^2\rangle = \frac{1}{n} \sum_{\{\sigma\}} \left( \sum_i \sigma_i \right)^2, \\
\langle |M|\rangle = \frac{1}{n} \sum_{\{\sigma\}} \left| \sum_i \sigma_i \right|
\]

% \section{Используемые алгоритмы}

Реализованы два алгоритма обновления спинов. Односпиновый и кластерный апдейт. Оба алгоритма работают на произвольном графе, используя таблицу соседей. Алгоритмы реализованы как отдельные библиотеки для Python, и написаны с использованием технологии Cython для ускорения работы. Кластерный апдейт является более эффективным по времени работы и количеству шагов, которые необходимо выполнить для хорошей сходимости модели.

\subsection{Проверка алгоритмов}

Чтобы убедиться что алгоритмы работают правильно мы проверили, что оба алгоритма дают одинаковые результаты на одних и тех же конформациях, так же сравнил их с точными решениями для одномерной модели Изинга.

Результаты замеров кластерным и односпиновым апдейтом совпадают в пределах погрешности.

\begin{figure}[H]
	\centering
	\includegraphics[width = 0.45\textwidth]{../images/1spin_&_cluster_ene.png} 
	\includegraphics[width = 0.45\textwidth]{../images/1spin_&_cluster_ene_dif.png} 
	\includegraphics[width = 0.45\textwidth]{../images/1spin_&_cluster_mag.png} 
	\includegraphics[width = 0.45\textwidth]{../images/1spin_&_cluster_mag_dif.png} 
	%% add magnetization
	\caption{кластерный и односпиновый апдейт}
\end{figure}

Для сравнения с точными значениями для одномерной модели Изинга, мы используем замкнутый квадратный контур. Данная конформация по свойствам полностью совпадает с одномерной моделью Изинга с открытыми граничными условиями.

\begin{figure}[H]
	\centering
	\begin{subfigure}[t]{0.45\textwidth}
		\includegraphics[width = \textwidth]{../images/1D_conf.png} 
		\caption{Конформация эмитирующая одномерную модель}
	\end{subfigure}
	\includegraphics[width = 0.45\textwidth]{../images/1D_ene.png}
	\includegraphics[width = 0.45\textwidth]{../images/1D_ene_diff.png} 
	%% add magnetization
	\caption{Сравнение с точным решением одномерной модели}
\end{figure}

Так же был написан код, точно вычисляющий энергию системы путём полного перебора всех её состояний. Сравнение на маленьких конформациях (длина 10) даёт одинаковые результаты.

%TODO втавить результаты поного перебора

Примеры с использованием кластерного апдейта добавлены в библиотеку \texttt{mc\_lib}.
\subsection{Текущие задачи}
\paragraph{Магнитная восприимчивость клубков}
Для определения точки магнитного перехода в конформациях мы используем магнитную восприимчивость. В точке магнитного перехода, магнитная восприимчивость должна иметь пи. И наоборот, если магнитный фазовый переход отсутствует, магнитная восприимчивость не должна иметь пиков. Так как мы предполагаем, что конформации вида клубок не имеют фазового магнитного перехода, мы хотим проверить что у них так же отсутствуют пики магнитной восприимчивости. Так же мы хотим сравнить магнитную восприимчивость конформаций клубок и одномерной модели изинга. 

\paragraph{Кластеризованные конформации}
Так же ранее было обнаружено что наличие в структуре конформации кластеров, отделённых от остальной части конформации одномерными цепочками, оказывает большое влияние на её магнитные свойства. Например даже плотные конформации могут иметь низкую намагниченность, если большая их часть разбита на два кластера равных размеров.

Мы рассмотрели влияние такой кластеризации на магнитные свойства конформаций, на сгенерированных наборах, которые мы рассматривали для определения точки перехода. Однако для более подробного изучения влияния кластеризации на магнитные свойства, будет эффективнее рассматривать специально сгенерированные конформации, с желаемой нам структурой.

\section{Магнитная восприимчивость клубков}
Магнитная восприимчивость это отношения изменения намагниченности к изменению внешнего магнитного поля. В пиках магнитной восприимчивости должна находится точка перехода.

Магнитная восприимчивость для данной модели вычисляется по следующей формуле.
\[
	\chi = \frac{\partial\langle |M|\rangle}{\partial h} = \beta \left(\langle M^2\rangle - \langle |M| \rangle^2 \right)
\]
Для одномерной цепочки Изинга с открытыми граничными условиями существует точное решение из которого можно вывести формулу Магнитной восприимчивости

\[
    \chi = \frac{\beta}{2} (2Ne^{2\beta J} - e^{4\beta J} + 1) + \frac{\beta}{2} \tanh^{N-1}(\beta J)(e^{4\beta J} - 2 e^{2\beta J} + 1)
\]
Где $N$ это количество спинов в цепочке. 

Мы рассмотрели конформации длинны 250, 500, 1000, 2000, по 1000 конформаций каждой длины

Действительно,у большинства конформаций полученных при $U=0.1$ отсутствуют пики, и график магнитной восприимчивости имеет такой же вид как и у одномерной модели изинига. Пример графиков представлен на рис. \ref{fig:MS_1D_comparison}. На этом же графике представлен пример масштабирования магнитной восприимчивости конформации, для сравнения с одномерной цепочкой.

\begin{figure}[ht]
	\centering
	%% Creator: Matplotlib, PGF backend
%%
%% To include the figure in your LaTeX document, write
%%   \input{<filename>.pgf}
%%
%% Make sure the required packages are loaded in your preamble
%%   \usepackage{pgf}
%%
%% Also ensure that all the required font packages are loaded; for instance,
%% the lmodern package is sometimes necessary when using math font.
%%   \usepackage{lmodern}
%%
%% Figures using additional raster images can only be included by \input if
%% they are in the same directory as the main LaTeX file. For loading figures
%% from other directories you can use the `import` package
%%   \usepackage{import}
%%
%% and then include the figures with
%%   \import{<path to file>}{<filename>.pgf}
%%
%% Matplotlib used the following preamble
%%   
%%   \usepackage{fontspec}
%%   \setmainfont{DejaVuSerif.ttf}[Path=\detokenize{/home/roman/anaconda3/envs/ising/lib/python3.8/site-packages/matplotlib/mpl-data/fonts/ttf/}]
%%   \setsansfont{DejaVuSans.ttf}[Path=\detokenize{/home/roman/anaconda3/envs/ising/lib/python3.8/site-packages/matplotlib/mpl-data/fonts/ttf/}]
%%   \setmonofont{DejaVuSansMono.ttf}[Path=\detokenize{/home/roman/anaconda3/envs/ising/lib/python3.8/site-packages/matplotlib/mpl-data/fonts/ttf/}]
%%   \makeatletter\@ifpackageloaded{underscore}{}{\usepackage[strings]{underscore}}\makeatother
%%
\begingroup%
\makeatletter%
\begin{pgfpicture}%
\pgfpathrectangle{\pgfpointorigin}{\pgfqpoint{5.217998in}{1.930900in}}%
\pgfusepath{use as bounding box, clip}%
\begin{pgfscope}%
\pgfsetbuttcap%
\pgfsetmiterjoin%
\definecolor{currentfill}{rgb}{1.000000,1.000000,1.000000}%
\pgfsetfillcolor{currentfill}%
\pgfsetlinewidth{0.000000pt}%
\definecolor{currentstroke}{rgb}{1.000000,1.000000,1.000000}%
\pgfsetstrokecolor{currentstroke}%
\pgfsetdash{}{0pt}%
\pgfpathmoveto{\pgfqpoint{0.000000in}{0.000000in}}%
\pgfpathlineto{\pgfqpoint{5.217998in}{0.000000in}}%
\pgfpathlineto{\pgfqpoint{5.217998in}{1.930900in}}%
\pgfpathlineto{\pgfqpoint{0.000000in}{1.930900in}}%
\pgfpathlineto{\pgfqpoint{0.000000in}{0.000000in}}%
\pgfpathclose%
\pgfusepath{fill}%
\end{pgfscope}%
\begin{pgfscope}%
\pgfsetbuttcap%
\pgfsetmiterjoin%
\definecolor{currentfill}{rgb}{1.000000,1.000000,1.000000}%
\pgfsetfillcolor{currentfill}%
\pgfsetlinewidth{0.000000pt}%
\definecolor{currentstroke}{rgb}{0.000000,0.000000,0.000000}%
\pgfsetstrokecolor{currentstroke}%
\pgfsetstrokeopacity{0.000000}%
\pgfsetdash{}{0pt}%
\pgfpathmoveto{\pgfqpoint{0.678396in}{0.467838in}}%
\pgfpathlineto{\pgfqpoint{2.696397in}{0.467838in}}%
\pgfpathlineto{\pgfqpoint{2.696397in}{1.830900in}}%
\pgfpathlineto{\pgfqpoint{0.678396in}{1.830900in}}%
\pgfpathlineto{\pgfqpoint{0.678396in}{0.467838in}}%
\pgfpathclose%
\pgfusepath{fill}%
\end{pgfscope}%
\begin{pgfscope}%
\pgfsetbuttcap%
\pgfsetroundjoin%
\definecolor{currentfill}{rgb}{0.000000,0.000000,0.000000}%
\pgfsetfillcolor{currentfill}%
\pgfsetlinewidth{0.803000pt}%
\definecolor{currentstroke}{rgb}{0.000000,0.000000,0.000000}%
\pgfsetstrokecolor{currentstroke}%
\pgfsetdash{}{0pt}%
\pgfsys@defobject{currentmarker}{\pgfqpoint{0.000000in}{-0.048611in}}{\pgfqpoint{0.000000in}{0.000000in}}{%
\pgfpathmoveto{\pgfqpoint{0.000000in}{0.000000in}}%
\pgfpathlineto{\pgfqpoint{0.000000in}{-0.048611in}}%
\pgfusepath{stroke,fill}%
}%
\begin{pgfscope}%
\pgfsys@transformshift{0.973962in}{0.467838in}%
\pgfsys@useobject{currentmarker}{}%
\end{pgfscope}%
\end{pgfscope}%
\begin{pgfscope}%
\definecolor{textcolor}{rgb}{0.000000,0.000000,0.000000}%
\pgfsetstrokecolor{textcolor}%
\pgfsetfillcolor{textcolor}%
\pgftext[x=0.973962in,y=0.370616in,,top]{\color{textcolor}\sffamily\fontsize{8.000000}{9.600000}\selectfont 0.2}%
\end{pgfscope}%
\begin{pgfscope}%
\pgfsetbuttcap%
\pgfsetroundjoin%
\definecolor{currentfill}{rgb}{0.000000,0.000000,0.000000}%
\pgfsetfillcolor{currentfill}%
\pgfsetlinewidth{0.803000pt}%
\definecolor{currentstroke}{rgb}{0.000000,0.000000,0.000000}%
\pgfsetstrokecolor{currentstroke}%
\pgfsetdash{}{0pt}%
\pgfsys@defobject{currentmarker}{\pgfqpoint{0.000000in}{-0.048611in}}{\pgfqpoint{0.000000in}{0.000000in}}{%
\pgfpathmoveto{\pgfqpoint{0.000000in}{0.000000in}}%
\pgfpathlineto{\pgfqpoint{0.000000in}{-0.048611in}}%
\pgfusepath{stroke,fill}%
}%
\begin{pgfscope}%
\pgfsys@transformshift{1.381639in}{0.467838in}%
\pgfsys@useobject{currentmarker}{}%
\end{pgfscope}%
\end{pgfscope}%
\begin{pgfscope}%
\definecolor{textcolor}{rgb}{0.000000,0.000000,0.000000}%
\pgfsetstrokecolor{textcolor}%
\pgfsetfillcolor{textcolor}%
\pgftext[x=1.381639in,y=0.370616in,,top]{\color{textcolor}\sffamily\fontsize{8.000000}{9.600000}\selectfont 0.4}%
\end{pgfscope}%
\begin{pgfscope}%
\pgfsetbuttcap%
\pgfsetroundjoin%
\definecolor{currentfill}{rgb}{0.000000,0.000000,0.000000}%
\pgfsetfillcolor{currentfill}%
\pgfsetlinewidth{0.803000pt}%
\definecolor{currentstroke}{rgb}{0.000000,0.000000,0.000000}%
\pgfsetstrokecolor{currentstroke}%
\pgfsetdash{}{0pt}%
\pgfsys@defobject{currentmarker}{\pgfqpoint{0.000000in}{-0.048611in}}{\pgfqpoint{0.000000in}{0.000000in}}{%
\pgfpathmoveto{\pgfqpoint{0.000000in}{0.000000in}}%
\pgfpathlineto{\pgfqpoint{0.000000in}{-0.048611in}}%
\pgfusepath{stroke,fill}%
}%
\begin{pgfscope}%
\pgfsys@transformshift{1.789316in}{0.467838in}%
\pgfsys@useobject{currentmarker}{}%
\end{pgfscope}%
\end{pgfscope}%
\begin{pgfscope}%
\definecolor{textcolor}{rgb}{0.000000,0.000000,0.000000}%
\pgfsetstrokecolor{textcolor}%
\pgfsetfillcolor{textcolor}%
\pgftext[x=1.789316in,y=0.370616in,,top]{\color{textcolor}\sffamily\fontsize{8.000000}{9.600000}\selectfont 0.6}%
\end{pgfscope}%
\begin{pgfscope}%
\pgfsetbuttcap%
\pgfsetroundjoin%
\definecolor{currentfill}{rgb}{0.000000,0.000000,0.000000}%
\pgfsetfillcolor{currentfill}%
\pgfsetlinewidth{0.803000pt}%
\definecolor{currentstroke}{rgb}{0.000000,0.000000,0.000000}%
\pgfsetstrokecolor{currentstroke}%
\pgfsetdash{}{0pt}%
\pgfsys@defobject{currentmarker}{\pgfqpoint{0.000000in}{-0.048611in}}{\pgfqpoint{0.000000in}{0.000000in}}{%
\pgfpathmoveto{\pgfqpoint{0.000000in}{0.000000in}}%
\pgfpathlineto{\pgfqpoint{0.000000in}{-0.048611in}}%
\pgfusepath{stroke,fill}%
}%
\begin{pgfscope}%
\pgfsys@transformshift{2.196993in}{0.467838in}%
\pgfsys@useobject{currentmarker}{}%
\end{pgfscope}%
\end{pgfscope}%
\begin{pgfscope}%
\definecolor{textcolor}{rgb}{0.000000,0.000000,0.000000}%
\pgfsetstrokecolor{textcolor}%
\pgfsetfillcolor{textcolor}%
\pgftext[x=2.196993in,y=0.370616in,,top]{\color{textcolor}\sffamily\fontsize{8.000000}{9.600000}\selectfont 0.8}%
\end{pgfscope}%
\begin{pgfscope}%
\pgfsetbuttcap%
\pgfsetroundjoin%
\definecolor{currentfill}{rgb}{0.000000,0.000000,0.000000}%
\pgfsetfillcolor{currentfill}%
\pgfsetlinewidth{0.803000pt}%
\definecolor{currentstroke}{rgb}{0.000000,0.000000,0.000000}%
\pgfsetstrokecolor{currentstroke}%
\pgfsetdash{}{0pt}%
\pgfsys@defobject{currentmarker}{\pgfqpoint{0.000000in}{-0.048611in}}{\pgfqpoint{0.000000in}{0.000000in}}{%
\pgfpathmoveto{\pgfqpoint{0.000000in}{0.000000in}}%
\pgfpathlineto{\pgfqpoint{0.000000in}{-0.048611in}}%
\pgfusepath{stroke,fill}%
}%
\begin{pgfscope}%
\pgfsys@transformshift{2.604669in}{0.467838in}%
\pgfsys@useobject{currentmarker}{}%
\end{pgfscope}%
\end{pgfscope}%
\begin{pgfscope}%
\definecolor{textcolor}{rgb}{0.000000,0.000000,0.000000}%
\pgfsetstrokecolor{textcolor}%
\pgfsetfillcolor{textcolor}%
\pgftext[x=2.604669in,y=0.370616in,,top]{\color{textcolor}\sffamily\fontsize{8.000000}{9.600000}\selectfont 1.0}%
\end{pgfscope}%
\begin{pgfscope}%
\definecolor{textcolor}{rgb}{0.000000,0.000000,0.000000}%
\pgfsetstrokecolor{textcolor}%
\pgfsetfillcolor{textcolor}%
\pgftext[x=1.687396in,y=0.207530in,,top]{\color{textcolor}\sffamily\fontsize{8.000000}{9.600000}\selectfont \(\displaystyle \beta\)}%
\end{pgfscope}%
\begin{pgfscope}%
\pgfsetbuttcap%
\pgfsetroundjoin%
\definecolor{currentfill}{rgb}{0.000000,0.000000,0.000000}%
\pgfsetfillcolor{currentfill}%
\pgfsetlinewidth{0.803000pt}%
\definecolor{currentstroke}{rgb}{0.000000,0.000000,0.000000}%
\pgfsetstrokecolor{currentstroke}%
\pgfsetdash{}{0pt}%
\pgfsys@defobject{currentmarker}{\pgfqpoint{-0.048611in}{0.000000in}}{\pgfqpoint{-0.000000in}{0.000000in}}{%
\pgfpathmoveto{\pgfqpoint{-0.000000in}{0.000000in}}%
\pgfpathlineto{\pgfqpoint{-0.048611in}{0.000000in}}%
\pgfusepath{stroke,fill}%
}%
\begin{pgfscope}%
\pgfsys@transformshift{0.678396in}{0.521526in}%
\pgfsys@useobject{currentmarker}{}%
\end{pgfscope}%
\end{pgfscope}%
\begin{pgfscope}%
\definecolor{textcolor}{rgb}{0.000000,0.000000,0.000000}%
\pgfsetstrokecolor{textcolor}%
\pgfsetfillcolor{textcolor}%
\pgftext[x=0.263086in, y=0.479317in, left, base]{\color{textcolor}\sffamily\fontsize{8.000000}{9.600000}\selectfont 0.000}%
\end{pgfscope}%
\begin{pgfscope}%
\pgfsetbuttcap%
\pgfsetroundjoin%
\definecolor{currentfill}{rgb}{0.000000,0.000000,0.000000}%
\pgfsetfillcolor{currentfill}%
\pgfsetlinewidth{0.803000pt}%
\definecolor{currentstroke}{rgb}{0.000000,0.000000,0.000000}%
\pgfsetstrokecolor{currentstroke}%
\pgfsetdash{}{0pt}%
\pgfsys@defobject{currentmarker}{\pgfqpoint{-0.048611in}{0.000000in}}{\pgfqpoint{-0.000000in}{0.000000in}}{%
\pgfpathmoveto{\pgfqpoint{-0.000000in}{0.000000in}}%
\pgfpathlineto{\pgfqpoint{-0.048611in}{0.000000in}}%
\pgfusepath{stroke,fill}%
}%
\begin{pgfscope}%
\pgfsys@transformshift{0.678396in}{0.860732in}%
\pgfsys@useobject{currentmarker}{}%
\end{pgfscope}%
\end{pgfscope}%
\begin{pgfscope}%
\definecolor{textcolor}{rgb}{0.000000,0.000000,0.000000}%
\pgfsetstrokecolor{textcolor}%
\pgfsetfillcolor{textcolor}%
\pgftext[x=0.263086in, y=0.818522in, left, base]{\color{textcolor}\sffamily\fontsize{8.000000}{9.600000}\selectfont 0.002}%
\end{pgfscope}%
\begin{pgfscope}%
\pgfsetbuttcap%
\pgfsetroundjoin%
\definecolor{currentfill}{rgb}{0.000000,0.000000,0.000000}%
\pgfsetfillcolor{currentfill}%
\pgfsetlinewidth{0.803000pt}%
\definecolor{currentstroke}{rgb}{0.000000,0.000000,0.000000}%
\pgfsetstrokecolor{currentstroke}%
\pgfsetdash{}{0pt}%
\pgfsys@defobject{currentmarker}{\pgfqpoint{-0.048611in}{0.000000in}}{\pgfqpoint{-0.000000in}{0.000000in}}{%
\pgfpathmoveto{\pgfqpoint{-0.000000in}{0.000000in}}%
\pgfpathlineto{\pgfqpoint{-0.048611in}{0.000000in}}%
\pgfusepath{stroke,fill}%
}%
\begin{pgfscope}%
\pgfsys@transformshift{0.678396in}{1.199937in}%
\pgfsys@useobject{currentmarker}{}%
\end{pgfscope}%
\end{pgfscope}%
\begin{pgfscope}%
\definecolor{textcolor}{rgb}{0.000000,0.000000,0.000000}%
\pgfsetstrokecolor{textcolor}%
\pgfsetfillcolor{textcolor}%
\pgftext[x=0.263086in, y=1.157728in, left, base]{\color{textcolor}\sffamily\fontsize{8.000000}{9.600000}\selectfont 0.004}%
\end{pgfscope}%
\begin{pgfscope}%
\pgfsetbuttcap%
\pgfsetroundjoin%
\definecolor{currentfill}{rgb}{0.000000,0.000000,0.000000}%
\pgfsetfillcolor{currentfill}%
\pgfsetlinewidth{0.803000pt}%
\definecolor{currentstroke}{rgb}{0.000000,0.000000,0.000000}%
\pgfsetstrokecolor{currentstroke}%
\pgfsetdash{}{0pt}%
\pgfsys@defobject{currentmarker}{\pgfqpoint{-0.048611in}{0.000000in}}{\pgfqpoint{-0.000000in}{0.000000in}}{%
\pgfpathmoveto{\pgfqpoint{-0.000000in}{0.000000in}}%
\pgfpathlineto{\pgfqpoint{-0.048611in}{0.000000in}}%
\pgfusepath{stroke,fill}%
}%
\begin{pgfscope}%
\pgfsys@transformshift{0.678396in}{1.539143in}%
\pgfsys@useobject{currentmarker}{}%
\end{pgfscope}%
\end{pgfscope}%
\begin{pgfscope}%
\definecolor{textcolor}{rgb}{0.000000,0.000000,0.000000}%
\pgfsetstrokecolor{textcolor}%
\pgfsetfillcolor{textcolor}%
\pgftext[x=0.263086in, y=1.496933in, left, base]{\color{textcolor}\sffamily\fontsize{8.000000}{9.600000}\selectfont 0.006}%
\end{pgfscope}%
\begin{pgfscope}%
\definecolor{textcolor}{rgb}{0.000000,0.000000,0.000000}%
\pgfsetstrokecolor{textcolor}%
\pgfsetfillcolor{textcolor}%
\pgftext[x=0.207530in,y=1.149369in,,bottom,rotate=90.000000]{\color{textcolor}\sffamily\fontsize{8.000000}{9.600000}\selectfont \(\displaystyle X\)}%
\end{pgfscope}%
\begin{pgfscope}%
\pgfpathrectangle{\pgfqpoint{0.678396in}{0.467838in}}{\pgfqpoint{2.018001in}{1.363061in}}%
\pgfusepath{clip}%
\pgfsetrectcap%
\pgfsetroundjoin%
\pgfsetlinewidth{1.505625pt}%
\definecolor{currentstroke}{rgb}{0.000000,0.000000,0.000000}%
\pgfsetstrokecolor{currentstroke}%
\pgfsetdash{}{0pt}%
\pgfpathmoveto{\pgfqpoint{0.770123in}{0.542217in}}%
\pgfpathlineto{\pgfqpoint{0.973962in}{0.572058in}}%
\pgfpathlineto{\pgfqpoint{1.177800in}{0.614085in}}%
\pgfpathlineto{\pgfqpoint{1.381639in}{0.672225in}}%
\pgfpathlineto{\pgfqpoint{1.585477in}{0.751539in}}%
\pgfpathlineto{\pgfqpoint{1.789316in}{0.858540in}}%
\pgfpathlineto{\pgfqpoint{1.993154in}{1.001571in}}%
\pgfpathlineto{\pgfqpoint{2.196993in}{1.191300in}}%
\pgfpathlineto{\pgfqpoint{2.400831in}{1.441325in}}%
\pgfpathlineto{\pgfqpoint{2.604669in}{1.768942in}}%
\pgfusepath{stroke}%
\end{pgfscope}%
\begin{pgfscope}%
\pgfpathrectangle{\pgfqpoint{0.678396in}{0.467838in}}{\pgfqpoint{2.018001in}{1.363061in}}%
\pgfusepath{clip}%
\pgfsetrectcap%
\pgfsetroundjoin%
\pgfsetlinewidth{1.505625pt}%
\definecolor{currentstroke}{rgb}{0.121569,0.466667,0.705882}%
\pgfsetstrokecolor{currentstroke}%
\pgfsetdash{}{0pt}%
\pgfpathmoveto{\pgfqpoint{0.770123in}{0.529796in}}%
\pgfpathlineto{\pgfqpoint{0.973962in}{0.542041in}}%
\pgfpathlineto{\pgfqpoint{1.177800in}{0.563116in}}%
\pgfpathlineto{\pgfqpoint{1.381639in}{0.596044in}}%
\pgfpathlineto{\pgfqpoint{1.585477in}{0.651218in}}%
\pgfpathlineto{\pgfqpoint{1.789316in}{0.735119in}}%
\pgfpathlineto{\pgfqpoint{1.993154in}{0.860588in}}%
\pgfpathlineto{\pgfqpoint{2.196993in}{1.015296in}}%
\pgfpathlineto{\pgfqpoint{2.400831in}{1.213409in}}%
\pgfpathlineto{\pgfqpoint{2.604669in}{1.487094in}}%
\pgfusepath{stroke}%
\end{pgfscope}%
\begin{pgfscope}%
\pgfsetrectcap%
\pgfsetmiterjoin%
\pgfsetlinewidth{0.803000pt}%
\definecolor{currentstroke}{rgb}{0.000000,0.000000,0.000000}%
\pgfsetstrokecolor{currentstroke}%
\pgfsetdash{}{0pt}%
\pgfpathmoveto{\pgfqpoint{0.678396in}{0.467838in}}%
\pgfpathlineto{\pgfqpoint{0.678396in}{1.830900in}}%
\pgfusepath{stroke}%
\end{pgfscope}%
\begin{pgfscope}%
\pgfsetrectcap%
\pgfsetmiterjoin%
\pgfsetlinewidth{0.803000pt}%
\definecolor{currentstroke}{rgb}{0.000000,0.000000,0.000000}%
\pgfsetstrokecolor{currentstroke}%
\pgfsetdash{}{0pt}%
\pgfpathmoveto{\pgfqpoint{2.696397in}{0.467838in}}%
\pgfpathlineto{\pgfqpoint{2.696397in}{1.830900in}}%
\pgfusepath{stroke}%
\end{pgfscope}%
\begin{pgfscope}%
\pgfsetrectcap%
\pgfsetmiterjoin%
\pgfsetlinewidth{0.803000pt}%
\definecolor{currentstroke}{rgb}{0.000000,0.000000,0.000000}%
\pgfsetstrokecolor{currentstroke}%
\pgfsetdash{}{0pt}%
\pgfpathmoveto{\pgfqpoint{0.678396in}{0.467838in}}%
\pgfpathlineto{\pgfqpoint{2.696397in}{0.467838in}}%
\pgfusepath{stroke}%
\end{pgfscope}%
\begin{pgfscope}%
\pgfsetrectcap%
\pgfsetmiterjoin%
\pgfsetlinewidth{0.803000pt}%
\definecolor{currentstroke}{rgb}{0.000000,0.000000,0.000000}%
\pgfsetstrokecolor{currentstroke}%
\pgfsetdash{}{0pt}%
\pgfpathmoveto{\pgfqpoint{0.678396in}{1.830900in}}%
\pgfpathlineto{\pgfqpoint{2.696397in}{1.830900in}}%
\pgfusepath{stroke}%
\end{pgfscope}%
\begin{pgfscope}%
\pgfsetbuttcap%
\pgfsetmiterjoin%
\definecolor{currentfill}{rgb}{1.000000,1.000000,1.000000}%
\pgfsetfillcolor{currentfill}%
\pgfsetfillopacity{0.800000}%
\pgfsetlinewidth{1.003750pt}%
\definecolor{currentstroke}{rgb}{0.800000,0.800000,0.800000}%
\pgfsetstrokecolor{currentstroke}%
\pgfsetstrokeopacity{0.800000}%
\pgfsetdash{}{0pt}%
\pgfpathmoveto{\pgfqpoint{0.756174in}{1.415839in}}%
\pgfpathlineto{\pgfqpoint{1.851204in}{1.415839in}}%
\pgfpathquadraticcurveto{\pgfqpoint{1.873427in}{1.415839in}}{\pgfqpoint{1.873427in}{1.438061in}}%
\pgfpathlineto{\pgfqpoint{1.873427in}{1.753122in}}%
\pgfpathquadraticcurveto{\pgfqpoint{1.873427in}{1.775344in}}{\pgfqpoint{1.851204in}{1.775344in}}%
\pgfpathlineto{\pgfqpoint{0.756174in}{1.775344in}}%
\pgfpathquadraticcurveto{\pgfqpoint{0.733952in}{1.775344in}}{\pgfqpoint{0.733952in}{1.753122in}}%
\pgfpathlineto{\pgfqpoint{0.733952in}{1.438061in}}%
\pgfpathquadraticcurveto{\pgfqpoint{0.733952in}{1.415839in}}{\pgfqpoint{0.756174in}{1.415839in}}%
\pgfpathlineto{\pgfqpoint{0.756174in}{1.415839in}}%
\pgfpathclose%
\pgfusepath{stroke,fill}%
\end{pgfscope}%
\begin{pgfscope}%
\pgfsetrectcap%
\pgfsetroundjoin%
\pgfsetlinewidth{1.505625pt}%
\definecolor{currentstroke}{rgb}{0.000000,0.000000,0.000000}%
\pgfsetstrokecolor{currentstroke}%
\pgfsetdash{}{0pt}%
\pgfpathmoveto{\pgfqpoint{0.778396in}{1.685370in}}%
\pgfpathlineto{\pgfqpoint{0.889507in}{1.685370in}}%
\pgfpathlineto{\pgfqpoint{1.000618in}{1.685370in}}%
\pgfusepath{stroke}%
\end{pgfscope}%
\begin{pgfscope}%
\definecolor{textcolor}{rgb}{0.000000,0.000000,0.000000}%
\pgfsetstrokecolor{textcolor}%
\pgfsetfillcolor{textcolor}%
\pgftext[x=1.089507in,y=1.646481in,left,base]{\color{textcolor}\sffamily\fontsize{8.000000}{9.600000}\selectfont exact 1D}%
\end{pgfscope}%
\begin{pgfscope}%
\pgfsetrectcap%
\pgfsetroundjoin%
\pgfsetlinewidth{1.505625pt}%
\definecolor{currentstroke}{rgb}{0.121569,0.466667,0.705882}%
\pgfsetstrokecolor{currentstroke}%
\pgfsetdash{}{0pt}%
\pgfpathmoveto{\pgfqpoint{0.778396in}{1.522284in}}%
\pgfpathlineto{\pgfqpoint{0.889507in}{1.522284in}}%
\pgfpathlineto{\pgfqpoint{1.000618in}{1.522284in}}%
\pgfusepath{stroke}%
\end{pgfscope}%
\begin{pgfscope}%
\definecolor{textcolor}{rgb}{0.000000,0.000000,0.000000}%
\pgfsetstrokecolor{textcolor}%
\pgfsetfillcolor{textcolor}%
\pgftext[x=1.089507in,y=1.483395in,left,base]{\color{textcolor}\sffamily\fontsize{8.000000}{9.600000}\selectfont conformation}%
\end{pgfscope}%
\begin{pgfscope}%
\pgfsetbuttcap%
\pgfsetmiterjoin%
\definecolor{currentfill}{rgb}{1.000000,1.000000,1.000000}%
\pgfsetfillcolor{currentfill}%
\pgfsetlinewidth{0.000000pt}%
\definecolor{currentstroke}{rgb}{0.000000,0.000000,0.000000}%
\pgfsetstrokecolor{currentstroke}%
\pgfsetstrokeopacity{0.000000}%
\pgfsetdash{}{0pt}%
\pgfpathmoveto{\pgfqpoint{3.099997in}{0.467838in}}%
\pgfpathlineto{\pgfqpoint{5.117998in}{0.467838in}}%
\pgfpathlineto{\pgfqpoint{5.117998in}{1.830900in}}%
\pgfpathlineto{\pgfqpoint{3.099997in}{1.830900in}}%
\pgfpathlineto{\pgfqpoint{3.099997in}{0.467838in}}%
\pgfpathclose%
\pgfusepath{fill}%
\end{pgfscope}%
\begin{pgfscope}%
\pgfsetbuttcap%
\pgfsetroundjoin%
\definecolor{currentfill}{rgb}{0.000000,0.000000,0.000000}%
\pgfsetfillcolor{currentfill}%
\pgfsetlinewidth{0.803000pt}%
\definecolor{currentstroke}{rgb}{0.000000,0.000000,0.000000}%
\pgfsetstrokecolor{currentstroke}%
\pgfsetdash{}{0pt}%
\pgfsys@defobject{currentmarker}{\pgfqpoint{0.000000in}{-0.048611in}}{\pgfqpoint{0.000000in}{0.000000in}}{%
\pgfpathmoveto{\pgfqpoint{0.000000in}{0.000000in}}%
\pgfpathlineto{\pgfqpoint{0.000000in}{-0.048611in}}%
\pgfusepath{stroke,fill}%
}%
\begin{pgfscope}%
\pgfsys@transformshift{3.395563in}{0.467838in}%
\pgfsys@useobject{currentmarker}{}%
\end{pgfscope}%
\end{pgfscope}%
\begin{pgfscope}%
\definecolor{textcolor}{rgb}{0.000000,0.000000,0.000000}%
\pgfsetstrokecolor{textcolor}%
\pgfsetfillcolor{textcolor}%
\pgftext[x=3.395563in,y=0.370616in,,top]{\color{textcolor}\sffamily\fontsize{8.000000}{9.600000}\selectfont 0.2}%
\end{pgfscope}%
\begin{pgfscope}%
\pgfsetbuttcap%
\pgfsetroundjoin%
\definecolor{currentfill}{rgb}{0.000000,0.000000,0.000000}%
\pgfsetfillcolor{currentfill}%
\pgfsetlinewidth{0.803000pt}%
\definecolor{currentstroke}{rgb}{0.000000,0.000000,0.000000}%
\pgfsetstrokecolor{currentstroke}%
\pgfsetdash{}{0pt}%
\pgfsys@defobject{currentmarker}{\pgfqpoint{0.000000in}{-0.048611in}}{\pgfqpoint{0.000000in}{0.000000in}}{%
\pgfpathmoveto{\pgfqpoint{0.000000in}{0.000000in}}%
\pgfpathlineto{\pgfqpoint{0.000000in}{-0.048611in}}%
\pgfusepath{stroke,fill}%
}%
\begin{pgfscope}%
\pgfsys@transformshift{3.803240in}{0.467838in}%
\pgfsys@useobject{currentmarker}{}%
\end{pgfscope}%
\end{pgfscope}%
\begin{pgfscope}%
\definecolor{textcolor}{rgb}{0.000000,0.000000,0.000000}%
\pgfsetstrokecolor{textcolor}%
\pgfsetfillcolor{textcolor}%
\pgftext[x=3.803240in,y=0.370616in,,top]{\color{textcolor}\sffamily\fontsize{8.000000}{9.600000}\selectfont 0.4}%
\end{pgfscope}%
\begin{pgfscope}%
\pgfsetbuttcap%
\pgfsetroundjoin%
\definecolor{currentfill}{rgb}{0.000000,0.000000,0.000000}%
\pgfsetfillcolor{currentfill}%
\pgfsetlinewidth{0.803000pt}%
\definecolor{currentstroke}{rgb}{0.000000,0.000000,0.000000}%
\pgfsetstrokecolor{currentstroke}%
\pgfsetdash{}{0pt}%
\pgfsys@defobject{currentmarker}{\pgfqpoint{0.000000in}{-0.048611in}}{\pgfqpoint{0.000000in}{0.000000in}}{%
\pgfpathmoveto{\pgfqpoint{0.000000in}{0.000000in}}%
\pgfpathlineto{\pgfqpoint{0.000000in}{-0.048611in}}%
\pgfusepath{stroke,fill}%
}%
\begin{pgfscope}%
\pgfsys@transformshift{4.210916in}{0.467838in}%
\pgfsys@useobject{currentmarker}{}%
\end{pgfscope}%
\end{pgfscope}%
\begin{pgfscope}%
\definecolor{textcolor}{rgb}{0.000000,0.000000,0.000000}%
\pgfsetstrokecolor{textcolor}%
\pgfsetfillcolor{textcolor}%
\pgftext[x=4.210916in,y=0.370616in,,top]{\color{textcolor}\sffamily\fontsize{8.000000}{9.600000}\selectfont 0.6}%
\end{pgfscope}%
\begin{pgfscope}%
\pgfsetbuttcap%
\pgfsetroundjoin%
\definecolor{currentfill}{rgb}{0.000000,0.000000,0.000000}%
\pgfsetfillcolor{currentfill}%
\pgfsetlinewidth{0.803000pt}%
\definecolor{currentstroke}{rgb}{0.000000,0.000000,0.000000}%
\pgfsetstrokecolor{currentstroke}%
\pgfsetdash{}{0pt}%
\pgfsys@defobject{currentmarker}{\pgfqpoint{0.000000in}{-0.048611in}}{\pgfqpoint{0.000000in}{0.000000in}}{%
\pgfpathmoveto{\pgfqpoint{0.000000in}{0.000000in}}%
\pgfpathlineto{\pgfqpoint{0.000000in}{-0.048611in}}%
\pgfusepath{stroke,fill}%
}%
\begin{pgfscope}%
\pgfsys@transformshift{4.618593in}{0.467838in}%
\pgfsys@useobject{currentmarker}{}%
\end{pgfscope}%
\end{pgfscope}%
\begin{pgfscope}%
\definecolor{textcolor}{rgb}{0.000000,0.000000,0.000000}%
\pgfsetstrokecolor{textcolor}%
\pgfsetfillcolor{textcolor}%
\pgftext[x=4.618593in,y=0.370616in,,top]{\color{textcolor}\sffamily\fontsize{8.000000}{9.600000}\selectfont 0.8}%
\end{pgfscope}%
\begin{pgfscope}%
\pgfsetbuttcap%
\pgfsetroundjoin%
\definecolor{currentfill}{rgb}{0.000000,0.000000,0.000000}%
\pgfsetfillcolor{currentfill}%
\pgfsetlinewidth{0.803000pt}%
\definecolor{currentstroke}{rgb}{0.000000,0.000000,0.000000}%
\pgfsetstrokecolor{currentstroke}%
\pgfsetdash{}{0pt}%
\pgfsys@defobject{currentmarker}{\pgfqpoint{0.000000in}{-0.048611in}}{\pgfqpoint{0.000000in}{0.000000in}}{%
\pgfpathmoveto{\pgfqpoint{0.000000in}{0.000000in}}%
\pgfpathlineto{\pgfqpoint{0.000000in}{-0.048611in}}%
\pgfusepath{stroke,fill}%
}%
\begin{pgfscope}%
\pgfsys@transformshift{5.026270in}{0.467838in}%
\pgfsys@useobject{currentmarker}{}%
\end{pgfscope}%
\end{pgfscope}%
\begin{pgfscope}%
\definecolor{textcolor}{rgb}{0.000000,0.000000,0.000000}%
\pgfsetstrokecolor{textcolor}%
\pgfsetfillcolor{textcolor}%
\pgftext[x=5.026270in,y=0.370616in,,top]{\color{textcolor}\sffamily\fontsize{8.000000}{9.600000}\selectfont 1.0}%
\end{pgfscope}%
\begin{pgfscope}%
\definecolor{textcolor}{rgb}{0.000000,0.000000,0.000000}%
\pgfsetstrokecolor{textcolor}%
\pgfsetfillcolor{textcolor}%
\pgftext[x=4.108997in,y=0.207530in,,top]{\color{textcolor}\sffamily\fontsize{8.000000}{9.600000}\selectfont \(\displaystyle \beta\)}%
\end{pgfscope}%
\begin{pgfscope}%
\pgfsetbuttcap%
\pgfsetroundjoin%
\definecolor{currentfill}{rgb}{0.000000,0.000000,0.000000}%
\pgfsetfillcolor{currentfill}%
\pgfsetlinewidth{0.803000pt}%
\definecolor{currentstroke}{rgb}{0.000000,0.000000,0.000000}%
\pgfsetstrokecolor{currentstroke}%
\pgfsetdash{}{0pt}%
\pgfsys@defobject{currentmarker}{\pgfqpoint{-0.048611in}{0.000000in}}{\pgfqpoint{-0.000000in}{0.000000in}}{%
\pgfpathmoveto{\pgfqpoint{-0.000000in}{0.000000in}}%
\pgfpathlineto{\pgfqpoint{-0.048611in}{0.000000in}}%
\pgfusepath{stroke,fill}%
}%
\begin{pgfscope}%
\pgfsys@transformshift{3.099997in}{0.519091in}%
\pgfsys@useobject{currentmarker}{}%
\end{pgfscope}%
\end{pgfscope}%
\begin{pgfscope}%
\definecolor{textcolor}{rgb}{0.000000,0.000000,0.000000}%
\pgfsetstrokecolor{textcolor}%
\pgfsetfillcolor{textcolor}%
\pgftext[x=2.684687in, y=0.476882in, left, base]{\color{textcolor}\sffamily\fontsize{8.000000}{9.600000}\selectfont 0.000}%
\end{pgfscope}%
\begin{pgfscope}%
\pgfsetbuttcap%
\pgfsetroundjoin%
\definecolor{currentfill}{rgb}{0.000000,0.000000,0.000000}%
\pgfsetfillcolor{currentfill}%
\pgfsetlinewidth{0.803000pt}%
\definecolor{currentstroke}{rgb}{0.000000,0.000000,0.000000}%
\pgfsetstrokecolor{currentstroke}%
\pgfsetdash{}{0pt}%
\pgfsys@defobject{currentmarker}{\pgfqpoint{-0.048611in}{0.000000in}}{\pgfqpoint{-0.000000in}{0.000000in}}{%
\pgfpathmoveto{\pgfqpoint{-0.000000in}{0.000000in}}%
\pgfpathlineto{\pgfqpoint{-0.048611in}{0.000000in}}%
\pgfusepath{stroke,fill}%
}%
\begin{pgfscope}%
\pgfsys@transformshift{3.099997in}{0.848519in}%
\pgfsys@useobject{currentmarker}{}%
\end{pgfscope}%
\end{pgfscope}%
\begin{pgfscope}%
\definecolor{textcolor}{rgb}{0.000000,0.000000,0.000000}%
\pgfsetstrokecolor{textcolor}%
\pgfsetfillcolor{textcolor}%
\pgftext[x=2.684687in, y=0.806310in, left, base]{\color{textcolor}\sffamily\fontsize{8.000000}{9.600000}\selectfont 0.002}%
\end{pgfscope}%
\begin{pgfscope}%
\pgfsetbuttcap%
\pgfsetroundjoin%
\definecolor{currentfill}{rgb}{0.000000,0.000000,0.000000}%
\pgfsetfillcolor{currentfill}%
\pgfsetlinewidth{0.803000pt}%
\definecolor{currentstroke}{rgb}{0.000000,0.000000,0.000000}%
\pgfsetstrokecolor{currentstroke}%
\pgfsetdash{}{0pt}%
\pgfsys@defobject{currentmarker}{\pgfqpoint{-0.048611in}{0.000000in}}{\pgfqpoint{-0.000000in}{0.000000in}}{%
\pgfpathmoveto{\pgfqpoint{-0.000000in}{0.000000in}}%
\pgfpathlineto{\pgfqpoint{-0.048611in}{0.000000in}}%
\pgfusepath{stroke,fill}%
}%
\begin{pgfscope}%
\pgfsys@transformshift{3.099997in}{1.177947in}%
\pgfsys@useobject{currentmarker}{}%
\end{pgfscope}%
\end{pgfscope}%
\begin{pgfscope}%
\definecolor{textcolor}{rgb}{0.000000,0.000000,0.000000}%
\pgfsetstrokecolor{textcolor}%
\pgfsetfillcolor{textcolor}%
\pgftext[x=2.684687in, y=1.135737in, left, base]{\color{textcolor}\sffamily\fontsize{8.000000}{9.600000}\selectfont 0.004}%
\end{pgfscope}%
\begin{pgfscope}%
\pgfsetbuttcap%
\pgfsetroundjoin%
\definecolor{currentfill}{rgb}{0.000000,0.000000,0.000000}%
\pgfsetfillcolor{currentfill}%
\pgfsetlinewidth{0.803000pt}%
\definecolor{currentstroke}{rgb}{0.000000,0.000000,0.000000}%
\pgfsetstrokecolor{currentstroke}%
\pgfsetdash{}{0pt}%
\pgfsys@defobject{currentmarker}{\pgfqpoint{-0.048611in}{0.000000in}}{\pgfqpoint{-0.000000in}{0.000000in}}{%
\pgfpathmoveto{\pgfqpoint{-0.000000in}{0.000000in}}%
\pgfpathlineto{\pgfqpoint{-0.048611in}{0.000000in}}%
\pgfusepath{stroke,fill}%
}%
\begin{pgfscope}%
\pgfsys@transformshift{3.099997in}{1.507374in}%
\pgfsys@useobject{currentmarker}{}%
\end{pgfscope}%
\end{pgfscope}%
\begin{pgfscope}%
\definecolor{textcolor}{rgb}{0.000000,0.000000,0.000000}%
\pgfsetstrokecolor{textcolor}%
\pgfsetfillcolor{textcolor}%
\pgftext[x=2.684687in, y=1.465165in, left, base]{\color{textcolor}\sffamily\fontsize{8.000000}{9.600000}\selectfont 0.006}%
\end{pgfscope}%
\begin{pgfscope}%
\pgfpathrectangle{\pgfqpoint{3.099997in}{0.467838in}}{\pgfqpoint{2.018001in}{1.363061in}}%
\pgfusepath{clip}%
\pgfsetrectcap%
\pgfsetroundjoin%
\pgfsetlinewidth{1.505625pt}%
\definecolor{currentstroke}{rgb}{0.000000,0.000000,0.000000}%
\pgfsetstrokecolor{currentstroke}%
\pgfsetdash{}{0pt}%
\pgfpathmoveto{\pgfqpoint{3.191724in}{0.539185in}}%
\pgfpathlineto{\pgfqpoint{3.395563in}{0.568167in}}%
\pgfpathlineto{\pgfqpoint{3.599401in}{0.608983in}}%
\pgfpathlineto{\pgfqpoint{3.803240in}{0.665446in}}%
\pgfpathlineto{\pgfqpoint{4.007078in}{0.742474in}}%
\pgfpathlineto{\pgfqpoint{4.210916in}{0.846391in}}%
\pgfpathlineto{\pgfqpoint{4.414755in}{0.985299in}}%
\pgfpathlineto{\pgfqpoint{4.618593in}{1.169558in}}%
\pgfpathlineto{\pgfqpoint{4.822432in}{1.412377in}}%
\pgfpathlineto{\pgfqpoint{5.026270in}{1.730550in}}%
\pgfusepath{stroke}%
\end{pgfscope}%
\begin{pgfscope}%
\pgfpathrectangle{\pgfqpoint{3.099997in}{0.467838in}}{\pgfqpoint{2.018001in}{1.363061in}}%
\pgfusepath{clip}%
\pgfsetrectcap%
\pgfsetroundjoin%
\pgfsetlinewidth{1.505625pt}%
\definecolor{currentstroke}{rgb}{0.121569,0.466667,0.705882}%
\pgfsetstrokecolor{currentstroke}%
\pgfsetdash{}{0pt}%
\pgfpathmoveto{\pgfqpoint{3.191724in}{0.529796in}}%
\pgfpathlineto{\pgfqpoint{3.395563in}{0.545647in}}%
\pgfpathlineto{\pgfqpoint{3.599401in}{0.572927in}}%
\pgfpathlineto{\pgfqpoint{3.803240in}{0.615550in}}%
\pgfpathlineto{\pgfqpoint{4.007078in}{0.686967in}}%
\pgfpathlineto{\pgfqpoint{4.210916in}{0.795571in}}%
\pgfpathlineto{\pgfqpoint{4.414755in}{0.957980in}}%
\pgfpathlineto{\pgfqpoint{4.618593in}{1.158238in}}%
\pgfpathlineto{\pgfqpoint{4.822432in}{1.414679in}}%
\pgfpathlineto{\pgfqpoint{5.026270in}{1.768942in}}%
\pgfusepath{stroke}%
\end{pgfscope}%
\begin{pgfscope}%
\pgfsetrectcap%
\pgfsetmiterjoin%
\pgfsetlinewidth{0.803000pt}%
\definecolor{currentstroke}{rgb}{0.000000,0.000000,0.000000}%
\pgfsetstrokecolor{currentstroke}%
\pgfsetdash{}{0pt}%
\pgfpathmoveto{\pgfqpoint{3.099997in}{0.467838in}}%
\pgfpathlineto{\pgfqpoint{3.099997in}{1.830900in}}%
\pgfusepath{stroke}%
\end{pgfscope}%
\begin{pgfscope}%
\pgfsetrectcap%
\pgfsetmiterjoin%
\pgfsetlinewidth{0.803000pt}%
\definecolor{currentstroke}{rgb}{0.000000,0.000000,0.000000}%
\pgfsetstrokecolor{currentstroke}%
\pgfsetdash{}{0pt}%
\pgfpathmoveto{\pgfqpoint{5.117998in}{0.467838in}}%
\pgfpathlineto{\pgfqpoint{5.117998in}{1.830900in}}%
\pgfusepath{stroke}%
\end{pgfscope}%
\begin{pgfscope}%
\pgfsetrectcap%
\pgfsetmiterjoin%
\pgfsetlinewidth{0.803000pt}%
\definecolor{currentstroke}{rgb}{0.000000,0.000000,0.000000}%
\pgfsetstrokecolor{currentstroke}%
\pgfsetdash{}{0pt}%
\pgfpathmoveto{\pgfqpoint{3.099997in}{0.467838in}}%
\pgfpathlineto{\pgfqpoint{5.117998in}{0.467838in}}%
\pgfusepath{stroke}%
\end{pgfscope}%
\begin{pgfscope}%
\pgfsetrectcap%
\pgfsetmiterjoin%
\pgfsetlinewidth{0.803000pt}%
\definecolor{currentstroke}{rgb}{0.000000,0.000000,0.000000}%
\pgfsetstrokecolor{currentstroke}%
\pgfsetdash{}{0pt}%
\pgfpathmoveto{\pgfqpoint{3.099997in}{1.830900in}}%
\pgfpathlineto{\pgfqpoint{5.117998in}{1.830900in}}%
\pgfusepath{stroke}%
\end{pgfscope}%
\begin{pgfscope}%
\pgfsetbuttcap%
\pgfsetmiterjoin%
\definecolor{currentfill}{rgb}{1.000000,1.000000,1.000000}%
\pgfsetfillcolor{currentfill}%
\pgfsetfillopacity{0.800000}%
\pgfsetlinewidth{1.003750pt}%
\definecolor{currentstroke}{rgb}{0.800000,0.800000,0.800000}%
\pgfsetstrokecolor{currentstroke}%
\pgfsetstrokeopacity{0.800000}%
\pgfsetdash{}{0pt}%
\pgfpathmoveto{\pgfqpoint{3.177775in}{1.415839in}}%
\pgfpathlineto{\pgfqpoint{4.664949in}{1.415839in}}%
\pgfpathquadraticcurveto{\pgfqpoint{4.687171in}{1.415839in}}{\pgfqpoint{4.687171in}{1.438061in}}%
\pgfpathlineto{\pgfqpoint{4.687171in}{1.753122in}}%
\pgfpathquadraticcurveto{\pgfqpoint{4.687171in}{1.775344in}}{\pgfqpoint{4.664949in}{1.775344in}}%
\pgfpathlineto{\pgfqpoint{3.177775in}{1.775344in}}%
\pgfpathquadraticcurveto{\pgfqpoint{3.155552in}{1.775344in}}{\pgfqpoint{3.155552in}{1.753122in}}%
\pgfpathlineto{\pgfqpoint{3.155552in}{1.438061in}}%
\pgfpathquadraticcurveto{\pgfqpoint{3.155552in}{1.415839in}}{\pgfqpoint{3.177775in}{1.415839in}}%
\pgfpathlineto{\pgfqpoint{3.177775in}{1.415839in}}%
\pgfpathclose%
\pgfusepath{stroke,fill}%
\end{pgfscope}%
\begin{pgfscope}%
\pgfsetrectcap%
\pgfsetroundjoin%
\pgfsetlinewidth{1.505625pt}%
\definecolor{currentstroke}{rgb}{0.000000,0.000000,0.000000}%
\pgfsetstrokecolor{currentstroke}%
\pgfsetdash{}{0pt}%
\pgfpathmoveto{\pgfqpoint{3.199997in}{1.685370in}}%
\pgfpathlineto{\pgfqpoint{3.311108in}{1.685370in}}%
\pgfpathlineto{\pgfqpoint{3.422219in}{1.685370in}}%
\pgfusepath{stroke}%
\end{pgfscope}%
\begin{pgfscope}%
\definecolor{textcolor}{rgb}{0.000000,0.000000,0.000000}%
\pgfsetstrokecolor{textcolor}%
\pgfsetfillcolor{textcolor}%
\pgftext[x=3.511108in,y=1.646481in,left,base]{\color{textcolor}\sffamily\fontsize{8.000000}{9.600000}\selectfont exact 1D}%
\end{pgfscope}%
\begin{pgfscope}%
\pgfsetrectcap%
\pgfsetroundjoin%
\pgfsetlinewidth{1.505625pt}%
\definecolor{currentstroke}{rgb}{0.121569,0.466667,0.705882}%
\pgfsetstrokecolor{currentstroke}%
\pgfsetdash{}{0pt}%
\pgfpathmoveto{\pgfqpoint{3.199997in}{1.522284in}}%
\pgfpathlineto{\pgfqpoint{3.311108in}{1.522284in}}%
\pgfpathlineto{\pgfqpoint{3.422219in}{1.522284in}}%
\pgfusepath{stroke}%
\end{pgfscope}%
\begin{pgfscope}%
\definecolor{textcolor}{rgb}{0.000000,0.000000,0.000000}%
\pgfsetstrokecolor{textcolor}%
\pgfsetfillcolor{textcolor}%
\pgftext[x=3.511108in,y=1.483395in,left,base]{\color{textcolor}\sffamily\fontsize{8.000000}{9.600000}\selectfont conformation scaled}%
\end{pgfscope}%
\end{pgfpicture}%
\makeatother%
\endgroup%

	\caption{сравнение магнитной восприимчивости типичной конформации при $U=0.1$ и магнитной восприимчивости одномерной моделии изинга, с открытыми граничными условиями. На втором графике магнитная восприимчивость конформации домножена на коэффициент, подобранный методом наименьших квадратов.}
	\label{fig:MS_1D_comparison}
\end{figure}


Чтобы убедиться что большинство конформаций имеют магнитную восприимчивость схожую с одномерной цепочкой мы вычислили среднеквадратичное отклонение магнитной восприимчивости конформаций от одномерной цепочки. Распределение полученных значений представлено на рис. \ref{fig:MS_1D_dif_distr}. На нём мы можем видеть пик в 0, означающий что большинство конформаций имеют магнитную восприимчивость близкую к одномерной модели. Так же интересным наблюдением является форма распределения, которая схожа с распределениями полученными при исследования кластеров и мостов в конформациях пи $U=1$.

\begin{figure}[ht]
	\centering
	%% Creator: Matplotlib, PGF backend
%%
%% To include the figure in your LaTeX document, write
%%   \input{<filename>.pgf}
%%
%% Make sure the required packages are loaded in your preamble
%%   \usepackage{pgf}
%%
%% Also ensure that all the required font packages are loaded; for instance,
%% the lmodern package is sometimes necessary when using math font.
%%   \usepackage{lmodern}
%%
%% Figures using additional raster images can only be included by \input if
%% they are in the same directory as the main LaTeX file. For loading figures
%% from other directories you can use the `import` package
%%   \usepackage{import}
%%
%% and then include the figures with
%%   \import{<path to file>}{<filename>.pgf}
%%
%% Matplotlib used the following preamble
%%   
%%   \usepackage{fontspec}
%%   \setmainfont{DejaVuSerif.ttf}[Path=\detokenize{/home/roman/anaconda3/envs/ising/lib/python3.8/site-packages/matplotlib/mpl-data/fonts/ttf/}]
%%   \setsansfont{DejaVuSans.ttf}[Path=\detokenize{/home/roman/anaconda3/envs/ising/lib/python3.8/site-packages/matplotlib/mpl-data/fonts/ttf/}]
%%   \setmonofont{DejaVuSansMono.ttf}[Path=\detokenize{/home/roman/anaconda3/envs/ising/lib/python3.8/site-packages/matplotlib/mpl-data/fonts/ttf/}]
%%   \makeatletter\@ifpackageloaded{underscore}{}{\usepackage[strings]{underscore}}\makeatother
%%
\begingroup%
\makeatletter%
\begin{pgfpicture}%
\pgfpathrectangle{\pgfpointorigin}{\pgfqpoint{4.469938in}{2.648979in}}%
\pgfusepath{use as bounding box, clip}%
\begin{pgfscope}%
\pgfsetbuttcap%
\pgfsetmiterjoin%
\definecolor{currentfill}{rgb}{1.000000,1.000000,1.000000}%
\pgfsetfillcolor{currentfill}%
\pgfsetlinewidth{0.000000pt}%
\definecolor{currentstroke}{rgb}{1.000000,1.000000,1.000000}%
\pgfsetstrokecolor{currentstroke}%
\pgfsetdash{}{0pt}%
\pgfpathmoveto{\pgfqpoint{-0.000000in}{0.000000in}}%
\pgfpathlineto{\pgfqpoint{4.469938in}{0.000000in}}%
\pgfpathlineto{\pgfqpoint{4.469938in}{2.648979in}}%
\pgfpathlineto{\pgfqpoint{-0.000000in}{2.648979in}}%
\pgfpathlineto{\pgfqpoint{-0.000000in}{0.000000in}}%
\pgfpathclose%
\pgfusepath{fill}%
\end{pgfscope}%
\begin{pgfscope}%
\pgfsetbuttcap%
\pgfsetmiterjoin%
\definecolor{currentfill}{rgb}{1.000000,1.000000,1.000000}%
\pgfsetfillcolor{currentfill}%
\pgfsetlinewidth{0.000000pt}%
\definecolor{currentstroke}{rgb}{0.000000,0.000000,0.000000}%
\pgfsetstrokecolor{currentstroke}%
\pgfsetstrokeopacity{0.000000}%
\pgfsetdash{}{0pt}%
\pgfpathmoveto{\pgfqpoint{0.755102in}{0.521603in}}%
\pgfpathlineto{\pgfqpoint{3.714836in}{0.521603in}}%
\pgfpathlineto{\pgfqpoint{3.714836in}{2.339018in}}%
\pgfpathlineto{\pgfqpoint{0.755102in}{2.339018in}}%
\pgfpathlineto{\pgfqpoint{0.755102in}{0.521603in}}%
\pgfpathclose%
\pgfusepath{fill}%
\end{pgfscope}%
\begin{pgfscope}%
\pgfpathrectangle{\pgfqpoint{0.755102in}{0.521603in}}{\pgfqpoint{2.959734in}{1.817415in}}%
\pgfusepath{clip}%
\pgfsetbuttcap%
\pgfsetmiterjoin%
\definecolor{currentfill}{rgb}{0.121569,0.466667,0.705882}%
\pgfsetfillcolor{currentfill}%
\pgfsetlinewidth{0.000000pt}%
\definecolor{currentstroke}{rgb}{0.000000,0.000000,0.000000}%
\pgfsetstrokecolor{currentstroke}%
\pgfsetstrokeopacity{0.000000}%
\pgfsetdash{}{0pt}%
\pgfpathmoveto{\pgfqpoint{0.889635in}{0.521603in}}%
\pgfpathlineto{\pgfqpoint{0.993123in}{0.521603in}}%
\pgfpathlineto{\pgfqpoint{0.993123in}{2.252475in}}%
\pgfpathlineto{\pgfqpoint{0.889635in}{2.252475in}}%
\pgfpathlineto{\pgfqpoint{0.889635in}{0.521603in}}%
\pgfpathclose%
\pgfusepath{fill}%
\end{pgfscope}%
\begin{pgfscope}%
\pgfpathrectangle{\pgfqpoint{0.755102in}{0.521603in}}{\pgfqpoint{2.959734in}{1.817415in}}%
\pgfusepath{clip}%
\pgfsetbuttcap%
\pgfsetmiterjoin%
\definecolor{currentfill}{rgb}{0.121569,0.466667,0.705882}%
\pgfsetfillcolor{currentfill}%
\pgfsetlinewidth{0.000000pt}%
\definecolor{currentstroke}{rgb}{0.000000,0.000000,0.000000}%
\pgfsetstrokecolor{currentstroke}%
\pgfsetstrokeopacity{0.000000}%
\pgfsetdash{}{0pt}%
\pgfpathmoveto{\pgfqpoint{0.993123in}{0.521603in}}%
\pgfpathlineto{\pgfqpoint{1.096610in}{0.521603in}}%
\pgfpathlineto{\pgfqpoint{1.096610in}{0.564209in}}%
\pgfpathlineto{\pgfqpoint{0.993123in}{0.564209in}}%
\pgfpathlineto{\pgfqpoint{0.993123in}{0.521603in}}%
\pgfpathclose%
\pgfusepath{fill}%
\end{pgfscope}%
\begin{pgfscope}%
\pgfpathrectangle{\pgfqpoint{0.755102in}{0.521603in}}{\pgfqpoint{2.959734in}{1.817415in}}%
\pgfusepath{clip}%
\pgfsetbuttcap%
\pgfsetmiterjoin%
\definecolor{currentfill}{rgb}{0.121569,0.466667,0.705882}%
\pgfsetfillcolor{currentfill}%
\pgfsetlinewidth{0.000000pt}%
\definecolor{currentstroke}{rgb}{0.000000,0.000000,0.000000}%
\pgfsetstrokecolor{currentstroke}%
\pgfsetstrokeopacity{0.000000}%
\pgfsetdash{}{0pt}%
\pgfpathmoveto{\pgfqpoint{1.096610in}{0.521603in}}%
\pgfpathlineto{\pgfqpoint{1.200097in}{0.521603in}}%
\pgfpathlineto{\pgfqpoint{1.200097in}{0.569535in}}%
\pgfpathlineto{\pgfqpoint{1.096610in}{0.569535in}}%
\pgfpathlineto{\pgfqpoint{1.096610in}{0.521603in}}%
\pgfpathclose%
\pgfusepath{fill}%
\end{pgfscope}%
\begin{pgfscope}%
\pgfpathrectangle{\pgfqpoint{0.755102in}{0.521603in}}{\pgfqpoint{2.959734in}{1.817415in}}%
\pgfusepath{clip}%
\pgfsetbuttcap%
\pgfsetmiterjoin%
\definecolor{currentfill}{rgb}{0.121569,0.466667,0.705882}%
\pgfsetfillcolor{currentfill}%
\pgfsetlinewidth{0.000000pt}%
\definecolor{currentstroke}{rgb}{0.000000,0.000000,0.000000}%
\pgfsetstrokecolor{currentstroke}%
\pgfsetstrokeopacity{0.000000}%
\pgfsetdash{}{0pt}%
\pgfpathmoveto{\pgfqpoint{1.200097in}{0.521603in}}%
\pgfpathlineto{\pgfqpoint{1.303584in}{0.521603in}}%
\pgfpathlineto{\pgfqpoint{1.303584in}{0.572198in}}%
\pgfpathlineto{\pgfqpoint{1.200097in}{0.572198in}}%
\pgfpathlineto{\pgfqpoint{1.200097in}{0.521603in}}%
\pgfpathclose%
\pgfusepath{fill}%
\end{pgfscope}%
\begin{pgfscope}%
\pgfpathrectangle{\pgfqpoint{0.755102in}{0.521603in}}{\pgfqpoint{2.959734in}{1.817415in}}%
\pgfusepath{clip}%
\pgfsetbuttcap%
\pgfsetmiterjoin%
\definecolor{currentfill}{rgb}{0.121569,0.466667,0.705882}%
\pgfsetfillcolor{currentfill}%
\pgfsetlinewidth{0.000000pt}%
\definecolor{currentstroke}{rgb}{0.000000,0.000000,0.000000}%
\pgfsetstrokecolor{currentstroke}%
\pgfsetstrokeopacity{0.000000}%
\pgfsetdash{}{0pt}%
\pgfpathmoveto{\pgfqpoint{1.303584in}{0.521603in}}%
\pgfpathlineto{\pgfqpoint{1.407071in}{0.521603in}}%
\pgfpathlineto{\pgfqpoint{1.407071in}{0.604153in}}%
\pgfpathlineto{\pgfqpoint{1.303584in}{0.604153in}}%
\pgfpathlineto{\pgfqpoint{1.303584in}{0.521603in}}%
\pgfpathclose%
\pgfusepath{fill}%
\end{pgfscope}%
\begin{pgfscope}%
\pgfpathrectangle{\pgfqpoint{0.755102in}{0.521603in}}{\pgfqpoint{2.959734in}{1.817415in}}%
\pgfusepath{clip}%
\pgfsetbuttcap%
\pgfsetmiterjoin%
\definecolor{currentfill}{rgb}{0.121569,0.466667,0.705882}%
\pgfsetfillcolor{currentfill}%
\pgfsetlinewidth{0.000000pt}%
\definecolor{currentstroke}{rgb}{0.000000,0.000000,0.000000}%
\pgfsetstrokecolor{currentstroke}%
\pgfsetstrokeopacity{0.000000}%
\pgfsetdash{}{0pt}%
\pgfpathmoveto{\pgfqpoint{1.407071in}{0.521603in}}%
\pgfpathlineto{\pgfqpoint{1.510559in}{0.521603in}}%
\pgfpathlineto{\pgfqpoint{1.510559in}{0.646759in}}%
\pgfpathlineto{\pgfqpoint{1.407071in}{0.646759in}}%
\pgfpathlineto{\pgfqpoint{1.407071in}{0.521603in}}%
\pgfpathclose%
\pgfusepath{fill}%
\end{pgfscope}%
\begin{pgfscope}%
\pgfpathrectangle{\pgfqpoint{0.755102in}{0.521603in}}{\pgfqpoint{2.959734in}{1.817415in}}%
\pgfusepath{clip}%
\pgfsetbuttcap%
\pgfsetmiterjoin%
\definecolor{currentfill}{rgb}{0.121569,0.466667,0.705882}%
\pgfsetfillcolor{currentfill}%
\pgfsetlinewidth{0.000000pt}%
\definecolor{currentstroke}{rgb}{0.000000,0.000000,0.000000}%
\pgfsetstrokecolor{currentstroke}%
\pgfsetstrokeopacity{0.000000}%
\pgfsetdash{}{0pt}%
\pgfpathmoveto{\pgfqpoint{1.510559in}{0.521603in}}%
\pgfpathlineto{\pgfqpoint{1.614046in}{0.521603in}}%
\pgfpathlineto{\pgfqpoint{1.614046in}{0.668062in}}%
\pgfpathlineto{\pgfqpoint{1.510559in}{0.668062in}}%
\pgfpathlineto{\pgfqpoint{1.510559in}{0.521603in}}%
\pgfpathclose%
\pgfusepath{fill}%
\end{pgfscope}%
\begin{pgfscope}%
\pgfpathrectangle{\pgfqpoint{0.755102in}{0.521603in}}{\pgfqpoint{2.959734in}{1.817415in}}%
\pgfusepath{clip}%
\pgfsetbuttcap%
\pgfsetmiterjoin%
\definecolor{currentfill}{rgb}{0.121569,0.466667,0.705882}%
\pgfsetfillcolor{currentfill}%
\pgfsetlinewidth{0.000000pt}%
\definecolor{currentstroke}{rgb}{0.000000,0.000000,0.000000}%
\pgfsetstrokecolor{currentstroke}%
\pgfsetstrokeopacity{0.000000}%
\pgfsetdash{}{0pt}%
\pgfpathmoveto{\pgfqpoint{1.614046in}{0.521603in}}%
\pgfpathlineto{\pgfqpoint{1.717533in}{0.521603in}}%
\pgfpathlineto{\pgfqpoint{1.717533in}{0.609478in}}%
\pgfpathlineto{\pgfqpoint{1.614046in}{0.609478in}}%
\pgfpathlineto{\pgfqpoint{1.614046in}{0.521603in}}%
\pgfpathclose%
\pgfusepath{fill}%
\end{pgfscope}%
\begin{pgfscope}%
\pgfpathrectangle{\pgfqpoint{0.755102in}{0.521603in}}{\pgfqpoint{2.959734in}{1.817415in}}%
\pgfusepath{clip}%
\pgfsetbuttcap%
\pgfsetmiterjoin%
\definecolor{currentfill}{rgb}{0.121569,0.466667,0.705882}%
\pgfsetfillcolor{currentfill}%
\pgfsetlinewidth{0.000000pt}%
\definecolor{currentstroke}{rgb}{0.000000,0.000000,0.000000}%
\pgfsetstrokecolor{currentstroke}%
\pgfsetstrokeopacity{0.000000}%
\pgfsetdash{}{0pt}%
\pgfpathmoveto{\pgfqpoint{1.717533in}{0.521603in}}%
\pgfpathlineto{\pgfqpoint{1.821020in}{0.521603in}}%
\pgfpathlineto{\pgfqpoint{1.821020in}{0.598827in}}%
\pgfpathlineto{\pgfqpoint{1.717533in}{0.598827in}}%
\pgfpathlineto{\pgfqpoint{1.717533in}{0.521603in}}%
\pgfpathclose%
\pgfusepath{fill}%
\end{pgfscope}%
\begin{pgfscope}%
\pgfpathrectangle{\pgfqpoint{0.755102in}{0.521603in}}{\pgfqpoint{2.959734in}{1.817415in}}%
\pgfusepath{clip}%
\pgfsetbuttcap%
\pgfsetmiterjoin%
\definecolor{currentfill}{rgb}{0.121569,0.466667,0.705882}%
\pgfsetfillcolor{currentfill}%
\pgfsetlinewidth{0.000000pt}%
\definecolor{currentstroke}{rgb}{0.000000,0.000000,0.000000}%
\pgfsetstrokecolor{currentstroke}%
\pgfsetstrokeopacity{0.000000}%
\pgfsetdash{}{0pt}%
\pgfpathmoveto{\pgfqpoint{1.821020in}{0.521603in}}%
\pgfpathlineto{\pgfqpoint{1.924507in}{0.521603in}}%
\pgfpathlineto{\pgfqpoint{1.924507in}{0.574861in}}%
\pgfpathlineto{\pgfqpoint{1.821020in}{0.574861in}}%
\pgfpathlineto{\pgfqpoint{1.821020in}{0.521603in}}%
\pgfpathclose%
\pgfusepath{fill}%
\end{pgfscope}%
\begin{pgfscope}%
\pgfpathrectangle{\pgfqpoint{0.755102in}{0.521603in}}{\pgfqpoint{2.959734in}{1.817415in}}%
\pgfusepath{clip}%
\pgfsetbuttcap%
\pgfsetmiterjoin%
\definecolor{currentfill}{rgb}{0.121569,0.466667,0.705882}%
\pgfsetfillcolor{currentfill}%
\pgfsetlinewidth{0.000000pt}%
\definecolor{currentstroke}{rgb}{0.000000,0.000000,0.000000}%
\pgfsetstrokecolor{currentstroke}%
\pgfsetstrokeopacity{0.000000}%
\pgfsetdash{}{0pt}%
\pgfpathmoveto{\pgfqpoint{1.924507in}{0.521603in}}%
\pgfpathlineto{\pgfqpoint{2.027995in}{0.521603in}}%
\pgfpathlineto{\pgfqpoint{2.027995in}{0.585512in}}%
\pgfpathlineto{\pgfqpoint{1.924507in}{0.585512in}}%
\pgfpathlineto{\pgfqpoint{1.924507in}{0.521603in}}%
\pgfpathclose%
\pgfusepath{fill}%
\end{pgfscope}%
\begin{pgfscope}%
\pgfpathrectangle{\pgfqpoint{0.755102in}{0.521603in}}{\pgfqpoint{2.959734in}{1.817415in}}%
\pgfusepath{clip}%
\pgfsetbuttcap%
\pgfsetmiterjoin%
\definecolor{currentfill}{rgb}{0.121569,0.466667,0.705882}%
\pgfsetfillcolor{currentfill}%
\pgfsetlinewidth{0.000000pt}%
\definecolor{currentstroke}{rgb}{0.000000,0.000000,0.000000}%
\pgfsetstrokecolor{currentstroke}%
\pgfsetstrokeopacity{0.000000}%
\pgfsetdash{}{0pt}%
\pgfpathmoveto{\pgfqpoint{2.027995in}{0.521603in}}%
\pgfpathlineto{\pgfqpoint{2.131482in}{0.521603in}}%
\pgfpathlineto{\pgfqpoint{2.131482in}{0.558884in}}%
\pgfpathlineto{\pgfqpoint{2.027995in}{0.558884in}}%
\pgfpathlineto{\pgfqpoint{2.027995in}{0.521603in}}%
\pgfpathclose%
\pgfusepath{fill}%
\end{pgfscope}%
\begin{pgfscope}%
\pgfpathrectangle{\pgfqpoint{0.755102in}{0.521603in}}{\pgfqpoint{2.959734in}{1.817415in}}%
\pgfusepath{clip}%
\pgfsetbuttcap%
\pgfsetmiterjoin%
\definecolor{currentfill}{rgb}{0.121569,0.466667,0.705882}%
\pgfsetfillcolor{currentfill}%
\pgfsetlinewidth{0.000000pt}%
\definecolor{currentstroke}{rgb}{0.000000,0.000000,0.000000}%
\pgfsetstrokecolor{currentstroke}%
\pgfsetstrokeopacity{0.000000}%
\pgfsetdash{}{0pt}%
\pgfpathmoveto{\pgfqpoint{2.131482in}{0.521603in}}%
\pgfpathlineto{\pgfqpoint{2.234969in}{0.521603in}}%
\pgfpathlineto{\pgfqpoint{2.234969in}{0.548232in}}%
\pgfpathlineto{\pgfqpoint{2.131482in}{0.548232in}}%
\pgfpathlineto{\pgfqpoint{2.131482in}{0.521603in}}%
\pgfpathclose%
\pgfusepath{fill}%
\end{pgfscope}%
\begin{pgfscope}%
\pgfpathrectangle{\pgfqpoint{0.755102in}{0.521603in}}{\pgfqpoint{2.959734in}{1.817415in}}%
\pgfusepath{clip}%
\pgfsetbuttcap%
\pgfsetmiterjoin%
\definecolor{currentfill}{rgb}{0.121569,0.466667,0.705882}%
\pgfsetfillcolor{currentfill}%
\pgfsetlinewidth{0.000000pt}%
\definecolor{currentstroke}{rgb}{0.000000,0.000000,0.000000}%
\pgfsetstrokecolor{currentstroke}%
\pgfsetstrokeopacity{0.000000}%
\pgfsetdash{}{0pt}%
\pgfpathmoveto{\pgfqpoint{2.234969in}{0.521603in}}%
\pgfpathlineto{\pgfqpoint{2.338456in}{0.521603in}}%
\pgfpathlineto{\pgfqpoint{2.338456in}{0.545569in}}%
\pgfpathlineto{\pgfqpoint{2.234969in}{0.545569in}}%
\pgfpathlineto{\pgfqpoint{2.234969in}{0.521603in}}%
\pgfpathclose%
\pgfusepath{fill}%
\end{pgfscope}%
\begin{pgfscope}%
\pgfpathrectangle{\pgfqpoint{0.755102in}{0.521603in}}{\pgfqpoint{2.959734in}{1.817415in}}%
\pgfusepath{clip}%
\pgfsetbuttcap%
\pgfsetmiterjoin%
\definecolor{currentfill}{rgb}{0.121569,0.466667,0.705882}%
\pgfsetfillcolor{currentfill}%
\pgfsetlinewidth{0.000000pt}%
\definecolor{currentstroke}{rgb}{0.000000,0.000000,0.000000}%
\pgfsetstrokecolor{currentstroke}%
\pgfsetstrokeopacity{0.000000}%
\pgfsetdash{}{0pt}%
\pgfpathmoveto{\pgfqpoint{2.338456in}{0.521603in}}%
\pgfpathlineto{\pgfqpoint{2.441944in}{0.521603in}}%
\pgfpathlineto{\pgfqpoint{2.441944in}{0.540243in}}%
\pgfpathlineto{\pgfqpoint{2.338456in}{0.540243in}}%
\pgfpathlineto{\pgfqpoint{2.338456in}{0.521603in}}%
\pgfpathclose%
\pgfusepath{fill}%
\end{pgfscope}%
\begin{pgfscope}%
\pgfpathrectangle{\pgfqpoint{0.755102in}{0.521603in}}{\pgfqpoint{2.959734in}{1.817415in}}%
\pgfusepath{clip}%
\pgfsetbuttcap%
\pgfsetmiterjoin%
\definecolor{currentfill}{rgb}{0.121569,0.466667,0.705882}%
\pgfsetfillcolor{currentfill}%
\pgfsetlinewidth{0.000000pt}%
\definecolor{currentstroke}{rgb}{0.000000,0.000000,0.000000}%
\pgfsetstrokecolor{currentstroke}%
\pgfsetstrokeopacity{0.000000}%
\pgfsetdash{}{0pt}%
\pgfpathmoveto{\pgfqpoint{2.441944in}{0.521603in}}%
\pgfpathlineto{\pgfqpoint{2.545431in}{0.521603in}}%
\pgfpathlineto{\pgfqpoint{2.545431in}{0.537581in}}%
\pgfpathlineto{\pgfqpoint{2.441944in}{0.537581in}}%
\pgfpathlineto{\pgfqpoint{2.441944in}{0.521603in}}%
\pgfpathclose%
\pgfusepath{fill}%
\end{pgfscope}%
\begin{pgfscope}%
\pgfpathrectangle{\pgfqpoint{0.755102in}{0.521603in}}{\pgfqpoint{2.959734in}{1.817415in}}%
\pgfusepath{clip}%
\pgfsetbuttcap%
\pgfsetmiterjoin%
\definecolor{currentfill}{rgb}{0.121569,0.466667,0.705882}%
\pgfsetfillcolor{currentfill}%
\pgfsetlinewidth{0.000000pt}%
\definecolor{currentstroke}{rgb}{0.000000,0.000000,0.000000}%
\pgfsetstrokecolor{currentstroke}%
\pgfsetstrokeopacity{0.000000}%
\pgfsetdash{}{0pt}%
\pgfpathmoveto{\pgfqpoint{2.545431in}{0.521603in}}%
\pgfpathlineto{\pgfqpoint{2.648918in}{0.521603in}}%
\pgfpathlineto{\pgfqpoint{2.648918in}{0.529592in}}%
\pgfpathlineto{\pgfqpoint{2.545431in}{0.529592in}}%
\pgfpathlineto{\pgfqpoint{2.545431in}{0.521603in}}%
\pgfpathclose%
\pgfusepath{fill}%
\end{pgfscope}%
\begin{pgfscope}%
\pgfpathrectangle{\pgfqpoint{0.755102in}{0.521603in}}{\pgfqpoint{2.959734in}{1.817415in}}%
\pgfusepath{clip}%
\pgfsetbuttcap%
\pgfsetmiterjoin%
\definecolor{currentfill}{rgb}{0.121569,0.466667,0.705882}%
\pgfsetfillcolor{currentfill}%
\pgfsetlinewidth{0.000000pt}%
\definecolor{currentstroke}{rgb}{0.000000,0.000000,0.000000}%
\pgfsetstrokecolor{currentstroke}%
\pgfsetstrokeopacity{0.000000}%
\pgfsetdash{}{0pt}%
\pgfpathmoveto{\pgfqpoint{2.648918in}{0.521603in}}%
\pgfpathlineto{\pgfqpoint{2.752405in}{0.521603in}}%
\pgfpathlineto{\pgfqpoint{2.752405in}{0.526929in}}%
\pgfpathlineto{\pgfqpoint{2.648918in}{0.526929in}}%
\pgfpathlineto{\pgfqpoint{2.648918in}{0.521603in}}%
\pgfpathclose%
\pgfusepath{fill}%
\end{pgfscope}%
\begin{pgfscope}%
\pgfpathrectangle{\pgfqpoint{0.755102in}{0.521603in}}{\pgfqpoint{2.959734in}{1.817415in}}%
\pgfusepath{clip}%
\pgfsetbuttcap%
\pgfsetmiterjoin%
\definecolor{currentfill}{rgb}{0.121569,0.466667,0.705882}%
\pgfsetfillcolor{currentfill}%
\pgfsetlinewidth{0.000000pt}%
\definecolor{currentstroke}{rgb}{0.000000,0.000000,0.000000}%
\pgfsetstrokecolor{currentstroke}%
\pgfsetstrokeopacity{0.000000}%
\pgfsetdash{}{0pt}%
\pgfpathmoveto{\pgfqpoint{2.752405in}{0.521603in}}%
\pgfpathlineto{\pgfqpoint{2.855892in}{0.521603in}}%
\pgfpathlineto{\pgfqpoint{2.855892in}{0.524266in}}%
\pgfpathlineto{\pgfqpoint{2.752405in}{0.524266in}}%
\pgfpathlineto{\pgfqpoint{2.752405in}{0.521603in}}%
\pgfpathclose%
\pgfusepath{fill}%
\end{pgfscope}%
\begin{pgfscope}%
\pgfpathrectangle{\pgfqpoint{0.755102in}{0.521603in}}{\pgfqpoint{2.959734in}{1.817415in}}%
\pgfusepath{clip}%
\pgfsetbuttcap%
\pgfsetmiterjoin%
\definecolor{currentfill}{rgb}{0.121569,0.466667,0.705882}%
\pgfsetfillcolor{currentfill}%
\pgfsetlinewidth{0.000000pt}%
\definecolor{currentstroke}{rgb}{0.000000,0.000000,0.000000}%
\pgfsetstrokecolor{currentstroke}%
\pgfsetstrokeopacity{0.000000}%
\pgfsetdash{}{0pt}%
\pgfpathmoveto{\pgfqpoint{2.855892in}{0.521603in}}%
\pgfpathlineto{\pgfqpoint{2.959380in}{0.521603in}}%
\pgfpathlineto{\pgfqpoint{2.959380in}{0.524266in}}%
\pgfpathlineto{\pgfqpoint{2.855892in}{0.524266in}}%
\pgfpathlineto{\pgfqpoint{2.855892in}{0.521603in}}%
\pgfpathclose%
\pgfusepath{fill}%
\end{pgfscope}%
\begin{pgfscope}%
\pgfpathrectangle{\pgfqpoint{0.755102in}{0.521603in}}{\pgfqpoint{2.959734in}{1.817415in}}%
\pgfusepath{clip}%
\pgfsetbuttcap%
\pgfsetmiterjoin%
\definecolor{currentfill}{rgb}{0.121569,0.466667,0.705882}%
\pgfsetfillcolor{currentfill}%
\pgfsetlinewidth{0.000000pt}%
\definecolor{currentstroke}{rgb}{0.000000,0.000000,0.000000}%
\pgfsetstrokecolor{currentstroke}%
\pgfsetstrokeopacity{0.000000}%
\pgfsetdash{}{0pt}%
\pgfpathmoveto{\pgfqpoint{2.959380in}{0.521603in}}%
\pgfpathlineto{\pgfqpoint{3.062867in}{0.521603in}}%
\pgfpathlineto{\pgfqpoint{3.062867in}{0.524266in}}%
\pgfpathlineto{\pgfqpoint{2.959380in}{0.524266in}}%
\pgfpathlineto{\pgfqpoint{2.959380in}{0.521603in}}%
\pgfpathclose%
\pgfusepath{fill}%
\end{pgfscope}%
\begin{pgfscope}%
\pgfpathrectangle{\pgfqpoint{0.755102in}{0.521603in}}{\pgfqpoint{2.959734in}{1.817415in}}%
\pgfusepath{clip}%
\pgfsetbuttcap%
\pgfsetmiterjoin%
\definecolor{currentfill}{rgb}{0.121569,0.466667,0.705882}%
\pgfsetfillcolor{currentfill}%
\pgfsetlinewidth{0.000000pt}%
\definecolor{currentstroke}{rgb}{0.000000,0.000000,0.000000}%
\pgfsetstrokecolor{currentstroke}%
\pgfsetstrokeopacity{0.000000}%
\pgfsetdash{}{0pt}%
\pgfpathmoveto{\pgfqpoint{3.062867in}{0.521603in}}%
\pgfpathlineto{\pgfqpoint{3.166354in}{0.521603in}}%
\pgfpathlineto{\pgfqpoint{3.166354in}{0.521603in}}%
\pgfpathlineto{\pgfqpoint{3.062867in}{0.521603in}}%
\pgfpathlineto{\pgfqpoint{3.062867in}{0.521603in}}%
\pgfpathclose%
\pgfusepath{fill}%
\end{pgfscope}%
\begin{pgfscope}%
\pgfpathrectangle{\pgfqpoint{0.755102in}{0.521603in}}{\pgfqpoint{2.959734in}{1.817415in}}%
\pgfusepath{clip}%
\pgfsetbuttcap%
\pgfsetmiterjoin%
\definecolor{currentfill}{rgb}{0.121569,0.466667,0.705882}%
\pgfsetfillcolor{currentfill}%
\pgfsetlinewidth{0.000000pt}%
\definecolor{currentstroke}{rgb}{0.000000,0.000000,0.000000}%
\pgfsetstrokecolor{currentstroke}%
\pgfsetstrokeopacity{0.000000}%
\pgfsetdash{}{0pt}%
\pgfpathmoveto{\pgfqpoint{3.166354in}{0.521603in}}%
\pgfpathlineto{\pgfqpoint{3.269841in}{0.521603in}}%
\pgfpathlineto{\pgfqpoint{3.269841in}{0.524266in}}%
\pgfpathlineto{\pgfqpoint{3.166354in}{0.524266in}}%
\pgfpathlineto{\pgfqpoint{3.166354in}{0.521603in}}%
\pgfpathclose%
\pgfusepath{fill}%
\end{pgfscope}%
\begin{pgfscope}%
\pgfpathrectangle{\pgfqpoint{0.755102in}{0.521603in}}{\pgfqpoint{2.959734in}{1.817415in}}%
\pgfusepath{clip}%
\pgfsetbuttcap%
\pgfsetmiterjoin%
\definecolor{currentfill}{rgb}{0.121569,0.466667,0.705882}%
\pgfsetfillcolor{currentfill}%
\pgfsetlinewidth{0.000000pt}%
\definecolor{currentstroke}{rgb}{0.000000,0.000000,0.000000}%
\pgfsetstrokecolor{currentstroke}%
\pgfsetstrokeopacity{0.000000}%
\pgfsetdash{}{0pt}%
\pgfpathmoveto{\pgfqpoint{3.269841in}{0.521603in}}%
\pgfpathlineto{\pgfqpoint{3.373328in}{0.521603in}}%
\pgfpathlineto{\pgfqpoint{3.373328in}{0.521603in}}%
\pgfpathlineto{\pgfqpoint{3.269841in}{0.521603in}}%
\pgfpathlineto{\pgfqpoint{3.269841in}{0.521603in}}%
\pgfpathclose%
\pgfusepath{fill}%
\end{pgfscope}%
\begin{pgfscope}%
\pgfpathrectangle{\pgfqpoint{0.755102in}{0.521603in}}{\pgfqpoint{2.959734in}{1.817415in}}%
\pgfusepath{clip}%
\pgfsetbuttcap%
\pgfsetmiterjoin%
\definecolor{currentfill}{rgb}{0.121569,0.466667,0.705882}%
\pgfsetfillcolor{currentfill}%
\pgfsetlinewidth{0.000000pt}%
\definecolor{currentstroke}{rgb}{0.000000,0.000000,0.000000}%
\pgfsetstrokecolor{currentstroke}%
\pgfsetstrokeopacity{0.000000}%
\pgfsetdash{}{0pt}%
\pgfpathmoveto{\pgfqpoint{3.373328in}{0.521603in}}%
\pgfpathlineto{\pgfqpoint{3.476816in}{0.521603in}}%
\pgfpathlineto{\pgfqpoint{3.476816in}{0.524266in}}%
\pgfpathlineto{\pgfqpoint{3.373328in}{0.524266in}}%
\pgfpathlineto{\pgfqpoint{3.373328in}{0.521603in}}%
\pgfpathclose%
\pgfusepath{fill}%
\end{pgfscope}%
\begin{pgfscope}%
\pgfpathrectangle{\pgfqpoint{0.755102in}{0.521603in}}{\pgfqpoint{2.959734in}{1.817415in}}%
\pgfusepath{clip}%
\pgfsetbuttcap%
\pgfsetmiterjoin%
\definecolor{currentfill}{rgb}{0.121569,0.466667,0.705882}%
\pgfsetfillcolor{currentfill}%
\pgfsetlinewidth{0.000000pt}%
\definecolor{currentstroke}{rgb}{0.000000,0.000000,0.000000}%
\pgfsetstrokecolor{currentstroke}%
\pgfsetstrokeopacity{0.000000}%
\pgfsetdash{}{0pt}%
\pgfpathmoveto{\pgfqpoint{3.476816in}{0.521603in}}%
\pgfpathlineto{\pgfqpoint{3.580303in}{0.521603in}}%
\pgfpathlineto{\pgfqpoint{3.580303in}{0.526929in}}%
\pgfpathlineto{\pgfqpoint{3.476816in}{0.526929in}}%
\pgfpathlineto{\pgfqpoint{3.476816in}{0.521603in}}%
\pgfpathclose%
\pgfusepath{fill}%
\end{pgfscope}%
\begin{pgfscope}%
\pgfsetbuttcap%
\pgfsetroundjoin%
\definecolor{currentfill}{rgb}{0.000000,0.000000,0.000000}%
\pgfsetfillcolor{currentfill}%
\pgfsetlinewidth{0.803000pt}%
\definecolor{currentstroke}{rgb}{0.000000,0.000000,0.000000}%
\pgfsetstrokecolor{currentstroke}%
\pgfsetdash{}{0pt}%
\pgfsys@defobject{currentmarker}{\pgfqpoint{0.000000in}{-0.048611in}}{\pgfqpoint{0.000000in}{0.000000in}}{%
\pgfpathmoveto{\pgfqpoint{0.000000in}{0.000000in}}%
\pgfpathlineto{\pgfqpoint{0.000000in}{-0.048611in}}%
\pgfusepath{stroke,fill}%
}%
\begin{pgfscope}%
\pgfsys@transformshift{0.877312in}{0.521603in}%
\pgfsys@useobject{currentmarker}{}%
\end{pgfscope}%
\end{pgfscope}%
\begin{pgfscope}%
\definecolor{textcolor}{rgb}{0.000000,0.000000,0.000000}%
\pgfsetstrokecolor{textcolor}%
\pgfsetfillcolor{textcolor}%
\pgftext[x=0.877312in,y=0.424381in,,top]{\color{textcolor}\sffamily\fontsize{10.000000}{12.000000}\selectfont 0.00}%
\end{pgfscope}%
\begin{pgfscope}%
\pgfsetbuttcap%
\pgfsetroundjoin%
\definecolor{currentfill}{rgb}{0.000000,0.000000,0.000000}%
\pgfsetfillcolor{currentfill}%
\pgfsetlinewidth{0.803000pt}%
\definecolor{currentstroke}{rgb}{0.000000,0.000000,0.000000}%
\pgfsetstrokecolor{currentstroke}%
\pgfsetdash{}{0pt}%
\pgfsys@defobject{currentmarker}{\pgfqpoint{0.000000in}{-0.048611in}}{\pgfqpoint{0.000000in}{0.000000in}}{%
\pgfpathmoveto{\pgfqpoint{0.000000in}{0.000000in}}%
\pgfpathlineto{\pgfqpoint{0.000000in}{-0.048611in}}%
\pgfusepath{stroke,fill}%
}%
\begin{pgfscope}%
\pgfsys@transformshift{1.709563in}{0.521603in}%
\pgfsys@useobject{currentmarker}{}%
\end{pgfscope}%
\end{pgfscope}%
\begin{pgfscope}%
\definecolor{textcolor}{rgb}{0.000000,0.000000,0.000000}%
\pgfsetstrokecolor{textcolor}%
\pgfsetfillcolor{textcolor}%
\pgftext[x=1.709563in,y=0.424381in,,top]{\color{textcolor}\sffamily\fontsize{10.000000}{12.000000}\selectfont 0.02}%
\end{pgfscope}%
\begin{pgfscope}%
\pgfsetbuttcap%
\pgfsetroundjoin%
\definecolor{currentfill}{rgb}{0.000000,0.000000,0.000000}%
\pgfsetfillcolor{currentfill}%
\pgfsetlinewidth{0.803000pt}%
\definecolor{currentstroke}{rgb}{0.000000,0.000000,0.000000}%
\pgfsetstrokecolor{currentstroke}%
\pgfsetdash{}{0pt}%
\pgfsys@defobject{currentmarker}{\pgfqpoint{0.000000in}{-0.048611in}}{\pgfqpoint{0.000000in}{0.000000in}}{%
\pgfpathmoveto{\pgfqpoint{0.000000in}{0.000000in}}%
\pgfpathlineto{\pgfqpoint{0.000000in}{-0.048611in}}%
\pgfusepath{stroke,fill}%
}%
\begin{pgfscope}%
\pgfsys@transformshift{2.541814in}{0.521603in}%
\pgfsys@useobject{currentmarker}{}%
\end{pgfscope}%
\end{pgfscope}%
\begin{pgfscope}%
\definecolor{textcolor}{rgb}{0.000000,0.000000,0.000000}%
\pgfsetstrokecolor{textcolor}%
\pgfsetfillcolor{textcolor}%
\pgftext[x=2.541814in,y=0.424381in,,top]{\color{textcolor}\sffamily\fontsize{10.000000}{12.000000}\selectfont 0.04}%
\end{pgfscope}%
\begin{pgfscope}%
\pgfsetbuttcap%
\pgfsetroundjoin%
\definecolor{currentfill}{rgb}{0.000000,0.000000,0.000000}%
\pgfsetfillcolor{currentfill}%
\pgfsetlinewidth{0.803000pt}%
\definecolor{currentstroke}{rgb}{0.000000,0.000000,0.000000}%
\pgfsetstrokecolor{currentstroke}%
\pgfsetdash{}{0pt}%
\pgfsys@defobject{currentmarker}{\pgfqpoint{0.000000in}{-0.048611in}}{\pgfqpoint{0.000000in}{0.000000in}}{%
\pgfpathmoveto{\pgfqpoint{0.000000in}{0.000000in}}%
\pgfpathlineto{\pgfqpoint{0.000000in}{-0.048611in}}%
\pgfusepath{stroke,fill}%
}%
\begin{pgfscope}%
\pgfsys@transformshift{3.374065in}{0.521603in}%
\pgfsys@useobject{currentmarker}{}%
\end{pgfscope}%
\end{pgfscope}%
\begin{pgfscope}%
\definecolor{textcolor}{rgb}{0.000000,0.000000,0.000000}%
\pgfsetstrokecolor{textcolor}%
\pgfsetfillcolor{textcolor}%
\pgftext[x=3.374065in,y=0.424381in,,top]{\color{textcolor}\sffamily\fontsize{10.000000}{12.000000}\selectfont 0.06}%
\end{pgfscope}%
\begin{pgfscope}%
\definecolor{textcolor}{rgb}{0.000000,0.000000,0.000000}%
\pgfsetstrokecolor{textcolor}%
\pgfsetfillcolor{textcolor}%
\pgftext[x=2.234969in,y=0.234413in,,top]{\color{textcolor}\sffamily\fontsize{10.000000}{12.000000}\selectfont error}%
\end{pgfscope}%
\begin{pgfscope}%
\pgfsetbuttcap%
\pgfsetroundjoin%
\definecolor{currentfill}{rgb}{0.000000,0.000000,0.000000}%
\pgfsetfillcolor{currentfill}%
\pgfsetlinewidth{0.803000pt}%
\definecolor{currentstroke}{rgb}{0.000000,0.000000,0.000000}%
\pgfsetstrokecolor{currentstroke}%
\pgfsetdash{}{0pt}%
\pgfsys@defobject{currentmarker}{\pgfqpoint{-0.048611in}{0.000000in}}{\pgfqpoint{-0.000000in}{0.000000in}}{%
\pgfpathmoveto{\pgfqpoint{-0.000000in}{0.000000in}}%
\pgfpathlineto{\pgfqpoint{-0.048611in}{0.000000in}}%
\pgfusepath{stroke,fill}%
}%
\begin{pgfscope}%
\pgfsys@transformshift{0.755102in}{0.521603in}%
\pgfsys@useobject{currentmarker}{}%
\end{pgfscope}%
\end{pgfscope}%
\begin{pgfscope}%
\definecolor{textcolor}{rgb}{0.000000,0.000000,0.000000}%
\pgfsetstrokecolor{textcolor}%
\pgfsetfillcolor{textcolor}%
\pgftext[x=0.569514in, y=0.468842in, left, base]{\color{textcolor}\sffamily\fontsize{10.000000}{12.000000}\selectfont 0}%
\end{pgfscope}%
\begin{pgfscope}%
\pgfsetbuttcap%
\pgfsetroundjoin%
\definecolor{currentfill}{rgb}{0.000000,0.000000,0.000000}%
\pgfsetfillcolor{currentfill}%
\pgfsetlinewidth{0.803000pt}%
\definecolor{currentstroke}{rgb}{0.000000,0.000000,0.000000}%
\pgfsetstrokecolor{currentstroke}%
\pgfsetdash{}{0pt}%
\pgfsys@defobject{currentmarker}{\pgfqpoint{-0.048611in}{0.000000in}}{\pgfqpoint{-0.000000in}{0.000000in}}{%
\pgfpathmoveto{\pgfqpoint{-0.000000in}{0.000000in}}%
\pgfpathlineto{\pgfqpoint{-0.048611in}{0.000000in}}%
\pgfusepath{stroke,fill}%
}%
\begin{pgfscope}%
\pgfsys@transformshift{0.755102in}{1.054179in}%
\pgfsys@useobject{currentmarker}{}%
\end{pgfscope}%
\end{pgfscope}%
\begin{pgfscope}%
\definecolor{textcolor}{rgb}{0.000000,0.000000,0.000000}%
\pgfsetstrokecolor{textcolor}%
\pgfsetfillcolor{textcolor}%
\pgftext[x=0.392784in, y=1.001418in, left, base]{\color{textcolor}\sffamily\fontsize{10.000000}{12.000000}\selectfont 200}%
\end{pgfscope}%
\begin{pgfscope}%
\pgfsetbuttcap%
\pgfsetroundjoin%
\definecolor{currentfill}{rgb}{0.000000,0.000000,0.000000}%
\pgfsetfillcolor{currentfill}%
\pgfsetlinewidth{0.803000pt}%
\definecolor{currentstroke}{rgb}{0.000000,0.000000,0.000000}%
\pgfsetstrokecolor{currentstroke}%
\pgfsetdash{}{0pt}%
\pgfsys@defobject{currentmarker}{\pgfqpoint{-0.048611in}{0.000000in}}{\pgfqpoint{-0.000000in}{0.000000in}}{%
\pgfpathmoveto{\pgfqpoint{-0.000000in}{0.000000in}}%
\pgfpathlineto{\pgfqpoint{-0.048611in}{0.000000in}}%
\pgfusepath{stroke,fill}%
}%
\begin{pgfscope}%
\pgfsys@transformshift{0.755102in}{1.586755in}%
\pgfsys@useobject{currentmarker}{}%
\end{pgfscope}%
\end{pgfscope}%
\begin{pgfscope}%
\definecolor{textcolor}{rgb}{0.000000,0.000000,0.000000}%
\pgfsetstrokecolor{textcolor}%
\pgfsetfillcolor{textcolor}%
\pgftext[x=0.392784in, y=1.533993in, left, base]{\color{textcolor}\sffamily\fontsize{10.000000}{12.000000}\selectfont 400}%
\end{pgfscope}%
\begin{pgfscope}%
\pgfsetbuttcap%
\pgfsetroundjoin%
\definecolor{currentfill}{rgb}{0.000000,0.000000,0.000000}%
\pgfsetfillcolor{currentfill}%
\pgfsetlinewidth{0.803000pt}%
\definecolor{currentstroke}{rgb}{0.000000,0.000000,0.000000}%
\pgfsetstrokecolor{currentstroke}%
\pgfsetdash{}{0pt}%
\pgfsys@defobject{currentmarker}{\pgfqpoint{-0.048611in}{0.000000in}}{\pgfqpoint{-0.000000in}{0.000000in}}{%
\pgfpathmoveto{\pgfqpoint{-0.000000in}{0.000000in}}%
\pgfpathlineto{\pgfqpoint{-0.048611in}{0.000000in}}%
\pgfusepath{stroke,fill}%
}%
\begin{pgfscope}%
\pgfsys@transformshift{0.755102in}{2.119331in}%
\pgfsys@useobject{currentmarker}{}%
\end{pgfscope}%
\end{pgfscope}%
\begin{pgfscope}%
\definecolor{textcolor}{rgb}{0.000000,0.000000,0.000000}%
\pgfsetstrokecolor{textcolor}%
\pgfsetfillcolor{textcolor}%
\pgftext[x=0.392784in, y=2.066569in, left, base]{\color{textcolor}\sffamily\fontsize{10.000000}{12.000000}\selectfont 600}%
\end{pgfscope}%
\begin{pgfscope}%
\definecolor{textcolor}{rgb}{0.000000,0.000000,0.000000}%
\pgfsetstrokecolor{textcolor}%
\pgfsetfillcolor{textcolor}%
\pgftext[x=0.337228in,y=1.430311in,,bottom,rotate=90.000000]{\color{textcolor}\sffamily\fontsize{10.000000}{12.000000}\selectfont count}%
\end{pgfscope}%
\begin{pgfscope}%
\pgfsetrectcap%
\pgfsetmiterjoin%
\pgfsetlinewidth{0.803000pt}%
\definecolor{currentstroke}{rgb}{0.000000,0.000000,0.000000}%
\pgfsetstrokecolor{currentstroke}%
\pgfsetdash{}{0pt}%
\pgfpathmoveto{\pgfqpoint{0.755102in}{0.521603in}}%
\pgfpathlineto{\pgfqpoint{0.755102in}{2.339018in}}%
\pgfusepath{stroke}%
\end{pgfscope}%
\begin{pgfscope}%
\pgfsetrectcap%
\pgfsetmiterjoin%
\pgfsetlinewidth{0.803000pt}%
\definecolor{currentstroke}{rgb}{0.000000,0.000000,0.000000}%
\pgfsetstrokecolor{currentstroke}%
\pgfsetdash{}{0pt}%
\pgfpathmoveto{\pgfqpoint{3.714836in}{0.521603in}}%
\pgfpathlineto{\pgfqpoint{3.714836in}{2.339018in}}%
\pgfusepath{stroke}%
\end{pgfscope}%
\begin{pgfscope}%
\pgfsetrectcap%
\pgfsetmiterjoin%
\pgfsetlinewidth{0.803000pt}%
\definecolor{currentstroke}{rgb}{0.000000,0.000000,0.000000}%
\pgfsetstrokecolor{currentstroke}%
\pgfsetdash{}{0pt}%
\pgfpathmoveto{\pgfqpoint{0.755102in}{0.521603in}}%
\pgfpathlineto{\pgfqpoint{3.714836in}{0.521603in}}%
\pgfusepath{stroke}%
\end{pgfscope}%
\begin{pgfscope}%
\pgfsetrectcap%
\pgfsetmiterjoin%
\pgfsetlinewidth{0.803000pt}%
\definecolor{currentstroke}{rgb}{0.000000,0.000000,0.000000}%
\pgfsetstrokecolor{currentstroke}%
\pgfsetdash{}{0pt}%
\pgfpathmoveto{\pgfqpoint{0.755102in}{2.339018in}}%
\pgfpathlineto{\pgfqpoint{3.714836in}{2.339018in}}%
\pgfusepath{stroke}%
\end{pgfscope}%
\begin{pgfscope}%
\definecolor{textcolor}{rgb}{0.000000,0.000000,0.000000}%
\pgfsetstrokecolor{textcolor}%
\pgfsetfillcolor{textcolor}%
\pgftext[x=2.234969in,y=2.422352in,,base]{\color{textcolor}\sffamily\fontsize{12.000000}{14.400000}\selectfont MSE of the magnetic susceptibility from a 1D chain}%
\end{pgfscope}%
\end{pgfpicture}%
\makeatother%
\endgroup%

	\caption{Распределение значений среднеквадратичного отклонения магнитной восприимчивости конформаций от одномерной цепочки длины 1000.}
	\label{fig:MS_1D_dif_distr}
\end{figure}

При рассмотрении пиков магнитной восприимчивости, распределение которых представлено на рис. \ref{fig:MS_peaks_distr}, видно что у большинства конформаций (больше 90\%) пик отсутствует и магнитная восприимчивость достигает максимума при $\beta = 1$. По данному графику можно предположить, что при увеличении длины конформаций так же увеличивается доля конформаций с пиком в $\beta = 1$, но особенность при $L=500$ не позволяет утверждать что-то однозначно. Для подтверждения требуется повторить замеры с большим количеством конформаций


\begin{figure}[ht]
	\centering
	%% Creator: Matplotlib, PGF backend
%%
%% To include the figure in your LaTeX document, write
%%   \input{<filename>.pgf}
%%
%% Make sure the required packages are loaded in your preamble
%%   \usepackage{pgf}
%%
%% Also ensure that all the required font packages are loaded; for instance,
%% the lmodern package is sometimes necessary when using math font.
%%   \usepackage{lmodern}
%%
%% Figures using additional raster images can only be included by \input if
%% they are in the same directory as the main LaTeX file. For loading figures
%% from other directories you can use the `import` package
%%   \usepackage{import}
%%
%% and then include the figures with
%%   \import{<path to file>}{<filename>.pgf}
%%
%% Matplotlib used the following preamble
%%   
%%   \usepackage{fontspec}
%%   \setmainfont{DejaVuSerif.ttf}[Path=\detokenize{/home/roman/anaconda3/envs/ising/lib/python3.8/site-packages/matplotlib/mpl-data/fonts/ttf/}]
%%   \setsansfont{DejaVuSans.ttf}[Path=\detokenize{/home/roman/anaconda3/envs/ising/lib/python3.8/site-packages/matplotlib/mpl-data/fonts/ttf/}]
%%   \setmonofont{DejaVuSansMono.ttf}[Path=\detokenize{/home/roman/anaconda3/envs/ising/lib/python3.8/site-packages/matplotlib/mpl-data/fonts/ttf/}]
%%   \makeatletter\@ifpackageloaded{underscore}{}{\usepackage[strings]{underscore}}\makeatother
%%
\begingroup%
\makeatletter%
\begin{pgfpicture}%
\pgfpathrectangle{\pgfpointorigin}{\pgfqpoint{3.702812in}{2.392613in}}%
\pgfusepath{use as bounding box, clip}%
\begin{pgfscope}%
\pgfsetbuttcap%
\pgfsetmiterjoin%
\definecolor{currentfill}{rgb}{1.000000,1.000000,1.000000}%
\pgfsetfillcolor{currentfill}%
\pgfsetlinewidth{0.000000pt}%
\definecolor{currentstroke}{rgb}{1.000000,1.000000,1.000000}%
\pgfsetstrokecolor{currentstroke}%
\pgfsetdash{}{0pt}%
\pgfpathmoveto{\pgfqpoint{0.000000in}{0.000000in}}%
\pgfpathlineto{\pgfqpoint{3.702812in}{0.000000in}}%
\pgfpathlineto{\pgfqpoint{3.702812in}{2.392613in}}%
\pgfpathlineto{\pgfqpoint{0.000000in}{2.392613in}}%
\pgfpathlineto{\pgfqpoint{0.000000in}{0.000000in}}%
\pgfpathclose%
\pgfusepath{fill}%
\end{pgfscope}%
\begin{pgfscope}%
\pgfsetbuttcap%
\pgfsetmiterjoin%
\definecolor{currentfill}{rgb}{1.000000,1.000000,1.000000}%
\pgfsetfillcolor{currentfill}%
\pgfsetlinewidth{0.000000pt}%
\definecolor{currentstroke}{rgb}{0.000000,0.000000,0.000000}%
\pgfsetstrokecolor{currentstroke}%
\pgfsetstrokeopacity{0.000000}%
\pgfsetdash{}{0pt}%
\pgfpathmoveto{\pgfqpoint{0.643077in}{0.467838in}}%
\pgfpathlineto{\pgfqpoint{3.602812in}{0.467838in}}%
\pgfpathlineto{\pgfqpoint{3.602812in}{2.285253in}}%
\pgfpathlineto{\pgfqpoint{0.643077in}{2.285253in}}%
\pgfpathlineto{\pgfqpoint{0.643077in}{0.467838in}}%
\pgfpathclose%
\pgfusepath{fill}%
\end{pgfscope}%
\begin{pgfscope}%
\pgfpathrectangle{\pgfqpoint{0.643077in}{0.467838in}}{\pgfqpoint{2.959734in}{1.817415in}}%
\pgfusepath{clip}%
\pgfsetbuttcap%
\pgfsetmiterjoin%
\definecolor{currentfill}{rgb}{0.121569,0.466667,0.705882}%
\pgfsetfillcolor{currentfill}%
\pgfsetlinewidth{0.000000pt}%
\definecolor{currentstroke}{rgb}{0.000000,0.000000,0.000000}%
\pgfsetstrokecolor{currentstroke}%
\pgfsetstrokeopacity{0.000000}%
\pgfsetdash{}{0pt}%
\pgfpathmoveto{\pgfqpoint{0.777611in}{0.467838in}}%
\pgfpathlineto{\pgfqpoint{0.819652in}{0.467838in}}%
\pgfpathlineto{\pgfqpoint{0.819652in}{0.471403in}}%
\pgfpathlineto{\pgfqpoint{0.777611in}{0.471403in}}%
\pgfpathlineto{\pgfqpoint{0.777611in}{0.467838in}}%
\pgfpathclose%
\pgfusepath{fill}%
\end{pgfscope}%
\begin{pgfscope}%
\pgfpathrectangle{\pgfqpoint{0.643077in}{0.467838in}}{\pgfqpoint{2.959734in}{1.817415in}}%
\pgfusepath{clip}%
\pgfsetbuttcap%
\pgfsetmiterjoin%
\definecolor{currentfill}{rgb}{0.121569,0.466667,0.705882}%
\pgfsetfillcolor{currentfill}%
\pgfsetlinewidth{0.000000pt}%
\definecolor{currentstroke}{rgb}{0.000000,0.000000,0.000000}%
\pgfsetstrokecolor{currentstroke}%
\pgfsetstrokeopacity{0.000000}%
\pgfsetdash{}{0pt}%
\pgfpathmoveto{\pgfqpoint{0.987819in}{0.467838in}}%
\pgfpathlineto{\pgfqpoint{1.029861in}{0.467838in}}%
\pgfpathlineto{\pgfqpoint{1.029861in}{0.467838in}}%
\pgfpathlineto{\pgfqpoint{0.987819in}{0.467838in}}%
\pgfpathlineto{\pgfqpoint{0.987819in}{0.467838in}}%
\pgfpathclose%
\pgfusepath{fill}%
\end{pgfscope}%
\begin{pgfscope}%
\pgfpathrectangle{\pgfqpoint{0.643077in}{0.467838in}}{\pgfqpoint{2.959734in}{1.817415in}}%
\pgfusepath{clip}%
\pgfsetbuttcap%
\pgfsetmiterjoin%
\definecolor{currentfill}{rgb}{0.121569,0.466667,0.705882}%
\pgfsetfillcolor{currentfill}%
\pgfsetlinewidth{0.000000pt}%
\definecolor{currentstroke}{rgb}{0.000000,0.000000,0.000000}%
\pgfsetstrokecolor{currentstroke}%
\pgfsetstrokeopacity{0.000000}%
\pgfsetdash{}{0pt}%
\pgfpathmoveto{\pgfqpoint{1.198027in}{0.467838in}}%
\pgfpathlineto{\pgfqpoint{1.240069in}{0.467838in}}%
\pgfpathlineto{\pgfqpoint{1.240069in}{0.467838in}}%
\pgfpathlineto{\pgfqpoint{1.198027in}{0.467838in}}%
\pgfpathlineto{\pgfqpoint{1.198027in}{0.467838in}}%
\pgfpathclose%
\pgfusepath{fill}%
\end{pgfscope}%
\begin{pgfscope}%
\pgfpathrectangle{\pgfqpoint{0.643077in}{0.467838in}}{\pgfqpoint{2.959734in}{1.817415in}}%
\pgfusepath{clip}%
\pgfsetbuttcap%
\pgfsetmiterjoin%
\definecolor{currentfill}{rgb}{0.121569,0.466667,0.705882}%
\pgfsetfillcolor{currentfill}%
\pgfsetlinewidth{0.000000pt}%
\definecolor{currentstroke}{rgb}{0.000000,0.000000,0.000000}%
\pgfsetstrokecolor{currentstroke}%
\pgfsetstrokeopacity{0.000000}%
\pgfsetdash{}{0pt}%
\pgfpathmoveto{\pgfqpoint{1.408236in}{0.467838in}}%
\pgfpathlineto{\pgfqpoint{1.450277in}{0.467838in}}%
\pgfpathlineto{\pgfqpoint{1.450277in}{0.498142in}}%
\pgfpathlineto{\pgfqpoint{1.408236in}{0.498142in}}%
\pgfpathlineto{\pgfqpoint{1.408236in}{0.467838in}}%
\pgfpathclose%
\pgfusepath{fill}%
\end{pgfscope}%
\begin{pgfscope}%
\pgfpathrectangle{\pgfqpoint{0.643077in}{0.467838in}}{\pgfqpoint{2.959734in}{1.817415in}}%
\pgfusepath{clip}%
\pgfsetbuttcap%
\pgfsetmiterjoin%
\definecolor{currentfill}{rgb}{0.121569,0.466667,0.705882}%
\pgfsetfillcolor{currentfill}%
\pgfsetlinewidth{0.000000pt}%
\definecolor{currentstroke}{rgb}{0.000000,0.000000,0.000000}%
\pgfsetstrokecolor{currentstroke}%
\pgfsetstrokeopacity{0.000000}%
\pgfsetdash{}{0pt}%
\pgfpathmoveto{\pgfqpoint{1.618444in}{0.467838in}}%
\pgfpathlineto{\pgfqpoint{1.660486in}{0.467838in}}%
\pgfpathlineto{\pgfqpoint{1.660486in}{0.467838in}}%
\pgfpathlineto{\pgfqpoint{1.618444in}{0.467838in}}%
\pgfpathlineto{\pgfqpoint{1.618444in}{0.467838in}}%
\pgfpathclose%
\pgfusepath{fill}%
\end{pgfscope}%
\begin{pgfscope}%
\pgfpathrectangle{\pgfqpoint{0.643077in}{0.467838in}}{\pgfqpoint{2.959734in}{1.817415in}}%
\pgfusepath{clip}%
\pgfsetbuttcap%
\pgfsetmiterjoin%
\definecolor{currentfill}{rgb}{0.121569,0.466667,0.705882}%
\pgfsetfillcolor{currentfill}%
\pgfsetlinewidth{0.000000pt}%
\definecolor{currentstroke}{rgb}{0.000000,0.000000,0.000000}%
\pgfsetstrokecolor{currentstroke}%
\pgfsetstrokeopacity{0.000000}%
\pgfsetdash{}{0pt}%
\pgfpathmoveto{\pgfqpoint{1.828653in}{0.467838in}}%
\pgfpathlineto{\pgfqpoint{1.870694in}{0.467838in}}%
\pgfpathlineto{\pgfqpoint{1.870694in}{0.489229in}}%
\pgfpathlineto{\pgfqpoint{1.828653in}{0.489229in}}%
\pgfpathlineto{\pgfqpoint{1.828653in}{0.467838in}}%
\pgfpathclose%
\pgfusepath{fill}%
\end{pgfscope}%
\begin{pgfscope}%
\pgfpathrectangle{\pgfqpoint{0.643077in}{0.467838in}}{\pgfqpoint{2.959734in}{1.817415in}}%
\pgfusepath{clip}%
\pgfsetbuttcap%
\pgfsetmiterjoin%
\definecolor{currentfill}{rgb}{0.121569,0.466667,0.705882}%
\pgfsetfillcolor{currentfill}%
\pgfsetlinewidth{0.000000pt}%
\definecolor{currentstroke}{rgb}{0.000000,0.000000,0.000000}%
\pgfsetstrokecolor{currentstroke}%
\pgfsetstrokeopacity{0.000000}%
\pgfsetdash{}{0pt}%
\pgfpathmoveto{\pgfqpoint{2.038861in}{0.467838in}}%
\pgfpathlineto{\pgfqpoint{2.080903in}{0.467838in}}%
\pgfpathlineto{\pgfqpoint{2.080903in}{0.467838in}}%
\pgfpathlineto{\pgfqpoint{2.038861in}{0.467838in}}%
\pgfpathlineto{\pgfqpoint{2.038861in}{0.467838in}}%
\pgfpathclose%
\pgfusepath{fill}%
\end{pgfscope}%
\begin{pgfscope}%
\pgfpathrectangle{\pgfqpoint{0.643077in}{0.467838in}}{\pgfqpoint{2.959734in}{1.817415in}}%
\pgfusepath{clip}%
\pgfsetbuttcap%
\pgfsetmiterjoin%
\definecolor{currentfill}{rgb}{0.121569,0.466667,0.705882}%
\pgfsetfillcolor{currentfill}%
\pgfsetlinewidth{0.000000pt}%
\definecolor{currentstroke}{rgb}{0.000000,0.000000,0.000000}%
\pgfsetstrokecolor{currentstroke}%
\pgfsetstrokeopacity{0.000000}%
\pgfsetdash{}{0pt}%
\pgfpathmoveto{\pgfqpoint{2.249069in}{0.467838in}}%
\pgfpathlineto{\pgfqpoint{2.291111in}{0.467838in}}%
\pgfpathlineto{\pgfqpoint{2.291111in}{0.467838in}}%
\pgfpathlineto{\pgfqpoint{2.249069in}{0.467838in}}%
\pgfpathlineto{\pgfqpoint{2.249069in}{0.467838in}}%
\pgfpathclose%
\pgfusepath{fill}%
\end{pgfscope}%
\begin{pgfscope}%
\pgfpathrectangle{\pgfqpoint{0.643077in}{0.467838in}}{\pgfqpoint{2.959734in}{1.817415in}}%
\pgfusepath{clip}%
\pgfsetbuttcap%
\pgfsetmiterjoin%
\definecolor{currentfill}{rgb}{0.121569,0.466667,0.705882}%
\pgfsetfillcolor{currentfill}%
\pgfsetlinewidth{0.000000pt}%
\definecolor{currentstroke}{rgb}{0.000000,0.000000,0.000000}%
\pgfsetstrokecolor{currentstroke}%
\pgfsetstrokeopacity{0.000000}%
\pgfsetdash{}{0pt}%
\pgfpathmoveto{\pgfqpoint{2.459278in}{0.467838in}}%
\pgfpathlineto{\pgfqpoint{2.501320in}{0.467838in}}%
\pgfpathlineto{\pgfqpoint{2.501320in}{0.483881in}}%
\pgfpathlineto{\pgfqpoint{2.459278in}{0.483881in}}%
\pgfpathlineto{\pgfqpoint{2.459278in}{0.467838in}}%
\pgfpathclose%
\pgfusepath{fill}%
\end{pgfscope}%
\begin{pgfscope}%
\pgfpathrectangle{\pgfqpoint{0.643077in}{0.467838in}}{\pgfqpoint{2.959734in}{1.817415in}}%
\pgfusepath{clip}%
\pgfsetbuttcap%
\pgfsetmiterjoin%
\definecolor{currentfill}{rgb}{0.121569,0.466667,0.705882}%
\pgfsetfillcolor{currentfill}%
\pgfsetlinewidth{0.000000pt}%
\definecolor{currentstroke}{rgb}{0.000000,0.000000,0.000000}%
\pgfsetstrokecolor{currentstroke}%
\pgfsetstrokeopacity{0.000000}%
\pgfsetdash{}{0pt}%
\pgfpathmoveto{\pgfqpoint{2.669486in}{0.467838in}}%
\pgfpathlineto{\pgfqpoint{2.711528in}{0.467838in}}%
\pgfpathlineto{\pgfqpoint{2.711528in}{0.467838in}}%
\pgfpathlineto{\pgfqpoint{2.669486in}{0.467838in}}%
\pgfpathlineto{\pgfqpoint{2.669486in}{0.467838in}}%
\pgfpathclose%
\pgfusepath{fill}%
\end{pgfscope}%
\begin{pgfscope}%
\pgfpathrectangle{\pgfqpoint{0.643077in}{0.467838in}}{\pgfqpoint{2.959734in}{1.817415in}}%
\pgfusepath{clip}%
\pgfsetbuttcap%
\pgfsetmiterjoin%
\definecolor{currentfill}{rgb}{0.121569,0.466667,0.705882}%
\pgfsetfillcolor{currentfill}%
\pgfsetlinewidth{0.000000pt}%
\definecolor{currentstroke}{rgb}{0.000000,0.000000,0.000000}%
\pgfsetstrokecolor{currentstroke}%
\pgfsetstrokeopacity{0.000000}%
\pgfsetdash{}{0pt}%
\pgfpathmoveto{\pgfqpoint{2.879695in}{0.467838in}}%
\pgfpathlineto{\pgfqpoint{2.921736in}{0.467838in}}%
\pgfpathlineto{\pgfqpoint{2.921736in}{0.487447in}}%
\pgfpathlineto{\pgfqpoint{2.879695in}{0.487447in}}%
\pgfpathlineto{\pgfqpoint{2.879695in}{0.467838in}}%
\pgfpathclose%
\pgfusepath{fill}%
\end{pgfscope}%
\begin{pgfscope}%
\pgfpathrectangle{\pgfqpoint{0.643077in}{0.467838in}}{\pgfqpoint{2.959734in}{1.817415in}}%
\pgfusepath{clip}%
\pgfsetbuttcap%
\pgfsetmiterjoin%
\definecolor{currentfill}{rgb}{0.121569,0.466667,0.705882}%
\pgfsetfillcolor{currentfill}%
\pgfsetlinewidth{0.000000pt}%
\definecolor{currentstroke}{rgb}{0.000000,0.000000,0.000000}%
\pgfsetstrokecolor{currentstroke}%
\pgfsetstrokeopacity{0.000000}%
\pgfsetdash{}{0pt}%
\pgfpathmoveto{\pgfqpoint{3.089903in}{0.467838in}}%
\pgfpathlineto{\pgfqpoint{3.131945in}{0.467838in}}%
\pgfpathlineto{\pgfqpoint{3.131945in}{0.467838in}}%
\pgfpathlineto{\pgfqpoint{3.089903in}{0.467838in}}%
\pgfpathlineto{\pgfqpoint{3.089903in}{0.467838in}}%
\pgfpathclose%
\pgfusepath{fill}%
\end{pgfscope}%
\begin{pgfscope}%
\pgfpathrectangle{\pgfqpoint{0.643077in}{0.467838in}}{\pgfqpoint{2.959734in}{1.817415in}}%
\pgfusepath{clip}%
\pgfsetbuttcap%
\pgfsetmiterjoin%
\definecolor{currentfill}{rgb}{0.121569,0.466667,0.705882}%
\pgfsetfillcolor{currentfill}%
\pgfsetlinewidth{0.000000pt}%
\definecolor{currentstroke}{rgb}{0.000000,0.000000,0.000000}%
\pgfsetstrokecolor{currentstroke}%
\pgfsetstrokeopacity{0.000000}%
\pgfsetdash{}{0pt}%
\pgfpathmoveto{\pgfqpoint{3.300111in}{0.467838in}}%
\pgfpathlineto{\pgfqpoint{3.342153in}{0.467838in}}%
\pgfpathlineto{\pgfqpoint{3.342153in}{2.159493in}}%
\pgfpathlineto{\pgfqpoint{3.300111in}{2.159493in}}%
\pgfpathlineto{\pgfqpoint{3.300111in}{0.467838in}}%
\pgfpathclose%
\pgfusepath{fill}%
\end{pgfscope}%
\begin{pgfscope}%
\pgfpathrectangle{\pgfqpoint{0.643077in}{0.467838in}}{\pgfqpoint{2.959734in}{1.817415in}}%
\pgfusepath{clip}%
\pgfsetbuttcap%
\pgfsetmiterjoin%
\definecolor{currentfill}{rgb}{1.000000,0.498039,0.054902}%
\pgfsetfillcolor{currentfill}%
\pgfsetlinewidth{0.000000pt}%
\definecolor{currentstroke}{rgb}{0.000000,0.000000,0.000000}%
\pgfsetstrokecolor{currentstroke}%
\pgfsetstrokeopacity{0.000000}%
\pgfsetdash{}{0pt}%
\pgfpathmoveto{\pgfqpoint{0.819652in}{0.467838in}}%
\pgfpathlineto{\pgfqpoint{0.861694in}{0.467838in}}%
\pgfpathlineto{\pgfqpoint{0.861694in}{0.469621in}}%
\pgfpathlineto{\pgfqpoint{0.819652in}{0.469621in}}%
\pgfpathlineto{\pgfqpoint{0.819652in}{0.467838in}}%
\pgfpathclose%
\pgfusepath{fill}%
\end{pgfscope}%
\begin{pgfscope}%
\pgfpathrectangle{\pgfqpoint{0.643077in}{0.467838in}}{\pgfqpoint{2.959734in}{1.817415in}}%
\pgfusepath{clip}%
\pgfsetbuttcap%
\pgfsetmiterjoin%
\definecolor{currentfill}{rgb}{1.000000,0.498039,0.054902}%
\pgfsetfillcolor{currentfill}%
\pgfsetlinewidth{0.000000pt}%
\definecolor{currentstroke}{rgb}{0.000000,0.000000,0.000000}%
\pgfsetstrokecolor{currentstroke}%
\pgfsetstrokeopacity{0.000000}%
\pgfsetdash{}{0pt}%
\pgfpathmoveto{\pgfqpoint{1.029861in}{0.467838in}}%
\pgfpathlineto{\pgfqpoint{1.071902in}{0.467838in}}%
\pgfpathlineto{\pgfqpoint{1.071902in}{0.467838in}}%
\pgfpathlineto{\pgfqpoint{1.029861in}{0.467838in}}%
\pgfpathlineto{\pgfqpoint{1.029861in}{0.467838in}}%
\pgfpathclose%
\pgfusepath{fill}%
\end{pgfscope}%
\begin{pgfscope}%
\pgfpathrectangle{\pgfqpoint{0.643077in}{0.467838in}}{\pgfqpoint{2.959734in}{1.817415in}}%
\pgfusepath{clip}%
\pgfsetbuttcap%
\pgfsetmiterjoin%
\definecolor{currentfill}{rgb}{1.000000,0.498039,0.054902}%
\pgfsetfillcolor{currentfill}%
\pgfsetlinewidth{0.000000pt}%
\definecolor{currentstroke}{rgb}{0.000000,0.000000,0.000000}%
\pgfsetstrokecolor{currentstroke}%
\pgfsetstrokeopacity{0.000000}%
\pgfsetdash{}{0pt}%
\pgfpathmoveto{\pgfqpoint{1.240069in}{0.467838in}}%
\pgfpathlineto{\pgfqpoint{1.282111in}{0.467838in}}%
\pgfpathlineto{\pgfqpoint{1.282111in}{0.487447in}}%
\pgfpathlineto{\pgfqpoint{1.240069in}{0.487447in}}%
\pgfpathlineto{\pgfqpoint{1.240069in}{0.467838in}}%
\pgfpathclose%
\pgfusepath{fill}%
\end{pgfscope}%
\begin{pgfscope}%
\pgfpathrectangle{\pgfqpoint{0.643077in}{0.467838in}}{\pgfqpoint{2.959734in}{1.817415in}}%
\pgfusepath{clip}%
\pgfsetbuttcap%
\pgfsetmiterjoin%
\definecolor{currentfill}{rgb}{1.000000,0.498039,0.054902}%
\pgfsetfillcolor{currentfill}%
\pgfsetlinewidth{0.000000pt}%
\definecolor{currentstroke}{rgb}{0.000000,0.000000,0.000000}%
\pgfsetstrokecolor{currentstroke}%
\pgfsetstrokeopacity{0.000000}%
\pgfsetdash{}{0pt}%
\pgfpathmoveto{\pgfqpoint{1.450277in}{0.467838in}}%
\pgfpathlineto{\pgfqpoint{1.492319in}{0.467838in}}%
\pgfpathlineto{\pgfqpoint{1.492319in}{0.467838in}}%
\pgfpathlineto{\pgfqpoint{1.450277in}{0.467838in}}%
\pgfpathlineto{\pgfqpoint{1.450277in}{0.467838in}}%
\pgfpathclose%
\pgfusepath{fill}%
\end{pgfscope}%
\begin{pgfscope}%
\pgfpathrectangle{\pgfqpoint{0.643077in}{0.467838in}}{\pgfqpoint{2.959734in}{1.817415in}}%
\pgfusepath{clip}%
\pgfsetbuttcap%
\pgfsetmiterjoin%
\definecolor{currentfill}{rgb}{1.000000,0.498039,0.054902}%
\pgfsetfillcolor{currentfill}%
\pgfsetlinewidth{0.000000pt}%
\definecolor{currentstroke}{rgb}{0.000000,0.000000,0.000000}%
\pgfsetstrokecolor{currentstroke}%
\pgfsetstrokeopacity{0.000000}%
\pgfsetdash{}{0pt}%
\pgfpathmoveto{\pgfqpoint{1.660486in}{0.467838in}}%
\pgfpathlineto{\pgfqpoint{1.702528in}{0.467838in}}%
\pgfpathlineto{\pgfqpoint{1.702528in}{0.487447in}}%
\pgfpathlineto{\pgfqpoint{1.660486in}{0.487447in}}%
\pgfpathlineto{\pgfqpoint{1.660486in}{0.467838in}}%
\pgfpathclose%
\pgfusepath{fill}%
\end{pgfscope}%
\begin{pgfscope}%
\pgfpathrectangle{\pgfqpoint{0.643077in}{0.467838in}}{\pgfqpoint{2.959734in}{1.817415in}}%
\pgfusepath{clip}%
\pgfsetbuttcap%
\pgfsetmiterjoin%
\definecolor{currentfill}{rgb}{1.000000,0.498039,0.054902}%
\pgfsetfillcolor{currentfill}%
\pgfsetlinewidth{0.000000pt}%
\definecolor{currentstroke}{rgb}{0.000000,0.000000,0.000000}%
\pgfsetstrokecolor{currentstroke}%
\pgfsetstrokeopacity{0.000000}%
\pgfsetdash{}{0pt}%
\pgfpathmoveto{\pgfqpoint{1.870694in}{0.467838in}}%
\pgfpathlineto{\pgfqpoint{1.912736in}{0.467838in}}%
\pgfpathlineto{\pgfqpoint{1.912736in}{0.467838in}}%
\pgfpathlineto{\pgfqpoint{1.870694in}{0.467838in}}%
\pgfpathlineto{\pgfqpoint{1.870694in}{0.467838in}}%
\pgfpathclose%
\pgfusepath{fill}%
\end{pgfscope}%
\begin{pgfscope}%
\pgfpathrectangle{\pgfqpoint{0.643077in}{0.467838in}}{\pgfqpoint{2.959734in}{1.817415in}}%
\pgfusepath{clip}%
\pgfsetbuttcap%
\pgfsetmiterjoin%
\definecolor{currentfill}{rgb}{1.000000,0.498039,0.054902}%
\pgfsetfillcolor{currentfill}%
\pgfsetlinewidth{0.000000pt}%
\definecolor{currentstroke}{rgb}{0.000000,0.000000,0.000000}%
\pgfsetstrokecolor{currentstroke}%
\pgfsetstrokeopacity{0.000000}%
\pgfsetdash{}{0pt}%
\pgfpathmoveto{\pgfqpoint{2.080903in}{0.467838in}}%
\pgfpathlineto{\pgfqpoint{2.122944in}{0.467838in}}%
\pgfpathlineto{\pgfqpoint{2.122944in}{0.491012in}}%
\pgfpathlineto{\pgfqpoint{2.080903in}{0.491012in}}%
\pgfpathlineto{\pgfqpoint{2.080903in}{0.467838in}}%
\pgfpathclose%
\pgfusepath{fill}%
\end{pgfscope}%
\begin{pgfscope}%
\pgfpathrectangle{\pgfqpoint{0.643077in}{0.467838in}}{\pgfqpoint{2.959734in}{1.817415in}}%
\pgfusepath{clip}%
\pgfsetbuttcap%
\pgfsetmiterjoin%
\definecolor{currentfill}{rgb}{1.000000,0.498039,0.054902}%
\pgfsetfillcolor{currentfill}%
\pgfsetlinewidth{0.000000pt}%
\definecolor{currentstroke}{rgb}{0.000000,0.000000,0.000000}%
\pgfsetstrokecolor{currentstroke}%
\pgfsetstrokeopacity{0.000000}%
\pgfsetdash{}{0pt}%
\pgfpathmoveto{\pgfqpoint{2.291111in}{0.467838in}}%
\pgfpathlineto{\pgfqpoint{2.333153in}{0.467838in}}%
\pgfpathlineto{\pgfqpoint{2.333153in}{0.467838in}}%
\pgfpathlineto{\pgfqpoint{2.291111in}{0.467838in}}%
\pgfpathlineto{\pgfqpoint{2.291111in}{0.467838in}}%
\pgfpathclose%
\pgfusepath{fill}%
\end{pgfscope}%
\begin{pgfscope}%
\pgfpathrectangle{\pgfqpoint{0.643077in}{0.467838in}}{\pgfqpoint{2.959734in}{1.817415in}}%
\pgfusepath{clip}%
\pgfsetbuttcap%
\pgfsetmiterjoin%
\definecolor{currentfill}{rgb}{1.000000,0.498039,0.054902}%
\pgfsetfillcolor{currentfill}%
\pgfsetlinewidth{0.000000pt}%
\definecolor{currentstroke}{rgb}{0.000000,0.000000,0.000000}%
\pgfsetstrokecolor{currentstroke}%
\pgfsetstrokeopacity{0.000000}%
\pgfsetdash{}{0pt}%
\pgfpathmoveto{\pgfqpoint{2.501320in}{0.467838in}}%
\pgfpathlineto{\pgfqpoint{2.543361in}{0.467838in}}%
\pgfpathlineto{\pgfqpoint{2.543361in}{0.485664in}}%
\pgfpathlineto{\pgfqpoint{2.501320in}{0.485664in}}%
\pgfpathlineto{\pgfqpoint{2.501320in}{0.467838in}}%
\pgfpathclose%
\pgfusepath{fill}%
\end{pgfscope}%
\begin{pgfscope}%
\pgfpathrectangle{\pgfqpoint{0.643077in}{0.467838in}}{\pgfqpoint{2.959734in}{1.817415in}}%
\pgfusepath{clip}%
\pgfsetbuttcap%
\pgfsetmiterjoin%
\definecolor{currentfill}{rgb}{1.000000,0.498039,0.054902}%
\pgfsetfillcolor{currentfill}%
\pgfsetlinewidth{0.000000pt}%
\definecolor{currentstroke}{rgb}{0.000000,0.000000,0.000000}%
\pgfsetstrokecolor{currentstroke}%
\pgfsetstrokeopacity{0.000000}%
\pgfsetdash{}{0pt}%
\pgfpathmoveto{\pgfqpoint{2.711528in}{0.467838in}}%
\pgfpathlineto{\pgfqpoint{2.753570in}{0.467838in}}%
\pgfpathlineto{\pgfqpoint{2.753570in}{0.467838in}}%
\pgfpathlineto{\pgfqpoint{2.711528in}{0.467838in}}%
\pgfpathlineto{\pgfqpoint{2.711528in}{0.467838in}}%
\pgfpathclose%
\pgfusepath{fill}%
\end{pgfscope}%
\begin{pgfscope}%
\pgfpathrectangle{\pgfqpoint{0.643077in}{0.467838in}}{\pgfqpoint{2.959734in}{1.817415in}}%
\pgfusepath{clip}%
\pgfsetbuttcap%
\pgfsetmiterjoin%
\definecolor{currentfill}{rgb}{1.000000,0.498039,0.054902}%
\pgfsetfillcolor{currentfill}%
\pgfsetlinewidth{0.000000pt}%
\definecolor{currentstroke}{rgb}{0.000000,0.000000,0.000000}%
\pgfsetstrokecolor{currentstroke}%
\pgfsetstrokeopacity{0.000000}%
\pgfsetdash{}{0pt}%
\pgfpathmoveto{\pgfqpoint{2.921736in}{0.467838in}}%
\pgfpathlineto{\pgfqpoint{2.963778in}{0.467838in}}%
\pgfpathlineto{\pgfqpoint{2.963778in}{0.492794in}}%
\pgfpathlineto{\pgfqpoint{2.921736in}{0.492794in}}%
\pgfpathlineto{\pgfqpoint{2.921736in}{0.467838in}}%
\pgfpathclose%
\pgfusepath{fill}%
\end{pgfscope}%
\begin{pgfscope}%
\pgfpathrectangle{\pgfqpoint{0.643077in}{0.467838in}}{\pgfqpoint{2.959734in}{1.817415in}}%
\pgfusepath{clip}%
\pgfsetbuttcap%
\pgfsetmiterjoin%
\definecolor{currentfill}{rgb}{1.000000,0.498039,0.054902}%
\pgfsetfillcolor{currentfill}%
\pgfsetlinewidth{0.000000pt}%
\definecolor{currentstroke}{rgb}{0.000000,0.000000,0.000000}%
\pgfsetstrokecolor{currentstroke}%
\pgfsetstrokeopacity{0.000000}%
\pgfsetdash{}{0pt}%
\pgfpathmoveto{\pgfqpoint{3.131945in}{0.467838in}}%
\pgfpathlineto{\pgfqpoint{3.173986in}{0.467838in}}%
\pgfpathlineto{\pgfqpoint{3.173986in}{0.467838in}}%
\pgfpathlineto{\pgfqpoint{3.131945in}{0.467838in}}%
\pgfpathlineto{\pgfqpoint{3.131945in}{0.467838in}}%
\pgfpathclose%
\pgfusepath{fill}%
\end{pgfscope}%
\begin{pgfscope}%
\pgfpathrectangle{\pgfqpoint{0.643077in}{0.467838in}}{\pgfqpoint{2.959734in}{1.817415in}}%
\pgfusepath{clip}%
\pgfsetbuttcap%
\pgfsetmiterjoin%
\definecolor{currentfill}{rgb}{1.000000,0.498039,0.054902}%
\pgfsetfillcolor{currentfill}%
\pgfsetlinewidth{0.000000pt}%
\definecolor{currentstroke}{rgb}{0.000000,0.000000,0.000000}%
\pgfsetstrokecolor{currentstroke}%
\pgfsetstrokeopacity{0.000000}%
\pgfsetdash{}{0pt}%
\pgfpathmoveto{\pgfqpoint{3.342153in}{0.467838in}}%
\pgfpathlineto{\pgfqpoint{3.384195in}{0.467838in}}%
\pgfpathlineto{\pgfqpoint{3.384195in}{2.143450in}}%
\pgfpathlineto{\pgfqpoint{3.342153in}{2.143450in}}%
\pgfpathlineto{\pgfqpoint{3.342153in}{0.467838in}}%
\pgfpathclose%
\pgfusepath{fill}%
\end{pgfscope}%
\begin{pgfscope}%
\pgfpathrectangle{\pgfqpoint{0.643077in}{0.467838in}}{\pgfqpoint{2.959734in}{1.817415in}}%
\pgfusepath{clip}%
\pgfsetbuttcap%
\pgfsetmiterjoin%
\definecolor{currentfill}{rgb}{0.172549,0.627451,0.172549}%
\pgfsetfillcolor{currentfill}%
\pgfsetlinewidth{0.000000pt}%
\definecolor{currentstroke}{rgb}{0.000000,0.000000,0.000000}%
\pgfsetstrokecolor{currentstroke}%
\pgfsetstrokeopacity{0.000000}%
\pgfsetdash{}{0pt}%
\pgfpathmoveto{\pgfqpoint{0.861694in}{0.467838in}}%
\pgfpathlineto{\pgfqpoint{0.903736in}{0.467838in}}%
\pgfpathlineto{\pgfqpoint{0.903736in}{0.478534in}}%
\pgfpathlineto{\pgfqpoint{0.861694in}{0.478534in}}%
\pgfpathlineto{\pgfqpoint{0.861694in}{0.467838in}}%
\pgfpathclose%
\pgfusepath{fill}%
\end{pgfscope}%
\begin{pgfscope}%
\pgfpathrectangle{\pgfqpoint{0.643077in}{0.467838in}}{\pgfqpoint{2.959734in}{1.817415in}}%
\pgfusepath{clip}%
\pgfsetbuttcap%
\pgfsetmiterjoin%
\definecolor{currentfill}{rgb}{0.172549,0.627451,0.172549}%
\pgfsetfillcolor{currentfill}%
\pgfsetlinewidth{0.000000pt}%
\definecolor{currentstroke}{rgb}{0.000000,0.000000,0.000000}%
\pgfsetstrokecolor{currentstroke}%
\pgfsetstrokeopacity{0.000000}%
\pgfsetdash{}{0pt}%
\pgfpathmoveto{\pgfqpoint{1.071902in}{0.467838in}}%
\pgfpathlineto{\pgfqpoint{1.113944in}{0.467838in}}%
\pgfpathlineto{\pgfqpoint{1.113944in}{0.467838in}}%
\pgfpathlineto{\pgfqpoint{1.071902in}{0.467838in}}%
\pgfpathlineto{\pgfqpoint{1.071902in}{0.467838in}}%
\pgfpathclose%
\pgfusepath{fill}%
\end{pgfscope}%
\begin{pgfscope}%
\pgfpathrectangle{\pgfqpoint{0.643077in}{0.467838in}}{\pgfqpoint{2.959734in}{1.817415in}}%
\pgfusepath{clip}%
\pgfsetbuttcap%
\pgfsetmiterjoin%
\definecolor{currentfill}{rgb}{0.172549,0.627451,0.172549}%
\pgfsetfillcolor{currentfill}%
\pgfsetlinewidth{0.000000pt}%
\definecolor{currentstroke}{rgb}{0.000000,0.000000,0.000000}%
\pgfsetstrokecolor{currentstroke}%
\pgfsetstrokeopacity{0.000000}%
\pgfsetdash{}{0pt}%
\pgfpathmoveto{\pgfqpoint{1.282111in}{0.467838in}}%
\pgfpathlineto{\pgfqpoint{1.324152in}{0.467838in}}%
\pgfpathlineto{\pgfqpoint{1.324152in}{0.467838in}}%
\pgfpathlineto{\pgfqpoint{1.282111in}{0.467838in}}%
\pgfpathlineto{\pgfqpoint{1.282111in}{0.467838in}}%
\pgfpathclose%
\pgfusepath{fill}%
\end{pgfscope}%
\begin{pgfscope}%
\pgfpathrectangle{\pgfqpoint{0.643077in}{0.467838in}}{\pgfqpoint{2.959734in}{1.817415in}}%
\pgfusepath{clip}%
\pgfsetbuttcap%
\pgfsetmiterjoin%
\definecolor{currentfill}{rgb}{0.172549,0.627451,0.172549}%
\pgfsetfillcolor{currentfill}%
\pgfsetlinewidth{0.000000pt}%
\definecolor{currentstroke}{rgb}{0.000000,0.000000,0.000000}%
\pgfsetstrokecolor{currentstroke}%
\pgfsetstrokeopacity{0.000000}%
\pgfsetdash{}{0pt}%
\pgfpathmoveto{\pgfqpoint{1.492319in}{0.467838in}}%
\pgfpathlineto{\pgfqpoint{1.534361in}{0.467838in}}%
\pgfpathlineto{\pgfqpoint{1.534361in}{0.476751in}}%
\pgfpathlineto{\pgfqpoint{1.492319in}{0.476751in}}%
\pgfpathlineto{\pgfqpoint{1.492319in}{0.467838in}}%
\pgfpathclose%
\pgfusepath{fill}%
\end{pgfscope}%
\begin{pgfscope}%
\pgfpathrectangle{\pgfqpoint{0.643077in}{0.467838in}}{\pgfqpoint{2.959734in}{1.817415in}}%
\pgfusepath{clip}%
\pgfsetbuttcap%
\pgfsetmiterjoin%
\definecolor{currentfill}{rgb}{0.172549,0.627451,0.172549}%
\pgfsetfillcolor{currentfill}%
\pgfsetlinewidth{0.000000pt}%
\definecolor{currentstroke}{rgb}{0.000000,0.000000,0.000000}%
\pgfsetstrokecolor{currentstroke}%
\pgfsetstrokeopacity{0.000000}%
\pgfsetdash{}{0pt}%
\pgfpathmoveto{\pgfqpoint{1.702528in}{0.467838in}}%
\pgfpathlineto{\pgfqpoint{1.744569in}{0.467838in}}%
\pgfpathlineto{\pgfqpoint{1.744569in}{0.467838in}}%
\pgfpathlineto{\pgfqpoint{1.702528in}{0.467838in}}%
\pgfpathlineto{\pgfqpoint{1.702528in}{0.467838in}}%
\pgfpathclose%
\pgfusepath{fill}%
\end{pgfscope}%
\begin{pgfscope}%
\pgfpathrectangle{\pgfqpoint{0.643077in}{0.467838in}}{\pgfqpoint{2.959734in}{1.817415in}}%
\pgfusepath{clip}%
\pgfsetbuttcap%
\pgfsetmiterjoin%
\definecolor{currentfill}{rgb}{0.172549,0.627451,0.172549}%
\pgfsetfillcolor{currentfill}%
\pgfsetlinewidth{0.000000pt}%
\definecolor{currentstroke}{rgb}{0.000000,0.000000,0.000000}%
\pgfsetstrokecolor{currentstroke}%
\pgfsetstrokeopacity{0.000000}%
\pgfsetdash{}{0pt}%
\pgfpathmoveto{\pgfqpoint{1.912736in}{0.467838in}}%
\pgfpathlineto{\pgfqpoint{1.954778in}{0.467838in}}%
\pgfpathlineto{\pgfqpoint{1.954778in}{0.487447in}}%
\pgfpathlineto{\pgfqpoint{1.912736in}{0.487447in}}%
\pgfpathlineto{\pgfqpoint{1.912736in}{0.467838in}}%
\pgfpathclose%
\pgfusepath{fill}%
\end{pgfscope}%
\begin{pgfscope}%
\pgfpathrectangle{\pgfqpoint{0.643077in}{0.467838in}}{\pgfqpoint{2.959734in}{1.817415in}}%
\pgfusepath{clip}%
\pgfsetbuttcap%
\pgfsetmiterjoin%
\definecolor{currentfill}{rgb}{0.172549,0.627451,0.172549}%
\pgfsetfillcolor{currentfill}%
\pgfsetlinewidth{0.000000pt}%
\definecolor{currentstroke}{rgb}{0.000000,0.000000,0.000000}%
\pgfsetstrokecolor{currentstroke}%
\pgfsetstrokeopacity{0.000000}%
\pgfsetdash{}{0pt}%
\pgfpathmoveto{\pgfqpoint{2.122944in}{0.467838in}}%
\pgfpathlineto{\pgfqpoint{2.164986in}{0.467838in}}%
\pgfpathlineto{\pgfqpoint{2.164986in}{0.467838in}}%
\pgfpathlineto{\pgfqpoint{2.122944in}{0.467838in}}%
\pgfpathlineto{\pgfqpoint{2.122944in}{0.467838in}}%
\pgfpathclose%
\pgfusepath{fill}%
\end{pgfscope}%
\begin{pgfscope}%
\pgfpathrectangle{\pgfqpoint{0.643077in}{0.467838in}}{\pgfqpoint{2.959734in}{1.817415in}}%
\pgfusepath{clip}%
\pgfsetbuttcap%
\pgfsetmiterjoin%
\definecolor{currentfill}{rgb}{0.172549,0.627451,0.172549}%
\pgfsetfillcolor{currentfill}%
\pgfsetlinewidth{0.000000pt}%
\definecolor{currentstroke}{rgb}{0.000000,0.000000,0.000000}%
\pgfsetstrokecolor{currentstroke}%
\pgfsetstrokeopacity{0.000000}%
\pgfsetdash{}{0pt}%
\pgfpathmoveto{\pgfqpoint{2.333153in}{0.467838in}}%
\pgfpathlineto{\pgfqpoint{2.375194in}{0.467838in}}%
\pgfpathlineto{\pgfqpoint{2.375194in}{0.467838in}}%
\pgfpathlineto{\pgfqpoint{2.333153in}{0.467838in}}%
\pgfpathlineto{\pgfqpoint{2.333153in}{0.467838in}}%
\pgfpathclose%
\pgfusepath{fill}%
\end{pgfscope}%
\begin{pgfscope}%
\pgfpathrectangle{\pgfqpoint{0.643077in}{0.467838in}}{\pgfqpoint{2.959734in}{1.817415in}}%
\pgfusepath{clip}%
\pgfsetbuttcap%
\pgfsetmiterjoin%
\definecolor{currentfill}{rgb}{0.172549,0.627451,0.172549}%
\pgfsetfillcolor{currentfill}%
\pgfsetlinewidth{0.000000pt}%
\definecolor{currentstroke}{rgb}{0.000000,0.000000,0.000000}%
\pgfsetstrokecolor{currentstroke}%
\pgfsetstrokeopacity{0.000000}%
\pgfsetdash{}{0pt}%
\pgfpathmoveto{\pgfqpoint{2.543361in}{0.467838in}}%
\pgfpathlineto{\pgfqpoint{2.585403in}{0.467838in}}%
\pgfpathlineto{\pgfqpoint{2.585403in}{0.482099in}}%
\pgfpathlineto{\pgfqpoint{2.543361in}{0.482099in}}%
\pgfpathlineto{\pgfqpoint{2.543361in}{0.467838in}}%
\pgfpathclose%
\pgfusepath{fill}%
\end{pgfscope}%
\begin{pgfscope}%
\pgfpathrectangle{\pgfqpoint{0.643077in}{0.467838in}}{\pgfqpoint{2.959734in}{1.817415in}}%
\pgfusepath{clip}%
\pgfsetbuttcap%
\pgfsetmiterjoin%
\definecolor{currentfill}{rgb}{0.172549,0.627451,0.172549}%
\pgfsetfillcolor{currentfill}%
\pgfsetlinewidth{0.000000pt}%
\definecolor{currentstroke}{rgb}{0.000000,0.000000,0.000000}%
\pgfsetstrokecolor{currentstroke}%
\pgfsetstrokeopacity{0.000000}%
\pgfsetdash{}{0pt}%
\pgfpathmoveto{\pgfqpoint{2.753570in}{0.467838in}}%
\pgfpathlineto{\pgfqpoint{2.795611in}{0.467838in}}%
\pgfpathlineto{\pgfqpoint{2.795611in}{0.467838in}}%
\pgfpathlineto{\pgfqpoint{2.753570in}{0.467838in}}%
\pgfpathlineto{\pgfqpoint{2.753570in}{0.467838in}}%
\pgfpathclose%
\pgfusepath{fill}%
\end{pgfscope}%
\begin{pgfscope}%
\pgfpathrectangle{\pgfqpoint{0.643077in}{0.467838in}}{\pgfqpoint{2.959734in}{1.817415in}}%
\pgfusepath{clip}%
\pgfsetbuttcap%
\pgfsetmiterjoin%
\definecolor{currentfill}{rgb}{0.172549,0.627451,0.172549}%
\pgfsetfillcolor{currentfill}%
\pgfsetlinewidth{0.000000pt}%
\definecolor{currentstroke}{rgb}{0.000000,0.000000,0.000000}%
\pgfsetstrokecolor{currentstroke}%
\pgfsetstrokeopacity{0.000000}%
\pgfsetdash{}{0pt}%
\pgfpathmoveto{\pgfqpoint{2.963778in}{0.467838in}}%
\pgfpathlineto{\pgfqpoint{3.005820in}{0.467838in}}%
\pgfpathlineto{\pgfqpoint{3.005820in}{0.483881in}}%
\pgfpathlineto{\pgfqpoint{2.963778in}{0.483881in}}%
\pgfpathlineto{\pgfqpoint{2.963778in}{0.467838in}}%
\pgfpathclose%
\pgfusepath{fill}%
\end{pgfscope}%
\begin{pgfscope}%
\pgfpathrectangle{\pgfqpoint{0.643077in}{0.467838in}}{\pgfqpoint{2.959734in}{1.817415in}}%
\pgfusepath{clip}%
\pgfsetbuttcap%
\pgfsetmiterjoin%
\definecolor{currentfill}{rgb}{0.172549,0.627451,0.172549}%
\pgfsetfillcolor{currentfill}%
\pgfsetlinewidth{0.000000pt}%
\definecolor{currentstroke}{rgb}{0.000000,0.000000,0.000000}%
\pgfsetstrokecolor{currentstroke}%
\pgfsetstrokeopacity{0.000000}%
\pgfsetdash{}{0pt}%
\pgfpathmoveto{\pgfqpoint{3.173986in}{0.467838in}}%
\pgfpathlineto{\pgfqpoint{3.216028in}{0.467838in}}%
\pgfpathlineto{\pgfqpoint{3.216028in}{0.467838in}}%
\pgfpathlineto{\pgfqpoint{3.173986in}{0.467838in}}%
\pgfpathlineto{\pgfqpoint{3.173986in}{0.467838in}}%
\pgfpathclose%
\pgfusepath{fill}%
\end{pgfscope}%
\begin{pgfscope}%
\pgfpathrectangle{\pgfqpoint{0.643077in}{0.467838in}}{\pgfqpoint{2.959734in}{1.817415in}}%
\pgfusepath{clip}%
\pgfsetbuttcap%
\pgfsetmiterjoin%
\definecolor{currentfill}{rgb}{0.172549,0.627451,0.172549}%
\pgfsetfillcolor{currentfill}%
\pgfsetlinewidth{0.000000pt}%
\definecolor{currentstroke}{rgb}{0.000000,0.000000,0.000000}%
\pgfsetstrokecolor{currentstroke}%
\pgfsetstrokeopacity{0.000000}%
\pgfsetdash{}{0pt}%
\pgfpathmoveto{\pgfqpoint{3.384195in}{0.467838in}}%
\pgfpathlineto{\pgfqpoint{3.426236in}{0.467838in}}%
\pgfpathlineto{\pgfqpoint{3.426236in}{2.180884in}}%
\pgfpathlineto{\pgfqpoint{3.384195in}{2.180884in}}%
\pgfpathlineto{\pgfqpoint{3.384195in}{0.467838in}}%
\pgfpathclose%
\pgfusepath{fill}%
\end{pgfscope}%
\begin{pgfscope}%
\pgfpathrectangle{\pgfqpoint{0.643077in}{0.467838in}}{\pgfqpoint{2.959734in}{1.817415in}}%
\pgfusepath{clip}%
\pgfsetbuttcap%
\pgfsetmiterjoin%
\definecolor{currentfill}{rgb}{0.839216,0.152941,0.156863}%
\pgfsetfillcolor{currentfill}%
\pgfsetlinewidth{0.000000pt}%
\definecolor{currentstroke}{rgb}{0.000000,0.000000,0.000000}%
\pgfsetstrokecolor{currentstroke}%
\pgfsetstrokeopacity{0.000000}%
\pgfsetdash{}{0pt}%
\pgfpathmoveto{\pgfqpoint{0.903736in}{0.467838in}}%
\pgfpathlineto{\pgfqpoint{0.945777in}{0.467838in}}%
\pgfpathlineto{\pgfqpoint{0.945777in}{0.471403in}}%
\pgfpathlineto{\pgfqpoint{0.903736in}{0.471403in}}%
\pgfpathlineto{\pgfqpoint{0.903736in}{0.467838in}}%
\pgfpathclose%
\pgfusepath{fill}%
\end{pgfscope}%
\begin{pgfscope}%
\pgfpathrectangle{\pgfqpoint{0.643077in}{0.467838in}}{\pgfqpoint{2.959734in}{1.817415in}}%
\pgfusepath{clip}%
\pgfsetbuttcap%
\pgfsetmiterjoin%
\definecolor{currentfill}{rgb}{0.839216,0.152941,0.156863}%
\pgfsetfillcolor{currentfill}%
\pgfsetlinewidth{0.000000pt}%
\definecolor{currentstroke}{rgb}{0.000000,0.000000,0.000000}%
\pgfsetstrokecolor{currentstroke}%
\pgfsetstrokeopacity{0.000000}%
\pgfsetdash{}{0pt}%
\pgfpathmoveto{\pgfqpoint{1.113944in}{0.467838in}}%
\pgfpathlineto{\pgfqpoint{1.155986in}{0.467838in}}%
\pgfpathlineto{\pgfqpoint{1.155986in}{0.467838in}}%
\pgfpathlineto{\pgfqpoint{1.113944in}{0.467838in}}%
\pgfpathlineto{\pgfqpoint{1.113944in}{0.467838in}}%
\pgfpathclose%
\pgfusepath{fill}%
\end{pgfscope}%
\begin{pgfscope}%
\pgfpathrectangle{\pgfqpoint{0.643077in}{0.467838in}}{\pgfqpoint{2.959734in}{1.817415in}}%
\pgfusepath{clip}%
\pgfsetbuttcap%
\pgfsetmiterjoin%
\definecolor{currentfill}{rgb}{0.839216,0.152941,0.156863}%
\pgfsetfillcolor{currentfill}%
\pgfsetlinewidth{0.000000pt}%
\definecolor{currentstroke}{rgb}{0.000000,0.000000,0.000000}%
\pgfsetstrokecolor{currentstroke}%
\pgfsetstrokeopacity{0.000000}%
\pgfsetdash{}{0pt}%
\pgfpathmoveto{\pgfqpoint{1.324152in}{0.467838in}}%
\pgfpathlineto{\pgfqpoint{1.366194in}{0.467838in}}%
\pgfpathlineto{\pgfqpoint{1.366194in}{0.467838in}}%
\pgfpathlineto{\pgfqpoint{1.324152in}{0.467838in}}%
\pgfpathlineto{\pgfqpoint{1.324152in}{0.467838in}}%
\pgfpathclose%
\pgfusepath{fill}%
\end{pgfscope}%
\begin{pgfscope}%
\pgfpathrectangle{\pgfqpoint{0.643077in}{0.467838in}}{\pgfqpoint{2.959734in}{1.817415in}}%
\pgfusepath{clip}%
\pgfsetbuttcap%
\pgfsetmiterjoin%
\definecolor{currentfill}{rgb}{0.839216,0.152941,0.156863}%
\pgfsetfillcolor{currentfill}%
\pgfsetlinewidth{0.000000pt}%
\definecolor{currentstroke}{rgb}{0.000000,0.000000,0.000000}%
\pgfsetstrokecolor{currentstroke}%
\pgfsetstrokeopacity{0.000000}%
\pgfsetdash{}{0pt}%
\pgfpathmoveto{\pgfqpoint{1.534361in}{0.467838in}}%
\pgfpathlineto{\pgfqpoint{1.576403in}{0.467838in}}%
\pgfpathlineto{\pgfqpoint{1.576403in}{0.480316in}}%
\pgfpathlineto{\pgfqpoint{1.534361in}{0.480316in}}%
\pgfpathlineto{\pgfqpoint{1.534361in}{0.467838in}}%
\pgfpathclose%
\pgfusepath{fill}%
\end{pgfscope}%
\begin{pgfscope}%
\pgfpathrectangle{\pgfqpoint{0.643077in}{0.467838in}}{\pgfqpoint{2.959734in}{1.817415in}}%
\pgfusepath{clip}%
\pgfsetbuttcap%
\pgfsetmiterjoin%
\definecolor{currentfill}{rgb}{0.839216,0.152941,0.156863}%
\pgfsetfillcolor{currentfill}%
\pgfsetlinewidth{0.000000pt}%
\definecolor{currentstroke}{rgb}{0.000000,0.000000,0.000000}%
\pgfsetstrokecolor{currentstroke}%
\pgfsetstrokeopacity{0.000000}%
\pgfsetdash{}{0pt}%
\pgfpathmoveto{\pgfqpoint{1.744569in}{0.467838in}}%
\pgfpathlineto{\pgfqpoint{1.786611in}{0.467838in}}%
\pgfpathlineto{\pgfqpoint{1.786611in}{0.467838in}}%
\pgfpathlineto{\pgfqpoint{1.744569in}{0.467838in}}%
\pgfpathlineto{\pgfqpoint{1.744569in}{0.467838in}}%
\pgfpathclose%
\pgfusepath{fill}%
\end{pgfscope}%
\begin{pgfscope}%
\pgfpathrectangle{\pgfqpoint{0.643077in}{0.467838in}}{\pgfqpoint{2.959734in}{1.817415in}}%
\pgfusepath{clip}%
\pgfsetbuttcap%
\pgfsetmiterjoin%
\definecolor{currentfill}{rgb}{0.839216,0.152941,0.156863}%
\pgfsetfillcolor{currentfill}%
\pgfsetlinewidth{0.000000pt}%
\definecolor{currentstroke}{rgb}{0.000000,0.000000,0.000000}%
\pgfsetstrokecolor{currentstroke}%
\pgfsetstrokeopacity{0.000000}%
\pgfsetdash{}{0pt}%
\pgfpathmoveto{\pgfqpoint{1.954778in}{0.467838in}}%
\pgfpathlineto{\pgfqpoint{1.996819in}{0.467838in}}%
\pgfpathlineto{\pgfqpoint{1.996819in}{0.485664in}}%
\pgfpathlineto{\pgfqpoint{1.954778in}{0.485664in}}%
\pgfpathlineto{\pgfqpoint{1.954778in}{0.467838in}}%
\pgfpathclose%
\pgfusepath{fill}%
\end{pgfscope}%
\begin{pgfscope}%
\pgfpathrectangle{\pgfqpoint{0.643077in}{0.467838in}}{\pgfqpoint{2.959734in}{1.817415in}}%
\pgfusepath{clip}%
\pgfsetbuttcap%
\pgfsetmiterjoin%
\definecolor{currentfill}{rgb}{0.839216,0.152941,0.156863}%
\pgfsetfillcolor{currentfill}%
\pgfsetlinewidth{0.000000pt}%
\definecolor{currentstroke}{rgb}{0.000000,0.000000,0.000000}%
\pgfsetstrokecolor{currentstroke}%
\pgfsetstrokeopacity{0.000000}%
\pgfsetdash{}{0pt}%
\pgfpathmoveto{\pgfqpoint{2.164986in}{0.467838in}}%
\pgfpathlineto{\pgfqpoint{2.207028in}{0.467838in}}%
\pgfpathlineto{\pgfqpoint{2.207028in}{0.467838in}}%
\pgfpathlineto{\pgfqpoint{2.164986in}{0.467838in}}%
\pgfpathlineto{\pgfqpoint{2.164986in}{0.467838in}}%
\pgfpathclose%
\pgfusepath{fill}%
\end{pgfscope}%
\begin{pgfscope}%
\pgfpathrectangle{\pgfqpoint{0.643077in}{0.467838in}}{\pgfqpoint{2.959734in}{1.817415in}}%
\pgfusepath{clip}%
\pgfsetbuttcap%
\pgfsetmiterjoin%
\definecolor{currentfill}{rgb}{0.839216,0.152941,0.156863}%
\pgfsetfillcolor{currentfill}%
\pgfsetlinewidth{0.000000pt}%
\definecolor{currentstroke}{rgb}{0.000000,0.000000,0.000000}%
\pgfsetstrokecolor{currentstroke}%
\pgfsetstrokeopacity{0.000000}%
\pgfsetdash{}{0pt}%
\pgfpathmoveto{\pgfqpoint{2.375194in}{0.467838in}}%
\pgfpathlineto{\pgfqpoint{2.417236in}{0.467838in}}%
\pgfpathlineto{\pgfqpoint{2.417236in}{0.467838in}}%
\pgfpathlineto{\pgfqpoint{2.375194in}{0.467838in}}%
\pgfpathlineto{\pgfqpoint{2.375194in}{0.467838in}}%
\pgfpathclose%
\pgfusepath{fill}%
\end{pgfscope}%
\begin{pgfscope}%
\pgfpathrectangle{\pgfqpoint{0.643077in}{0.467838in}}{\pgfqpoint{2.959734in}{1.817415in}}%
\pgfusepath{clip}%
\pgfsetbuttcap%
\pgfsetmiterjoin%
\definecolor{currentfill}{rgb}{0.839216,0.152941,0.156863}%
\pgfsetfillcolor{currentfill}%
\pgfsetlinewidth{0.000000pt}%
\definecolor{currentstroke}{rgb}{0.000000,0.000000,0.000000}%
\pgfsetstrokecolor{currentstroke}%
\pgfsetstrokeopacity{0.000000}%
\pgfsetdash{}{0pt}%
\pgfpathmoveto{\pgfqpoint{2.585403in}{0.467838in}}%
\pgfpathlineto{\pgfqpoint{2.627445in}{0.467838in}}%
\pgfpathlineto{\pgfqpoint{2.627445in}{0.474969in}}%
\pgfpathlineto{\pgfqpoint{2.585403in}{0.474969in}}%
\pgfpathlineto{\pgfqpoint{2.585403in}{0.467838in}}%
\pgfpathclose%
\pgfusepath{fill}%
\end{pgfscope}%
\begin{pgfscope}%
\pgfpathrectangle{\pgfqpoint{0.643077in}{0.467838in}}{\pgfqpoint{2.959734in}{1.817415in}}%
\pgfusepath{clip}%
\pgfsetbuttcap%
\pgfsetmiterjoin%
\definecolor{currentfill}{rgb}{0.839216,0.152941,0.156863}%
\pgfsetfillcolor{currentfill}%
\pgfsetlinewidth{0.000000pt}%
\definecolor{currentstroke}{rgb}{0.000000,0.000000,0.000000}%
\pgfsetstrokecolor{currentstroke}%
\pgfsetstrokeopacity{0.000000}%
\pgfsetdash{}{0pt}%
\pgfpathmoveto{\pgfqpoint{2.795611in}{0.467838in}}%
\pgfpathlineto{\pgfqpoint{2.837653in}{0.467838in}}%
\pgfpathlineto{\pgfqpoint{2.837653in}{0.467838in}}%
\pgfpathlineto{\pgfqpoint{2.795611in}{0.467838in}}%
\pgfpathlineto{\pgfqpoint{2.795611in}{0.467838in}}%
\pgfpathclose%
\pgfusepath{fill}%
\end{pgfscope}%
\begin{pgfscope}%
\pgfpathrectangle{\pgfqpoint{0.643077in}{0.467838in}}{\pgfqpoint{2.959734in}{1.817415in}}%
\pgfusepath{clip}%
\pgfsetbuttcap%
\pgfsetmiterjoin%
\definecolor{currentfill}{rgb}{0.839216,0.152941,0.156863}%
\pgfsetfillcolor{currentfill}%
\pgfsetlinewidth{0.000000pt}%
\definecolor{currentstroke}{rgb}{0.000000,0.000000,0.000000}%
\pgfsetstrokecolor{currentstroke}%
\pgfsetstrokeopacity{0.000000}%
\pgfsetdash{}{0pt}%
\pgfpathmoveto{\pgfqpoint{3.005820in}{0.467838in}}%
\pgfpathlineto{\pgfqpoint{3.047861in}{0.467838in}}%
\pgfpathlineto{\pgfqpoint{3.047861in}{0.478534in}}%
\pgfpathlineto{\pgfqpoint{3.005820in}{0.478534in}}%
\pgfpathlineto{\pgfqpoint{3.005820in}{0.467838in}}%
\pgfpathclose%
\pgfusepath{fill}%
\end{pgfscope}%
\begin{pgfscope}%
\pgfpathrectangle{\pgfqpoint{0.643077in}{0.467838in}}{\pgfqpoint{2.959734in}{1.817415in}}%
\pgfusepath{clip}%
\pgfsetbuttcap%
\pgfsetmiterjoin%
\definecolor{currentfill}{rgb}{0.839216,0.152941,0.156863}%
\pgfsetfillcolor{currentfill}%
\pgfsetlinewidth{0.000000pt}%
\definecolor{currentstroke}{rgb}{0.000000,0.000000,0.000000}%
\pgfsetstrokecolor{currentstroke}%
\pgfsetstrokeopacity{0.000000}%
\pgfsetdash{}{0pt}%
\pgfpathmoveto{\pgfqpoint{3.216028in}{0.467838in}}%
\pgfpathlineto{\pgfqpoint{3.258070in}{0.467838in}}%
\pgfpathlineto{\pgfqpoint{3.258070in}{0.467838in}}%
\pgfpathlineto{\pgfqpoint{3.216028in}{0.467838in}}%
\pgfpathlineto{\pgfqpoint{3.216028in}{0.467838in}}%
\pgfpathclose%
\pgfusepath{fill}%
\end{pgfscope}%
\begin{pgfscope}%
\pgfpathrectangle{\pgfqpoint{0.643077in}{0.467838in}}{\pgfqpoint{2.959734in}{1.817415in}}%
\pgfusepath{clip}%
\pgfsetbuttcap%
\pgfsetmiterjoin%
\definecolor{currentfill}{rgb}{0.839216,0.152941,0.156863}%
\pgfsetfillcolor{currentfill}%
\pgfsetlinewidth{0.000000pt}%
\definecolor{currentstroke}{rgb}{0.000000,0.000000,0.000000}%
\pgfsetstrokecolor{currentstroke}%
\pgfsetstrokeopacity{0.000000}%
\pgfsetdash{}{0pt}%
\pgfpathmoveto{\pgfqpoint{3.426236in}{0.467838in}}%
\pgfpathlineto{\pgfqpoint{3.468278in}{0.467838in}}%
\pgfpathlineto{\pgfqpoint{3.468278in}{2.198710in}}%
\pgfpathlineto{\pgfqpoint{3.426236in}{2.198710in}}%
\pgfpathlineto{\pgfqpoint{3.426236in}{0.467838in}}%
\pgfpathclose%
\pgfusepath{fill}%
\end{pgfscope}%
\begin{pgfscope}%
\pgfsetbuttcap%
\pgfsetroundjoin%
\definecolor{currentfill}{rgb}{0.000000,0.000000,0.000000}%
\pgfsetfillcolor{currentfill}%
\pgfsetlinewidth{0.803000pt}%
\definecolor{currentstroke}{rgb}{0.000000,0.000000,0.000000}%
\pgfsetstrokecolor{currentstroke}%
\pgfsetdash{}{0pt}%
\pgfsys@defobject{currentmarker}{\pgfqpoint{0.000000in}{-0.048611in}}{\pgfqpoint{0.000000in}{0.000000in}}{%
\pgfpathmoveto{\pgfqpoint{0.000000in}{0.000000in}}%
\pgfpathlineto{\pgfqpoint{0.000000in}{-0.048611in}}%
\pgfusepath{stroke,fill}%
}%
\begin{pgfscope}%
\pgfsys@transformshift{0.959013in}{0.467838in}%
\pgfsys@useobject{currentmarker}{}%
\end{pgfscope}%
\end{pgfscope}%
\begin{pgfscope}%
\definecolor{textcolor}{rgb}{0.000000,0.000000,0.000000}%
\pgfsetstrokecolor{textcolor}%
\pgfsetfillcolor{textcolor}%
\pgftext[x=0.959013in,y=0.370616in,,top]{\color{textcolor}\sffamily\fontsize{8.000000}{9.600000}\selectfont 0.5}%
\end{pgfscope}%
\begin{pgfscope}%
\pgfsetbuttcap%
\pgfsetroundjoin%
\definecolor{currentfill}{rgb}{0.000000,0.000000,0.000000}%
\pgfsetfillcolor{currentfill}%
\pgfsetlinewidth{0.803000pt}%
\definecolor{currentstroke}{rgb}{0.000000,0.000000,0.000000}%
\pgfsetstrokecolor{currentstroke}%
\pgfsetdash{}{0pt}%
\pgfsys@defobject{currentmarker}{\pgfqpoint{0.000000in}{-0.048611in}}{\pgfqpoint{0.000000in}{0.000000in}}{%
\pgfpathmoveto{\pgfqpoint{0.000000in}{0.000000in}}%
\pgfpathlineto{\pgfqpoint{0.000000in}{-0.048611in}}%
\pgfusepath{stroke,fill}%
}%
\begin{pgfscope}%
\pgfsys@transformshift{1.465070in}{0.467838in}%
\pgfsys@useobject{currentmarker}{}%
\end{pgfscope}%
\end{pgfscope}%
\begin{pgfscope}%
\definecolor{textcolor}{rgb}{0.000000,0.000000,0.000000}%
\pgfsetstrokecolor{textcolor}%
\pgfsetfillcolor{textcolor}%
\pgftext[x=1.465070in,y=0.370616in,,top]{\color{textcolor}\sffamily\fontsize{8.000000}{9.600000}\selectfont 0.6}%
\end{pgfscope}%
\begin{pgfscope}%
\pgfsetbuttcap%
\pgfsetroundjoin%
\definecolor{currentfill}{rgb}{0.000000,0.000000,0.000000}%
\pgfsetfillcolor{currentfill}%
\pgfsetlinewidth{0.803000pt}%
\definecolor{currentstroke}{rgb}{0.000000,0.000000,0.000000}%
\pgfsetstrokecolor{currentstroke}%
\pgfsetdash{}{0pt}%
\pgfsys@defobject{currentmarker}{\pgfqpoint{0.000000in}{-0.048611in}}{\pgfqpoint{0.000000in}{0.000000in}}{%
\pgfpathmoveto{\pgfqpoint{0.000000in}{0.000000in}}%
\pgfpathlineto{\pgfqpoint{0.000000in}{-0.048611in}}%
\pgfusepath{stroke,fill}%
}%
\begin{pgfscope}%
\pgfsys@transformshift{1.971127in}{0.467838in}%
\pgfsys@useobject{currentmarker}{}%
\end{pgfscope}%
\end{pgfscope}%
\begin{pgfscope}%
\definecolor{textcolor}{rgb}{0.000000,0.000000,0.000000}%
\pgfsetstrokecolor{textcolor}%
\pgfsetfillcolor{textcolor}%
\pgftext[x=1.971127in,y=0.370616in,,top]{\color{textcolor}\sffamily\fontsize{8.000000}{9.600000}\selectfont 0.7}%
\end{pgfscope}%
\begin{pgfscope}%
\pgfsetbuttcap%
\pgfsetroundjoin%
\definecolor{currentfill}{rgb}{0.000000,0.000000,0.000000}%
\pgfsetfillcolor{currentfill}%
\pgfsetlinewidth{0.803000pt}%
\definecolor{currentstroke}{rgb}{0.000000,0.000000,0.000000}%
\pgfsetstrokecolor{currentstroke}%
\pgfsetdash{}{0pt}%
\pgfsys@defobject{currentmarker}{\pgfqpoint{0.000000in}{-0.048611in}}{\pgfqpoint{0.000000in}{0.000000in}}{%
\pgfpathmoveto{\pgfqpoint{0.000000in}{0.000000in}}%
\pgfpathlineto{\pgfqpoint{0.000000in}{-0.048611in}}%
\pgfusepath{stroke,fill}%
}%
\begin{pgfscope}%
\pgfsys@transformshift{2.477184in}{0.467838in}%
\pgfsys@useobject{currentmarker}{}%
\end{pgfscope}%
\end{pgfscope}%
\begin{pgfscope}%
\definecolor{textcolor}{rgb}{0.000000,0.000000,0.000000}%
\pgfsetstrokecolor{textcolor}%
\pgfsetfillcolor{textcolor}%
\pgftext[x=2.477184in,y=0.370616in,,top]{\color{textcolor}\sffamily\fontsize{8.000000}{9.600000}\selectfont 0.8}%
\end{pgfscope}%
\begin{pgfscope}%
\pgfsetbuttcap%
\pgfsetroundjoin%
\definecolor{currentfill}{rgb}{0.000000,0.000000,0.000000}%
\pgfsetfillcolor{currentfill}%
\pgfsetlinewidth{0.803000pt}%
\definecolor{currentstroke}{rgb}{0.000000,0.000000,0.000000}%
\pgfsetstrokecolor{currentstroke}%
\pgfsetdash{}{0pt}%
\pgfsys@defobject{currentmarker}{\pgfqpoint{0.000000in}{-0.048611in}}{\pgfqpoint{0.000000in}{0.000000in}}{%
\pgfpathmoveto{\pgfqpoint{0.000000in}{0.000000in}}%
\pgfpathlineto{\pgfqpoint{0.000000in}{-0.048611in}}%
\pgfusepath{stroke,fill}%
}%
\begin{pgfscope}%
\pgfsys@transformshift{2.983242in}{0.467838in}%
\pgfsys@useobject{currentmarker}{}%
\end{pgfscope}%
\end{pgfscope}%
\begin{pgfscope}%
\definecolor{textcolor}{rgb}{0.000000,0.000000,0.000000}%
\pgfsetstrokecolor{textcolor}%
\pgfsetfillcolor{textcolor}%
\pgftext[x=2.983242in,y=0.370616in,,top]{\color{textcolor}\sffamily\fontsize{8.000000}{9.600000}\selectfont 0.9}%
\end{pgfscope}%
\begin{pgfscope}%
\pgfsetbuttcap%
\pgfsetroundjoin%
\definecolor{currentfill}{rgb}{0.000000,0.000000,0.000000}%
\pgfsetfillcolor{currentfill}%
\pgfsetlinewidth{0.803000pt}%
\definecolor{currentstroke}{rgb}{0.000000,0.000000,0.000000}%
\pgfsetstrokecolor{currentstroke}%
\pgfsetdash{}{0pt}%
\pgfsys@defobject{currentmarker}{\pgfqpoint{0.000000in}{-0.048611in}}{\pgfqpoint{0.000000in}{0.000000in}}{%
\pgfpathmoveto{\pgfqpoint{0.000000in}{0.000000in}}%
\pgfpathlineto{\pgfqpoint{0.000000in}{-0.048611in}}%
\pgfusepath{stroke,fill}%
}%
\begin{pgfscope}%
\pgfsys@transformshift{3.489299in}{0.467838in}%
\pgfsys@useobject{currentmarker}{}%
\end{pgfscope}%
\end{pgfscope}%
\begin{pgfscope}%
\definecolor{textcolor}{rgb}{0.000000,0.000000,0.000000}%
\pgfsetstrokecolor{textcolor}%
\pgfsetfillcolor{textcolor}%
\pgftext[x=3.489299in,y=0.370616in,,top]{\color{textcolor}\sffamily\fontsize{8.000000}{9.600000}\selectfont 1.0}%
\end{pgfscope}%
\begin{pgfscope}%
\definecolor{textcolor}{rgb}{0.000000,0.000000,0.000000}%
\pgfsetstrokecolor{textcolor}%
\pgfsetfillcolor{textcolor}%
\pgftext[x=2.122944in,y=0.207530in,,top]{\color{textcolor}\sffamily\fontsize{8.000000}{9.600000}\selectfont \(\displaystyle \beta\)}%
\end{pgfscope}%
\begin{pgfscope}%
\pgfsetbuttcap%
\pgfsetroundjoin%
\definecolor{currentfill}{rgb}{0.000000,0.000000,0.000000}%
\pgfsetfillcolor{currentfill}%
\pgfsetlinewidth{0.803000pt}%
\definecolor{currentstroke}{rgb}{0.000000,0.000000,0.000000}%
\pgfsetstrokecolor{currentstroke}%
\pgfsetdash{}{0pt}%
\pgfsys@defobject{currentmarker}{\pgfqpoint{-0.048611in}{0.000000in}}{\pgfqpoint{-0.000000in}{0.000000in}}{%
\pgfpathmoveto{\pgfqpoint{-0.000000in}{0.000000in}}%
\pgfpathlineto{\pgfqpoint{-0.048611in}{0.000000in}}%
\pgfusepath{stroke,fill}%
}%
\begin{pgfscope}%
\pgfsys@transformshift{0.643077in}{0.467838in}%
\pgfsys@useobject{currentmarker}{}%
\end{pgfscope}%
\end{pgfscope}%
\begin{pgfscope}%
\definecolor{textcolor}{rgb}{0.000000,0.000000,0.000000}%
\pgfsetstrokecolor{textcolor}%
\pgfsetfillcolor{textcolor}%
\pgftext[x=0.475163in, y=0.425629in, left, base]{\color{textcolor}\sffamily\fontsize{8.000000}{9.600000}\selectfont 0}%
\end{pgfscope}%
\begin{pgfscope}%
\pgfsetbuttcap%
\pgfsetroundjoin%
\definecolor{currentfill}{rgb}{0.000000,0.000000,0.000000}%
\pgfsetfillcolor{currentfill}%
\pgfsetlinewidth{0.803000pt}%
\definecolor{currentstroke}{rgb}{0.000000,0.000000,0.000000}%
\pgfsetstrokecolor{currentstroke}%
\pgfsetdash{}{0pt}%
\pgfsys@defobject{currentmarker}{\pgfqpoint{-0.048611in}{0.000000in}}{\pgfqpoint{-0.000000in}{0.000000in}}{%
\pgfpathmoveto{\pgfqpoint{-0.000000in}{0.000000in}}%
\pgfpathlineto{\pgfqpoint{-0.048611in}{0.000000in}}%
\pgfusepath{stroke,fill}%
}%
\begin{pgfscope}%
\pgfsys@transformshift{0.643077in}{0.824352in}%
\pgfsys@useobject{currentmarker}{}%
\end{pgfscope}%
\end{pgfscope}%
\begin{pgfscope}%
\definecolor{textcolor}{rgb}{0.000000,0.000000,0.000000}%
\pgfsetstrokecolor{textcolor}%
\pgfsetfillcolor{textcolor}%
\pgftext[x=0.333778in, y=0.782142in, left, base]{\color{textcolor}\sffamily\fontsize{8.000000}{9.600000}\selectfont 200}%
\end{pgfscope}%
\begin{pgfscope}%
\pgfsetbuttcap%
\pgfsetroundjoin%
\definecolor{currentfill}{rgb}{0.000000,0.000000,0.000000}%
\pgfsetfillcolor{currentfill}%
\pgfsetlinewidth{0.803000pt}%
\definecolor{currentstroke}{rgb}{0.000000,0.000000,0.000000}%
\pgfsetstrokecolor{currentstroke}%
\pgfsetdash{}{0pt}%
\pgfsys@defobject{currentmarker}{\pgfqpoint{-0.048611in}{0.000000in}}{\pgfqpoint{-0.000000in}{0.000000in}}{%
\pgfpathmoveto{\pgfqpoint{-0.000000in}{0.000000in}}%
\pgfpathlineto{\pgfqpoint{-0.048611in}{0.000000in}}%
\pgfusepath{stroke,fill}%
}%
\begin{pgfscope}%
\pgfsys@transformshift{0.643077in}{1.180865in}%
\pgfsys@useobject{currentmarker}{}%
\end{pgfscope}%
\end{pgfscope}%
\begin{pgfscope}%
\definecolor{textcolor}{rgb}{0.000000,0.000000,0.000000}%
\pgfsetstrokecolor{textcolor}%
\pgfsetfillcolor{textcolor}%
\pgftext[x=0.333778in, y=1.138655in, left, base]{\color{textcolor}\sffamily\fontsize{8.000000}{9.600000}\selectfont 400}%
\end{pgfscope}%
\begin{pgfscope}%
\pgfsetbuttcap%
\pgfsetroundjoin%
\definecolor{currentfill}{rgb}{0.000000,0.000000,0.000000}%
\pgfsetfillcolor{currentfill}%
\pgfsetlinewidth{0.803000pt}%
\definecolor{currentstroke}{rgb}{0.000000,0.000000,0.000000}%
\pgfsetstrokecolor{currentstroke}%
\pgfsetdash{}{0pt}%
\pgfsys@defobject{currentmarker}{\pgfqpoint{-0.048611in}{0.000000in}}{\pgfqpoint{-0.000000in}{0.000000in}}{%
\pgfpathmoveto{\pgfqpoint{-0.000000in}{0.000000in}}%
\pgfpathlineto{\pgfqpoint{-0.048611in}{0.000000in}}%
\pgfusepath{stroke,fill}%
}%
\begin{pgfscope}%
\pgfsys@transformshift{0.643077in}{1.537378in}%
\pgfsys@useobject{currentmarker}{}%
\end{pgfscope}%
\end{pgfscope}%
\begin{pgfscope}%
\definecolor{textcolor}{rgb}{0.000000,0.000000,0.000000}%
\pgfsetstrokecolor{textcolor}%
\pgfsetfillcolor{textcolor}%
\pgftext[x=0.333778in, y=1.495169in, left, base]{\color{textcolor}\sffamily\fontsize{8.000000}{9.600000}\selectfont 600}%
\end{pgfscope}%
\begin{pgfscope}%
\pgfsetbuttcap%
\pgfsetroundjoin%
\definecolor{currentfill}{rgb}{0.000000,0.000000,0.000000}%
\pgfsetfillcolor{currentfill}%
\pgfsetlinewidth{0.803000pt}%
\definecolor{currentstroke}{rgb}{0.000000,0.000000,0.000000}%
\pgfsetstrokecolor{currentstroke}%
\pgfsetdash{}{0pt}%
\pgfsys@defobject{currentmarker}{\pgfqpoint{-0.048611in}{0.000000in}}{\pgfqpoint{-0.000000in}{0.000000in}}{%
\pgfpathmoveto{\pgfqpoint{-0.000000in}{0.000000in}}%
\pgfpathlineto{\pgfqpoint{-0.048611in}{0.000000in}}%
\pgfusepath{stroke,fill}%
}%
\begin{pgfscope}%
\pgfsys@transformshift{0.643077in}{1.893891in}%
\pgfsys@useobject{currentmarker}{}%
\end{pgfscope}%
\end{pgfscope}%
\begin{pgfscope}%
\definecolor{textcolor}{rgb}{0.000000,0.000000,0.000000}%
\pgfsetstrokecolor{textcolor}%
\pgfsetfillcolor{textcolor}%
\pgftext[x=0.333778in, y=1.851682in, left, base]{\color{textcolor}\sffamily\fontsize{8.000000}{9.600000}\selectfont 800}%
\end{pgfscope}%
\begin{pgfscope}%
\pgfsetbuttcap%
\pgfsetroundjoin%
\definecolor{currentfill}{rgb}{0.000000,0.000000,0.000000}%
\pgfsetfillcolor{currentfill}%
\pgfsetlinewidth{0.803000pt}%
\definecolor{currentstroke}{rgb}{0.000000,0.000000,0.000000}%
\pgfsetstrokecolor{currentstroke}%
\pgfsetdash{}{0pt}%
\pgfsys@defobject{currentmarker}{\pgfqpoint{-0.048611in}{0.000000in}}{\pgfqpoint{-0.000000in}{0.000000in}}{%
\pgfpathmoveto{\pgfqpoint{-0.000000in}{0.000000in}}%
\pgfpathlineto{\pgfqpoint{-0.048611in}{0.000000in}}%
\pgfusepath{stroke,fill}%
}%
\begin{pgfscope}%
\pgfsys@transformshift{0.643077in}{2.250404in}%
\pgfsys@useobject{currentmarker}{}%
\end{pgfscope}%
\end{pgfscope}%
\begin{pgfscope}%
\definecolor{textcolor}{rgb}{0.000000,0.000000,0.000000}%
\pgfsetstrokecolor{textcolor}%
\pgfsetfillcolor{textcolor}%
\pgftext[x=0.263086in, y=2.208195in, left, base]{\color{textcolor}\sffamily\fontsize{8.000000}{9.600000}\selectfont 1000}%
\end{pgfscope}%
\begin{pgfscope}%
\definecolor{textcolor}{rgb}{0.000000,0.000000,0.000000}%
\pgfsetstrokecolor{textcolor}%
\pgfsetfillcolor{textcolor}%
\pgftext[x=0.207530in,y=1.376546in,,bottom,rotate=90.000000]{\color{textcolor}\sffamily\fontsize{8.000000}{9.600000}\selectfont count}%
\end{pgfscope}%
\begin{pgfscope}%
\pgfsetrectcap%
\pgfsetmiterjoin%
\pgfsetlinewidth{0.803000pt}%
\definecolor{currentstroke}{rgb}{0.000000,0.000000,0.000000}%
\pgfsetstrokecolor{currentstroke}%
\pgfsetdash{}{0pt}%
\pgfpathmoveto{\pgfqpoint{0.643077in}{0.467838in}}%
\pgfpathlineto{\pgfqpoint{0.643077in}{2.285253in}}%
\pgfusepath{stroke}%
\end{pgfscope}%
\begin{pgfscope}%
\pgfsetrectcap%
\pgfsetmiterjoin%
\pgfsetlinewidth{0.803000pt}%
\definecolor{currentstroke}{rgb}{0.000000,0.000000,0.000000}%
\pgfsetstrokecolor{currentstroke}%
\pgfsetdash{}{0pt}%
\pgfpathmoveto{\pgfqpoint{3.602812in}{0.467838in}}%
\pgfpathlineto{\pgfqpoint{3.602812in}{2.285253in}}%
\pgfusepath{stroke}%
\end{pgfscope}%
\begin{pgfscope}%
\pgfsetrectcap%
\pgfsetmiterjoin%
\pgfsetlinewidth{0.803000pt}%
\definecolor{currentstroke}{rgb}{0.000000,0.000000,0.000000}%
\pgfsetstrokecolor{currentstroke}%
\pgfsetdash{}{0pt}%
\pgfpathmoveto{\pgfqpoint{0.643077in}{0.467838in}}%
\pgfpathlineto{\pgfqpoint{3.602812in}{0.467838in}}%
\pgfusepath{stroke}%
\end{pgfscope}%
\begin{pgfscope}%
\pgfsetrectcap%
\pgfsetmiterjoin%
\pgfsetlinewidth{0.803000pt}%
\definecolor{currentstroke}{rgb}{0.000000,0.000000,0.000000}%
\pgfsetstrokecolor{currentstroke}%
\pgfsetdash{}{0pt}%
\pgfpathmoveto{\pgfqpoint{0.643077in}{2.285253in}}%
\pgfpathlineto{\pgfqpoint{3.602812in}{2.285253in}}%
\pgfusepath{stroke}%
\end{pgfscope}%
\begin{pgfscope}%
\pgfsetbuttcap%
\pgfsetmiterjoin%
\definecolor{currentfill}{rgb}{1.000000,1.000000,1.000000}%
\pgfsetfillcolor{currentfill}%
\pgfsetfillopacity{0.800000}%
\pgfsetlinewidth{1.003750pt}%
\definecolor{currentstroke}{rgb}{0.800000,0.800000,0.800000}%
\pgfsetstrokecolor{currentstroke}%
\pgfsetstrokeopacity{0.800000}%
\pgfsetdash{}{0pt}%
\pgfpathmoveto{\pgfqpoint{0.720855in}{1.544021in}}%
\pgfpathlineto{\pgfqpoint{1.514182in}{1.544021in}}%
\pgfpathquadraticcurveto{\pgfqpoint{1.536404in}{1.544021in}}{\pgfqpoint{1.536404in}{1.566243in}}%
\pgfpathlineto{\pgfqpoint{1.536404in}{2.207476in}}%
\pgfpathquadraticcurveto{\pgfqpoint{1.536404in}{2.229698in}}{\pgfqpoint{1.514182in}{2.229698in}}%
\pgfpathlineto{\pgfqpoint{0.720855in}{2.229698in}}%
\pgfpathquadraticcurveto{\pgfqpoint{0.698633in}{2.229698in}}{\pgfqpoint{0.698633in}{2.207476in}}%
\pgfpathlineto{\pgfqpoint{0.698633in}{1.566243in}}%
\pgfpathquadraticcurveto{\pgfqpoint{0.698633in}{1.544021in}}{\pgfqpoint{0.720855in}{1.544021in}}%
\pgfpathlineto{\pgfqpoint{0.720855in}{1.544021in}}%
\pgfpathclose%
\pgfusepath{stroke,fill}%
\end{pgfscope}%
\begin{pgfscope}%
\pgfsetbuttcap%
\pgfsetmiterjoin%
\definecolor{currentfill}{rgb}{0.121569,0.466667,0.705882}%
\pgfsetfillcolor{currentfill}%
\pgfsetlinewidth{0.000000pt}%
\definecolor{currentstroke}{rgb}{0.000000,0.000000,0.000000}%
\pgfsetstrokecolor{currentstroke}%
\pgfsetstrokeopacity{0.000000}%
\pgfsetdash{}{0pt}%
\pgfpathmoveto{\pgfqpoint{0.743077in}{2.100835in}}%
\pgfpathlineto{\pgfqpoint{0.965299in}{2.100835in}}%
\pgfpathlineto{\pgfqpoint{0.965299in}{2.178613in}}%
\pgfpathlineto{\pgfqpoint{0.743077in}{2.178613in}}%
\pgfpathlineto{\pgfqpoint{0.743077in}{2.100835in}}%
\pgfpathclose%
\pgfusepath{fill}%
\end{pgfscope}%
\begin{pgfscope}%
\definecolor{textcolor}{rgb}{0.000000,0.000000,0.000000}%
\pgfsetstrokecolor{textcolor}%
\pgfsetfillcolor{textcolor}%
\pgftext[x=1.054188in,y=2.100835in,left,base]{\color{textcolor}\sffamily\fontsize{8.000000}{9.600000}\selectfont L=250}%
\end{pgfscope}%
\begin{pgfscope}%
\pgfsetbuttcap%
\pgfsetmiterjoin%
\definecolor{currentfill}{rgb}{1.000000,0.498039,0.054902}%
\pgfsetfillcolor{currentfill}%
\pgfsetlinewidth{0.000000pt}%
\definecolor{currentstroke}{rgb}{0.000000,0.000000,0.000000}%
\pgfsetstrokecolor{currentstroke}%
\pgfsetstrokeopacity{0.000000}%
\pgfsetdash{}{0pt}%
\pgfpathmoveto{\pgfqpoint{0.743077in}{1.937749in}}%
\pgfpathlineto{\pgfqpoint{0.965299in}{1.937749in}}%
\pgfpathlineto{\pgfqpoint{0.965299in}{2.015527in}}%
\pgfpathlineto{\pgfqpoint{0.743077in}{2.015527in}}%
\pgfpathlineto{\pgfqpoint{0.743077in}{1.937749in}}%
\pgfpathclose%
\pgfusepath{fill}%
\end{pgfscope}%
\begin{pgfscope}%
\definecolor{textcolor}{rgb}{0.000000,0.000000,0.000000}%
\pgfsetstrokecolor{textcolor}%
\pgfsetfillcolor{textcolor}%
\pgftext[x=1.054188in,y=1.937749in,left,base]{\color{textcolor}\sffamily\fontsize{8.000000}{9.600000}\selectfont L=500}%
\end{pgfscope}%
\begin{pgfscope}%
\pgfsetbuttcap%
\pgfsetmiterjoin%
\definecolor{currentfill}{rgb}{0.172549,0.627451,0.172549}%
\pgfsetfillcolor{currentfill}%
\pgfsetlinewidth{0.000000pt}%
\definecolor{currentstroke}{rgb}{0.000000,0.000000,0.000000}%
\pgfsetstrokecolor{currentstroke}%
\pgfsetstrokeopacity{0.000000}%
\pgfsetdash{}{0pt}%
\pgfpathmoveto{\pgfqpoint{0.743077in}{1.774663in}}%
\pgfpathlineto{\pgfqpoint{0.965299in}{1.774663in}}%
\pgfpathlineto{\pgfqpoint{0.965299in}{1.852441in}}%
\pgfpathlineto{\pgfqpoint{0.743077in}{1.852441in}}%
\pgfpathlineto{\pgfqpoint{0.743077in}{1.774663in}}%
\pgfpathclose%
\pgfusepath{fill}%
\end{pgfscope}%
\begin{pgfscope}%
\definecolor{textcolor}{rgb}{0.000000,0.000000,0.000000}%
\pgfsetstrokecolor{textcolor}%
\pgfsetfillcolor{textcolor}%
\pgftext[x=1.054188in,y=1.774663in,left,base]{\color{textcolor}\sffamily\fontsize{8.000000}{9.600000}\selectfont L=1000}%
\end{pgfscope}%
\begin{pgfscope}%
\pgfsetbuttcap%
\pgfsetmiterjoin%
\definecolor{currentfill}{rgb}{0.839216,0.152941,0.156863}%
\pgfsetfillcolor{currentfill}%
\pgfsetlinewidth{0.000000pt}%
\definecolor{currentstroke}{rgb}{0.000000,0.000000,0.000000}%
\pgfsetstrokecolor{currentstroke}%
\pgfsetstrokeopacity{0.000000}%
\pgfsetdash{}{0pt}%
\pgfpathmoveto{\pgfqpoint{0.743077in}{1.611577in}}%
\pgfpathlineto{\pgfqpoint{0.965299in}{1.611577in}}%
\pgfpathlineto{\pgfqpoint{0.965299in}{1.689355in}}%
\pgfpathlineto{\pgfqpoint{0.743077in}{1.689355in}}%
\pgfpathlineto{\pgfqpoint{0.743077in}{1.611577in}}%
\pgfpathclose%
\pgfusepath{fill}%
\end{pgfscope}%
\begin{pgfscope}%
\definecolor{textcolor}{rgb}{0.000000,0.000000,0.000000}%
\pgfsetstrokecolor{textcolor}%
\pgfsetfillcolor{textcolor}%
\pgftext[x=1.054188in,y=1.611577in,left,base]{\color{textcolor}\sffamily\fontsize{8.000000}{9.600000}\selectfont L=2000}%
\end{pgfscope}%
\end{pgfpicture}%
\makeatother%
\endgroup%

	\caption{Распределение пиков магнитной восприимчивости}
	\label{fig:MS_peaks_distr}
\end{figure}

\section{Кластеризованные конформации}
Под кластеризованными конформации имеются ввиду конформации, у которых есть плотные, двумерные кластеры, соединённые друг с другом одномерными сегментами (мостами). В качестве модели таких конформаций мы рассматриваем конформации состоящие из прямоугольных кластеров, соединённых прямыми сегментами. Пример такой конформации представлен на Рис \ref{fig:cluster_conf_example}.

\begin{figure}[ht]
	\centering
	\includegraphics*[width=0.4\textwidth]{../images/3Cluster_conformation_short.png}
	\caption{Пример модели кластеризованной конформации}
	\label{fig:cluster_conf_example}
\end{figure}

Для начала я написал простой алгоритм генерации подобных конформаций, укладывающий конформацию внутри кластеров "змейкой" и соединяющий кластеры мостом снизу.
Я сгенерировал конформации с размерами кластеров: 250, 500, 1000, 2000, количеством кластеров: 2, 3, 4, и длинами мостов между ними от 1 до 10.

\begin{figure}[ht]
	\centering
    \begin{subfigure}[t]{0.3\textwidth}
        \includegraphics*[width=\textwidth]{../images/magnetization_clusterized_W20_H50_N2.png}
        \caption*{2 кластера}
    \end{subfigure}
    \begin{subfigure}[t]{0.3\textwidth}
        \includegraphics*[width=\textwidth]{../images/magnetization_clusterized_W20_H50_N3.png}
        \caption*{3 кластера}

    \end{subfigure}
    \begin{subfigure}[t]{0.3\textwidth}
        \includegraphics*[width=\textwidth]{../images/magnetization_clusterized_W20_H50_N4.png}
        \caption*{4 кластера}
    \end{subfigure}
	\caption{Квадрат намагниченность конформаций с кластерами размера 2000, цветами отмечена длина мостов между кластерами}
	\label{fig:cluster_magnetization}
\end{figure}

Поведение намагниченности у конформаций мало различается при изменении размеров кластеров.
Как видно на рис. \ref{fig:cluster_magnetization}, увеличение как количества кластеров, так и длинны мостов между ними, уменьшает намагниченность конформации. Так же уже при наличии двух кластеров, намагниенность конформации оказывается значительно ниже чем 1.0.

Не смотря на более низкую намагниченность чем у обычной двумерной решётки, у данных конформаций всё ещё присутствует резкое изменение в поведении в том месте, где мы предполагаем точку перехода для конформаций вида глобула.

Так же мы рассмотрели магнитную восприимчивость конформаций

\begin{figure}[ht]
	\centering
    \begin{subfigure}[t]{0.3\textwidth}
        \includegraphics*[width=\textwidth]{../images/magnetic_susceptibility/clusterized_W20_H50_N2.png}
        \caption*{2 кластера}
    \end{subfigure}
    \begin{subfigure}[t]{0.3\textwidth}
        \includegraphics*[width=\textwidth]{../images/magnetic_susceptibility/clusterized_W20_H50_N3.png}
        \caption*{3 кластера}

    \end{subfigure}
    \begin{subfigure}[t]{0.3\textwidth}
        \includegraphics*[width=\textwidth]{../images/magnetic_susceptibility/clusterized_W20_H50_N4.png}
        \caption*{4 кластера}
    \end{subfigure}
	\caption{агнитная восприимчивость конформаций с кластерами размера 2000, цветами отмечена длина мостов между кластерами}
	\label{fig:cluster_magnetc_sus}
\end{figure}

Магнитная восприимчивость так же как и намагниченность имеет схожее поведение для различных размеров кластеров. Однако на рис. \ref*{fig:cluster_magnetc_sus} видно сильное отличие между конформациями с двумя кластерами и конформациями с 3 и 4 кластерами. Главное из которых заключатся в увеличении магнитной восприимчивости при удлинении мостов между кластерами.

Мы предположли что данная особенность может возникать из-за нашего способа генерации конформаций, а именно из-за расположения мостов в конформации. Ведь во всех рассмотреных конформациях мосты всегда ноходятся снизу кластеров.

Чтобы проверить это было необходимо изменить алгоритм генерации конформаций так, чтобы мосты можно было размещать не только с одной стороны конформации, но в любом желаемом нам месте.
Несколько попыток написать такой алгоритм, оставляя при этом конформацию в виде не самопересекающейся последовательности вершин оказались неудачными. Затем мы доказали, что для некоторых размеров кластеров, являются невозможными определённые варианты расположения мостов.

Например если левый мост входит в кластер на высоте $p_1$, а правый мост на высоте $p_2$ 
и кластер имеет ширину $W$ вершин и высоту $H$ вершин, тогда $p_2 - p_1 \equiv (H-1)W \mod 2$.

Доказательство: 
Пусть внутри конформации последовательность вершин делает $l$ шагов влево, $r$ шагов вправо, $u$ шагов вверх, $d$ шагов вниз. Всего шагов внутри кластера: $ W H - 1 = l + r + u + d $. При этом $W - 1 = r - l$ и $p_2 - p_1 = u - d$. Заметим что $ r + l = r - l + 2l \equiv r - l \mod 2$ аналогично с $u+d \equiv u-d$. Тогда

\[
    H W - 1 = l + r + u + d \equiv r - l + u - d = W - 1 + p_2 - p_1 
\]
\[
    W H - 1 \equiv W - 1 + p_2 - p_1
\]
\[    
    (H - 1)W \equiv p_2 - p_1
\]

В итоге с учётом найденных ограничений было решено строить структуру модели, пренебрегая возможностью провести через неё не самопересекающуюся последовательность. 
Используя новый алгоритм были сгенерированы ещё два набора конформаций. В первом наборе мосты чередуют своё расположение: сверху, снизу... И набор со случайным расположением мостов в конформациях.

Однако в обоих наборах отличия значений от прошлого нобора были незначительными, в пределах погрешности. Поведение конформаций с 3 и 4 кластерами почти не изменилось(рис. \ref*{fig:cluster_ud_magnetc_sus}). 

\begin{figure}[ht]
	\centering
    \begin{subfigure}[t]{0.3\textwidth}
        \includegraphics*[width=\textwidth]{../images/magnetic_susceptibility/clusterized_W41_H41_N2.png}
        \caption*{2 кластера}
    \end{subfigure}
    \begin{subfigure}[t]{0.3\textwidth}
        \includegraphics*[width=\textwidth]{../images/magnetic_susceptibility/clusterized_W41_H41_N3.png}
        \caption*{3 кластера}
    \end{subfigure}
    \begin{subfigure}[t]{0.3\textwidth}
        \includegraphics*[width=\textwidth]{../images/magnetic_susceptibility/clusterized_W41_H41_N4.png}
        \caption*{4 кластера}
    \end{subfigure}
	\caption{агнитная восприимчивость конформаций с чередующимися мостами, цветами отмечена длина мостов между кластерами}
	\label{fig:cluster_ud_magnetc_sus}
\end{figure}

Из полученных результатов можно сделать вывод, что расположение мостов не имеет большого влияния на свойства конформации. Наибольшее влияние имеет количество или относительные размеры кластеров.


\section{Заключение}
Результаты полученные для конформаций вида клубок, соответствуют полученным ранее результатам, подтверждают отсутствие фазового перехода и демонстрируют схожесть конформаций данного типа с одномерной цепочкой. Исследование кластеризованных конформаций подтверждает влияние кластеров на магнитные свойства конформации. Для объяснения особенностей магнитной восприимчивости в конформациях с двумя кластерами требуются дополнительные исследования. 

Данные результаты и подготовленные данные будут использованы далее в ВКР.

В ходе преддипломной практики мной были получены новые навыки работы с суперкомпьютером, опыт в разработке и прототипировании алгоритмов, с теоретическим обоснованием возможности их осуществления.

\begin{thebibliography}{9}
\bibitem{github}
 Ссылка на репозиторий:https://github.com/MoskalenkoRomanBorisovich/Ising-on-random-conformation 
 
\bibitem{ising}
Ernst Ising, Contribution to the Theory of Ferromagnetism, 1925


\bibitem{wolf_algorithm}
 U. Wolff, Collective Monte Carlo Updating for Spin Systems. Physical Review Letters. 62 (4): 361–364, 1989
 
\end{thebibliography}

\end{document}